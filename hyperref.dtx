% \iffalse
%% File: hyperref.dtx Copyright 1995-2001 Sebastian Rahtz,
% with portions written by David Carlisle and Heiko Oberdiek.
%% RCS: $Id: hyperref.dtx 6.71 2000/10/04 rahtz Exp rahtz $
%%
%% This file is part of the `Hyperref Bundle'.
%% -------------------------------------------
%%
%% It may be distributed under the conditions of the LaTeX Project Public
%% License, either version 1.2 of this license or (at your option) any
%% later version.  The latest version of this license is in
%%    http://www.latex-project.org/lppl.txt
%% and version 1.2 or later is part of all distributions of LaTeX
%% version 1999/12/01 or later.
%%
%% The list of all files belonging to the `Hyperref Bundle' is
%% given in the file `manifest.txt'.
%%
%<package|nohyperref|driver|check>\NeedsTeXFormat{LaTeX2e}
%<package>\ProvidesPackage{hyperref}
%<nohyperref>\ProvidesPackage{nohyperref}
%<driver>\ProvidesFile{hyperref.drv}
%<check>\ProvidesFile{hycheck.tex}
%<hypertex>\ProvidesFile{hypertex.def}
%<pdftex>\ProvidesFile{hpdftex.def}
%<pdfmark>\ProvidesFile{pdfmark.def}
%<vtexpdfmark>\ProvidesFile{hvtexmrk.def}
%<dvips>\ProvidesFile{hdvips.def}
%<dvipsone>\ProvidesFile{hdvipson.def}
%<textures>\ProvidesFile{htexture.def}
%<dviwindo>\ProvidesFile{hdviwind.def}
%<dvipdfm>\ProvidesFile{hdvipdfm.def}
%<dvipdf>\ProvidesFile{hdvipdf.def}
%<vtex>\ProvidesFile{hvtex.def}
%<vtexhtml>\ProvidesFile{hvtexhtml.def}
%<tex4ht>\ProvidesFile{htex4ht.def}
%<tex4htcfg>\ProvidesFile{htex4ht.cfg}
%<pd1enc>\ProvidesFile{pd1enc.def}
%<puenc>\ProvidesFile{puenc.def}
%<!none>  [2002/09/12 v6.72y
%<package>  Hypertext links for LaTeX]
%<nohyperref>  Dummy hyperref (SR)]
%<driver>  Hyperref documentation driver file]
%<check>  Hyperref test file]
%<hypertex>  Hyperref driver for HyperTeX specials]
%<pdftex>  Hyperref driver for pdfTeX]
%<pdfmark>  Hyperref definitions for pdfmark specials]
%<vtexpdfmark> Hyperref driver for VTeX in PDF/PS mode (pdfmark specials)]
%<dvips>  Hyperref driver for dvips]
%<dvipsone>  Hyperref driver for dvipsone]
%<textures>  Hyperref driver for Textures]
%<dviwindo>  Hyperref driver for dviwindo]
%<dvipdfm>  Hyperref driver for dvipdfm]
%<dvipdf>  Hyperref driver for dvipdf]
%<vtex>  Hyperref driver for VTeX in PDF/PS mode]
%<vtexhtml>  Hyperref driver for VTeX in HTML mode]
%<tex4ht>  Hyperref driver for TeX4ht]
%<tex4htcfg>  Hyperref configuration file for TeX4ht]
%<pd1enc>  Hyperref: PDFDocEncoding definition (HO)]
%<puenc>  Hyperref: PDF Unicode definition (HO)]
%<*driver>
\documentclass{ltxdoc}
\usepackage{array,times}
\def\ttdefault{cmtt}
\usepackage[colorlinks,hyperindex]{hyperref}
\pdfstringdefDisableCommands{\let\\\textbackslash}%
\EnableCrossrefs
\CodelineIndex
\begin{document}
  \GetFileInfo{hyperref.sty}
  \title{Hypertext marks in \LaTeX}
  \author{Sebastian Rahtz\\
    Email: \texttt{sebastian.rahtz@oucs.ox.ac.uk}}
  \date{processed \today}
  \maketitle
  \tableofcontents
  \let\Email\texttt
  \DocInput{hyperref.dtx}
  \PrintIndex
\end{document}
%</driver>
% \fi
% \CheckSum{16604}
%
% \MakeShortVerb{|}
% \StopEventually{}
%
% \DoNotIndex{\def,\edef,\gdef,\xdef,\global,\long,\let}
% \DoNotIndex{\expandafter,\string,\the,\ifx,\else,\fi}
% \DoNotIndex{\csname,\endcsname,\relax,\begingroup,\endgroup}
% \DoNotIndex{\DeclareTextCommand,\DeclareTextCompositeCommand}
% \DoNotIndex{\space,\@empty,\special}
%
% \section{File hycheck.tex}
%
%    Many commands of \LaTeX\ or other packages cannot be
%    overloaded, but have to be redefined by hyperref directly.
%    If these commands change in newer versions, these
%    changes are not noticed by hyperref.
%    With this test file this situation can be checked.
%    It defines the command \cmd{\checkcommand} that
%    is more powerful than \LaTeX's \cmd{\CheckCommand},
%    because it takes \cmd{\DeclareRobustCommand} and
%    optional parameters better into account.
%
%    \begin{macrocode}
%<*check>
\documentclass{article}
\makeatletter
%    \end{macrocode}
%
%    \begin{macro}{\checklatex}
%    Optional argument: release date of \LaTeX.
%    \begin{macrocode}
\newcommand*{\checklatex}[1][]{%
  \typeout{}%
  \typeout{* Format: `LaTeX2e' #1}%
  \typeout{\space\space Loaded: `\fmtname' \fmtversion}%
}%
%    \end{macrocode}
%    \end{macro}
%
%    \begin{macro}{\checkpackage}
%    The argument of \cmd{\checkpackage} is the package name
%    without extension optionally followed by a release date.
%    \begin{macrocode}
\newcommand*{\checkpackage}[1]{%
  \def\HyC@package{#1}%
  \let\HyC@date\@empty
  \@ifnextchar[\HyC@getDate\HyC@checkPackage
}
%    \end{macrocode}
%    \end{macro}
%    \begin{macro}{\HyC@getDate}
%    The release date is scanned.
%    \begin{macrocode}
\def\HyC@getDate[#1]{%
  \def\HyC@date{#1}%
  \HyC@checkPackage
}
%    \end{macrocode}
%    \end{macro}
%    \begin{macro}{\HyC@checkPackage}
%    \begin{macrocode}
\def\HyC@checkPackage{%
  \typeout{}
  \begingroup
    \edef\x{\endgroup
      \noexpand\RequirePackage{\HyC@package}%
      \ifx\HyC@date\@empty\relax\else[\HyC@date]\fi%
    }%
  \x
  \typeout{}%
  \typeout{%
    * Package `\HyC@package'%
    \ifx\HyC@date\@empty
    \else
      \space\HyC@date
    \fi
  }%
  \@ifundefined{ver@\HyC@package.sty}{%
  }{%
    \typeout{%
      \space\space Loaded: `\HyC@package' %
      \csname ver@\HyC@package.sty\endcsname
    }%
  }%
}
%    \end{macrocode}
%    \end{macro}
%
%    \begin{macro}{\checkcommand}
%    The macro \cmd{\checkcommand} parses the next
%    tokens as a \LaTeX\ definition and compares
%    this definition with the current meaning of
%    that command.
%    \begin{macrocode}
\newcommand*{\checkcommand}[1]{%
  \begingroup
  \ifx\long#1\relax
    \expandafter\HyC@checklong
  \else
    \def\HyC@defcmd{#1}%
    \expandafter\let\expandafter\HyC@next
      \csname HyC@\expandafter\@gobble\string#1\endcsname
    \expandafter\HyC@checkcommand
  \fi
}
%    \end{macrocode}
%    \end{macro}
%    \begin{macro}{\HyC@checklong}
%    The definition command \cmd{\def} or \cmd{\edef}
%    is read.
%    \begin{macrocode}
\def\HyC@checklong#1{%
  \def\HyC@defcmd{\long#1}%
  \expandafter\let\expandafter\HyC@next
    \csname HyC@\expandafter\@gobble\string#1\endcsname
  \HyC@checkcommand
}
%    \end{macrocode}
%    \end{macro}
%    \begin{macro}{\HyC@checkcommand}
%    The optional star of \LaTeX's definitions is parsed.
%    \begin{macrocode}
\def\HyC@checkcommand{%
  \ifx\HyC@next\relax
    \PackageError{hycheck}{%
      Unknown command `\expandafter\strip@prefix\meaning\HyC@cmd'%
    }\@ehd
    \expandafter\endinput
  \fi
  \@ifstar{%
    \def\HyC@star{*}%
    \HyC@check
  }{%
    \let\HyC@star\@empty
    \HyC@check
  }%
}
%    \end{macrocode}
%    \end{macro}
%    \begin{macro}{\HyC@check}
%    The macro \cmd{\HyC@check} reads the
%    definition command.
%    \begin{macrocode}
\def\HyC@check#1{%
  \def\HyC@cmd{#1}%
  \let\HyC@org@cmd#1%
  \let#1\relax
  \let\HyC@param\@empty
  \HyC@Toks{}%
  \let\HyC@org@optcmd\HyC@noValue
  \let\HyC@org@robustcmd\HyC@noValue
  \let\HyC@org@robustoptcmd\HyC@noValue
  \HyC@next
}
%    \end{macrocode}
%    \end{macro}
%    \begin{macro}{\HyC@noValue}
%    \begin{macrocode}
\def\HyC@noValue{NoValue}
%    \end{macrocode}
%    \end{macro}
%    \begin{macro}{\HyC@newcommand}
%    The code for \cmd{\newcommand}.
%    \begin{macrocode}
\def\HyC@newcommand{%
  \let\HyC@@cmd\HyC@cmd
  \@ifnextchar[\HyC@nc@opt\HyC@nc@noopt
}
%    \end{macrocode}
%    \end{macro}
%    \begin{macro}{\HyC@Toks}
%    A register for storing the default value of an
%    optional argument.
%    \begin{macrocode}
\newtoks\HyC@Toks
%    \end{macrocode}
%    \end{macro}
%    \begin{macro}{\HyC@nc@noopt}
%    This macro \cmd{\HyC@nc@noopt} is called, if the
%    parser has reached the definition text.
%    \begin{macrocode}
\long\def\HyC@nc@noopt#1{%
  \edef\x{%
    \expandafter\noexpand\HyC@defcmd
    \HyC@star
    \expandafter\noexpand\HyC@cmd
    \HyC@param\the\HyC@Toks
  }%
  \x{#1}%
  \HyC@doCheck
}
%    \end{macrocode}
%    \end{macro}
%    \begin{macro}{\HyC@nc@opt}
%    This macro scans the first optional argument
%    of a \LaTeX\ definition (number of arguments).
%    \begin{macrocode}
\def\HyC@nc@opt[#1]{%
  \def\HyC@param{[#1]}%
  \@ifnextchar[\HyC@nc@default\HyC@nc@noopt
}
%    \end{macrocode}
%    \end{macro}
%    \begin{macro}{\HyC@nc@default}
%    Macro \cmd{\HyC@nc@default} scans the
%    default for an optional argument.
%    \begin{macrocode}
\def\HyC@nc@default[#1]{%
  \HyC@Toks={[{#1}]}%
  \edef\HyC@optcmd{%
    \expandafter\noexpand
    \csname\expandafter\string\HyC@@cmd\endcsname
  }%
  \expandafter\let\expandafter\HyC@org@optcmd\HyC@optcmd
  \HyC@nc@noopt
}
%    \end{macrocode}
%    \end{macro}
%
%    \begin{macro}{\HyC@DeclareRobustCommand}
%    \cmd{\DeclareRobustCommand}|{\cmd}| makes the command
%    \cmd{\cmd} robust, that then calls |\cmd|\verb*| |
%    with an space at the end of the command name, defined by
%    \cmd{\newcommand}. Therefore the further parsing
%    is done by \cmd{\HyC@nc@opt} or \cmd{\Hy@nc@noopt}
%    of the \cmd{\HyC@newcommand} chain.
%    \begin{macrocode}
\def\HyC@DeclareRobustCommand{%
  \edef\HyC@robustcmd{%
    \expandafter\noexpand
    \csname\expandafter\expandafter\expandafter\@gobble
      \expandafter\string\HyC@cmd\space\endcsname
  }%
  \expandafter\let\expandafter\HyC@org@robustcmd\HyC@robustcmd
  \expandafter\let\HyC@robustcmd\relax
  \let\HyC@@cmd\HyC@robustcmd
  \@ifnextchar[\HyC@nc@opt\HyC@nc@noopt
}
%    \end{macrocode}
%    \end{macro}
%
%    \begin{macro}{\HyC@def}
%    \begin{macro}{\HyC@edef}
%    The parameter text of \cmd{\def} or \cmd{\edef} is
%    stored in the token register \cmd{\HyC@Toks}.
%    \begin{macrocode}
\def\HyC@def#1#{%
  \HyC@Toks={#1}%
  \HyC@nc@noopt
}
\let\HyC@edef\HyC@def
%    \end{macrocode}
%    \end{macro}
%    \end{macro}
%
%    \begin{macro}{\HyC@doCheck}
%    This command performs the checks and prints the result.
%    \begin{macrocode}
\def\HyC@doCheck{%
  \typeout{* Checking `\HyC@string\HyC@cmd':}%
  \HyC@checkItem{cmd}%
  \HyC@checkItem{robustcmd}%
  \HyC@checkItem{optcmd}%
  \HyC@checkItem{robustoptcmd}%
  \endgroup
}
%    \end{macrocode}
%    \end{macro}
%    \begin{macro}{\HyC@checkItem}
%    A single check.
%    \begin{macrocode}
\def\HyC@checkItem#1{%
  \expandafter\ifx\csname HyC@org@#1\endcsname\HyC@noValue
  \else
    \expandafter\expandafter\expandafter\ifx
    \csname HyC@#1\expandafter\endcsname
    \csname HyC@org@#1\endcsname
      \expandafter\HyC@checkOk\csname HyC@#1\endcsname
    \else
      \expandafter\HyC@checkFailed
        \csname HyC@#1\expandafter\endcsname
        \csname HyC@org@#1\endcsname
    \fi
  \fi
}
%    \end{macrocode}
%    \end{macro}
%    \begin{macro}{\HyC@string}
%    \begin{macro}{\HyC@meaning}
%    Some shorthands.
%    \begin{macrocode}
\def\HyC@string#1{\expandafter\string#1}
\def\HyC@meaning#1{\expandafter\meaning#1}
%    \end{macrocode}
%    \end{macro}
%    \end{macro}
%    \begin{macro}{\HyC@checkOk}
%    The result, if the check succeeds.
%    \begin{macrocode}
\def\HyC@checkOk#1{%
  \typeout{\space\space`\HyC@string#1' ok.}}
%    \end{macrocode}
%    \end{macro}
%    \begin{macro}{\HyC@checkFailed}
%    The result, if the check fails.
%    \begin{macrocode}
\def\HyC@checkFailed#1#2{%
  \typeout{\space\space`\HyC@string#1' failed.}%
  \typeout{\space\space* original: \meaning#2}%
  \typeout{\space\space* expected: \HyC@meaning#1}%
}
%    \end{macrocode}
%    \end{macro}
%    \begin{macrocode}
% **************************************************
%</check>
%    \end{macrocode}
%
%    \begin{macrocode}
%<*package>
%    \end{macrocode}
% \section{Package options and setup}\label{options}
%
% \subsection{Save catcodes}
%    There are many packages that change the standard catcodes.
%
%    First we save the original meaning of |`| and |=|
%    in the token register |\toks@|, because we need the two
%    characters in the macros \cmd{\Hy@SetCatcodes} and
%    \cmd{\Hy@RestoreCatcodes}.
%    \begin{macrocode}
\begingroup
  \@makeother\`%
  \@makeother\=%
  \edef\x{%
    \edef\noexpand\x{%
      \endgroup
      \noexpand\toks@{%
        \catcode 96=\noexpand\the\catcode`\noexpand\`\relax
        \catcode 61=\noexpand\the\catcode`\noexpand\=\relax
      }%
    }%
    \noexpand\x
  }%
\x
\@makeother\`
\@makeother\=
%    \end{macrocode}
%    \begin{macro}{\Hy@SetCatcodes}
%    \begin{macrocode}
\def\Hy@SetCatcodes{%
  \@makeother\`%
  \@makeother\=%
  \catcode`\$=3
  \catcode`\&=4
  \catcode`\^=7
  \catcode`\_=8
  \@makeother\|%
  \@makeother\:%
  \@makeother\(%
  \@makeother\)%
  \@makeother\[%
  \@makeother\]%
  \@makeother\/%
  \@makeother\!%
  \@makeother\<%
  \@makeother\>%
  \@makeother\.%
  \@makeother\;%
  \@makeother\+%
  \@makeother\-%
  \@makeother\"%
  \@makeother\'%
}
%    \end{macrocode}
%    \end{macro}
%    \begin{macro}{\Hy@RestoreCatcodes}
%    \begin{macrocode}
\begingroup
  \def\x#1{\catcode`\noexpand#1=\the\catcode`#1\relax}%
  \xdef\Hy@RestoreCatcodes{%
    \the\toks@
    \x\$%
    \x\&%
    \x\^%
    \x\_%
    \x\|%
    \x\:%
    \x\(%
    \x\)%
    \x\[%
    \x\]%
    \x\/%
    \x\!%
    \x\<%
    \x\>%
    \x\.%
    \x\;%
    \x\+%
    \x\-%
    \x\"%
    \x\'%
  }%
\endgroup
%    \end{macrocode}
%    \end{macro}
%    \begin{macrocode}
\Hy@SetCatcodes
%    \end{macrocode}
%
% It needs the December 95 release of \LaTeX, because it uses
% |\protected@write|, and it defines commands in options; and the page
% setup internal code changed at that point. It'll probably break
% with the later releases!
%    \begin{macrocode}
\RequirePackage{keyval}[1997/11/10]
\def\Hy@Warning#1{\PackageWarning{hyperref}{#1}}
\def\Hy@WarningNoLine#1{\PackageWarningNoLine{hyperref}{#1}}
\def\Hy@Info#1{\PackageInfo{hyperref}{#1}}
%    \end{macrocode}
%
% \subsection{Compatibility with format dumps}
%
%    \begin{macro}{\AfterBeginDocument}
%    For use with pre-compiled formats, created using the
%    |ldump|  package, there needs to be 2 hooks for adding
%    material delayed until |\begin{document}|.
%    These are called \cmd{\AfterBeginDocument} and
%    \cmd{\AtBeginDocument}.
%    If |ldump| is not loaded, then a single hook suffices
%    for normal \LaTeX{} processing.
%
%    The default definition of |\AfterBeginDocument| cannot
%    be done by |\let| because of problems with |xypic|.
%    \begin{macrocode}
\@ifundefined{AfterBeginDocument}{%
  \def\AfterBeginDocument{\AtBeginDocument}%
}{}%
%    \end{macrocode}
%    \end{macro}
%
% \subsection{Switches}
%    \begin{macrocode}
\newif\ifHy@typexml
\newif\ifHy@activeanchor
\newif\ifHy@backref
\newif\ifHy@bookmarks
\newif\ifHy@bookmarksnumbered
\newif\ifHy@bookmarksopen
\newif\ifHy@breaklinks
\newif\ifHy@centerwindow
\newif\ifHy@CJKbookmarks
\newif\ifHy@colorlinks
\newif\ifHy@draft
\newif\ifHy@figures
\newif\ifHy@fitwindow
\newif\ifHy@frenchlinks
\newif\ifHy@hyperfootnotes
\newif\ifHy@hyperindex
\newif\ifHy@hypertexnames
\newif\ifHy@implicit
\newif\ifHy@linktocpage
\newif\ifHy@menubar
\newif\ifHy@naturalnames
\newif\ifHy@nesting
\newif\ifHy@newwindow
\newif\ifHy@pageanchor
\newif\ifHy@pagelabels
\newif\ifHy@pdfpagehidden
\newif\ifHy@pdfstring
\newif\ifHy@plainpages
\newif\ifHy@psize
\newif\ifHy@raiselinks
\newif\ifHy@seminarslides
\newif\ifHy@texht
\newif\ifHy@toolbar
\newif\ifHy@unicode
\newif\ifHy@usetitle
\newif\ifHy@verbose
\newif\ifHy@windowui
%    \end{macrocode}
%    Defaults for the switches are now set.
%    \begin{macrocode}
\Hy@backreffalse
\Hy@bookmarksnumberedfalse
\Hy@bookmarksopenfalse
\Hy@bookmarkstrue
\Hy@breaklinksfalse
\Hy@centerwindowfalse
\Hy@CJKbookmarksfalse
\Hy@figuresfalse
\Hy@fitwindowfalse
\Hy@hyperfootnotestrue
\Hy@hyperindextrue
\Hy@hypertexnamestrue
\Hy@implicittrue
\Hy@linktocpagefalse
\Hy@menubartrue
\Hy@naturalnamesfalse
\Hy@nestingfalse
\Hy@newwindowfalse
\Hy@pageanchortrue
\Hy@pagelabelsfalse
\Hy@pdfpagehiddenfalse
\Hy@pdfstringfalse
\Hy@plainpagestrue
\Hy@raiselinksfalse
\Hy@texhtfalse
\Hy@toolbartrue
\Hy@typexmlfalse
\Hy@unicodefalse
\Hy@usetitlefalse
\Hy@verbosefalse
\Hy@windowuitrue
%    \end{macrocode}
%
% \section{Common help macros}
%
%    \begin{macro}{\Hy@StepCount}
%    \begin{macrocode}
\def\Hy@StepCount#1{\advance#1 by 1 }%
%    \end{macrocode}
%    \end{macro}
%    \begin{macro}{\Hy@GlobalStepCount}
%    \begin{macrocode}
\def\Hy@GlobalStepCount#1{\global\advance#1 by 1 }%
%    \end{macrocode}
%    \end{macro}
%
%    \begin{macrocode}
\newdimen\@linkdim
\let\Hy@driver\@empty
\let\MaybeStopEarly\relax
\newcount\Hy@linkcounter
\newcount\Hy@pagecounter
\Hy@linkcounter0
\Hy@pagecounter0
%    \end{macrocode}
%
% \subsection{Macros for recursions}
%
%    \begin{macro}{\Hy@ReturnAfterElseFi}
%    \begin{macro}{\Hy@ReturnAfterFi}
%    The commands \cs{Hy@ReturnAfterElseFi} and \cs{Hy@ReturnAfterFi}
%    avoid a too deep \cs{if}-nesting especially for recursive macros.
%    \begin{macrocode}
\long\def\Hy@ReturnAfterElseFi#1\else#2\fi{\fi#1}
\long\def\Hy@ReturnAfterFi#1\fi{\fi#1}
%    \end{macrocode}
%    \end{macro}
%    \end{macro}
%
% \subsection{Coordinate transformations}
%
%    At some places numbers in pdf units are
%    expected (eg: FitBH, ...). The following macros
%    perform the transformation from TeX units (pt)
%    to PDF units (bp).
%
%    \begin{macro}{\hypercalcbp}
%    The user macro \cmd{\hypercalcbp} can be used, for example,
%    inside option values:
%\begin{verbatim}
%pdfstartview={FitBH \hypercalcbp{\paperheight-\topmargin-1in}}
%\end{verbatim}
%    \begin{itemize}
%    \item
%    It cannot be used inside \cmd{\usepackage}, because
%    LaTeX expands the options before package hyperref
%    is loaded and \cmd{\hypercalcbp} is defined.
%    \item
%    With e-TeX extensions an expandable implementation
%    is very easy; \cmd{\hypercalcbp} can be used
%    everywhere and is expanded at use.
%    \item
%    Without e-TeX's features \cmd{\hypercalcbp} cannot be
%    implemented expandable (practically) and have to
%    be supported by \cmd{\hypercalcbpdef}.
%    Limitations:
%      \begin{itemize}
%      \item Works only in options that use \cmd{\hypercalcbpdef}
%            (currently only |pdfstartview|).
%      \item For calculations package |calc| have to be loaded.
%      \item The expansion of the argument is done at definition time.
%      \end{itemize}
%    \end{itemize}
%    Example (\TeX):
%\begin{verbatim}
%\usepackage{calc}
%\usepackage[...]{hyperref}
%\hypersetup{
%  pdfstartview={FitBH \hypercalcbp{\paperheight-\topmargin-1in
%    -\headheight-\headsep}
%}
%\end{verbatim}
%    \begin{macro}{\hypercalcbp}
%    \begin{macrocode}
\begingroup\expandafter\expandafter\expandafter\endgroup
\expandafter\ifx\csname dimexpr\endcsname\relax
  \def\hypercalcbpdef#1#2{%
    \begingroup
      \toks@{}%
      \HyCal@scan#2\hypercalcbp\@nil
    \expandafter\endgroup
    \expandafter\def\expandafter#1\expandafter{\the\toks@}%
  }
  \def\HyCal@scan#1\hypercalcbp#2\@nil{%
    \toks@\expandafter{\the\toks@ #1}%
    \ifx\\#2\\%
    \else
      \Hy@ReturnAfterFi{%
        \HyCal@do#2\@nil
      }%
    \fi
  }
  \def\HyCal@do#1#2\@nil{%
    \@ifpackageloaded{calc}{}{%
      \Hy@Warning{%
        For calculations \string\hypercalcbp\space needs\MessageBreak
        package calc or e-TeX%
      }%
    }%
    \setlength{\dimen@}{#1}%
    \setlength{\dimen@}{0.99626401\dimen@}%
    \edef\x{%
      \toks@{%
        \the\toks@
        \strip@pt\dimen@
      }%
    }\x
    \HyCal@scan#2\@nil
  }
\else
  \def\hypercalcbp#1{%
    \strip@pt\dimexpr 0.99626401\dimexpr #1\relax\relax
  }
  \def\hypercalcbpdef{\def}
\fi
%    \end{macrocode}
%    \end{macro}
%    \end{macro}
%
% \section{Dealing with PDF strings}\label{sec:pdfstring}
%    The PDF string stuff done by Heiko Oberdiek.\\
%    Email: \Email{oberdiek@ruf.uni-freiburg.de}.
%
%    Naming convention: All internal commands that are only
%    needed by \cs{pdfstringdef} are prefixed with \cs{HyPsd@}.
%
% \subsection{Description of PDF strings}
%    The PDF specification defines several places to hold
%    text strings (bookmark names, document information,
%    text annotations, etc.).
%    The PDF strings have following properties:
%    \begin{itemize}
%    \item They are surrounded by parentheses. The hexadecimal form
%      is not supported.
%    \item Like PostScript language strings they use the same
%      escaping mechanism:\\
%      \begin{tabular}{ll}
%      |\\|& the backslash itself\\
%      \cs{)}, \cs{(}& unbalanced parentheses\\
%      \cs{n}, \cs{r}, \cs{t}, \cs{b}, \cs{f}& special white space
%        escape sequences\\
%      |\|\textit{|ddd|}& octal character code \textit{|ddd|}
%      \end{tabular}
%    \item Strings are stored either in PDFDocEncoding, which is a superset of
%      ISOLatin1 and is compatible with Unicode with character codes
%      below 256, or in Unicode.
%    \end{itemize}
%
% \subsection{Definition of
%    \texorpdfstring{\cs{pdfstringdef}}{\\pdfstringdef}}
%    The central macro for dealing with PDF strings is \cs{pdfstringdef}.
%    It defines a command |#1| to be the result of the conversion
%    from the string in |#2| to a legal PDFDocEncoded string.
%    Currently the definition is global, but this can be changed in
%    the future.
%
%    Important: In \TeX's view PDF strings are written to a file and
%    are expanded only in its mouth. Stomach commands that cannot
%    be expanded further aren't executed, they are written verbatim.
%    But the PDF reader that reads such a string isn't a \TeX{}
%    interpreter!
%
%    The macro \cs{pdfstringdef} consists of three main parts:
%    \begin{enumerate}
%    \item Preprocessing. Here the expansion is prepared. The encoding
%          is set and many commands are redefined, so that they work
%          appropriate.
%    \item Expansion. The \TeX{} string is expanded the first time
%          to get a PDF string.
%    \item Postprocessing. The result of the expansion is checked and
%          converted to the final form.
%    \end{enumerate}
%
% \begin{macro}{\pdfstringdef}
%    \cs{pdfstringdef} works on the tokens in |#2| and converts them to
%    a PDF string as far as possible:
%    \begin{itemize}
%    \item The result should obey the rules of the PDF specification
%          for strings.
%    \item The string can safely processed by \TeX, because the
%          tokens have only catcodes 10 until 12.
%    \end{itemize}
%    The result is stored in the command token given in |#1|.
%    \begin{macrocode}
\def\pdfstringdef#1#2{%
%    \end{macrocode}
%    Many redefinitions are needed, so all the work is done in a group.
%    \begin{macrocode}
  \begingroup
%    \end{macrocode}
%
% \subsubsection{Preprocessing}
% \paragraph{Octal escape sequences.}
%    To avoid problems with eight bit or non printable characters, the octal
%    escape notation is supported. So most glyphs in the encoding definitions
%    for PD1 and PU produce these octal escape sequences.
%    All three octal digits have to be used:
%    \begin{itemize}
%    \item Wrong results are avoided, if digits follow that are not part of the
%          octal sequence.
%    \item Macros rely on the fact that the octal sequences always consist of
%          three digits (vtex driver, Unicode support).
%    \end{itemize}
%    The escape sequences start with a backslash. By \cs{string} it will be
%    printed. Therefore it is ensured that the \TeX{} escape character indeed
%    prints as a normal backslash.
%    Eventually this line can be removed, because this is standard
%    \LaTeX{} behaviour.
%    \begin{macrocode}
    \escapechar`\\%
%    \end{macrocode}
%    From the view of \TeX{} a octal sequence consists of the command tokens
%    \cs{0} until \cs{3} and two digits.
%    For saving tokens \cs{0}, \cs{1}, \cs{2}, and \cs{3} are directly
%    used without a preceding \cs{string} in the glyph definitions.
%    This is done here locally
%    by defining the \cs{0} until \cs{3} commands. So the user
%    can use octal escape sequences directly, the disadvantage is
%    that a previous definition of this short commands does not apply.
%    \begin{macrocode}
    \edef\0{\string\0}%
    \edef\1{\string\1}%
    \edef\2{\string\2}%
    \edef\3{\string\3}%
%    \end{macrocode}
% \paragraph{Setting font encoding.}
%    The unicode encoding
%    uses \cs{8} and \cs{9} as marker for the higher byte.
%    \cs{8} is an abbreviation for the higher bytes 0 until 7
%    that can be expressed by one digit. \cs{8} will be
%    converted to \cs{00}. However \cs{9} only marks the next
%    three digits as higher byte and will be removed later.
%
%    The encoding is set by \cs{enc@update} for optimizing reasons.
%    \begin{macrocode}
    \ifHy@unicode
      \edef\8{\string\8}%
      \edef\9{\string\9}%
      \fontencoding{PU}%
    \else
      \fontencoding{PD1}%
    \fi
    \enc@update
%    \end{macrocode}
%
% \paragraph{Internal encoding commands.}
%    \cs{pdfstringdef} interpretes text strings which are not allowed
%    to contain mathematical stuff. The text glyph commands will produce
%    a warning, if called in math mode. But this warning disturbs while
%    expanding. Therefore we check for math mode here, before
%    \cs{@inmathwarn} will be disabled (see below).
%    \begin{macrocode}
    \@inmathwarn\pdfstringdef
%    \end{macrocode}
%    If a glyph is used, that isn't in the PD1/PU encoding there will
%    be an infinite error loop, because the NFSS encoding stuff
%    have to be expanded unprotected (\cs{edef}), so that the
%    assigments of \cs{@changed@cmd} don't take place.
%    To patch this behaviour I only found \cs{@inmathwarn}
%    as a usable hook. While an \cs{edef} a warning message by
%    \cs{@inmathwarn} or \cs{TextSymbolUnavailable} cannot be give out,
%    so \cs{@inmathwarn} should be disabled. And with the help of it
%    the assignments in \cs{@changed@cmd} can easily be catched
%    (see below).
%    \begin{macrocode}
    \let\@inmathwarn\HyPsd@inmathwarn
%    \end{macrocode}
%
%    Unknown composite characters are built with \cs{add@accent},
%    so it is redefined to provide a warning.
%    \begin{macrocode}
    \let\add@accent\HyPsd@add@accent
%    \end{macrocode}
%
% \paragraph{Commands that don't use NFSS directly.}
%    There are several commands that prints characters in the
%    printable ASCII area that don't obey the NFSS, so they have
%    to be redefined here.
%    \begin{macrocode}
    \let\{\textbraceleft
    \let\}\textbraceright
    \let\\\textbackslash
    \let\#\textnumbersign
    \let\$\textdollar
    \let\%\textpercent
    \let\&\textampersand
%    \let\~\textasciitilde
    \let\_\textunderscore
    \let\P\textparagraph
    \let\ldots\textellipsis
    \let\dots\textellipsis
%    \end{macrocode}
%
% \paragraph{Newline}
%    \cmd{\newline} or \cmd{\\} do not work in bookmarks, in text
%    annotations they should expand to \cmd{\r}. In pdf strings
%    \cmd{\\} stands for a backslash. Therefore the commands
%    are disabled now. The user can redefine them for a result
%    what he want:
%    \begin{description}
%    \item[backslash:]
%      |\pdfstringdefDisableCommands{\let\\\textbackslash}|
%    \item[new line:]
%      |\pdfstringdefDisableCommands{\let\\\textCR}|
%    \item[disabled:]
%      |\pdfstringdefDisableCommands{\let\\\empty}|
%    \end{description}
%    At any case, however, the optional argument or the star
%    cannot be scanned in a 100\% sure manner.
%    \begin{macrocode}
    \def\\{\pdfstringdefWarn\\}%
    \def\newline{\pdfstringdefWarn\newline}%
%    \end{macrocode}
%
% \paragraph{Logos.}
%    Because the box shifting
%    used in the \TeX{} logo does not work while writing to a file,
%    the standard \TeX{} logos are redefined.
%    \begin{macrocode}
    \def\TeX{TeX}%
    \def\LaTeX{La\TeX}%
    \def\LaTeXe{\LaTeX2e}%
    \def\eTeX{e-\TeX}%
    \def\MF{Metafont}%
    \def\MP{Metapost}%
%    \end{macrocode}
%
% \paragraph{Standard font commands.}
%    Because font changes do not work, the standard font
%    switching commands are disabled.
%    \begin{macrocode}
    \let\emph\@firstofone
    \let\textbf\@firstofone
    \let\textit\@firstofone
    \let\textmd\@firstofone
    \let\textnormal\@firstofone
    \let\textrm\@firstofone
    \let\textsc\@firstofone
    \let\textsf\@firstofone
    \let\textsl\@firstofone
    \let\texttt\@firstofone
    \let\textup\@firstofone
    \let\ttfamily\@empty
    \let\sffamily\@empty
    \let\itshape\@empty
    \let\upshape\@empty
    \let\bfseries\@empty
    \let\rm\@empty
    \let\Huge\@empty
    \let\LARGE\@empty
    \let\Large\@empty
    \let\footnotesize\@empty
    \let\huge\@empty
    \let\large\@empty
    \let\normalsize\@empty
    \let\scriptsize\@empty
    \let\small\@empty
    \let\tiny\@empty
%    \end{macrocode}
%
% \paragraph{Package babel.}
%    Whereever ``naturalnames'' is used, disable \cs{textlatin}
%    (from Babel 3.6k). Thanks to Felix Neubauer
%    (Email: \Email{Felix.Neubauer@gmx.net}).
%    \begin{macrocode}
    \let\textlatin\@firstofone
    \@ifundefined{language@group}{}{%
      \csname HyPsd@babel@\language@group\endcsname
    }%
%    \end{macrocode}
%
%    Disable \cs{cyr}, used in russianb.ldf.
%    \begin{macrocode}
    \let\cyr\relax
%    \end{macrocode}
%
% \paragraph{Package german.}
%    \begin{macrocode}
    \let\glqq\textglqq
    \let\grqq\textgrqq
    \let\glq\textglq
    \let\grq\textgrq
    \let\flqq\textflqq
    \let\frqq\textfrqq
    \let\flq\textflq
    \let\frq\textfrq
%    \end{macrocode}
%
% \paragraph{Package french.} The support is deferred, because
%    it needs |\GenericError| to be disabled (see below).
%
% \paragraph{AMS classes.}
%    \begin{macrocode}
    \HyPSD@AMSclassfix
%    \end{macrocode}
%
% \paragraph{Redefintion of \cs{hspace}}
%    \cs{hspace} don't work in bookmarks, the following fix
%    tries to set a space if the argument is a positive length.
%    \begin{macrocode}
    \let\hspace\HyPsd@hspace
%    \end{macrocode}
%
% \paragraph{Commands of referencing and indexing systems.}
%    Some \LaTeX{} commands that are legal in \cs{section} commands
%    have to be disabled here.
%    \begin{macrocode}
    \let\label\@gobble
    \let\index\@gobble
    \let\glossary\@gobble
    \let\href\@secondoftwo
%    \end{macrocode}
%
%    The \cs{ref} and \cs{pageref} is much more complicate because of their
%    star form.
%    \begin{macrocode}
    \let\ref\HyPsd@ref
    \let\pageref\HyPsd@pageref
%    \end{macrocode}
%
% \paragraph{Miscellaneous commands.}
%    \begin{macrocode}
    \let\leavevmode\@empty
%    \end{macrocode}
%    \cs{halign} causes error messages because of the template
%    character |#|.
%    \begin{macrocode}
    \def\halign{\pdfstringdefWarn\halign\@gobble}%
%    \end{macrocode}
%
% \paragraph{Patch for cjk bookmarks.}
%    \begin{macrocode}
    \ifHy@CJKbookmarks
      \HyPsd@CJKhook
    \fi
%    \end{macrocode}
%
% \paragraph{User hook.}
%    The switch \cs{Hy@pdfstring} is turned on. So user commands
%    can detect that they are processed not to be typesetted within
%    \TeX's stomach,
%    but to be expanded by the mouth to give a PDF string.
%    At this place before interpreting the string in |#2| additional
%    redefinitions can by added by the hook \cs{pdfstringdefPreHook}.
%
%    The position in the middle of the redefinitions is a compromise:
%    The user should be able to provide his own (perhaps better)
%    redefinitions, but some commands should have their original
%    meaning, because they can be used in the hook (\cs{bgroup},
%    or \cs{@protected@testopt}, and \cs{@ifnextchar}
%    for \cs{renewcommand}).
%    \begin{macrocode}
    \Hy@pdfstringtrue
    \pdfstringdefPreHook
%    \end{macrocode}
%
% \paragraph{Spaces.}
%    For checking the token of the string, spaces must be masked, because
%    they cannot by catched by undelimited arguments.
%    \begin{macrocode}
    \HyPsd@LetUnexpandableSpace\space
    \HyPsd@LetUnexpandableSpace\ %
    \HyPsd@LetUnexpandableSpace~%
    \HyPsd@LetUnexpandableSpace\nobreakspace
%    \end{macrocode}
%
% \paragraph{Package xspace.}
%    \begin{macrocode}
    \@ifundefined{@xspace}{%
      \let\xspace\HyPsd@ITALCORR
    }{%
      \let\xspace\HyPsd@XSPACE
    }%
    \let\/\HyPsd@ITALCORR
    \let\bgroup\/%
    \let\egroup\/%
%    \end{macrocode}
%
% \paragraph{Redefinitions of miscellaneous commands.}
%    Hyphenation does not make sense.
%    \begin{macrocode}
    \let\discretionary\@gobbletwo
%    \end{macrocode}
%
%  \cs{@ifstar} is defined in \LaTeX\ as follows:
%\begin{verbatim}
%\def\@ifstar#1{\@ifnextchar *{\@firstoftwo{#1}}}
%\end{verbatim}
%    \cs{@ifnextchar} doesn't work, because it uses stomach
%    commands like \cs{let} and \cs{futurelet}. But it
%    doesn't break. Whereas |\@firstoftwo{#1}}| gives an
%    error message because \cs{@firstoftwo} misses its second
%    argument.
%
%    A mimicry of \cs{@ifnextchar} only with expandible commands
%    would be very extensive and the result would be only an
%    approximation. So here a cheaper solution follows
%    in order to get rid of the error message at least:
%    \begin{macrocode}
    \let\@ifnextchar\HyPsd@ifnextchar
    \let\@protected@testopt\HyPsd@protected@testopt
%    \end{macrocode}
%
% \subsubsection{Expansion}
%    There are several possibilities to
%    expand tokens within \LaTeX:
%    \begin{description}
%    \item[\cs{protected@edef}:]
%      The weakest form isn't usable, because
%      it does not expand the font encoding commands. They are
%      made roboust and protect themselves.
%    \item[\cs{csname}:] First the string is
%      expanded whithin a \cs{csname} and \cs{endcsname}.
%      Then the command name is converted to characters
%      with catcode 12 by \cs{string} and the first
%      escape character removed by \cs{@gobble}.
%      This method has the great \emph{advantage} that
%      stomach tokens that aren't allowed in PDF strings are detected
%      by \TeX{} and reported as errors in order to force the user
%      to write correct things. So he get no wrong results by
%      forgetting the proofreading of his text.
%      But the \emph{disadvantage} is that old wrong code cannot
%      processed without errors. Mainly the error message is very cryptic
%      and for the normal user hard to understand. \TeX{} provides
%      no way to catch the error caused by \cs{csname} or allows to
%      support the user with a descriptive error message. Therefore
%      the experienced user had to enable this behaviour by an
%      option |exactdef| in previous versions less or equal 6.50.
%    \item[\cs{edef}] This version uses this standard form for expansion.
%      It is stronger than \LaTeX's \cs{protected@edef}.
%      So the font encoding mechanism works and the glyph commands
%      are converted to the correct tokens for PDF strings whith the
%      definitions of the PD1 encoding.
%      Because the protecting mechanism of \LaTeX{} doesn't work
%      within an \cs{edef}, there are situations thinkable where
%      code can break. For example, assignments and definitions aren't
%      performed and so undefined command errors or argument
%      parsing errors can occur. But this is only a compatibility problem
%      with old texts. Now there are possibilities to write
%      code that gives correct PDF strings (see \cs{texorpdfstring}).
%      In the most cases unexpandable commands and tokens
%      (math shift, grouping characters) remains. They
%      don't cause an error like with \cs{csname}. However a PDF reader
%      isn't \TeX{}, so these tokens are viewed verbatim. So
%      this version detects them now, and removes them with an
%      descriptive warning for the user. As additional features
%      xspace support is possible and grouping characters can be
%      used without problems, because they are removed silently.
%    \end{description}
%
% \paragraph{Generic messages.}
%    While expanding via \cs{xdef} the |\Generic...| messages
%    don't work and causes problems (error messages, invalid |.out|
%    file). So they are disabled while expanding and removed silently,
%    because a user warning would be too expensive (memory and runtime,
%    |\pdfstringdef| is slow enough).
%    \begin{macrocode}
    \begingroup
      \let\GenericError\@gobblefour
      \let\GenericWarning\@gobbletwo
      \let\GenericInfo\@gobbletwo
%    \end{macrocode}
%
% \paragraph{Package french.}
%    This fix only works, if \cs{GenericError} is disabled.
%    \begin{macrocode}
      \ifx\nofrenchguillemets\@undefined
      \else
        \nofrenchguillemets
      \fi
%    \end{macrocode}
%
% \paragraph{Definition commands and expansion.}
%    Redefining the defining commands (see sec. \ref{defcmd}).
%    The original meaning of \cs{xdef} is saved in \cs{Hy@temp}.
%    \begin{macrocode}
      \let\Hy@temp\xdef
      \let\def\HyPsd@DefCommand
      \let\gdef\HyPsd@DefCommand
      \let\edef\HyPsd@DefCommand
      \let\xdef\HyPsd@DefCommand
      \let\futurelet\HyPsd@LetCommand
      \let\let\HyPsd@LetCommand
      \Hy@temp#1{#2}%
    \endgroup
%    \end{macrocode}
%
% \subsubsection{Postprocessing}
%    If the string is empty time can be saved by omitting the
%    postprocessing process.
%    \begin{macrocode}
    \ifx#1\@empty
    \else
%    \end{macrocode}
%
% \paragraph{Protecting spaces and removing grouping characters.}
%    In order to check the tokens we must separate them. This will be
%    done with \TeX's argument parsing. With this method
%    we must the following item takes into account, that makes
%    makes things a litte more complicate:
%    \begin{itemize}
%    \item \TeX{} does not accept a space as an undelimited argument,
%    it cancels space tokens while looking for an undelimited
%    argument. Therefore we must protect the spaces now.
%    \item An argument can be a single token or a group of many tokens.
%    And within curly braces tokens aren't find by \TeX's
%    argument scanning process. Third curly braces as grouping characters
%    cannot be expanded further, so they don't vanish by the string
%    expansion above. So these characters with catcode 1 and 2 are
%    removed in the following and replaced by an marker for the xspace
%    support.
%    \item \TeX{} silently removes the outmost pair of braces of an
%      argument. To prevent this on unwanted places, in the following
%    the character \verb+|+ is appended to the string to make an outer
%    brace to an inner one.
%    \end{itemize}
%    First the top level spaces are protected by replacing. Then the
%    string is scanned to detect token groups. Each token group
%    will now be space protected and again scanned for another
%    token groups.
%    \begin{macrocode}
      \HyPsd@ProtectSpaces#1%
      \let\HyPsd@String\@empty
      \expandafter\HyPsd@RemoveBraces\expandafter{#1|}%
      \global\let#1\HyPsd@String
%    \end{macrocode}
%
% \paragraph{Check tokens.}
%    After removing the spaces and the grouping characters
%    the string now should only consists of the following tokens/catcodes:\\
%    \begin{tabular}{rl}
%      0&command names with start with an escape character.\\
%      3&math shift\\
%      4&alignment tabs\\
%      6&parameter, but this is unlikely.\\
%      7&superscript\\
%      8&subscript\\
%     11&letter\\
%     12&other\\
%     13&commands that are active characters.
%    \end{tabular}
%
%    After \cs{HyPsd@CheckCatcodes} the command \cs{HyPsd@RemoveMask} is
%    reused to remove the group protection character \verb+|+.
%    This character is needed to ensure that the string at least
%    consists of one token if \cs{HyPsd@CheckCatcodes}
%    is called.
%
%    Because of internal local assignments and tabulars
%    group braces are used.
%    \begin{macrocode}
      \let\HyPsd@SPACEOPTI\relax
      {%
         \let\HyPsd@String\@empty
         \expandafter\HyPsd@CheckCatcodes#1\HyPsd@End
         \global\let#1\HyPsd@String
      }%
      \expandafter\HyPsd@RemoveMask\expandafter
        |\expandafter\@empty#1\HyPsd@End#1%
%    \end{macrocode}
%    \cs{HyPsd@CheckCatcodes} should no have removed the tokens with
%    catcode 3, 4, 7, and 8. Because a parameter token (6) would
%    cause to many errors before, there should now be only tokens
%    with catcodes 11 or 12. So I think there is no need for
%    a safety step like:
%\begin{verbatim}
%\xdef#1{\expandafter\strip@prefix\meaning#1}%
%\end{verbatim}
% \paragraph{Looking for wrong glyphs.}
%    The case that glyphs aren't defined in the PD1 encoding
%    is catched above in such a way, that the glyph name and
%    a marker is inserted into the string. Now we can safely
%    scan the string for this marker and provide a descriptive
%    warning.
%    \begin{macrocode}
      \expandafter\HyPsd@Subst\expandafter{\HyPsd@GLYPHERR}{\relax}#1%
      \let\HyPsd@String\@empty
      \expandafter\HyPsd@GlyphProcess#1\relax\@empty
      \global\let#1\HyPsd@String
%    \end{macrocode}
%
% \paragraph{Backslash.}
%    The double backslash disturbs parsing octal sequenzes, for
%    example in an string like |abc\\051| the sequence \cs{051}
%    is detected although the second \cs{} belongs to the
%    first backslash.
%    \begin{macrocode}
      \HyPsd@StringSubst{\\}{\textbackslash}#1%
%    \end{macrocode}
%
% \paragraph{Spaces.}
%    All spaces have already the form \cs{040}.
%    The last postprocessing step will
%    be an optimizing of the spaces, so we already introduce
%    already the necessary command \cs{HyPsd@SPACEOPTI}.
%    But first it is defined to be \cs{relax} in order to
%    prevent a too early expansion by an \cs{edef}.
%    Secondly a \cs{relax} serves as a marker for
%    a token that is detected by \cs{xspace}.
%
%    The code of |frenchb.ldf| can produce an additional
%    space before \cs{guillemotright}, because \cs{lastskip}
%    and \cs{unskip} do not work. Therefore it is removed here.
%    \begin{macrocode}
      \ifHy@unicode
        \expandafter\HyPsd@StringSubst\csname 80\040\endcsname
          \HyPsd@SPACEOPTI#1%
        \edef\Hy@temp@A{\HyPsd@SPACEOPTI\HyPsd@SPACEOPTI\80\273}%
        \expandafter\HyPsd@Subst\expandafter{\Hy@temp@A}%
          {\HyPsd@SPACEOPTI\80\273}#1%
      \else
        \HyPsd@StringSubst{\040}\HyPsd@SPACEOPTI#1%
        \expandafter\HyPsd@Subst\expandafter{%
          \expandafter\HyPsd@SPACEOPTI\expandafter\HyPsd@SPACEOPTI
          \string\273}{\HyPsd@SPACEOPTI\273}#1%
      \fi
%    \end{macrocode}
%
% \paragraph{Right parenthesis.}
%    Also \cs{xspace} detects a right parenthesis.
%    For the \cs{xspace} support and the following
%    parenthesis check the different parenthesis
%    notations |)|, \cs{)}, and \cs{051} are converted
%    to one type \cs{)} and before \cs{HyPsd@empty}
%    with the meaning of \cs{relax} is introduced for
%    \cs{xspace}. By redefining to \cs{@empty} \cs{HyPsd@empty}
%    can easily removed later.
%    \begin{macrocode}
      \ifHy@unicode
        \HyPsd@StringSubst{\)}{\80\051}#1%
        \HyPsd@Subst){\80\051}#1%
        \let\HyPsd@empty\relax
        \expandafter\HyPsd@StringSubst\csname 80\051\endcsname
          {\HyPsd@empty\80\051}#1%
      \else
        \HyPsd@StringSubst{\)}{\051}#1%
        \HyPsd@Subst){\051}#1%
        \let\HyPsd@empty\relax
        \HyPsd@StringSubst{\051}{\HyPsd@empty\string\)}#1%
      \fi
%    \end{macrocode}
%
% \paragraph{Support for package \texttt{xspace}.}
%    \cs{xspace} looks for the next token and decides if it
%    expands to a space or not. Following tokens prevent its
%    transformation to a space: Beginning and end of group,
%    handled above by replacing by an italic correction,
%    several punctuation marks, a closing parentheses, and
%    several spaces.
%
%    Without package |xspace| there are tokens with catcode 11 and 12,
%    \cs{HyPsd@empty} and \cs{HyPsd@SPACEOPTI}. With package |xspace|
%    marker for the italic correction \cs{/} and \cs{xspace} come with.
%    In the package |xspace| case the two markers are replaced by
%    commands and an \cs{edef} performs the \cs{xspace} processing.
%
%    In the opposite of the original \cs{xspace} \cs{HyPsd@xspace} uses
%    an argument instead of a \cs{futurelet}, so we have to provide
%    such an argument, if \cs{HyPsd@xspace} comes last. Because
%    \cs{HyPsd@Subst} with several equal tokens (|--|) needs a safe
%    last token, in both cases
%    the string gets an additional \cs{HyPsd@empty}.
%    \begin{macrocode}
      \expandafter\HyPsd@Subst\expandafter{\/}\HyPsd@empty#1%
      \@ifundefined{@xspace}{%
      }{%
        \let\HyPsd@xspace\relax
        \expandafter\HyPsd@Subst\expandafter
          {\HyPsd@XSPACE}\HyPsd@xspace#1%
        \let\HyPsd@xspace\HyPsd@doxspace
      }%
      \xdef#1{#1\HyPsd@empty}%
%    \end{macrocode}
%
% \paragraph{Ligatures.}
%    \TeX{} forms ligatures in its stomach, but the PDF strings are
%    treated only by \TeX's mouth. The PDFDocEncoding contains
%    some ligatures, but the current
%    version 3 of the AcrobatReader lacks the |fi| and |fl| glyphs, and
%    the Linux version lacks the |emdash| and |endash| glyphs.
%    So the necessary code is provided here, but currently disabled,
%    hoping that version 4 of the AcrobatReader is better.
%    To break the ligatures the user can use an empty group,
%    because it leads to an insertion of an \cs{HyPsd@empty}.
%    If this ligature code will be enabled some day, then the italic
%    correction should also break the ligatures. Currently this occurs
%    only, if package |xspace| is loaded.
%    \begin{macrocode}
%      \HyPsd@Subst{---}\textemdash#1%
%      \HyPsd@Subst{--}\textendash#1%
%      \HyPsd@Subst{fi}\textfi#1%
%      \HyPsd@Subst{fl}\textfl#1%
      \HyPsd@Subst{!`}\textexclamdown#1%
      \HyPsd@Subst{?`}\textquestiondown#1%
%    \end{macrocode}
%     With the next \cs{edef} we get rid of the token \cs{HyPsd@empty}.
%    \begin{macrocode}
      \let\HyPsd@empty\@empty
%    \end{macrocode}
%
% \paragraph{Left parentheses.}
%    Left parentheses are now converted to safe forms to avoid
%    problems with unmatched ones (\cs{(} with PDFDocEncoding,
%    the octal sequence with Unicode.
%
%    An optimization is possible. Matched parentheses can replaced
%    by a |()| pair. But this code is removed to save \TeX{} memory
%    and time.
%    \begin{macrocode}
      \ifHy@unicode
        \HyPsd@StringSubst\(\textparenleft#1%
        \HyPsd@Subst(\textparenleft#1%
      \else
        \HyPsd@StringSubst\({\050}#1%
        \HyPsd@Subst({\050}#1%
        \HyPsd@StringSubst{\050}{\string\(}#1%
      \fi
%    \end{macrocode}
%
% \paragraph{Optimizing spaces.}
%    Spaces are often used, but they have a very long form \cs{040}.
%    They are converted back to real spaces, but not all, so that
%    no space follows after another. In the bookmark case several
%    spaces are written to the |.out| file, but if the entries
%    are read back, several spaces are merged to a single one.
%
%    With Unicode the spaces are replaced by their octal sequences.
%    \begin{macrocode}
      \ifHy@unicode
        \edef\HyPsd@SPACEOPTI{\80\040}%
      \else
        \let\HyPsd@SPACEOPTI\HyPsd@spaceopti
      \fi
      \xdef#1{#1\@empty}%
    \fi
%    \end{macrocode}
%
% \paragraph{Converting to Unicode.}
%    At last the eight bit letters have to be converted to Unicode,
%    the masks \cs{8} and \cs{9} are removed and the Unicode
%    marker is added.
%    \begin{macrocode}
    \ifHy@unicode
      \HyPsd@ConvertToUnicode#1%
    \fi
%    \end{macrocode}
%
% \paragraph{User hook.}
%    The hook \cs{pdfstringdefPostHook} can be used
%    for the purpose to postprocess the string further.
%    \begin{macrocode}
    \pdfstringdefPostHook#1%
  \endgroup
}
%    \end{macrocode}
%    \end{macro}
%
% \subsection{Encodings}
% \subsubsection{PD1 encoding}
%    The PD1 encoding implements the PDFDocEncoding for use with
%    \LaTeXe's NFSS. Because the informational strings are not set by
%    \TeX's typesetting mechanism but for interpreting by the PDF reader,
%    the glyphs of the PD1 encoding are implemented to be safely written
%    to a file (PDF output file, |.out| file).
%
%    The  PD1 encoding can be specified as an option of the 'fontenc'  package
%    or loaded here. It does not matter what font family is selected,
%    as \TeX{} does not process it anyway. So use CM.
%    \begin{macrocode}
\@ifundefined{T@PD1}{\input{pd1enc.def}}{}
\DeclareFontFamily{PD1}{pdf}{}
\DeclareFontShape{PD1}{pdf}{m}{n}{ <-> cmr10 }{}
\DeclareFontSubstitution{PD1}{pdf}{m}{n}
%    \end{macrocode}
%
% \subsubsection{PU encoding}
%    The PU encoding implements the Unicode encoding for use with
%    \LaTeX's NFSS. Because of large memory requirements the
%    encoding file for Unicode support is only loaded, if option
%    |unicode| is specified as package option.
%    \begin{macro}{\HyPsd@InitUnicode}
%    Because the file |puenc.def| takes a lot of memory, the loading
%    is defined in the macro \cs{HyPsd@InitUnicode} called by
%    the package option |unicode|.
%    \begin{macrocode}
\def\HyPsd@InitUnicode{%
  \@ifundefined{T@PU}{\input{puenc.def}}{}%
  \DeclareFontFamily{PU}{pdf}{}%
  \DeclareFontShape{PU}{pdf}{m}{n}{ <-> cmr10 }{}%
  \DeclareFontSubstitution{PU}{pdf}{m}{n}%
  \let\HyPsd@InitUnicode\relax
}
%    \end{macrocode}
%    \end{macro}
%
% \subsection{Additional user commands}
%
% \subsubsection{^^A
%   \texorpdfstring{\cs{texorpdfstring}}{\\texorpdfstring}^^A
% }
%    \begin{macro}{\texorpdfstring}
%    While expanding the string in \cs{pdfstringdef} the switch
%    \cs{ifHy@pdfstring} is set. This is used by the
%    full expandible macro \cs{texorpdfstring}. It expects
%    two arguments, the first contains the string that will be
%    set and processed by \TeX's stomach, the second
%    contains the replacement for PDF strings.
%    \begin{macrocode}
\newcommand*{\texorpdfstring}{%
   \ifHy@pdfstring
     \expandafter\@secondoftwo
   \else
     \expandafter\@firstoftwo
   \fi
}
%    \end{macrocode}
%    \end{macro}
%
% \subsubsection{Hooks for
%    \texorpdfstring{\cs{pdfstringdef}}{\\pdfstringdef}^^A
% }
%    \begin{macro}{\pdfstringdefPreHook}
%    \begin{macro}{\pdfstringdefPostHook}
%    Default definition of the hooks for \cs{pdfstringdef}.
%    The construct \cs{@ifundefined} with \cs{let} is a little bit
%    faster than \cs{providecommand}.
%    \begin{macrocode}
\@ifundefined{pdfstringdefPreHook}{%
  \let\pdfstringdefPreHook\@empty
}{}
\@ifundefined{pdfstringdefPostHook}{%
  \let\pdfstringdefPostHook\@gobble
}{}
%    \end{macrocode}
%    \end{macro}
%    \end{macro}
%
%    \begin{macro}{\pdfstringdefDisableCommands}
%    In \cmd{\pdfstringdefPreHook} the user can add
%    code that is executed before the string, that have
%    to be converted by \cmd{\pdfstringdef}, is expanded.
%    So replacements for problematic macros can be given.
%    The code in \cmd{\pdfstringdefPreHook} should not
%    be replaced perhaps by an \cmd{\renewcommand},
%    because a previous meaning gets lost.
%
%    Macro \cmd{\pdfstringdefDisableCommands} avoids this,
%    because it reuses the old meaning of the hook and appends
%    the new code to \cmd{\pdfstringdefPreHook}, e.g.:
%\begin{verbatim}
%\pdfstringdefDisableCommands{%
%  \let~\textasciitilde
%  \def\url{\pdfstringdefWarn\url}%
%  \let\textcolor\@gobble
%}%
%\end{verbatim}
%    In the argument of \cmd{\pdfstringdefDisableCommands} the
%    character |@| can be used in command names. So it is easy
%    to use useful \LaTeX{} commands like \cmd{\@gobble} or
%    \cmd{\@firstofone}.
%    \begin{macrocode}
\def\pdfstringdefDisableCommands{%
  \begingroup
    \makeatletter
    \HyPsd@DisableCommands
}
%    \end{macrocode}
%    \end{macro}
%    \begin{macro}{\HyPsd@DisableCommands}
%    \begin{macrocode}
\long\def\HyPsd@DisableCommands#1{%
    \toks0=\expandafter{\pdfstringdefPreHook}%
    \toks1={#1}%
    \xdef\pdfstringdefPreHook{\the\toks0 \the\toks1}%
  \endgroup
}
%    \end{macrocode}
%    \end{macro}
%
%    \begin{macro}{\pdfstringdefWarn}
%    The purpose of \cmd{\pdfstringdefWarn} is to produce
%    a warning message, so the user can see, that something
%    can go wrong with the conversion to PDF strings.
%
%    The prefix |\<>-| is added to the token. \cmd{\noexpand}
%    protects the probably undefined one during the first
%    expansion step. Then \cmd{\HyPsd@CheckCatcodes} can
%    detect the not allowed token, \cmd{\HyPsd@CatcodeWarning}
%    prints a warning message, after \cmd{\HyPsd@RemovePrefix}
%    has removed the prefix.
%
%    \cmd{\pdfstringdefWarn} is intended for document authors or
%    package writers, examples for use can be seen in the definition
%    of \cmd{\HyPsd@ifnextchar} or \cmd{\HyPsd@protected@testopt}.
%    \begin{macrocode}
\def\pdfstringdefWarn#1{%
   \expandafter\noexpand\csname<>-\string#1\endcsname
}
%    \end{macrocode}
%    \end{macro}
%
% \subsection{Help macros for expansion}
%
% \subsubsection{Babel languages}
%    \begin{macrocode}
\newif\ifHy@next
%    \end{macrocode}
%
%    Nothing to do for english.
%    \begin{macrocode}
\@ifpackagewith{babel}{danish}{%
  \def\HyPsd@babel@danish{%
    \declare@shorthand{danish}{"|}{}%
    \declare@shorthand{danish}{"~}{-}%
  }%
}{}
\@ifpackagewith{babel}{dutch}{%
  \def\HyPsd@babel@dutch{%
    \declare@shorthand{dutch}{"|}{}%
    \declare@shorthand{dutch}{"~}{-}%
  }%
}{}
\@ifpackagewith{babel}{finnish}{%
  \def\HyPsd@babel@finnish{%
    \declare@shorthand{finnish}{"|}{}%
  }%
}{}
\Hy@nextfalse
\@ifpackagewith{babel}{frenchb}{\Hy@nexttrue}{}
\@ifpackagewith{babel}{francais}{\Hy@nexttrue}{}
\ifHy@next
  \def\HyPsd@babel@frenchb{%
    \def\guill@spacing{ }%
  }%
\fi
\Hy@nextfalse
\@ifpackagewith{babel}{german}{\Hy@nexttrue}{}%
\@ifpackagewith{babel}{germanb}{\Hy@nexttrue}{}%
\@ifpackagewith{babel}{austrian}{\Hy@nexttrue}{}%
\ifHy@next
  \def\HyPsd@babel@german{%
    \declare@shorthand{german}{"f}{f}%
    \declare@shorthand{german}{"|}{}%
    \declare@shorthand{german}{"~}{-}%
  }%
\fi
\Hy@nextfalse
\@ifpackagewith{babel}{ngerman}{\Hy@nexttrue}{}%
\@ifpackagewith{babel}{ngermanb}{\Hy@nexttrue}{}%
\@ifpackagewith{babel}{naustrian}{\Hy@nexttrue}{}%
\ifHy@next
  \def\HyPsd@babel@ngerman{%
    \declare@shorthand{german}{"|}{}%
    \declare@shorthand{german}{"~}{-}%
  }%
\fi
\Hy@nextfalse
\@ifpackagewith{babel}{usorbian}{\Hy@nexttrue}{}%
\@ifpackagewith{babel}{uppersorbian}{\Hy@nexttrue}{}%
\ifHy@next
  \def\HyPsd@babel@usorbian{%
    \declare@shorthand{usorbian}{"f}{f}%
    \declare@shorthand{usorbian}{"|}{}%
  }%
\fi
\Hy@nextfalse
\@ifpackagewith{babel}{brazil}{\Hy@nexttrue}{}%
\@ifpackagewith{babel}{brazilian}{\Hy@nexttrue}{}%
\@ifpackagewith{babel}{portuges}{\Hy@nexttrue}{}%
\@ifpackagewith{babel}{portuguese}{\Hy@nexttrue}{}%
\ifHy@next
  \def\HyPsd@babel@portuges{%
    \declare@shorthand{portuges}{"|}{}%
  }%
\fi
\Hy@nextfalse
\@ifpackagewith{babel}{russian}{\Hy@nexttrue}{}
\@ifpackagewith{babel}{russianb}{\Hy@nexttrue}{}
\ifHy@next
  \def\HyPsd@babel@russian{%
    \declare@shorthand{russian}{"|}{}%
    \declare@shorthand{russian}{"~}{-}%
  }%
\fi
\Hy@nextfalse
\@ifpackagewith{babel}{ukrainian}{\Hy@nexttrue}{}
\@ifpackagewith{babel}{ukraineb}{\Hy@nexttrue}{}
\ifHy@next
  \def\HyPsd@babel@ukrainian{%
    \declare@shorthand{ukrainian}{"|}{}%
    \declare@shorthand{ukrainian}{"~}{-}%
  }%
\fi
\@ifpackagewith{babel}{macedonian}{%
  \def\HyPsd@babel@macedonian{%
    \declare@shorthand{macedonian}{"|}{}%
    \declare@shorthand{macedonian}{"~}{-}%
  }%
}{}
\@ifpackagewith{babel}{slovene}{%
  \def\HyPsd@babel@slovene{%
    \declare@shorthand{slovene}{"|}{}%
  }%
}{}
\@ifpackagewith{babel}{swedish}{%
  \def\HyPsd@babel@swedish{%
    \declare@shorthand{swedish}{"|}{}%
    \declare@shorthand{swedish}{"~}{-}%
  }%
}{}
%    \end{macrocode}
%
% \subsubsection{CJK bookmarks}
%
%    \begin{macro}{\HyPsd@CJKhook}
%    Some internal commands of package cjk are redefined
%    to avoid error messages. For a rudimental support
%    of CJK bookmarks the active characters are
%    redefined so that they print themselves.
%
%    After preprocessing of Big5 encoded data the
%    following string for a double-byte character
%    is emitted:
%\begin{verbatim}
%^^7f<arg1>^^7f<arg2>^^7f
%\end{verbatim}
%    \verb|<arg1>| is the first byte in the range (always $>$ 0x80);
%    \verb|<arg2>| is the second byte in decimal notation
%    ($\ge$ 0x40).
%    \begin{macrocode}
\begingroup
  \catcode"7F=\active
  \toks@{%
    \let\CJK@ignorespaces\empty
    \def\CJK@char#1{\@gobbletwo}%
    \let\CJK@charx\@gobblefour
    \let\CJK@punctchar\@gobblefour
    \def\CJK@punktcharx#1{\@gobblefour}%
    \catcode"7F=\active
    \def^^7f#1^^7f#2^^7f{%
      \string #1\HyPsd@DecimalToOctal{#2}%
    }%
    % ... ?
    \ifHy@unicode
      \def\Hy@cjkpu{\80}%
    \else
      \let\Hy@cjkpu\@empty
    \fi
    \HyPsd@CJKActiveChars
  }%
  \count@=127
  \@whilenum\count@<255 \do{%
    \advance\count@ by 1
    \lccode`\~=\count@
    \lowercase{%
      \toks@\expandafter{\the\toks@ ~}%
    }%
  }%
  \toks@\expandafter{\the\toks@ !}%
  \xdef\HyPsd@CJKhook{\the\toks@}%
\endgroup
%    \end{macrocode}
%    \end{macro}
%    \begin{macro}{\HyPsd@CJKActiveChars}
%    The macro \cmd{\HyPsd@CJKActiveChars} is only defined
%    to limit the memory consumption of \cmd{\HyPsd@CJKhook}.
%    \begin{macrocode}
\def\HyPsd@CJKActiveChars#1{%
  \ifx#1!%
    \let\HyPsd@CJKActiveChars\relax
  \else
    \edef#1{\noexpand\Hy@cjkpu\string#1}%
  \fi
  \HyPsd@CJKActiveChars
}
%    \end{macrocode}
%    \end{macro}
%    \begin{macro}{\HyPsd@DecimalToOctal}
%    A character, given by the decimal number is converted
%    to a PDF character.
%    \begin{macrocode}
\def\HyPsd@DecimalToOctal#1{%
  \ifcase #1 %
        \000\or \001\or \002\or \003\or \004\or \005\or \006\or \007%
    \or \010\or \011\or \012\or \013\or \014\or \015\or \016\or \017%
    \or \020\or \021\or \022\or \023\or \024\or \025\or \026\or \027%
    \or \030\or \031\or \032\or \033\or \034\or \035\or \036\or \037%
    \or \040\or \041\or \042\or \043\or \044\or \045\or \046\or \047%
    \or \050\or \051\or \052\or \053\or \054\or \055\or \056\or \057%
    \or    0\or    1\or    2\or    3\or    4\or    5\or    6\or    7%
    \or    8\or    9\or \072\or \073\or \074\or \075\or \076\or \077%
    \or    @\or    A\or    B\or    C\or    D\or    E\or    F\or    G%
    \or    H\or    I\or    J\or    K\or    L\or    M\or    N\or    O%
    \or    P\or    Q\or    R\or    S\or    T\or    U\or    V\or    W%
    \or    X\or    Y\or    Z\or \133\or \134\or \135\or \136\or \137%
    \or \140\or    a\or    b\or    c\or    d\or    e\or    f\or    g%
    \or    h\or    i\or    j\or    k\or    l\or    m\or    n\or    o%
    \or    p\or    q\or    r\or    s\or    t\or    u\or    v\or    w%
    \or    x\or    y\or    z\or \173\or \174\or \175\or \176\or \177%
    \or \200\or \201\or \202\or \203\or \204\or \205\or \206\or \207%
    \or \210\or \211\or \212\or \213\or \214\or \215\or \216\or \217%
    \or \220\or \221\or \222\or \223\or \224\or \225\or \226\or \227%
    \or \230\or \231\or \232\or \233\or \234\or \235\or \236\or \237%
    \or \240\or \241\or \242\or \243\or \244\or \245\or \246\or \247%
    \or \250\or \251\or \252\or \253\or \254\or \255\or \256\or \257%
    \or \260\or \261\or \262\or \263\or \264\or \265\or \266\or \267%
    \or \270\or \271\or \272\or \273\or \274\or \275\or \276\or \277%
    \or \300\or \301\or \302\or \303\or \304\or \305\or \306\or \307%
    \or \310\or \311\or \312\or \313\or \314\or \315\or \316\or \317%
    \or \320\or \321\or \322\or \323\or \324\or \325\or \326\or \327%
    \or \330\or \331\or \332\or \333\or \334\or \335\or \336\or \337%
    \or \340\or \341\or \342\or \343\or \344\or \345\or \346\or \347%
    \or \350\or \351\or \352\or \353\or \354\or \355\or \356\or \357%
    \or \360\or \361\or \362\or \363\or \364\or \365\or \366\or \367%
    \or \370\or \371\or \372\or \373\or \374\or \375\or \376\or \377%
  \fi
}
%    \end{macrocode}
%    \end{macro}
%
% \subsubsection{\texorpdfstring{\cs{@inmathwarn}}{\\@inmathwarn}-Patch}
%    \begin{macro}{\HyPsd@inmathwarn}
%    The patch of \cs{@inmathwarn} is needed to get rid of the
%    infinite error loop with glyphs of other encodings
%    (see the explanation above). Potentially the patch is
%    dangerous, if the code in |ltoutenc.dtx| changes.
%    Checked with \LaTeXe{} versions [1998/06/01] and
%    [1998/12/01]. I expect that versions below [1995/12/01]
%    don't work.
%
%    To understand the patch easier, the original code of
%    \cs{@current@cmd} and \cs{@changed@cmd} follows
%    (\LaTeXe{} release [1998/12/01]).
%    In the normal case \cs{pdfstringdef} is executed in a context
%    where \cs{protect} has the meaning of \cs{@typesetprotect}
%    (=\cs{relax}).
%\begin{verbatim}
%\def\@current@cmd#1{%
%   \ifx\protect\@typeset@protect
%      \@inmathwarn#1%
%   \else
%      \noexpand#1\expandafter\@gobble
%   \fi}
%\def\@changed@cmd#1#2{%
%   \ifx\protect\@typeset@protect
%      \@inmathwarn#1%
%      \expandafter\ifx\csname\cf@encoding\string#1\endcsname\relax
%         \expandafter\ifx\csname ?\string#1\endcsname\relax
%            \expandafter\def\csname ?\string#1\endcsname{%
%               \TextSymbolUnavailable#1%
%            }%
%         \fi
%         \global\expandafter\let
%               \csname\cf@encoding \string#1\expandafter\endcsname
%               \csname ?\string#1\endcsname
%      \fi
%      \csname\cf@encoding\string#1%
%         \expandafter\endcsname
%   \else
%      \noexpand#1%
%   \fi}
%\gdef\TextSymbolUnavailable#1{%
%   \@latex@error{%
%      Command \protect#1 unavailable in encoding \cf@encoding%
%   }\@eha}
%\def\@inmathwarn#1{%
%   \ifmmode
%      \@latex@warning{Command \protect#1 invalid in math mode}%
%   \fi}
%\end{verbatim}
%    \begin{macrocode}
\def\HyPsd@inmathwarn#1#2{%
  \ifx#2\expandafter
    \expandafter\ifx\csname\cf@encoding\string#1\endcsname\relax
      \HyPsd@GLYPHERR
      \expandafter\@gobble\string#1%
      >%
      \expandafter\expandafter\expandafter\HyPsd@EndWithElse
    \else
      \expandafter\expandafter\expandafter\HyPsd@GobbleFiFi
    \fi
  \else
    \expandafter#2%
  \fi
}
\def\HyPsd@GobbleFiFi#1\fi#2\fi{}
\def\HyPsd@EndWithElse#1\else{\else}
%    \end{macrocode}
%    \end{macro}
%
% \subsubsection{\texorpdfstring{\cs{add@accent}}{\\@addaccent}-Patch}
%    Unknown composite characters are built with \cs{add@accent},
%    so it is redefined to provide a warning.
%    \begin{macro}{\HyPsd@add@accent}
%    \begin{macrocode}
\def\HyPsd@add@accent#1#2{%
  \HyPsd@GLYPHERR\expandafter\@gobble\string#1+\string#2>%
  #2%
}%
%    \end{macrocode}
%    \end{macro}
%
% \subsubsection{Unexpandable spaces}
%    \begin{macro}{\HyPsd@LetUnexpandableSpace}
%    In \cmd{\HyPsd@@ProtectSpaces} the space tokens are replaced
%    by not expandable commands, that work like spaces:
%    \begin{itemize}
%    \item So they can catched by undelimited arguments.
%    \item And they work in number, dimen, and skip
%          assignments.
%    \end{itemize}
%    These properties are used in \cmd{\HyPsd@CheckCatcodes}.
%    \begin{macrocode}
\def\HyPsd@LetUnexpandableSpace#1{%
  \expandafter\futurelet\expandafter#1\expandafter\@gobble\space\relax
}
%    \end{macrocode}
%    \end{macro}
%    \begin{macro}{\HyPsd@UnexpandableSpace}
%    \cmd{\HyPsd@UnexpandableSpace} is used
%    in \cmd{\HyPsd@@ProtectSpaces}.
%    In \cmd{HyPsd@@ProtectSpaces} the space tokens are replaced
%    by unexpandable commands \cmd{\HyPsd@UnexpandableSpace},
%    but that have the effect of spaces.
%    \begin{macrocode}
\HyPsd@LetUnexpandableSpace\HyPsd@UnexpandableSpace
%    \end{macrocode}
%    \end{macro}
%
% \subsubsection{Marker for commands}
%    \begin{macro}{\HyPsd@XSPACE}
%    \begin{macro}{\HyPsd@ITALCORR}
%    \begin{macro}{\HyPsd@GLYPHERR}
%    Some commands and informations cannot be utilized before
%    the string expansion and the checking process.
%    Command names are filtered out, so we need another way
%    to transport the information: An unusual |#| with catcode
%    12 marks the beginning of the extra information.
%    \begin{macrocode}
\edef\HyPsd@XSPACE{\string#\string X}
\edef\HyPsd@ITALCORR{\string#\string I}
\edef\HyPsd@GLYPHERR{\string#\string G}
%    \end{macrocode}
%    \end{macro}
%    \end{macro}
%    \end{macro}
%
% \subsubsection{\texorpdfstring{\cs{hspace}}{\\hspace} fix}
%    \begin{macro}{\HyPsd@hspace}
%    \begin{macrocode}
\def\HyPsd@hspace#1{\HyPsd@@hspace#1*\END}
%    \end{macrocode}
%    \end{macro}
%    \begin{macro}{\HyPsd@@hspace}
%    \cs{HyPsd@@hspace} checks whether \cs{hspace}
%    is called in its star form.
%    \begin{macrocode}
\def\HyPsd@@hspace#1*#2\END{%
  \ifx\\#2\\%
    \HyPsd@hspacetest{#1}%
  \else
    \expandafter\HyPsd@hspacetest
  \fi
}
%    \end{macrocode}
%    \end{macro}
%    \begin{macro}{\HyPsd@hspacetest}
%    \cs{HyPsd@hyspacetest} replaces the \cs{hspace} by a space, if
%    the length is greater than zero.
%    \begin{macrocode}
\def\HyPsd@hspacetest#1{\ifdim#1>\z@\space\fi}
%    \end{macrocode}
%    \end{macro}
%
% \subsubsection{Fix for AMS classes}
%
%    \begin{macrocode}
\@ifundefined{tocsection}{%
  \let\HyPSD@AMSclassfix\relax
}{%
  \def\HyPSD@AMSclassfix{%
    \let\tocpart\HyPSD@tocsection
    \let\tocchapter\HyPSD@tocsection
    \let\tocappendix\HyPSD@tocsection
    \let\tocsection\HyPSD@tocsection
    \let\tocsubsection\HyPSD@tocsection
    \let\tocsubsubsection\HyPSD@tocsection
    \let\tocparagraph\HyPSD@tocsection
  }%
  \def\HyPSD@tocsection#1#2#3{%
    \if @#2@\else\if @#1@\else#1 \fi#2. \fi
    #3%
  }%
}
%    \end{macrocode}
%
% \subsubsection{Reference commands}
%
%    \begin{macro}{\HyPsd@ref}
%    Macro \cs{HyPsd@ref} calls the macro \cs{HyPsd@@ref} for star checking.
%    The same methods like in \cs{HyPsd@hspace} is used.
%    \begin{macrocode}
\def\HyPsd@ref#1{\HyPsd@@ref#1*\END}%
%    \end{macrocode}
%    \end{macro}
%    \begin{macro}{\HyPsd@@ref}
%    Macro \cs{HyPsd@@ref} checks if a star is present.
%    \begin{macrocode}
\def\HyPsd@@ref#1*#2\END{%
  \ifx\\#2\\%
    \HyPsd@@@ref{#1}%
  \else
    \expandafter\HyPsd@@@ref
  \fi
}%
%    \end{macrocode}
%    \end{macro}
%    \begin{macro}{\HyPsd@@@ref}
%    \cs{HyPsd@@@ref} does the work and extracts the first argument.
%    \begin{macrocode}
\def\HyPsd@@@ref#1{%
  \expandafter\ifx\csname r@#1\endcsname\relax
    ??%
  \else
    \expandafter\expandafter\expandafter\@car\csname r@#1\endcsname\@nil
  \fi
}
%    \end{macrocode}
%    \end{macro}
%
%    \begin{macro}{\HyPsd@pageref}
%    Macro \cs{HyPsd@pageref} calls the macro \cs{HyPsd@@pageref} for star checking.
%    The same methods like in \cs{HyPsd@hspace} is used.
%    \begin{macrocode}
\def\HyPsd@pageref#1{\HyPsd@@pageref#1*\END}
%    \end{macrocode}
%    \end{macro}
%    \begin{macro}{\HyPsd@@pageref}
%    Macro \cs{HyPsd@@pageref} checks if a star is present.
%    \begin{macrocode}
\def\HyPsd@@pageref#1*#2\END{%
  \ifx\\#2\\%
    \HyPsd@@@pageref{#1}%
  \else
    \expandafter\HyPsd@@@pageref
  \fi
}
%    \end{macrocode}
%    \end{macro}
%    \begin{macro}{\HyPsd@@@pageref}
%    \cs{HyPsd@@@pageref} does the work and extracts the second argument.
%    \begin{macrocode}
\def\HyPsd@@@pageref#1{%
  \expandafter\ifx\csname r@#1\endcsname\relax
    ??%
  \else
    \expandafter\expandafter\expandafter\expandafter
    \expandafter\expandafter\expandafter\@car
    \expandafter\expandafter\expandafter\@gobble
    \csname r@#1\endcsname\@nil
  \fi
}
%    \end{macrocode}
%    \end{macro}
%
% \subsubsection{Redefining the defining commands}
% \label{defcmd}
%    Definitions aren't allowed, because they aren't executed in
%    an only expanding context. So the command to be defined
%    isn't defined and can perhaps be undefined. This would causes
%    TeX to stop with an error message.
%    With a deep trick it is possible to define commands in such
%    a context: \cs{csname} does the job, it defines the command
%    to be \cs{relax}, if it has no meaning.
%
%    Active characters cannot be defined with this trick. It is
%    possible to define all undefined active characters
%    (perhaps that they have the meaning of \cs{relax}).
%    To avoid side effects this should be done in \cs{pdfstringdef}
%    shortly before the \cs{xdef} job. But checking and defining
%    all possible active characters of the full range (0 until 255)
%    would take a while. \cs{pdfstringdef} is slow enough, so
%    this isn't done.
%
%    \cs{HyPsd@DefCommand} and \cs{HyPsd@LetCommand} expands to the
%    commands \cs{<def>-command} and \cs{<let>-command}
%    with the meaning of \cs{def} and \cs{let}. So it is detected by
%    \cs{HyPsd@CheckCatcodes} and the command name \cs{<def>-command}
%    or \cs{<let>-command} should indicate a forbidden definition
%    command.
%
%    The command to be defined is converted to a string and back
%    to a command name with the help of \cs{csname}. If the
%    command is already defined, \cs{noexpand} prevents a
%    further expansion, even though the command would
%    expand to legal stuff. If the command don't have the meaning
%    of \cs{relax}, \cs{HyPsd@CheckCatcodes} will produce a warning.
%    (The command itself can be legal, but the warning is legitimate
%    because of the position after a defining command.)
%
%    The difference between \cs{HyPsd@DefCommand} and
%    \cs{HyPsdLetCommand} is that the first one also cancels this
%    arguments, the parameter and definition text. The right side
%    of the \cs{let} commands cannot be canceled with an undelimited
%    parameter because of a possible space token after \cs{futurelet}.
%
%    \begin{macro}{\HyPsd@DefCommand}
%    \begin{macro}{\HyPsd@LetCommand}
%    \begin{macrocode}
\begingroup
  \def\x#1#2{%
    \endgroup
    \let#1\def
    \def\HyPsd@DefCommand##1##2##{%
      #1%
      \expandafter\noexpand
        \csname\expandafter\@gobble\string##1\@empty\endcsname
      \@gobble
    }%
    \let#2\let
    \def\HyPsd@LetCommand##1{%
      #2%
      \expandafter\noexpand
        \csname\expandafter\@gobble\string##1\@empty\endcsname
    }%
  }%
\expandafter\x\csname <def>-command\expandafter\endcsname
              \csname <let>-command\endcsname
%    \end{macrocode}
%    \end{macro}
%    \end{macro}
%
% \subsubsection{^^A
%   \texorpdfstring{\cs{ifnextchar}}{\\ifnextchar}^^A
% }
%    \begin{macro}{\HyPsd@ifnextchar}
%    In \cs{pdfstringdef} \cs{@ifnextchar} is disabled
%    via a \cs{let} command to save time. First a
%    warning message is given, then the three arguments
%    are canceled. \cs{@ifnextchar} cannot work in a correct
%    manner, because it uses \cs{futurelet}, but this is a
%    stomach feature, that doesn't work in an expanding context.
%    \begin{macrocode}
\def\HyPsd@ifnextchar{%
  \pdfstringdefWarn\@ifnextchar
  \expandafter\@gobbletwo\@gobble
}
%    \end{macrocode}
%    \end{macro}
%
% \subsubsection{^^A
%   \texorpdfstring{\cs{@protected@testoptifnextchar}}^^A
%     {\\@protected@testopt}^^A
% }
%    \begin{macro}{\HyPsd@protected@testopt}
%    Macros with optional arguments doesn't work properly, because
%    they call \cmd{\@ifnextchar} to detect the optional argument
%    (see the explanation of \cmd{\HyPsd@ifnextchar}).
%    But a warning, that \cmd{\@ifnextchar} doesn't work, doesn't
%    help the user very much. Therefore \cmd{\@protected@testopt}
%    is also disabled, because its first argument is the problematic
%    macro with the optional argument and it is called before
%    \cmd{\@ifnextchar}.
%    \begin{macrocode}
\def\HyPsd@protected@testopt#1{%
  \pdfstringdefWarn#1%
  \@gobbletwo
}
%    \end{macrocode}
%    \end{macro}
%
% \subsection{Help macros for postprocessing}
%
% \subsubsection{Generic warning.}
%    \begin{macro}{\HyPsd@Warning}
%    For several reasons \cs{space} is masked and does not have its
%    normal meaning. But it is used in warning messages, so it is
%    redefined locally:
%    \begin{macrocode}
\def\HyPsd@Warning#1{%
  \begingroup
    \def\space{ }%
    \Hy@Warning{#1}%
  \endgroup
}
%    \end{macrocode}
%    \end{macro}
%
% \subsubsection{Protecting spaces}
%    \begin{macro}{\HyPsd@ProtectSpaces}
%    \cs{HyPsd@ProtectSpaces} calls with the expanded
%    string \cs{HyPsd@@ProtectSpaces}. The expanded string is
%    protected by \verb+|+ at the beginning and end of
%    the expanded string. Because of this there can be no group
%    at the beginning or end of the string and grouping characters
%    are not removed by the call of \cs{HyPsd@@ProtectSpaces}.
%    \begin{macrocode}
\def\HyPsd@ProtectSpaces#1{%
  \expandafter\HyPsd@@ProtectSpaces
    \expandafter|\expandafter\@empty#1| \HyPsd@End#1%
}
%    \end{macrocode}
%    \end{macro}
%    \begin{macro}{\HyPsd@@ProtectSpaces}
%    The string can contain command tokens, so it is better
%    to use an \cs{def} instead of an \cs{edef}.
%    \begin{macrocode}
\def\HyPsd@@ProtectSpaces#1 #2\HyPsd@End#3{%
  \ifx\scrollmode#2\scrollmode
    \HyPsd@RemoveMask#1\HyPsd@End#3%
  \else
    \gdef#3{#1\HyPsd@UnexpandableSpace#2}%
    \Hy@ReturnAfterFi{%
      \expandafter\HyPsd@@ProtectSpaces#3\HyPsd@End#3%
    }%
  \fi
}
%    \end{macrocode}
%    \end{macro}
%
% \paragraph{Remove mask.}
%    \begin{macro}{\HyPsd@RemoveMask}
%    \cs{HyPsd@RemoveMask} removes the protecting \verb+|+.
%    It is used by \cs{HyPsd@@ProtectSpaces} and by the code in
%    \cs{pdfstringdef} that removes the grouping chararcters.
%    \begin{macrocode}
\def\HyPsd@RemoveMask|#1|\HyPsd@End#2{%
  \toks@\expandafter{#1}%
  \xdef#2{\the\toks@}%
}
%    \end{macrocode}
%    \end{macro}
%
% \subsubsection{Remove grouping braces}
%    \begin{macro}{\HyPsd@RemoveBraces}
%    |#1| contains the expanded string, the result will
%    be locally written in command \cs{HyPsd@String}.
%    \begin{macrocode}
\def\HyPsd@RemoveBraces#1{%
  \ifx\scrollmode#1\scrollmode
  \else
    \Hy@ReturnAfterFi{%
      \HyPsd@@RemoveBraces#1\HyPsd@End{#1}%
    }%
  \fi
}
%    \end{macrocode}
%    \end{macro}
%    \begin{macro}{\HyPsd@@RemoveBraces}
%    \cs{HyPsd@@RemoveBraces} is called with the expanded string,
%    the end marked by \cs{HyPsd@End}, the expanded string again, but
%    enclosed in braces and the string command. The first expanded
%    string is scanned by the parameter text |#1#2|.
%    By a comparison with the original form in |#3| we can decide
%    whether |#1| is a single token or a group. To avoid the
%    case that |#2| is a group, the string is extended by a \verb+|+
%    before.
%
%    While removing the grouping braces an italic correction
%    marker is inserted for supporting package |xspace| and
%    letting ligatures broken.
%
%    Because the string is already expanded, the \cs{if} commands
%    should disappeared. So we can move some parts out
%    of the argument of \cs{Hy@ReturnAfterFi}.
%    \begin{macrocode}
\def\HyPsd@@RemoveBraces#1#2\HyPsd@End#3{%
  \def\Hy@temp@A{#1#2}%
  \def\Hy@temp@B{#3}%
  \ifx\Hy@temp@A\Hy@temp@B
    \expandafter\def\expandafter\HyPsd@String\expandafter{%
      \HyPsd@String#1%
    }%
    \Hy@ReturnAfterElseFi{%
      \ifx\scrollmode#2\scrollmode
      \else
        \Hy@ReturnAfterFi{%
          \HyPsd@RemoveBraces{#2}%
        }%
      \fi
    }%
  \else
    \def\Hy@temp@A{#1}%
    \HyPsd@AppendItalcorr\HyPsd@String
    \Hy@ReturnAfterFi{%
      \ifx\Hy@temp@A\@empty
        \Hy@ReturnAfterElseFi{%
          \HyPsd@RemoveBraces{#2}%
        }%
      \else
        \Hy@ReturnAfterFi{%
          \HyPsd@ProtectSpaces\Hy@temp@A
          \HyPsd@AppendItalcorr\Hy@temp@A
          \expandafter\HyPsd@RemoveBraces\expandafter
            {\Hy@temp@A#2}%
        }%
      \fi
    }%
  \fi
}
%    \end{macrocode}
%    \end{macro}
%    \begin{macro}{\HyPsd@AppendItalcorr}
%    \begin{macro}{\HyPsd@@AppendItalcorr}
%    The string can contain commands yet, so it is better
%    to use \cs{def} instead of a shorter \cs{edef}.
%    The two help macros limit the count of \cs{expandafter}.
%    \begin{macrocode}
\def\HyPsd@AppendItalcorr#1{%
  \expandafter\HyPsd@@AppendItalcorr\expandafter{\/}#1%
}
\def\HyPsd@@AppendItalcorr#1#2{%
  \expandafter\def\expandafter#2\expandafter{#2#1}%
}
%    \end{macrocode}
%    \end{macro}
%    \end{macro}
%
% \subsubsection{Catcode check}
% \paragraph{Check catcodes.}
%    \begin{macro}{\HyPsd@CheckCatcodes}
%    Because \cs{ifcat} expands its arguments, this is
%    prevented by \cs{noexpand}. In case of command tokens
%    and active characters \cs{ifcat} now sees a \cs{relax}.
%    After protecting spaces and removing braces |#1| should
%    be a single token, no group of several tokens, nor an
%    empty group. (So the \cs{expandafter}\cs{relax} between
%    \cs{ifcat} and \cs{noexpand} is only for safety and
%    it should be possible to remove it.)
%
%    \cs{protect} and \cs{relax} should be removed silently.
%    But it is too dangerous and breaks some code giving them
%    the meaning of \cs{@empty}. So commands with the meaning
%    of \cs{protect} are removed here. (\cs{protect} should
%    have the meaning of \cs{@typeset@protect} that
%    is equal to \cs{relax}).
%
%    \begin{macrocode}
\def\HyPsd@CheckCatcodes#1#2\HyPsd@End{%
  \global\let\HyPsd@Rest\relax
  \ifcat\relax\noexpand#1\relax
    \ifx#1\protect
    \else
      \ifx#1\penalty
        \setbox\z@=\hbox{%
          \afterassignment\HyPsd@AfterCountRemove
          \count@=#2\HyPsd@End
        }%
      \else
        \ifx#1\kern
          \setbox\z@=\hbox{%
            \afterassignment\HyPsd@AfterDimenRemove
            \dimen@=#2\HyPsd@End
          }%
        \else
          \ifx#1\hskip
            \setbox\z@=\hbox{%
              \afterassignment\HyPsd@AfterSkipRemove
              \skip@=#2\HyPsd@End
            }%
          \else
            \HyPsd@CatcodeWarning{#1}%
          \fi
        \fi
      \fi
    \fi
  \else
    \ifcat#1 %SPACE
      \expandafter\def\expandafter\HyPsd@String\expandafter{%
        \HyPsd@String\HyPsd@SPACEOPTI
      }%
    \else
      \ifcat$#1%
        \HyPsd@CatcodeWarning{math shift}%
      \else
        \ifcat&#1%
          \HyPsd@CatcodeWarning{alignment tab}%
        \else
          \ifcat^#1%
            \HyPsd@CatcodeWarning{superscript}%
          \else
            \ifcat_#1%
              \HyPsd@CatcodeWarning{subscript}%
            \else
              \expandafter\def\expandafter\HyPsd@String\expandafter{%
                \HyPsd@String#1%
              }%
            \fi
          \fi
        \fi
      \fi
    \fi
  \fi
  \ifx\HyPsd@Rest\relax
    \Hy@ReturnAfterElseFi{%
      \ifx\scrollmode#2\scrollmode
      \else
        \Hy@ReturnAfterFi{%
          \HyPsd@CheckCatcodes#2\HyPsd@End
        }%
      \fi
    }%
  \else
    \Hy@ReturnAfterFi{%
      \ifx\HyPsd@Rest\@empty
      \else
        \expandafter\HyPsd@CheckCatcodes\HyPsd@Rest\HyPsd@End
      \fi
    }%
  \fi
}
%    \end{macrocode}
%    \end{macro}
%
% \paragraph{Remove counts, dimens, skips.}
%    \begin{macro}{\HyPsd@AfterCountRemove}
%    Counts like \cs{penalty} are removed silently.
%    \begin{macrocode}
\def\HyPsd@AfterCountRemove#1\HyPsd@End{%
  \gdef\HyPsd@Rest{#1}%
}
%    \end{macrocode}
%    \end{macro}
%    \begin{macro}{\HyPsd@AfterDimenRemove}
%    If the value of the dimen (\cs{kern}) is zero, it can be
%    removed silently. All other values are difficult to interpret.
%    Negative values do not work in bookmarks. Should positive
%    values be removed or should they be replaced by space(s)?
%    The following code replaces positive values greater than
%    |1ex| with a space and removes them else.
%    \begin{macrocode}
\def\HyPsd@AfterDimenRemove#1\HyPsd@End{%
  \ifdim\ifx\HyPsd@String\@empty\z@\else\dimen@\fi>1ex
    \HyPsd@ReplaceSpaceWarning{\string\kern\space\the\dimen@}%
    \gdef\HyPsd@Rest{\HyPsd@UnexpandableSpace #1}%
  \else
    \ifdim\dimen@=\z@
    \else
      \HyPsd@RemoveSpaceWarning{\string\kern\space\the\dimen@}%
    \fi
    \gdef\HyPsd@Rest{#1}%
  \fi
}
%    \end{macrocode}
%    \end{macro}
%    \begin{macro}{\HyPsd@AfterSkipRemove}
%    The glue part of skips do not work in PDF strings and are ignored.
%    Skips (\cs{hskip}), that are not zero, have the same
%    interpreting problems like dimens (see above).
%    \begin{macrocode}
\def\HyPsd@AfterSkipRemove#1\HyPsd@End{%
  \ifdim\ifx\HyPsd@String\@empty\z@\else\skip@\fi>1ex
    \HyPsd@ReplaceSpaceWarning{\string\hskip\space\the\skip@}%
    \gdef\HyPsd@Rest{\HyPsd@UnexpandableSpace #1}%
  \else
    \ifdim\skip@=\z@
    \else
      \HyPsd@RemoveSpaceWarning{\string\kern\space\the\skip@}%
    \fi
    \gdef\HyPsd@Rest{#1}%
  \fi
}
%    \end{macrocode}
%    \end{macro}
%
% \paragraph{Catcode warnings.}
%    \begin{macro}{\HyPsd@CatcodeWarning}
%    \cs{HyPsd@CatcodeWarning} produces a warning for the user.
%    \begin{macrocode}
\def\HyPsd@CatcodeWarning#1{%
  \HyPsd@Warning{%
    Token not allowed in a PDFDocEncoded string,%
    \MessageBreak removing `\HyPsd@RemoveCmdPrefix#1'%
  }%
}
\begingroup
  \catcode`\|=0
  \catcode`\\=12
%    \end{macrocode}
%    \SpecialEscapechar{\|}
%    \vspace{-2\MacrocodeTopsep}
%    \vspace{-\parskip}
%    \vspace{-\partopsep}
%    \begin{macrocode}
  |gdef|HyPsd@RemoveCmdPrefix#1{%
    |expandafter|HyPsd@@RemoveCmdPrefix
      |string#1|@empty\<>-|@empty|@empty
  }%
  |gdef|HyPsd@@RemoveCmdPrefix#1\<>-#2|@empty#3|@empty{#1#2}%
|endgroup
%    \end{macrocode}
%    \SpecialEscapechar{\\}
%    \end{macro}
%    \begin{macro}{\HyPsd@RemoveSpaceWarning}
%    \begin{macrocode}
\def\HyPsd@RemoveSpaceWarning#1{%
  \HyPsd@Warning{%
    Token not allowed in a PDFDocEncoded string:%
    \MessageBreak #1\MessageBreak
    removed%
  }%
}
%    \end{macrocode}
%    \end{macro}
%    \begin{macro}{\HyPsd@ReplaceSpaceWarning}
%    \begin{macrocode}
\def\HyPsd@ReplaceSpaceWarning#1{%
  \HyPsd@Warning{%
    Token not allowed in a PDFDocEncoded string:%
    \MessageBreak #1\MessageBreak
    replaced by space%
  }%
}
%    \end{macrocode}
%    \end{macro}
%
% \subsubsection{Check for wrong glyphs}
%    A wrong glyph is marked with \cs{relax}, the glyph
%    name follows, delimited by |>|. \cs{@empty} ends
%    the string.
%    \begin{macrocode}
\def\HyPsd@GlyphProcess#1\relax#2\@empty{%
  \expandafter\def\expandafter\HyPsd@String\expandafter{%
    \HyPsd@String#1%
  }%
  \ifx\\#2\\%
  \else
    \Hy@ReturnAfterFi{%
      \HyPsd@GlyphProcessWarning#2\@empty
    }%
  \fi
}
\def\HyPsd@GlyphProcessWarning#1>#2\@empty{%
  \HyPsd@Warning{%
    Glyph not defined in %
    P\ifHy@unicode U\else D1\fi\space encoding,\MessageBreak
    removing `\@backslashchar#1'%
  }%
  \HyPsd@GlyphProcess#2\@empty
}
%    \end{macrocode}
%
% \paragraph{Spaces.}
%    \begin{macro}{\HyPsd@spaceopti}
%    In the string the spaces are represented by \cs{HyPsd@spaceopti}
%    tokens. Within an \cs{edef} it prints itself as
%    a simple space and looks for its next argument.
%    If another space follows, so it replaces the next \cs{HyPsd@spaceopti}
%    by an protected space \cs{040}.
%    \begin{macrocode}
\def\HyPsd@spaceopti#1{ % first space
  \ifx\HyPsd@spaceopti#1%
    \040%
  \else
    #1%
  \fi
}%
%    \end{macrocode}
%    \end{macro}
%
% \subsubsection{Replacing tokens}
% \begin{macro}{\HyPsd@Subst}
%    To save tokens \cs{HyPsd@StringSubst} is an wrapper for the
%    command \cs{HyPsd@Subst} that does all the work:
%    In string stored in command |#3| it replaces the tokens
%    |#1| with |#2|.\\
%    \begin{tabular}{ll}
%    |#1|& Exact the tokens that should be replaced.\\
%    |#2|& The replacement (don't need to be expanded).\\
%    |#3|& Command with the string.
%    \end{tabular}
%    \begin{macrocode}
\def\HyPsd@Subst#1#2#3{%
  \def\HyPsd@@Replace##1#1##2\END{%
    ##1%
    \ifx\\##2\\%
    \else
      #2%
      \Hy@ReturnAfterFi{%
        \HyPsd@@Replace##2\END
      }%
    \fi
  }%
  \xdef#3{%
    \expandafter\HyPsd@@Replace#3#1\END
  }%
}
%    \end{macrocode}
% \end{macro}
% \begin{macro}{\HyPsd@StringSubst}
%    To save tokens in \cs{pdfstringdef} \cs{HyPsd@StringSubst} is a wrapper,
%    that expands argument |#1| before calling \cs{HyPsd@Subst}.
%    \begin{macrocode}
\def\HyPsd@StringSubst#1{%
  \expandafter\HyPsd@Subst\expandafter{\string#1}%
}
%    \end{macrocode}
% \end{macro}
%
% \subsubsection{Support for package \texttt{xspace}}
%    \begin{macro}{\HyPsd@doxspace}
%    \cs{xspace} does not work, because it uses a \cs{futurelet}
%    that cannot be executed in \TeX's mouth. So this implementation
%    uses an argument to examine the next token. In a previous version
%    I reused \cs{@xspace}, but this version is shorter and easier
%    to understand.
%    \begin{macrocode}
\def\HyPsd@doxspace#1{%
  \ifx#1\relax\else
   \ifx#1.\else
    \ifx#1:\else
     \ifx#1,\else
      \ifx#1;\else
       \ifx#1!\else
        \ifx#1?\else
         \ifx#1/\else
          \ifx#1-\else
           \ifx#1'\else
             \HyPsd@SPACEOPTI
           \fi
          \fi
         \fi
        \fi
       \fi
      \fi
     \fi
    \fi
   \fi
  \fi
  #1%
}%
%    \end{macrocode}
%    \end{macro}
%
% \subsubsection{Converting to Unicode}
%    Eight bit characters are converted to the sixteen bit ones,
%    \cs{8} is replaced by \cs{00}, and \cs{9} is removed.
%    The result should be a valid Unicode PDF string without the
%    Unicode marker at the beginning.
%    \begin{macrocode}
\begingroup
  \catcode`\|=0
  \catcode`\\=12
%    \end{macrocode}
%    \SpecialEscapechar{\|}
%    \begin{macro}{\HyPsd@ConvertToUnicode}
%    \begin{macrocode}
  |gdef|HyPsd@ConvertToUnicode#1{%
    |xdef#1{%
      \376\377%
      |expandafter|HyPsd@DoConvert#1|@empty|@empty|@empty
    }%
  }
%    \end{macrocode}
%    \end{macro}
%    \begin{macro}{\HyPsd@DoConvert}
%    \begin{macrocode}
  |gdef|HyPsd@DoConvert#1{%
    |ifx#1|@empty
    |else
      |Hy@ReturnAfterFi{%
        |ifx#1\%
          \%
          |expandafter|HyPsd@DoEscape
        |else
          \000#1%
          |expandafter|HyPsd@DoConvert
        |fi
      }%
    |fi
  }
%    \end{macrocode}
%    \end{macro}
%    \begin{macro}{\HyPsd@DoEscape}
%    \begin{macrocode}
  |gdef|HyPsd@DoEscape#1{%
    |ifx#19%
      |expandafter|HyPsd@GetTwoBytes
    |else
      |Hy@ReturnAfterFi{%
        |ifx#18%
          00%
          |expandafter|HyPsd@GetTwoBytes
        |else
          #1%
          |expandafter|HyPsd@GetOneByte
        |fi
      }%
    |fi
  }
%    \end{macrocode}
%    \end{macro}
%    \begin{macro}{\HyPsd@GetTwoBytes}
%    \begin{macrocode}
  |gdef|HyPsd@GetTwoBytes#1\#2#3#4{%
    #1\#2#3#4%
    |HyPsd@DoConvert
  }
%    \end{macrocode}
%    \end{macro}
%    \begin{macro}{\HyPsd@GetOneBye}
%    \begin{macrocode}
  |gdef|HyPsd@GetOneByte#1#2{%
    #1#2%
    |HyPsd@DoConvert
  }
|endgroup
%    \end{macrocode}
%    \end{macro}
%    \SpecialEscapechar{\\}
%    \begin{macro}{\HyPsd@@GetNextTwoTokens}
%    \TeX{} does only allow nine parameters, so we need another macro
%    to get more arguments.
%    \begin{macrocode}
\def\HyPsd@GetNextTwoTokens#1#2#3\END#4{%
  \xdef#4{#4#1#2}%
  \HyPsd@@ConvertToUnicode#3\END#4%
}
%    \end{macrocode}
%    \end{macro}
%
% \section{Support of other packages}
%
% \subsection{Package subfigure}
%    Added fix for version 2.1. Here \cmd{\sub@label} is defined.
%    \begin{macrocode}
\@ifpackageloaded{subfigure}{%
  \@ifundefined{sub@label}{%
    \Hy@hypertexnamesfalse
  }{%
    \renewcommand*{\sub@label}[1]{%
      \@bsphack
      \subfig@oldlabel{#1}%
      \begingroup
        \edef\@currentlabstr{%
          \expandafter\strip@prefix\meaning\@currentlabelname
        }%
        \protected@write\@auxout{}{%
          \string\newlabel{sub@#1}{%
            {\@nameuse{@@thesub\@captype}}%
            {\thepage}%
            {\expandafter\strip@period\@currentlabstr\relax.\relax\@@@}%
            {\@currentHref}%
            {}%
          }%
        }%
      \endgroup
      \@esphack
    }%
    \@ifpackagelater{subfigure}{2002/03/26}{}{%
      \providecommand*{\toclevel@subfigure}{1}%
      \providecommand*{\toclevel@subtable}{1}%
    }%
  }%
}{}
%    \end{macrocode}
%
% \subsection{Package xr and xr-hyper}
%    The beta version of xr that supports \cmd{\XR@addURL} is called
%    \verb|xr-hyper|. Therefore we test for the macro itself and not
%    for the package name:
%    \begin{macrocode}
\@ifundefined{XR@addURL}{%
}{%
%    \end{macrocode}
% If reading external aux files check whether they have a non zero
% fourth field in |\newlabel| and if so, add the URL as the fifth field.
%    \begin{macrocode}
  \def\XR@addURL#1{\XR@@dURL#1{}{}{}{}\\}%
  \def\XR@@dURL#1#2#3#4#5\\{%
    {#1}{#2}%
    \if!#4!%
    \else
      {#3}{#4}{\XR@URL}%
    \fi
  }%
}
%    \end{macrocode}
%
%    \begin{macrocode}
\def\Hy@true{true}
\def\Hy@false{false}
%    \end{macrocode}
%
%    Providing dummy definitions.
%    \begin{macrocode}
\let\literalps@out\@gobble
\newcommand\pdfbookmark[3][]{}
\let\Acrobatmenu\@gobble
\def\Hy@writebookmark#1#2#3#4#5{}%
%    \end{macrocode}
%
% \section{Help macros for links}
% Anchors get created on the baseline of where they occur.   If an
% XYZ PDF view is set, this means that the link places the top of the
% screen \emph{on the baseline} of the target. If this is an equation,
% for instance, it means that you cannot see anything. Some links, of
% course, are created at the start of environments, and so it works. To
% allow for this, anchors are raised, where possible, by some small
% amount. This defaults to |\baselineskip|, but users can set it to
% something else in two ways (thanks to Heiko Oberdiek for suggesting this):
% \begin{enumerate}
% \item Redefine |\HyperRaiseLinkDefault| to be eg the height of a |\strut|
% \item Redefine  |\HyperRaiseLinkHook| to do something complicated;
%  it must give a value to |\HyperRaiseLinkLength|, which is what
%  actually gets used
% \end{enumerate}
%    \begin{macrocode}
\let\HyperRaiseLinkLength\@tempdima
\let\HyperRaiseLinkHook\@empty
\def\HyperRaiseLinkDefault{\baselineskip}
%    \end{macrocode}
%    |\HyperRaiseLinkHook| allows the user to reassign
%    |\HyperRaiseLinkLength|.
%    \begin{macrocode}
\def\Hy@raisedlink#1{%
  \setlength\HyperRaiseLinkLength\HyperRaiseLinkDefault
  \HyperRaiseLinkHook
  \ifvmode
    #1%
  \else
    \smash{\raise\HyperRaiseLinkLength\hbox{#1}}%
  \fi
}
%    \end{macrocode}
%
%    \begin{macro}{\Hy@SaveLastskip}
%    \begin{macro}{\Hy@RestoreLastskip}
%    Inserting a \cmd{\special} command to set a
%    destination destroys the \cmd{\lastskip} value.
%
%    \begin{macrocode}
\def\Hy@SaveLastskip{%
  \let\Hy@RestoreLastskip\relax
  \ifvmode
    \ifdim\lastskip=\z@
      \let\Hy@RestoreLastskip\nobreak
    \else
      \begingroup
        \skip@=-\lastskip
        \edef\x{%
          \endgroup
          \def\noexpand\Hy@RestoreLastskip{%
            \noexpand\ifvmode
              \noexpand\nobreak
              \vskip\the\skip@
              \vskip\the\lastskip\relax
            \noexpand\fi
          }%
        }%
      \x
    \fi
  \else
    \ifhmode
      \ifdim\lastskip=\z@
        \let\Hy@RestoreLastskip\nobreak
      \else
        \begingroup
          \skip@=-\lastskip
          \edef\x{%
            \endgroup
            \def\noexpand\Hy@RestoreLastskip{%
              \noexpand\ifhmode
                \noexpand\nobreak
                \hskip\the\skip@
                \hskip\the\lastskip\relax
              \noexpand\fi
            }%
          }%
        \x
      \fi
    \fi
  \fi
}%
%    \end{macrocode}
%    \end{macro}
%    \end{macro}
%
% \section{Options}
%
% \subsection{Help macros}
%
%    \begin{macro}{\Hy@boolkey}
%    \begin{macrocode}
\def\Hy@boolkey{\@dblarg\Hy@@boolkey}
\def\Hy@@boolkey[#1]#2#3{%
  \lowercase{\def\Hy@tempa{#3}}%
  \ifx\Hy@tempa\@empty
    \let\Hy@tempa\Hy@true
  \fi
  \ifx\Hy@tempa\Hy@true
  \else
    \ifx\Hy@tempa\Hy@false
    \else
      \let\Hy@tempa\relax
    \fi
  \fi
  \ifx\Hy@tempa\relax
    \Hy@WarnOptionValue{#3}{#1}{`true' or 'false'}%
  \else
    \Hy@Info{Option `#1' set `\Hy@tempa'}%
    \csname Hy@#2\Hy@tempa\endcsname
  \fi
}
%    \end{macrocode}
%    \end{macro}
%    \begin{macro}{\Hy@WarnOptionValue}
%    \begin{macrocode}
\def\Hy@WarnOptionValue#1#2#3{%
  \Hy@Warning{%
    Unexpected value `#1'\MessageBreak
    of option `#2' instead of\MessageBreak
    #3%
  }
}
%    \end{macrocode}
%    \end{macro}
%
%    \begin{macro}{\Hy@DisableOption}
%    \begin{macrocode}
\def\Hy@DisableOption#1{%
  \@ifundefined{KV@Hyp@#1@default}{%
    \define@key{Hyp}{#1}%
  }{%
    \define@key{Hyp}{#1}[]%
  }%
  {\Hy@WarnOptionDisabled{#1}}%
}
%    \end{macrocode}
%    \end{macro}
%
%    \begin{macro}{\Hy@WarnOptionDisabled}
%    \begin{macrocode}
\def\Hy@WarnOptionDisabled#1{%
  \Hy@Warning{%
    Option `#1' has already been used,\MessageBreak
    setting the option has no effect%
  }%
}
%    \end{macrocode}
%    \end{macro}
%
% \subsection{Defining the options}
%    \begin{macrocode}
\define@key{Hyp}{implicit}[true]{%
  \Hy@boolkey{implicit}{#1}%
}
\define@key{Hyp}{draft}[true]{%
  \Hy@boolkey{draft}{#1}%
}
\let\KV@Hyp@nolinks\KV@Hyp@draft
\define@key{Hyp}{a4paper}[true]{%
  \def\special@paper{210mm,297mm}%
  \def\Hy@pageheight{842}%
}
\define@key{Hyp}{a5paper}[true]{%
  \def\special@paper{148mm,210mm}%
  \def\Hy@pageheight{595}%
}
\define@key{Hyp}{b5paper}[true]{%
  \def\special@paper{176mm,250mm}%
  \def\Hy@pageheight{709}%
}
\define@key{Hyp}{letterpaper}[true]{%
  \def\special@paper{8.5in,11in}%
  \def\Hy@pageheight{792}%
}
\define@key{Hyp}{legalpaper}[true]{%
  \def\special@paper{8.5in,14in}%
  \def\Hy@pageheight{1008}%
}
\define@key{Hyp}{executivepaper}[true]{%
  \def\special@paper{7.25in,10.5in}%
  \def\Hy@pageheight{720}%
}
\define@key{Hyp}{debug}[true]{%
  \Hy@boolkey[debug]{verbose}{#1}%
}
\define@key{Hyp}{linktocpage}[true]{%
  \Hy@boolkey{linktocpage}{#1}%
}
\define@key{Hyp}{extension}{\def\XR@ext{#1}}
\def\XR@ext{dvi}
\define@key{Hyp}{verbose}[true]{%
  \Hy@boolkey{verbose}{#1}%
}
\define@key{Hyp}{typexml}[true]{%
  \Hy@boolkey{typexml}{#1}%
}
%    \end{macrocode}
% If we are going to PDF via HyperTeX |\special| commands,
% the dvips (-z option)  processor does not know
% the \emph{height} of a link, as it works solely on the
% position of the closing |\special|. If we use this option,
% the |\special| is raised up by the right amount, to fool
% the dvi processor.
%    \begin{macrocode}
\define@key{Hyp}{raiselinks}[true]{%
  \Hy@boolkey{raiselinks}{#1}%
}
%    \end{macrocode}
% Most PDF-creating drivers do not allow links to be broken
%    \begin{macrocode}
\define@key{Hyp}{breaklinks}[true]{%
  \Hy@boolkey{breaklinks}{#1}%
}
%    \end{macrocode}
% Determines whether an automatic anchor is put on each page
%    \begin{macrocode}
\define@key{Hyp}{pageanchor}[true]{%
  \Hy@boolkey{pageanchor}{#1}%
}
%    \end{macrocode}
% Are the page links done as plain arabic numbers, or do
% they follow the formatting of the package? The latter loses
% if you put in typesetting like |\textbf| or the like.
%    \begin{macrocode}
\define@key{Hyp}{plainpages}[true]{%
  \Hy@boolkey{plainpages}{#1}%
}
%    \end{macrocode}
% Are the names for anchors made as per the HyperTeX system,
% or do they simply use what \LaTeX\ provides?
%    \begin{macrocode}
\define@key{Hyp}{naturalnames}[true]{%
  \Hy@boolkey{naturalnames}{#1}%
}
%    \end{macrocode}
% Completely ignore the names as per the HyperTeX system,
% and use unique counters.
%    \begin{macrocode}
\define@key{Hyp}{hypertexnames}[true]{%
  \Hy@boolkey{hypertexnames}{#1}%
}
%    \end{macrocode}
% Currently, |dvips| doesn't allow anchors nested within targets,
% so this option tries to stop that happening. Other processors
% may be able to cope.
%    \begin{macrocode}
\define@key{Hyp}{nesting}[true]{%
  \Hy@boolkey{nesting}{#1}%
}
%    \end{macrocode}
%
%    \begin{macrocode}
\define@key{Hyp}{unicode}[true]{%
  \Hy@boolkey{unicode}{#1}%
  \HyPsd@InitUnicode
}
%    \end{macrocode}
%
% \section{Options for different drivers}\label{drivers}
%    \begin{macrocode}
\define@key{Hyp}{hyperref}[true]{}
\define@key{Hyp}{tex4ht}[true]{%
  \def\XR@ext{html}%
  \Hy@texhttrue
  \def\Hy@raisedlink{}%
  \setkeys{Hyp}{colorlinks=true}%
  \def\BeforeTeXIVht{\usepackage{color}}%
  \def\Hy@driver{htex4ht}%
  \def\MaybeStopEarly{%
    \typeout{Hyperref stopped early}%
    \AfterBeginDocument{\PDF@FinishDoc}%
    \endinput
  }%
}
\define@key{Hyp}{pdftex}[true]{%
  \def\Hy@driver{hpdftex}%
  \def\XR@ext{pdf}%
  \PassOptionsToPackage{pdftex}{color}%
  \Hy@breaklinkstrue
}
\define@key{Hyp}{dvipdf}[true]{%
}
\define@key{Hyp}{nativepdf}[true]{%
  \def\Hy@driver{hdvips}%
  \def\Hy@raisedlink{}%
  \def\XR@ext{pdf}%
}
\define@key{Hyp}{dvipdfm}[true]{%
  \def\Hy@driver{hdvipdfm}%
  \def\XR@ext{pdf}%
  \Hy@breaklinkstrue
}
\define@key{Hyp}{pdfmark}[true]{%
  \def\Hy@driver{hdvips}%
  \def\Hy@raisedlink{}%
  \def\XR@ext{pdf}%
}
\define@key{Hyp}{dvips}[true]{%
  \def\Hy@driver{hdvips}%
  \def\Hy@raisedlink{}%
  \def\XR@ext{pdf}%
}
\define@key{Hyp}{hypertex}[true]{%
  \def\Hy@driver{hypertex}%
}
\let\Hy@MaybeStopNow\relax
\define@key{Hyp}{vtex}[true]{%
  \begingroup\expandafter\expandafter\expandafter\endgroup
  \expandafter\ifx\csname OpMode\endcsname\relax
    \@latex@error{Non-VTeX processor}{}%
    \global\let\Hy@MaybeStopNow\endinput
    \endinput
  \else
    \ifnum 0\ifnum\OpMode<1 1\fi \ifnum\OpMode>3 1\fi =0 %
      \def\XR@ext{pdf}%
      \def\Hy@driver{hvtex}%
    \else
      \ifnum\OpMode=10\relax
        \def\XR@ext{htm}%
        \def\Hy@driver{hvtexhtm}%
        \def\MaybeStopEarly{%
           \typeout{Hyperref stopped early}%
           \AfterBeginDocument{\PDF@FinishDoc}%
           \endinput
        }%
      \else
        \@latex@error{Mode (\the\OpMode) has no hyperref driver}{}%
        \global\let\Hy@MaybeStopNow\endinput
        \endinput
      \fi
    \fi
  \fi
}
\define@key{Hyp}{vtexpdfmark}[true]{%
  \begingroup\expandafter\expandafter\expandafter\endgroup
  \expandafter\ifx\csname OpMode\endcsname\relax
    \@latex@error{Non-VTeX processor}{}%
    \let\Hy@MaybeStopNow\endinput
    \endinput
  \else
    \ifnum 0\ifnum\OpMode<1 1\fi \ifnum\OpMode>3 1\fi =0 %
      \def\XR@ext{pdf}%
      \def\Hy@driver{hvtexmrk}%
      \def\Hy@raisedlink{}%
    \else
      \@latex@error{Mode (\the\OpMode) has no hyperref driver}{}%
      \let\Hy@MaybeStopNow\endinput
      \endinput
    \fi
  \fi
}
\define@key{Hyp}{dviwindo}[true]{%
  \def\Hy@driver{hdviwind}%
  \setkeys{Hyp}{colorlinks}%
  \PassOptionsToPackage{dviwindo}{color}%
}
\define@key{Hyp}{dvipsone}[true]{%
  \def\XR@ext{pdf}%
  \def\Hy@driver{hdvipson}%
  \def\Hy@raisedlink{}%
}
\define@key{Hyp}{textures}[true]{%
  \def\XR@ext{pdf}%
  \def\Hy@driver{htexture}%
}
\define@key{Hyp}{latex2html}[true]{%
  \AtBeginDocument{\@@latextohtmlX}%
}
%    \end{macrocode}
% No more special treatment for ps2pdf. Let it sink or swim.
%    \begin{macrocode}
\define@key{Hyp}{ps2pdf}[true]{%
  \def\Hy@driver{hdvips}%
  \def\Hy@raisedlink{}%
  \providecommand\@pdfborder{0 0 1}%
}
%    \end{macrocode}
%
% \section{Options to add extra features}\label{features}
%    Make included figures (assuming they use the standard graphics
%     package) be hypertext links. Off by default. Needs more work.
%    \begin{macrocode}
\define@key{Hyp}{hyperfigures}[true]{%
  \Hy@boolkey[hyperfigures]{figures}{#1}%
}
%    \end{macrocode}
%
%    The automatic footnote linking can be disabled
%    by option hyperfootnotes.
%    \begin{macrocode}
\define@key{Hyp}{hyperfootnotes}[true]{%
  \Hy@boolkey{hyperfootnotes}{#1}%
}
%    \end{macrocode}
%
%    Set up back-referencing to be hyper links, by page,
%     slide or section number,
%    \begin{macrocode}
\def\back@none{none}
\def\back@section{section}
\def\back@page{page}
\def\back@slide{slide}
\define@key{Hyp}{backref}[section]{%
  \lowercase{\def\Hy@tempa{#1}}%
  \ifx\Hy@tempa\@empty
    \let\Hy@tempa\back@section
  \fi
  \ifx\Hy@tempa\Hy@false
    \let\Hy@tempa\back@none
  \fi
  \ifx\Hy@tempa\back@slide
    \let\Hy@tempa\back@section
  \fi
  \ifx\Hy@tempa\back@page
    \PassOptionsToPackage{hyperpageref}{backref}%
    \Hy@backreftrue
  \else
    \ifx\Hy@tempa\back@section
      \PassOptionsToPackage{hyperref}{backref}%
      \Hy@backreftrue
    \else
      \ifx\Hy@tempa\back@none
        \Hy@backreffalse
      \else
        \Hy@WarnOptionValue{#1}{backref}{%
          `section', `slide', `page', `none',\MessageBreak
          or `false'}%
      \fi
    \fi
  \fi
}
\define@key{Hyp}{pagebackref}[true]{%
  \lowercase{\def\Hy@tempa{#1}}%
  \ifx\Hy@tempa\@empty
    \let\Hy@tempa\Hy@true
  \fi
  \ifx\Hy@tempa\Hy@true
    \PassOptionsToPackage{hyperpageref}{backref}%
    \Hy@backreftrue
  \else
    \ifx\Hy@tempa\Hy@false
      \Hy@backreffalse
    \else
      \Hy@WarnOptionValue{#1}{pagebackref}{`true' or `false'}%
    \fi
  \fi
}
%    \end{macrocode}
% Make index entries be links back to the relevant pages. By default
% this is turned on, but may be stopped.
%    \begin{macrocode}
\define@key{Hyp}{hyperindex}[true]{%
  \Hy@boolkey{hyperindex}{#1}%
}
%    \end{macrocode}
%
% \section{Language options}
%
%    \begin{macrocode}
\def\Hy@setcaptions#1#2{%
  \@ifpackageloaded{babel}{%
    \expandafter\addto\csname captions#2\expandafter\endcsname
    \expandafter{#1}%
  }{%
    #1%
  }%
}
\def\Hy@autorefenglish{\Hy@setcaptions\Hy@captionsenglish}
\def\Hy@autorefgerman{\Hy@setcaptions\Hy@captionsgerman}
\def\Hy@autorefportuges{\Hy@setcaptions\Hy@captionsportuges}
\def\Hy@autorefspanish{\Hy@setcaptions\Hy@captionsspanish}
\def\Hy@captionsenglish{%
  \def\equationautorefname{Equation}%
  \def\footnoteautorefname{footnote}%
  \def\itemautorefname{item}%
  \def\figureautorefname{Figure}%
  \def\tableautorefname{Table}%
  \def\partautorefname{Part}%
  \def\appendixautorefname{Appendix}%
  \def\chapterautorefname{chapter}%
  \def\sectionautorefname{section}%
  \def\subsectionautorefname{subsection}%
  \def\subsubsectionautorefname{subsubsection}%
  \def\paragraphautorefname{paragraph}%
  \def\subparagraphautorefname{subparagraph}%
  \def\FancyVerbLineautorefname{line}%
  \def\theoremautorefname{Theorem}%
}
\def\Hy@captionsgerman{%
  \def\equationautorefname{Gleichung}%
  \def\footnoteautorefname{Fu\ss note}%
  \def\itemautorefname{Punkt}%
  \def\figureautorefname{Abbildung}%
  \def\tableautorefname{Tabelle}%
  \def\partautorefname{Teil}%
  \def\appendixautorefname{Anhang}%
  \def\chapterautorefname{Kapitel}%
  \def\sectionautorefname{Abschnitt}%
  \def\subsectionautorefname{Unterabschnitt}%
  \def\subsubsectionautorefname{Unterunterabschnitt}%
  \def\paragraphautorefname{Absatz}%
  \def\subparagraphautorefname{Unterabsatz}%
  \def\FancyVerbLineautorefname{Zeile}%
  \def\theoremautorefname{Theorem}%
}
\def\Hy@captionsportuges{%
  \def\equationautorefname{Equa\c c\~ao}%
  \def\footnoteautorefname{Nota de rodap\'e}%
  \def\itemautorefname{Item}%
  \def\figureautorefname{Figura}%
  \def\tableautorefname{Tabela}%
  \def\partautorefname{Parte}%
  \def\appendixautorefname{Ap\^endice}%
  \def\chapterautorefname{Cap\'itulo}%
  \def\sectionautorefname{Se\c c\~ao}%
  \def\subsectionautorefname{Subse\c c\~ao}%
  \def\subsubsectionautorefname{Subsubse\c c\~ao}%
  \def\paragraphautorefname{par\'agrafo}%
  \def\subparagraphautorefname{subpar\'agrafo}%
  \def\FancyVerbLineautorefname{linha}%
  \def\theoremautorefname{Teorema}%
}
\def\Hy@captionsspanish{%
  \def\equationautorefname{Ecuaci\'on}%
  \def\footnoteautorefname{Nota a pie de p\'agina}%
  \def\itemautorefname{Elemento}%
  \def\figureautorefname{Figura}%
  \def\tableautorefname{Tabla}%
  \def\partautorefname{Parte}%
  \def\appendixautorefname{Ap\'endice}%
  \def\chapterautorefname{Cap\'itulo}%
  \def\sectionautorefname{Secci\'on}%
  \def\subsectionautorefname{Subsecci\'on}%
  \def\subsubsectionautorefname{Subsubsecci\'on}%
  \def\paragraphautorefname{P\'arrafo}%
  \def\subparagraphautorefname{Subp\'arrafo}%
  \def\FancyVerbLineautorefname{L\'inea}%
  \def\theoremautorefname{Teorema}%
}
\define@key{Hyp}{english}[]{\Hy@autorefenglish{english}}
\define@key{Hyp}{UKenglish}[]{\Hy@autorefenglish{UKenglish}}
\define@key{Hyp}{british}[]{\Hy@autorefenglish{british}}
\define@key{Hyp}{USenglish}[]{\Hy@autorefenglish{USenglish}}
\define@key{Hyp}{american}[]{\Hy@autorefenglish{american}}
\define@key{Hyp}{german}[]{\Hy@autorefgerman{german}}
\define@key{Hyp}{austrian}[]{\Hy@autorefgerman{austrian}}
\define@key{Hyp}{ngerman}[]{\Hy@autorefgerman{ngerman}}
\define@key{Hyp}{naustrian}[]{\Hy@autorefgerman{naustrian}}
\define@key{Hyp}{brazil}[]{\Hy@autorefportuges{brazil}}
\define@key{Hyp}{brazilian}[]{\Hy@autorefportuges{brazilian}}
\define@key{Hyp}{portuguese}[]{\Hy@autorefportuges{portuguese}}
\define@key{Hyp}{spanish}[]{\Hy@autorefspanish{spanish}}
%    \end{macrocode}
%
% \section{Options to change appearance of links}\label{appearance}
% Colouring links at the \LaTeX\ level is useful for debugging, perhaps.
%    \begin{macrocode}
\define@key{Hyp}{colorlinks}[true]{%
  \Hy@boolkey{colorlinks}{#1}%
  \ifHy@colorlinks
    \def\@pdfborder{0 0 0}%
  \fi
}
\define@key{Hyp}{frenchlinks}[true]{%
  \Hy@boolkey{frenchlinks}{#1}%
}
%    \end{macrocode}
%
% \section{Bookmarking}
%    \begin{macrocode}
\define@key{Hyp}{bookmarks}[true]{%
  \Hy@boolkey{bookmarks}{#1}%
}
\define@key{Hyp}{bookmarksopen}[true]{%
  \Hy@boolkey{bookmarksopen}{#1}%
}
% `bookmarksopenlevel' to specify the open level. From Heiko Oberdiek.
\define@key{Hyp}{bookmarksopenlevel}{%
  \def\@bookmarksopenlevel{#1}%
}
\def\@bookmarksopenlevel{\maxdimen}
% `bookmarkstype' to specify which `toc' file to mimic
\define@key{Hyp}{bookmarkstype}{%
  \def\Hy@bookmarkstype{#1}%
}
\def\Hy@bookmarkstype{toc}
%    \end{macrocode}
% Richard Curnow <richard@curnow.demon.co.uk> suggested this
% functionality. It adds section numbers etc to bookmarks.
%    \begin{macrocode}
\define@key{Hyp}{bookmarksnumbered}[true]{%
  \Hy@boolkey{bookmarksnumbered}{#1}%
}
%    \end{macrocode}
%
%    Option CJKbookmarks enables the patch for
%    CJK bookmarks.
%    \begin{macrocode}
\define@key{Hyp}{CJKbookmarks}[true]{%
  \Hy@boolkey{CJKbookmarks}{#1}%
}
%    \end{macrcode}
%
%    \begin{macrocode}
\define@key{Hyp}{linkcolor}{\def\@linkcolor{#1}}
\define@key{Hyp}{anchorcolor}{\def\@anchorcolor{#1}}
\define@key{Hyp}{citecolor}{\def\@citecolor{#1}}
\define@key{Hyp}{urlcolor}{\def\@urlcolor{#1}}
\define@key{Hyp}{menucolor}{\def\@menucolor{#1}}
\define@key{Hyp}{filecolor}{\def\@filecolor{#1}}
\define@key{Hyp}{pagecolor}{\def\@pagecolor{#1}}
%    \end{macrocode}
% Default values:
%    \begin{macrocode}
\def\@linkcolor{red}
\def\@anchorcolor{black}
\def\@citecolor{green}
\def\@filecolor{cyan}
\def\@urlcolor{magenta}
\def\@menucolor{red}
\def\@pagecolor{red}
\def\hyperbaseurl#1{\def\@baseurl{#1}}
\define@key{Hyp}{baseurl}{\hyperbaseurl{#1}}
\let\@baseurl\@empty
\def\hyperlinkfileprefix#1{\def\Hy@linkfileprefix{#1}}
\define@key{Hyp}{linkfileprefix}{\hyperlinkfileprefix{#1}}
\hyperlinkfileprefix{file:}
%    \end{macrocode}
%
% \section{PDF-specific options}\label{pdfopt}
%
%    \begin{macro}{\@pdfpagetransition}
%    The value of option |pdfpagetransition| is stored in
%    \cmd{\@pdfpagetransition}. Its initial value is set
%    to \cmd{\relax} in order to be able to differentiate
%    between a not used option and an option with an empty
%    value.
%    \begin{macrocode}
\let\@pdfpagetransition\relax
\define@key{Hyp}{pdfpagetransition}{\def\@pdfpagetransition{#1}}
%    \end{macrocode}
%    \end{macro}
%    \begin{macro}{\@pdfpageduration}
%    The value of option |pdfpageduration| is stored in
%    \cmd{\@pdfpageduration}. Its initial value is set
%    to \cmd{\relax} in order to be able to differentiate
%    between a not used option and an option with an empty
%    value.
%    \begin{macrocode}
\let\@pdfpageduration\relax
\define@key{Hyp}{pdfpageduration}{\def\@pdfpageduration{#1}}
%    \end{macrocode}
%    \end{macro}
%
%    The entry for the |/Hid| key in the page object is
%    only necessary, if it is used and set to true for
%    at least one time. If it is always false, then
%    the |/Hid| key is not written to the pdf page
%    object in order not to enlarge the pdf file.
%    \begin{macrocode}
\newif\ifHy@useHidKey
\Hy@useHidKeyfalse
\define@key{Hyp}{pdfpagehidden}[true]{%
  \Hy@boolkey{pdfpagehidden}{#1}%
  \ifHy@pdfpagehidden
    \global\Hy@useHidKeytrue
  \fi
}
%    \end{macrocode}
%
%    \begin{macrocode}
\define@key{Hyp}{linkbordercolor}{\def\@linkbordercolor{#1}}
\define@key{Hyp}{urlbordercolor}{\def\@urlbordercolor{#1}}
\define@key{Hyp}{menubordercolor}{\def\@menubordercolor{#1}}
\define@key{Hyp}{filebordercolor}{\def\@filebordercolor{#1}}
\define@key{Hyp}{runbordercolor}{\def\@runbordercolor{#1}}
\define@key{Hyp}{citebordercolor}{\def\@citebordercolor{#1}}
\define@key{Hyp}{pagebordercolor}{\def\@pagebordercolor{#1}}
\define@key{Hyp}{pdfhighlight}{\def\@pdfhighlight{#1}}
\define@key{Hyp}{pdfborder}{\def\@pdfborder{#1}}
\define@key{Hyp}{pdfpagemode}{%
  \def\Hy@tempa{#1}%
  \ifx\Hy@tempa\@empty
    \let\@pdfpagemode\@empty
  \else
    \def\@pdfpagemode{/#1 }%
  \fi
}
\define@key{Hyp}{pdfusetitle}[true]{%
  \Hy@boolkey[pdfusetitle]{usetitle}{#1}%
}
\define@key{Hyp}{pdftitle}{\pdfstringdef\@pdftitle{#1}}
\define@key{Hyp}{pdfauthor}{\pdfstringdef\@pdfauthor{#1}}
\define@key{Hyp}{pdfproducer}{\pdfstringdef\@pdfproducer{#1}}
\define@key{Hyp}{pdfcreator}{\pdfstringdef\@pdfcreator{#1}}
\define@key{Hyp}{pdfsubject}{\pdfstringdef\@pdfsubject{#1}}
\define@key{Hyp}{pdfkeywords}{\pdfstringdef\@pdfkeywords{#1}}
\define@key{Hyp}{pdfview}{\calculate@pdfview#1 \\}
\define@key{Hyp}{pdflinkmargin}{\setpdflinkmargin{#1}}
\let\setpdflinkmargin\@gobble
\def\calculate@pdfview#1 #2\\{%
  \def\@pdfview{#1}%
  \ifx\\#2\\%
    \def\@pdfviewparams{ -32768}%
  \else
    \def\@pdfviewparams{ #2}%
  \fi
}
\define@key{Hyp}{pdfstartpage}{\def\@pdfstartpage{#1}}
\define@key{Hyp}{pdfstartview}{%
  \ifx\\#1\\%
    \def\@pdfstartview{}%
  \else
    \hypercalcbpdef\@pdfstartview{ /#1 }%
  \fi
}
\define@key{Hyp}{pdfpagescrop}{\edef\@pdfpagescrop{#1}}
\define@key{Hyp}{pdftoolbar}[true]{%
  \Hy@boolkey[pdftoolbar]{toolbar}{#1}%
}
\define@key{Hyp}{pdfmenubar}[true]{%
  \Hy@boolkey[pdfmenubar]{menubar}{#1}%
}
\define@key{Hyp}{pdfwindowui}[true]{%
  \Hy@boolkey[pdfwindowui]{windowui}{#1}%
}
\define@key{Hyp}{pdffitwindow}[true]{%
  \Hy@boolkey[pdffitwindow]{fitwindow}{#1}%
}
\define@key{Hyp}{pdfcenterwindow}[true]{%
  \Hy@boolkey[pdfcenterwindow]{centerwindow}{#1}%
}
\define@key{Hyp}{pdfnewwindow}[true]{%
  \Hy@boolkey[pdfnewwindow]{newwindow}{#1}%
}
\define@key{Hyp}{pdfpagelayout}{\def\pdf@pagelayout{#1}}
\def\pdf@pagelayout{}
\define@key{Hyp}{pdfpagelabels}[true]{%
  \Hy@boolkey[pdfpagelabels]{pagelabels}{#1}%
}
%    \end{macrocode}
% Default values:
%    \begin{macrocode}
\def\@linkbordercolor{1 0 0}
\def\@urlbordercolor{0 1 1}
\def\@menubordercolor{1 0 0}
\def\@filebordercolor{0 .5 .5}
\def\@runbordercolor{0 .7 .7}
\def\@citebordercolor{0 1 0}
\def\@pagebordercolor{1 1 0}
\def\@pdfhighlight{/I}
\def\@pdfpagemode{}
\def\@pdftitle{}
\def\@pdfauthor{}
\def\@pdfproducer{}
\def\@pdfcreator{LaTeX with hyperref package}
\def\@pdfsubject{}
\def\@pdfkeywords{}
\def\@pdfpagescrop{}
\def\@pdfstartview{ /Fit }
\def\@pdfstartpage{1}
\let\PDF@SetupDoc\@empty
\let\PDF@FinishDoc\@empty
\let\phantomsection\@empty
\@ifundefined{stockwidth}{%
  \edef\special@paper{\the\paperwidth,\the\paperheight}
}{%
  \edef\special@paper{\the\stockwidth,\the\stockheight}
}
\begingroup
  \dimen@=\@ifundefined{stockheight}{\paperheight}{\stockheight}\relax
  \dimen@=0.99626401\dimen@
  \xdef\Hy@pageheight{\strip@pt\dimen@}
\endgroup
%    \end{macrocode}
%
%    \begin{macrocode}
\def\hypersetup{\setkeys{Hyp}}
%    \end{macrocode}
%
% Allow the user to use |\ExecuteOptions| in the cfg file even though
% this package does not use the normal option mechanism.
% Use |\hyper@normalise| as a scratch macro, since it is going to
% be defined in a couple of lines anyway.
%    \begin{macrocode}
\let\hyper@normalise\ExecuteOptions
\let\ExecuteOptions\hypersetup
\Hy@RestoreCatcodes
\InputIfFileExists{hyperref.cfg}{}{}
\Hy@SetCatcodes
\let\ExecuteOptions\hyper@normalise
\Hy@MaybeStopNow
%    \end{macrocode}
% To add flexibility, we will not use the ordinary processing of
% package options, but put them through the \emph{keyval} package.
% This section was written by David Carlisle.
%    \begin{macrocode}
\def\ProcessOptionsWithKV#1{%
  \let\@tempc\relax
  \let\Hy@tempa\@empty
%    \end{macrocode}
% Add any global options that are known to KV to the start of the list
% being built in |\Hy@tempa| and mark them used (by removing them from
% the unused option list).
%    \begin{macrocode}
  \@for\CurrentOption:=\@classoptionslist\do{%
    \@ifundefined{KV@#1@\CurrentOption}%
    {}%
    {%
      \edef\Hy@tempa{\Hy@tempa,\CurrentOption,}%
      \@expandtwoargs\@removeelement\CurrentOption
        \@unusedoptionlist\@unusedoptionlist
    }%
  }%
%    \end{macrocode}
%
% Now stick the package options at the end of the list and wrap in a call
% to |\setkeys|. Can simply use |\edef|, normally KV takes care to avoid
% expansion, but the package system has already fully expanded the package
% option list before passing it to the package, so no more harm can occur
% here.
%    \begin{macrocode}
  \edef\Hy@tempa{%
    \noexpand\setkeys{#1}{%
      \Hy@tempa\@ptionlist{\@currname.\@currext}%
    }%
  }%
%    \end{macrocode}
%
% Do it.
%    \begin{macrocode}
  \Hy@tempa
}
%    \end{macrocode}
%
%    Add option |tex4ht| if package |tex4ht| is loaded.
%    \begin{macrocode}
\@ifpackageloaded{tex4ht}{%
  \@ifpackagewith{hyperref}{tex4ht}{}{%
    \PassOptionsToPackage{tex4ht}{hyperref}%
  }%
}{}
%    \end{macrocode}
%
%    \begin{macrocode}
\let\ReadBookmarks\relax
\ProcessOptionsWithKV{Hyp}
%    \end{macrocode}
%
%    After processing options.
%
%    \begin{macrocode}
\AtBeginDocument{%
  \ifHy@draft
    \let\hyper@@anchor\@gobble
    \gdef\hyper@link#1#2#3{#3}%
    \let\hyper@anchorstart\@gobble
    \let\hyper@anchorend\@empty
    \let\hyper@linkstart\@gobbletwo
    \let\hyper@linkend\@empty
    \def\hyper@linkurl#1#2{#1}%
    \def\hyper@linkfile#1#2#3{#1}%
    \def\Acrobatmenu#1#2{#2}%
    \let\PDF@SetupDoc\@empty
    \let\PDF@FinishDoc\@empty
    \let\@fifthoffive\@secondoftwo
    \let\@secondoffive\@secondoftwo
    \let\ReadBookmarks\relax
    \let\WriteBookmarks\relax
    \Hy@WarningNoLine{ draft mode on}%
  \fi
  \Hy@DisableOption{draft}%
}%
%    \end{macrocode}
%
%    If option |unicode| is already processed, then the
%    command \cs{HyPsd@InitUnicode} is cleared after
%    execution. Then the option code only sets
%    the meaning of the switch \cs{ifHy@unicode}.
%    \begin{macrocode}
\ifx\HyPsd@InitUnicode\relax
  \define@key{Hyp}{unicode}[true]{%
    \Hy@boolkey{unicode}{#1}%
  }
\else
  \define@key{Hyp}{unicode}[true]{%
    \Hy@boolkey{unicode}{#1}%
    \ifHy@unicode
      \Hy@WarningNoLine{Set package option `unicode' first\MessageBreak
                  in order to load unicode support%
      }%
      \Hy@unicodefalse
    \fi
  }
\fi
%    \end{macrocode}
%    The macro \cs{HyPsd@InitUnicode} is no longer needed.
%    \begin{macrocode}
\let\HyPsd@InitUnicode\@undefined
%    \end{macrocode}
%
% \subsubsection{Patch for babel's
%    \texorpdfstring{\cs{texttilde}}{\\texttilde}}
%    Babel does not define \cmd{\texttilde} in NFSS2 manner,
%    so the NFSS2 definitions of PD1 or PU encoding is not
%    compatible. To fix this, \cmd{\texttilde} is defined
%    in babel manner.
%    \begin{macrocode}
\Hy@nextfalse
\@ifpackagewith{babel}{spanish}{\Hy@nexttrue}{}
\@ifpackagewith{babel}{galician}{\Hy@nexttrue}{}
\@ifpackagewith{babel}{estonian}{\Hy@nexttrue}{}
\ifHy@next
  \let\texttilde\~%
\fi
%    \end{macrocode}
%
%    \begin{macrocode}
\ifHy@figures
  \Hy@Info{Hyper figures ON}
\else
  \Hy@Info{Hyper figures OFF}
\fi
\ifHy@nesting
  \Hy@Info{Link nesting ON}
\else
  \Hy@Info{Link nesting OFF}
\fi
\ifHy@hyperindex
  \Hy@Info{Hyper index ON}
\else
  \Hy@Info{Hyper index OFF}
\fi
\ifHy@plainpages
  \Hy@Info{Plain pages ON}
\else
  \Hy@Info{Plain pages OFF}
\fi
\ifHy@backref
  \Hy@Info{Backreferencing ON}
\else
  \Hy@Info{Backreferencing OFF}
\fi
\ifHy@typexml
   \AtEndOfPackage{\RequirePackage{color}\RequirePackage{nameref}}
\fi
\Hy@DisableOption{typexml}
\ifHy@implicit
  \typeout{Implicit mode ON; LaTeX internals redefined}%
\else
  \typeout{Implicit mode OFF; no redefinition of LaTeX internals}%
  \def\MaybeStopEarly{%
    \typeout{Hyperref stopped early}%
    \AfterBeginDocument{\PDF@FinishDoc}%
    \endinput
  }%
  \AtBeginDocument{%
    \let\autoref\ref
    \ifx\@pdfpagemode\@empty
      \gdef\@pdfpagemode{/UseNone}%
    \fi
    \global\Hy@backreffalse
  }%
  \AtEndOfPackage{%
    \global\let\ReadBookmarks\relax
    \global\let\WriteBookmarks\relax
  }%
\fi
\Hy@DisableOption{implicit}
%    \end{macrocode}
%
% \subsubsection{Driver loading}
%
%    \begin{macrocode}
\AtEndOfPackage{%
  \@ifpackageloaded{tex4ht}{%
    \def\Hy@driver{htex4ht}%
    \Hy@texhttrue
  }{}%
  \ifx\Hy@driver\@empty
    \providecommand*{\Hy@defaultdriver}{hypertex}%
    \begingroup\expandafter\expandafter\expandafter\endgroup
    \expandafter\ifx\csname pdfoutput\endcsname\relax
      \begingroup\expandafter\expandafter\expandafter\endgroup
      \expandafter\ifx\csname OpMode\endcsname\relax
        \let\Hy@driver\Hy@defaultdriver
      \else
        \ifnum 0\ifnum\OpMode<1 1\fi \ifnum\OpMode>3 1\fi =0 %
          \def\Hy@driver{hvtex}%
          \def\XR@ext{pdf}%
        \else
          \ifnum\OpMode=10\relax
            \def\XR@ext{htm}%
            \def\Hy@driver{hvtexhtm}%
            \def\MaybeStopEarly{%
              \typeout{Hyperref stopped early}%
              \AfterBeginDocument{\PDF@FinishDoc}%
              \endinput
            }%
          \else
            \let\Hy@driver\Hy@defaultdriver
          \fi
        \fi
      \fi
    \else
      \ifcase\pdfoutput
        \let\Hy@driver\Hy@defaultdriver
      \else
        \def\Hy@driver{hpdftex}%
        \def\XR@ext{pdf}%
        \PassOptionsToPackage{pdftex}{color}%
        \Hy@breaklinkstrue
      \fi
    \fi
    \typeout{*hyperref using default driver \Hy@driver*}%
  \else
    \typeout{*hyperref using driver \Hy@driver*}%
  \fi
  \input{\Hy@driver.def}%
  \let\@unprocessedoptions\relax
  \Hy@RestoreCatcodes
}
\Hy@DisableOption{tex4ht}
\Hy@DisableOption{pdftex}
\Hy@DisableOption{dvipdf}
\Hy@DisableOption{nativepdf}
\Hy@DisableOption{dvipdfm}
\Hy@DisableOption{pdfmark}
\Hy@DisableOption{dvips}
\Hy@DisableOption{hypertex}
\Hy@DisableOption{vtex}
\Hy@DisableOption{vtexpdfmark}
\Hy@DisableOption{dviwindo}
\Hy@DisableOption{dvipsone}
\Hy@DisableOption{textures}
\Hy@DisableOption{latex2html}
\Hy@DisableOption{ps2pdf}
%    \end{macrocode}
%
% \subsubsection{Bookmarks}
%    \begin{macrocode}
\def\WriteBookmarks{0}
\def\@bookmarkopenstatus#1{%
  \ifHy@bookmarksopen
%    \end{macrocode}
%    The purpose of the |\@firstofone|-number-space-construct
%    is that no |\relax| will be inserted by \TeX{} before the |\else|:
%    \begin{macrocode}
    \ifnum#1<\expandafter\@firstofone\expandafter
             {\number\@bookmarksopenlevel} % explicit space
    \else
      -%
    \fi
  \else
    -%
  \fi
}
\ifHy@bookmarks
  \Hy@Info{Bookmarks ON}%
  \ifx\@pdfpagemode\@empty
    \def\@pdfpagemode{/UseOutlines }%
  \fi
\else
  \def\@bookmarkopenstatus#1{}%
  \Hy@Info{Bookmarks OFF}%
  \AtEndOfPackage{%
    \global\let\ReadBookmarks\relax
    \global\let\WriteBookmarks\relax
  }
  \ifx\@pdfpagemode\@empty
    \def\@pdfpagemode{/UseNone}%
  \fi
\fi
\Hy@DisableOption{bookmarks}
%    \end{macrocode}
%
%    \begin{macrocode}
\AtBeginDocument{%
  \ifHy@colorlinks
    \ifHy@typexml\else\RequirePackage{color}\fi
    \def\Hy@colorlink#1{\begingroup\color{#1}}%
    \def\Hy@endcolorlink{\endgroup}%
    \Hy@Info{Link coloring ON}%
  \else
    \ifHy@frenchlinks
      \def\Hy@colorlink#1{\begingroup\fontshape{sc}\selectfont}%
      \def\Hy@endcolorlink{\endgroup}%
      \Hy@Info{French linking ON}%
    \else
%    \end{macrocode}
%    for grouping consistency:
%    \begin{macrocode}
      \def\Hy@colorlink#1{\begingroup}%
      \def\Hy@endcolorlink{\endgroup}%
      \Hy@Info{Link coloring OFF}%
    \fi
  \fi
  \Hy@DisableOption{colorlinks}%
  \Hy@DisableOption{frenchlinks}%
  \ifHy@texht
    \long\def\@firstoffive#1#2#3#4#5{#1}%
    \long\def\@secondoffive#1#2#3#4#5{#2}%
    \long\def\@thirdoffive#1#2#3#4#5{#3}%
    \long\def\@fourthoffive#1#2#3#4#5{#4}%
    \long\def\@fifthoffive#1#2#3#4#5{#5}%
    \providecommand*\@safe@activestrue{}%
    \providecommand*\@safe@activesfalse{}%
    \def\T@ref#1{%
      \@safe@activestrue
      \expandafter\@setref\csname r@#1\endcsname\@firstoffive{#1}%
      \@safe@activesfalse
    }%
    \def\T@pageref#1{%
      \@safe@activestrue
      \expandafter\@setref\csname r@#1\endcsname\@secondoffive{#1}%
      \@safe@activesfalse
    }%
  \else
    \ifHy@typexml\else\RequirePackage{nameref}\fi
  \fi
  \DeclareRobustCommand\ref{\@ifstar\@refstar\T@ref}%
  \DeclareRobustCommand\pageref{%
    \@ifstar\@pagerefstar\T@pageref
  }%
}
\AfterBeginDocument{%
  \ifHy@texht
  \else
    \ReadBookmarks
  \fi
}
%    \end{macrocode}
%    \begin{macrocode}
\ifHy@backref
  \RequirePackage{backref}
\else
  \let\Hy@backout\@gobble
\fi
\Hy@DisableOption{backref}
\Hy@DisableOption{pagebackref}
%    \end{macrocode}
%    \begin{macrocode}
\Hy@activeanchorfalse
%    \end{macrocode}
%
% \section{User hypertext macros}\label{usermacros}
% We need to normalise all user commands taking a URL argument;
% Within the argument the following special definitions apply:
% |\#|, |\%|, |~| produce |#|, |%|, |~| respectively.
% for consistency |\~| produces |~| as well.
% At the \emph{top level only} ie not within the argument of another
% command, you can use |#| and |%| unescaped, to produce themselves.
% even if, say, |#| is entered as |#| it will be converted to |\#|
% so it does not die if written to an aux file etc. |\#| will write
% as |#| locally while making |\special|s.
%    \begin{macrocode}
\begingroup
  \endlinechar=-1
  \catcode`\^^M\active
  \catcode`\%\active
  \catcode`\#\active
  \catcode`\_\active
  \gdef\hyper@normalise{
    \begingroup
    \catcode`\^^M\active
    \def^^M{ }
    \catcode`\%\active
    \let%\@percentchar
    \let\%\@percentchar
    \catcode`\#\active
    \def#{\hyper@hash}
    \def\#{\hyper@hash}
    \edef\textunderscore{\string_}
    \let\_\textunderscore
    \catcode`\_\active
    \let_\textunderscore
    \let~\hyper@tilde
    \let\~\hyper@tilde
    \let\textasciitilde\hyper@tilde
    \ifx\@safe@activestrue\@undefined\else\@safe@activestrue\fi
    \hyper@n@rmalise
  }
  \catcode`\#=6
  \gdef\hyper@n@rmalise#1#2{
    \edef\Hy@tempa{
      \endgroup
      \noexpand#1{\Hy@RemovePercentCr#2%^^M\@nil}
    }
    \Hy@tempa
  }
  \gdef\Hy@RemovePercentCr#1%^^M#2\@nil{
    #1
    \ifx\limits#2\limits
    \else
      \Hy@ReturnAfterFi{
        \Hy@RemovePercentCr #2\@nil
      }
    \fi
  }
\endgroup
\providecommand\hyper@chars{%
  \let\#\hyper@hash
  \let\%\@percentchar
}
%    \end{macrocode}
%
%    \begin{macrocode}
\def\hyperlink#1#2{%
  \hyper@@link{}{#1}{#2}%
}
%    \end{macrocode}
%
%    \begin{macrocode}
\DeclareRobustCommand*{\href}{\hyper@normalise\href@}
\begingroup
  \catcode`\$=6
  \catcode`\#=12
  \gdef\href@$1{\expandafter\href@split$1##\\}
  \gdef\href@split$1#$2#$3\\{%
    \hyper@@link{$1}{$2}%
  }
\endgroup
%    \end{macrocode}
%    Load package |url.sty| and save the meaning of
%    the original \cmd{\url} in \cmd{\nolinkurl}.
%    \begin{macrocode}
\RequirePackage{url}
\let\HyOrg@url\url
\def\Hurl{\begingroup \Url}
\let\nolinkurl\Hurl
\DeclareRobustCommand*{\url}{\hyper@normalise\url@}
\def\url@#1{\hyper@linkurl{\Hurl{#1}}{#1}}
%    \end{macrocode}
%
%    \begin{macrocode}
\DeclareRobustCommand*{\hyperimage}{\hyper@normalise\hyper@image}
\providecommand\hyper@image[2]{#2}
%    \end{macrocode}
%
%    \begin{macrocode}
\def\hypertarget#1#2{%
  \ifHy@nesting
    \hyper@@anchor{#1}{#2}%
  \else
    \hyper@@anchor{#1}{\relax}#2%
  \fi
}
%    \end{macrocode}
% |\hyperref| is more complicated, as it includes the concept of a
% category of link, used to make the name. This is not really used in this
% package.  |\hyperdef| sets up an anchor in the same way. They each have
% three  parameters of category, linkname, and marked text, and |\hyperrref|
% also has a first parameter of URL.
% If there is an optional first parameter to |\hyperdef|,
% it is the name of a \LaTeX\ label which can be used in
% a short form of |\hyperref| later, to avoid
% remembering the name and category.
%    \begin{macrocode}
\DeclareRobustCommand*{\hyperref}{%
  \@ifnextchar[\label@hyperref\@hyperref
}
\def\@hyperref{\hyper@normalise\@@hyperref}
\def\@@hyperref#1#2#3{%
  \edef\ref@one{\ifx\\#2\\\else#2.\fi#3}%
  \expandafter\tryhyper@link\ref@one\\{#1}%
}
\def\tryhyper@link#1\\#2{%
 \hyper@@link{#2}{#1}%
}
%    \end{macrocode}
%
%    \begin{macrocode}
\def\hyperdef{\@ifnextchar[{\label@hyperdef}{\@hyperdef}}
\def\@hyperdef#1#2#3{%, category, name, text
  \ifHy@nesting
    \hyper@@anchor{#1.#2}{#3}%
  \else
    \hyper@@anchor{#1.#2}{\relax}#3%
  \fi
}
%    \end{macrocode}
% We also have a need to give a \LaTeX\ \emph{label} to a
% hyper reference, to ease the pain of referring to it later.
%    \begin{macrocode}
\def\label@hyperref[#1]{%
  \expandafter\label@@hyperref\csname r@#1\endcsname{#1}%
}%
\def\label@@hyperref#1#2#3{%
  \ifx#1\relax
    \protect\G@refundefinedtrue
    \@latex@warning{%
      Hyper reference `#2' on page \thepage \space undefined%
    }%
    \hyper@@link{}{??}{#3}%
  \else
    \hyper@@link{\expandafter\@fifthoffive#1}%
      {\expandafter\@fourthoffive#1\@empty\@empty}{#3}%
  \fi
}
\def\label@hyperdef[#1]#2#3#4{% label name, category, name,
                                % anchor text
  \@bsphack
  \protected@write\@auxout{}%
    {\string\newlabel{#1}{{}{}{}\##2.#3}}%
  \@esphack
  \ifHy@nesting
    \hyper@@anchor{#2.#3}{#4}%
  \else
    \hyper@@anchor{#2.#3}{\relax}#4%
  \fi
}
%    \end{macrocode}
% \section{Underlying basic hypertext macros}\label{coremacros}
% Links have an optional type, a filename (possibly a URL),
% an internal name, and some marked text.
% If the second parameter is empty, its an internal link,
% otherwise we need to open another file or a URL.
% A link start has a type, and a URL.
%    \begin{macrocode}
\def\hyper@@link{\let\Hy@reserved@a\relax
  \@ifnextchar[{\hyper@link@}{\hyper@link@[link]}%
}
\def\hyper@link@[#1]#2#3#4{%
  \protected@edef\Hy@tempa{#2}%
  \ifx\Hy@tempa\@empty
    \hyper@link{#1}{#3}{#4}%
  \else
    \expandafter\hyper@readexternallink#2\\{#1}{#3}{#4}%
  \fi
}
%    \end{macrocode}
% The problem here is that the first (URL) parameter may be a
% local \texttt{file:} reference
% (in which case some browsers treat it differently)
% or a genuine URL, in which case we'll have to activate
% a real Web browser.
% Note that a simple name is also a URL, as that is interpreted
% as a relative file name. We have to worry about |#| signs in a local
% file as well.
%
%    \begin{macrocode}
\def\hyper@readexternallink#1\\#2#3#4{%
%    \end{macrocode}
% Parameters are:
% \begin{enumerate}
% \item The URL or file name
% \item The type
% \item The internal name
% \item The link string
% \end{enumerate}
% We need to get the 1st parameter properly expanded,
% so we delimit the arguments rather than passing it inside a group.
%    \begin{macrocode}
  \expandafter\@hyper@readexternallink{#2}{#3}{#4}#1::\\{#1}%
}
%    \end{macrocode}
% Now (potentially), we are passed:
% 1) The link type
% 2) The internal name,
% 3) the link string,
% 4) the URL type (http, mailto, file etc),
% 5) the URL details
% 6) anything after a real : in the URL
% 7) the whole URL again
%    \begin{macrocode}
\def\@pdftempwordfile{file}%
\def\@pdftempwordrun{run}%
\def\@hyper@readexternallink#1#2#3#4:#5:#6\\#7{%
%    \end{macrocode}
% If there are no colons at all (|#6| is blank), its a local
% file; if the URL type (|#4|) is blank, its probably a Mac filename,
% so treat it like a \texttt{file:} URL. The only flaw is if
% its a relative Mac path, with several colon-separated elements ---
% then we lose. Such names must be prefixed with an explicit |dvi:|
%    \begin{macrocode}
  \ifx\\#6\\%
    \expandafter\@hyper@linkfile file:#7\\{#3}{#2}%
  \else
    \ifx\\#4\\%
      \expandafter\@hyper@linkfile file:#7\\{#3}{#2}%
    \else
%    \end{macrocode}
% If the URL type is `file', pass it for local opening
%    \begin{macrocode}
      \def\@pdftempa{#4}%
      \ifx\@pdftempa\@pdftempwordfile
        \expandafter\@hyper@linkfile#7\\{#3}{#2}%
      \else
%    \end{macrocode}
% if it starts `run:', its to launch an application.
%    \begin{macrocode}
        \ifx\@pdftempa\@pdftempwordrun
          \expandafter\@hyper@launch#7\\{#3}{#2}%
        \else
%    \end{macrocode}
% otherwise its a URL
%    \begin{macrocode}
          \hyper@linkurl{#3}{#7\ifx\\#2\\\else\##2\fi}%
        \fi
      \fi
    \fi
  \fi
}
%    \end{macrocode}
%  By default, turn |run:| into |file:|
%    \begin{macrocode}
\def\@hyper@launch run:#1\\#2#3{% filename, anchor text, linkname
   \hyper@linkurl{#2}{file:#1\ifx\\#3\\\else\##3\fi}%
}
%    \end{macrocode}
% D P Story <story@uakron.edu> pointed out that relative paths
% starting .. fell over. Switched to using |\filename@parse| to
% solve this.
%    \begin{macrocode}
\def\@hyper@linkfile file:#1\\#2#3{%
     %file url,link string, name
  \filename@parse{#1}%
  \ifx\filename@ext\relax
    \edef\filename@ext{\XR@ext}%
  \fi
  \def\use@file{\filename@area\filename@base.\filename@ext}%
  \ifx\filename@ext\XR@ext
    \hyper@linkfile{#2}{\use@file}{#3}%
  \else
    \ifx\@baseurl\@empty
      \hyper@linkurl{#2}{%
        \Hy@linkfileprefix\use@file\ifx\\#3\\\else\##3\fi
      }%
    \else
      \hyper@linkurl{#2}{\use@file\ifx\\#3\\\else\##3\fi}%
    \fi
  \fi
}
%    \end{macrocode}
% Anchors have a name, and marked text.
% We have to be careful with the marked text, as if we break
% off part of something to put a |\special| around it, all hell breaks
% loose. Therefore, we check the category code of the first token,
% and only proceed if its safe. Tanmoy sorted this out.
%
% A curious case arises if the original parameter
% was in braces. That means that |#2| comes here a multiple
% letters, and the |noexpand| just looks at the first one,
% putting the rest in the output. Yuck.
%    \begin{macrocode}
\long\def\hyper@@anchor#1#2{\@hyper@@anchor#1\relax#2\relax}
\long\def\@hyper@@anchor#1\relax#2#3\relax{%
  \ifx\\#1\\%
    #2\Hy@WarningNoLine{empty link? #1: #2#3}%
  \else
    \def\anchor@spot{#2#3}%
    \let\put@me@back\@empty
    \ifx\relax#2\relax
    \else
      \ifHy@nesting
      \else
        \ifcat a\noexpand#2\relax
        \else
          \ifcat 0\noexpand#2 \relax
          \else
%            \typeout{Anchor start is not alphanumeric %
%              on input line\the\inputlineno%
%            }%
            \let\anchor@spot\@empty
            \def\put@me@back{#2#3}%
          \fi
        \fi
      \fi
    \fi
    \ifHy@activeanchor
      \anchor@spot
    \else
      \hyper@anchor{#1}%
    \fi
    \expandafter\put@me@back
  \fi
  \let\anchor@spot\@empty
}
%    \end{macrocode}
% \section{Compatibility with the \emph{\LaTeX{}2html} package}\label{latex2html}
% Map our macro names on to Nikos', so that documents prepared
% for that system will work without change.
%
% Note, however, that the whole complicated structure for
% segmenting documents is not supported; it is assumed that the user
% will load |html.sty| first, and then |hyperref.sty|, so that the
% definitions in |html.sty| take effect, and are then overridden
% in a few circumstances by this package.
%    \begin{macrocode}
\let\htmladdimg\hyperimage
%    \end{macrocode}
%
%    \begin{macrocode}
\def\htmladdnormallink#1#2{\href{#2}{#1}}
\def\htmladdnormallinkfoot#1#2{\href{#2}{#1}\footnote{#2}}
\def\htmlref#1#2{% anchor text, label
  \label@hyperref[{#2}]{#1}%
}
%    \end{macrocode}
% This is really too much. The \LaTeX2html package defines its own
% |\hyperref| command, with a different syntax. Was this always here?
% Its weird, anyway. We interpret it in the `printed' way, since
% we are about fidelity to the page.
%    \begin{macrocode}
\def\@@latextohtmlX{%
  \let\hhyperref\hyperref
  \def\hyperref##1##2##3##4{% anchor text for HTML
                     % text to print before label in print
                     % label
                     % post-label text in print
    ##2\ref{##4}##3%
  }%
}
%    \end{macrocode}
%
% \section{Forms creation}
% Allow for creation of PDF or HTML forms. The effects here are
% limited somewhat by the need to support both output formats,
% so it may not be as clever as something which only wants
% to make PDF forms.
%
% I could not have started this without the encouragement of T V Raman.
%    \begin{macrocode}
\newif\ifFld@checked
\newif\ifFld@hidden
\newif\ifFld@multiline
\newif\ifFld@readonly
\newif\ifFld@disabled
\newif\ifFld@password
\newif\ifFld@radio
\newif\ifFld@combo
\newif\ifFld@popdown
\Fld@multilinefalse
\Fld@checkedfalse
\Fld@hiddenfalse
\Fld@readonlyfalse
\Fld@disabledfalse
\Fld@radiofalse
\Fld@combofalse
\Fld@popdownfalse
\Fld@passwordfalse
\newcount\Fld@menulength
\newdimen\Field@Width
\newdimen\Fld@charsize
\Fld@charsize=10\p@
\def\Fld@maxlen{0}
\def\Fld@align{0}
\def\Fld@color{0 0 0}
\def\Fld@bcolor{1 1 1}
\def\Fld@bordercolor{1 0 0}
\def\Fld@bordersep{1\p@}
\def\Fld@borderwidth{1}
\def\Fld@borderstyle{S}
\def\Fld@cbsymbol{4}
\newtoks\Choice@toks
\def\Form{\@ifnextchar[{\@Form}{\@Form[]}}
\def\endForm{\@endForm}
\newif\ifForm@html
\Form@htmlfalse
\def\Form@boolkey#1#2{%
  \csname Form@#2\ifx\relax#1\relax true\else#1\fi\endcsname
}
\define@key{Form}{action}{%
  \def\Form@action{#1}%
}
\def\enc@@html{html}
\define@key{Form}{encoding}{%
  \def\Hy@tempa{#1}%
  \ifx\Hy@tempa\enc@@html
    \Form@htmltrue
  \else
    \typeout{hyperref: form `encoding' key set to #1 %
      -- unknown type%
    }%
    \Form@htmlfalse
  \fi
}
\define@key{Form}{method}{%
  \def\Form@method{#1}%
}
\def\Form@method{}
\def\Field@boolkey#1#2{%
  \csname Fld@#2\ifx\relax#1\relax true\else#1\fi\endcsname
}
\newtoks\Field@toks
\Field@toks={ }%
\def\Field@addtoks#1#2{%
  \edef\@processme{\Field@toks{\the\Field@toks\space #1="#2"}}%
  \@processme
}
\def\Fld@checkequals#1=#2=#3\\{%
  \def\@currDisplay{#1}%
  \ifx\\#2\\%
    \def\@currValue{#1}%
  \else
    \def\@currValue{#2}%
  \fi
}
\define@key{Field}{loc}{%
  \def\Fld@loc{#1}%
}
\define@key{Field}{multiline}[true]{%
  \lowercase{\Field@boolkey{#1}}{multiline}%
}
\define@key{Field}{checked}[true]{%
  \lowercase{\Field@boolkey{#1}}{checked}%
}
\define@key{Field}{hidden}[true]{%
  \lowercase{\Field@boolkey{#1}}{hidden}%
}
\define@key{Field}{readonly}[true]{%
  \lowercase{\Field@boolkey{#1}}{readonly}%
}
\define@key{Field}{disabled}[true]{%
  \lowercase{\Field@boolkey{#1}}{disabled}%
}
\define@key{Field}{password}[true]{%
  \lowercase{\Field@boolkey{#1}}{password}%
}
\define@key{Field}{radio}[true]{%
  \lowercase{\Field@boolkey{#1}}{radio}%
}
\define@key{Field}{combo}[true]{%
  \lowercase{\Field@boolkey{#1}}{combo}%
}
\define@key{Field}{popdown}[true]{%
  \lowercase{\Field@boolkey{#1}}{popdown}%
}
\define@key{Field}{accesskey}{%
  \Field@addtoks{accesskey}{#1}%
}
\define@key{Field}{tabkey}{%
  \Field@addtoks{tabkey}{#1}%
}
\define@key{Field}{name}{%
  \def\Fld@name{#1}%
}
\define@key{Field}{width}{%
  \def\Fld@width{#1}%
  \Field@Width#1\setbox0=\hbox{m}%
}
\define@key{Field}{maxlen}{%
  \def\Fld@maxlen{#1}%
}
\define@key{Field}{menulength}{%
  \Fld@menulength=#1\relax
}
\define@key{Field}{height}{%
  \def\Fld@height{#1}%
}
\define@key{Field}{charsize}{%
  \Fld@charsize#1%
}
\define@key{Field}{fillcolor}{%
  \def\Fld@fillcolor{#1}%
}
\define@key{Field}{bordercolor}{%
  \def\Fld@bordercolor{#1}%
}
\define@key{Field}{color}{%
  \def\Fld@color{#1}%
}
\define@key{Field}{borderwidth}{%
  \def\Fld@borderwidth{#1}%
}
\define@key{Field}{borderstyle}{%
  \def\Fld@borderstyle{#1}%
}
\define@key{Field}{bordersep}{%
  \def\Fld@bordersep{#1}%
}
\define@key{Field}{default}{%
  \def\Fld@default{#1}%
}
\define@key{Field}{align}{%
  \def\Fld@align{#1}%
}
\define@key{Field}{value}{%
  \def\Fld@value{#1}%
}
\define@key{Field}{backgroundcolor}{%
  \def\Fld@bcolor{#1}%
}
\define@key{Field}{checkboxsymbol}{%
  \def\Fld@cbsymbol{#1}%
}
\def\Fld@format@code{}
\def\Fld@validate@code{}
\def\Fld@calculate@code{}
\def\Fld@keystroke@code{}
\def\Fld@onfocus@code{}
\def\Fld@onblur@code{}
\def\Fld@onmousedown@code{}
\def\Fld@onmouseup@code{}
\def\Fld@onenter@code{}
\def\Fld@onexit@code{}
\define@key{Field}{keystroke}{%
  \def\Fld@keystroke@code{#1}%
}
\define@key{Field}{format}{%
  \def\Fld@format@code{#1}%
}
\define@key{Field}{validate}{%
  \def\Fld@validate@code{#1}%
}
\define@key{Field}{calculate}{%
  \def\Fld@calculate@code{#1}%
}
\define@key{Field}{onfocus}{%
  \def\Fld@onfocus@code{#1}%
}
\define@key{Field}{onblur}{%
  \def\Fld@onblur@code{#1}%
}
\define@key{Field}{onenter}{%
  \def\Fld@onenter@code{#1}%
}
\define@key{Field}{onexit}{%
  \def\Fld@onexit@code{#1}%
}
\define@key{Field}{onselect}{%
  \Field@addtoks{onselect}{#1}%
}
\define@key{Field}{onchange}{%
  \Field@addtoks{onchange}{#1}%
}
\define@key{Field}{onclick}{%
  \def\Fld@onclick{#1}%
  \Field@addtoks{onclick}{#1}%
}
\define@key{Field}{ondblclick}{%
  \Field@addtoks{ondblclick}{#1}%
}
\define@key{Field}{onmousedown}{%
  \Field@addtoks{onmousedown}{#1}%
}
\define@key{Field}{onmouseup}{%
  \Field@addtoks{onmouseup}{#1}%
}
\define@key{Field}{onmouseover}{%
  \Field@addtoks{onmouseover}{#1}%
}
\define@key{Field}{onmousemove}{%
  \Field@addtoks{onmousemove}{#1}%
}
\define@key{Field}{onmouseout}{%
  \Field@addtoks{onmouseout}{#1}%
}
\define@key{Field}{onkeypress}{%
  \Field@addtoks{onkeypress}{#1}%
}
\define@key{Field}{onkeydown}{%
  \Field@addtoks{onkeydown}{#1}%
}
\define@key{Field}{onkeyup}{%
  \Field@addtoks{onkeyup}{#1}%
}
%
\DeclareRobustCommand\TextField{%
  \@ifnextchar[{\@TextField}{\@TextField[]}%
}
\DeclareRobustCommand\ChoiceMenu{%
  \@ifnextchar[{\@ChoiceMenu}{\@ChoiceMenu[]}%
}
\DeclareRobustCommand\CheckBox{%
  \@ifnextchar[{\@CheckBox}{\@CheckBox[]}%
}
\DeclareRobustCommand\PushButton{%
  \@ifnextchar[{\@PushButton}{\@PushButton[]}%
}
\DeclareRobustCommand\Gauge{%
  \@ifnextchar[{\@Gauge}{\@Gauge[]}%
}
\DeclareRobustCommand\Submit{%
  \@ifnextchar[{\@Submit}{\@Submit[]}%
}
\DeclareRobustCommand\Reset{%
  \@ifnextchar[{\@Reset}{\@Reset[]}%
}
\def\LayoutTextField#1#2{% label, field
  #1 #2%
}
\def\LayoutChoiceField#1#2{% label, field
  #1 #2%
}
\def\LayoutCheckField#1#2{% label, field
  #1 #2%
}
\def\LayoutPushButtonField#1{% button
  #1%
}
\def\MakeRadioField#1#2{\vbox to #2{\hbox to #1{\hfill}\vfill}}
\def\MakeCheckField#1#2{\vbox to #2{\hbox to #1{\hfill}\vfill}}
\def\MakeTextField#1#2{\vbox to #2{\hbox to #1{\hfill}\vfill}}
\def\MakeChoiceField#1#2{\vbox to #2{\hbox to #1{\hfill}\vfill}}
\def\MakeButtonField#1{%
  \sbox0{%
    \hskip\Fld@borderwidth bp#1\hskip\Fld@borderwidth bp%
  }%
  \@tempdima\ht0
  \advance\@tempdima by \Fld@borderwidth bp
  \advance\@tempdima by \Fld@borderwidth bp
  \ht0\@tempdima
  \@tempdima\dp0
  \advance\@tempdima by \Fld@borderwidth bp
  \advance\@tempdima by \Fld@borderwidth bp
  \dp0\@tempdima
  \box0\relax
}
\def\DefaultHeightofSubmit{14pt}
\def\DefaultWidthofSubmit{2cm}
\def\DefaultHeightofReset{14pt}
\def\DefaultWidthofReset{2cm}
\def\DefaultHeightofCheckBox{\baselineskip}
\def\DefaultWidthofCheckBox{\baselineskip}
\def\DefaultHeightofChoiceMenu{\baselineskip}
\def\DefaultWidthofChoiceMenu{\baselineskip}
\def\DefaultHeightofText{\baselineskip}
\def\DefaultWidthofText{3cm}
%    \end{macrocode}
%
% \section{Setup}
%    \begin{macrocode}
\ifHy@figures
  \Hy@Info{Hyper figures ON}
\else
  \Hy@Info{Hyper figures OFF}
\fi
\ifHy@nesting
  \Hy@Info{Link nesting ON}
\else
  \Hy@Info{Link nesting OFF}
\fi
\ifHy@hyperindex
  \Hy@Info{Hyper index ON}
\else
  \Hy@Info{Hyper index OFF}
\fi
\ifHy@backref
  \Hy@Info{backreferencing ON}
\else
  \Hy@Info{backreferencing OFF}
\fi
\ifHy@colorlinks
  \Hy@Info{Link coloring ON}
\else
  \Hy@Info{Link coloring OFF}
\fi
%    \end{macrocode}
% \section{Low-level utility macros}
% We need unrestricted access to the |#|, |~| and |"| characters, so make
% them nice macros.
%    \begin{macrocode}
\edef\hyper@hash{\string#}
\edef\hyper@tilde{\string~}
\edef\hyper@quote{\string"}
\let\@currentHref\@empty
%    \end{macrocode}
% We give the start of document a special label; this is used
% in backreferencing-by-section, to allow for cites before
% any sectioning commands. Set up PDF info.
%    \begin{macrocode}
\AfterBeginDocument{%
  \Hy@pdfstringtrue
  \PDF@SetupDoc
  \let\PDF@SetupDoc\@empty
  \Hy@DisableOption{4}%
  \Hy@DisableOption{pdfpagescrop}%
  \Hy@DisableOption{pdfpagemode}%
  \Hy@DisableOption{pdfstartview}%
  \Hy@DisableOption{pdfstartpage}%
  \Hy@DisableOption{pdftoolbar}%
  \Hy@DisableOption{pdfmenubar}%
  \Hy@DisableOption{pdfwindowui}%
  \Hy@DisableOption{pdffitwindow}%
  \Hy@DisableOption{pdfcenterwindow}%
  \Hy@DisableOption{pdfpagelayout}%
  \Hy@DisableOption{baseurl}%
  \ifHy@texht\else\hyper@anchorstart{Doc-Start}\hyper@anchorend\fi
  \Hy@pdfstringfalse
}
%    \end{macrocode}
%
% \section{Localized nullifying of package}
% Sometimes we just don't want the wretched package interfering
% with us. Define an environment we can put in manually, or include
% in a style file, which stops the hypertext functions doing anything.
% This is used, for instance, in the Elsevier classes, to stop
% |hyperref| playing havoc in the front matter.
%    \begin{macrocode}
\def\NoHyper{%
  \def\hyper@link@[##1]##2##3##4{##4}%
  \def\hyper@@anchor##1{}%
  \global\let\hyper@livelink\hyper@link
  \gdef\hyper@link##1##2##3{##3}%
  \def\hyper@anchorstart##1{}%
  \let\hyper@anchorend\@empty
  \def\hyper@linkstart##1##2{}%
  \let\hyper@linkend\@empty
  \def\hyper@linkurl##1##2{##1}%
  \def\hyper@linkfile##1##2##3{##1}%
  \let\Hy@backout\@gobble
}
\def\stop@hyper{%
  \def\hyper@link@[##1]##2##3##4{##4}%
  \let\Hy@backout\@gobble
  \def\hyper@@anchor##1{}%
  \def\hyper@link##1##2##3{##3}%
  \def\hyper@anchorstart##1{}%
  \let\hyper@anchorend\@empty
  \def\hyper@linkstart##1##2{}%
  \let\hyper@linkend\@empty
  \def\hyper@linkurl##1##2{##1}%
  \def\hyper@linkfile##1##2##3{##1}%
}
\def\endNoHyper{%
  \global\let\hyper@link\hyper@livelink
}
%</package>
%    \end{macrocode}
%
% \section{Package nohyperref}
%
%    This package is introduced by Sebastian Rahtz.
%
%    Package nohyperref is a dummy package that defines
%    some low level and some top-level commands.
%    It is done for jadetex, which calls hyperref
%    low-level commands, but it would also be useful with people using
%    normal hyperref, who really do not want the package loaded at all.
%
%    Some low-level commands:
%    \begin{macrocode}
%<*nohyperref>
\let\hyper@@anchor\@gobble
\def\hyper@link#1#2#3{#3}%
\let\hyper@anchorstart\@gobble
\let\hyper@anchorend\@empty
\let\hyper@linkstart\@gobbletwo
\let\hyper@linkend\@empty
\def\hyper@linkurl#1#2{#1}%
\def\hyper@linkfile#1#2#3{#1}%
\let\PDF@SetupDoc\@empty
\let\PDF@FinishDoc\@empty
%    \end{macrocode}
%    Some top-level commands:
%    \begin{macrocode}
\let\Acrobatmenu\@gobble
\let\pdfstringdefDisableCommands\@gobbletwo
\let\texorpdfstring\@firstoftwo
\let\pdfbookmark\@undefined
\newcommand\pdfbookmark[3][]{}
\let\phantomsection\@empty
\let\hypersetup\@gobble
\let\hyperbaseurl\@gobble
\let\href\@gobble
\let\hyperdef\@gobbletwo
\let\hyperlink\@gobble
\let\hypertarget\@gobble
\def\hyperref{\@ifnextchar[\@gobbleopt{\expandafter\@gobbletwo\@gobble}}
\long\def\@gobbleopt[#1]{}
\let\hyperpage\@empty
%</nohyperref>
%    \end{macrocode}
%
% \section{The Mangling Of Aux and Toc Files}
% Some extra tests so that the hyperref package may be removed or added
% to a document without having to remove .aux and .toc files
% (this section is by David Carlisle)
% All the code is delayed to |\begin{document}|
%    \begin{macrocode}
%<*package>
\AfterBeginDocument{%
%    \end{macrocode}
% First the code to deal with removing the hyperref package from
% a document.
%
% Write some stuff into the aux file so if the next run is done
% without hyperref, then |\contentsline| and |\newlabel| are defined
% to cope with the extra arguments.
%    \begin{macrocode}
  \if@filesw
   \ifHy@typexml
     \immediate\closeout\@mainaux
     \immediate\openout\@mainaux\jobname.aux
     \immediate\write\@auxout{<relaxxml>\relax}%
   \fi
   \immediate\write\@auxout{%
%    \end{macrocode}
%
%    \begin{macrocode}
      \string\ifx\string\hyper@anchor\string\@undefined^^J%
%    \end{macrocode}
%
%    \begin{macrocode}
        \global\let\string\oldcontentsline\string\contentsline^^J%
        \gdef\string\contentsline%
          \string#1\string#2\string#3\string#4{%
          \string\oldcontentsline%
            {\string#1}{\string#2}{\string#3}}^^J%
        \global\let\string\oldnewlabel\string\newlabel^^J%
        \gdef\string\newlabel\string#1\string#2{%
           \string\newlabelxx{\string#1}\string#2}^^J%
        \gdef\string\newlabelxx%
           \string#1\string#2\string#3\string#4\string#5\string#6{%
           \string\oldnewlabel{\string#1}{{\string#2}{\string#3}}}^^J%
%    \end{macrocode}
%
% But the new aux file will be read again at the end, with the normal
% definitions expected, so better put things back as they were.
%    \begin{macrocode}
        \string\AtEndDocument{%
          \let\string\contentsline\string\oldcontentsline^^J%
          \let\string\newlabel\string\oldnewlabel}^^J%
%    \end{macrocode}
%
% If the document is being run with hyperref put this definition
% into the aux file, so we can spot it on the next run.
%    \begin{macrocode}
      \string\else^^J%
        \global\let\string\hyper@last\relax^^J%
      \string\fi^^J%
    }%
  \fi
%    \end{macrocode}
%
% Now the code to deal with adding the hyperref package to a document
% with aux and toc written the standard way.
%
% If hyperref was used last time, do nothing. If it was not used,
% or an old version of hyperref was used, don't use that TOC at all
% but generate a warning. Not ideal, but better than failing
% with pre-5.0 hyperref TOCs.
%    \begin{macrocode}
  \ifx\hyper@last\@undefined
    \def\@starttoc#1{%
      \begingroup
        \makeatletter
        \IfFileExists{\jobname.#1}{%
          \Hy@WarningNoLine{%
            old #1 file detected, not used; run LaTeX again%
          }%
        }{}%
        \if@filesw
          \expandafter\newwrite\csname tf@#1\endcsname
          \immediate\openout\csname tf@#1\endcsname \jobname.#1\relax
        \fi
        \@nobreakfalse
      \endgroup
    }%
    \def\newlabel#1#2{\@newl@bel r{#1}{#2{}{}{}{}}}%
  \fi
}
%    \end{macrocode}
%
% \section{Title strings}
%
%    If options |pdftitle| and |pdfauthor| are not used,
%    these informations for the pdf information dictionary
%    can be extracted by the \title and \author.
%    \begin{macrocode}
\ifHy@usetitle
  \let\HyOrg@title\title
  \let\HyOrg@author\author
  \def\title{\@ifnextchar[{\Hy@scanopttitle}{\Hy@scantitle}}%
  \def\Hy@scanopttitle[#1]{%
    \gdef\Hy@title{#1}%
    \HyOrg@title[{#1}]%
  }%
  \def\Hy@scantitle#1{%
    \gdef\Hy@title{#1}%
    \HyOrg@title{#1}%
  }
  \def\author{\@ifnextchar[{\Hy@scanoptauthor}{\Hy@scanauthor}}%
  \def\Hy@scanoptauthor[#1]{%
    \gdef\Hy@author{#1}%
    \HyOrg@author[{#1}]%
  }%
  \def\Hy@scanauthor#1{%
    \gdef\Hy@author{#1}%
    \HyOrg@author{#1}%
  }
%    \end{macrocode}
%
%    The case, that \title, or \author are given before
%    hyperref is loaded, is much more complicate, because
%    LaTeX initializes the macros \@title and \@author with
%    LaTeX error and warning messages.
%    \begin{macrocode}
  \begingroup
    \def\process@me#1\@nil#2{%
      \expandafter\let\expandafter\x\csname @#2\endcsname
      \edef\y{\expandafter\strip@prefix\meaning\x}%
      \def\c##1#1##2\@nil{%
        \ifx\\##1\\%
        \else
         \expandafter\gdef\csname Hy@#2\expandafter\endcsname
              \expandafter{\x}%
        \fi
      }%
      \expandafter\c\y\relax#1\@nil
    }%
    \expandafter\process@me\string\@latex@\@nil{title}%
    \expandafter\process@me\string\@latex@\@nil{author}%
  \endgroup
\fi
\Hy@DisableOption{pdfusetitle}
%    \end{macrocode}
%
%    Macro |\Hy@UseMaketitleInfos| is used in the driver files,
%    before the information entries are used.
%
%    The newline macro |\newline| or |\\| is much more
%    complicate. In the title a good replacement can be
%    a space, but can be already a space after |\\| in
%    the title string. So this space is removed by
%    scanning for the next non-empty argument.
%
%    In the macro |\author| the newline can perhaps
%    separate the different authors, so the newline
%    expands here to a comma with space.
%
%    The possible arguments such as space or the optional
%    argument after the newline macros are not detected.
%
%    \begin{macrocode}
\def\Hy@UseMaketitleString#1{%
  \@ifundefined{Hy@#1}{}{%
    \begingroup
      \let\Hy@saved@hook\pdfstringdefPreHook
      \pdfstringdefDisableCommands{%
        \expandafter\let\expandafter\\\csname Hy@newline@#1\endcsname
        \let\newline\\%
      }%
      \expandafter\ifx\csname @pdf#1\endcsname\@empty
        \expandafter\pdfstringdef\csname @pdf#1\endcsname{%
          \csname Hy@#1\endcsname\@empty
        }%
      \fi
      \global\let\pdfstringdefPreHook\Hy@saved@hook
    \endgroup
  }%
}
\def\Hy@newline@title#1{ #1}
\def\Hy@newline@author#1{, #1}
\def\Hy@UseMaketitleInfos{%
  \Hy@UseMaketitleString{title}%
  \Hy@UseMaketitleString{author}%
}
%    \end{macrocode}
%
% \section{Page numbers}
%    This stuff is done by Heiko Oberdiek.
%
% \subsection{PDF /PageLabels}
%    Internal macros of this module are marked with |\HyPL@|.
%
%    \begin{macrocode}
\ifHy@pagelabels
%    \end{macrocode}
%
%    \begin{macro}{\thispdfpagelabel}
%    The command \cmd{\thispdfpagelabel} allows to label a special
%    page without the redefinition of \cmd{\thepage} for the page.
%    \begin{macrocode}
  \def\thispdfpagelabel#1{%
    \gdef\HyPL@thisLabel{#1}%
  }
  \global\let\HyPL@thisLabel\relax
%    \end{macrocode}
%    \end{macro}
%
%    \begin{macro}{\HyPL@Labels}
%    The page labels are collected in \cmd{\HyPL@Labels} and
%    set at the end of the document.
%    \begin{macrocode}
  \def\HyPL@Labels{}
%    \end{macrocode}
%    \end{macro}
%    \begin{macro}{\Hy@abspage}
%    We have to know the the absolute page number and introduce
%    a new counter for that.
%    \begin{macrocode}
  \newcount\Hy@abspage
  \Hy@abspage=0
%    \end{macrocode}
%    \end{macro}
%    For comparisons with the values of the previous page, some
%    variables are needed:
%    \begin{macrocode}
  \def\HyPL@LastType{init}%
  \def\HyPL@LastNumber{0}%
  \def\HyPL@LastPrefix{}%
%    \end{macrocode}
%    Definitions for the PDF names of the \LaTeX{} pendents.
%    \begin{macrocode}
  \def\HyPL@arabic{D}%
  \def\HyPL@Roman{R}%
  \def\HyPL@roman{r}%
  \def\HyPL@Alph{A}%
  \def\HyPL@alph{a}%
%    \end{macrocode}
%
%    \begin{macro}{\HyPL@EveryPage}
%    If a page is shipout and the page number is known,
%    \cmd{\HyPL@EveryPage} has to be called. It stores the
%    current page label.
%    \begin{macrocode}
  \def\HyPL@EveryPage{%
    \begingroup
      \ifx\HyPL@thisLabel\relax
        \let\HyPL@page\thepage
      \else
        \let\HyPL@page\HyPL@thisLabel
        \global\let\HyPL@thisLabel\relax
      \fi
      \let\HyPL@Type\relax
      \ifnum\the\c@page>0
        \expandafter\HyPL@CheckThePage\HyPL@page\@nil
      \fi
      \let\Hy@temp Y%
      \ifx\HyPL@Type\HyPL@LastType
      \else
        \let\Hy@temp N%
      \fi
      \ifx\HyPL@Type\relax
         \pdfstringdef\HyPL@Prefix{\HyPL@page}%
      \else
         \pdfstringdef\HyPL@Prefix\HyPL@Prefix
      \fi
      \ifx\HyPL@Prefix\HyPL@LastPrefix
      \else
        \let\Hy@temp N%
      \fi
      \if Y\Hy@temp
        \advance\c@page by -1
        \ifnum\HyPL@LastNumber=\the\c@page\relax
        \else
          \let\Hy@temp N%
        \fi
        \Hy@StepCount\c@page
      \fi
      \if N\Hy@temp
        \ifx\HyPL@Type\relax
          \HyPL@StorePageLabel{/P (\HyPL@Prefix)}%
        \else
          \HyPL@StorePageLabel{%
            \ifx\HyPL@Prefix\@empty
            \else
              /P (\HyPL@Prefix)
            \fi
            /S /\csname HyPL\HyPL@Type\endcsname
            \ifnum\the\c@page=1
            \else
              \space/St \the\c@page
            \fi
          }%
        \fi
      \fi
      \xdef\HyPL@LastNumber{\the\c@page}%
      \global\let\HyPL@LastType\HyPL@Type
      \global\let\HyPL@LastPrefix\HyPL@Prefix
    \endgroup
    \Hy@GlobalStepCount\Hy@abspage
  }
%    \end{macrocode}
%    \end{macro}
%
%    \begin{macro}{\HyPL@CheckThePage}
%    Macro \cmd{\HyPL@CheckThePage} calls \cmd{\HyPL@@CheckThePage}
%    that does the job.
%    \begin{macrocode}
  \def\HyPL@CheckThePage#1\@nil{%
    \HyPL@@CheckThePage{#1}#1\csname\endcsname\c@page\@nil
  }
%    \end{macrocode}
%    \end{macro}
%    \begin{macro}{\HyPL@@CheckThePage}
%    The first check is, is \cmd{\thepage} is defined
%    such as in \LaTeX, e.\,g.: |\csname @arabic\endcsname\c@page|.
%    In the current implemenation the check fails, if there is
%    another \cmd{\csname} before.
%
%    The second check tries to detect |\arabic{page}| at the
%    end of the definition text of \cmd{\thepage}.
%    \begin{macrocode}
  \def\HyPL@@CheckThePage#1#2\csname#3\endcsname\c@page#4\@nil{%
    \def\Hy@tempa{#4}%
    \def\Hy@tempb{\csname\endcsname\c@page}%
    \ifx\Hy@tempa\Hy@tempb
      \expandafter\ifx\csname HyPL#3\endcsname\relax
      \else
        \def\HyPL@Type{#3}%
        \def\HyPL@Prefix{#2}%
      \fi
    \else
      \begingroup
        \let\Hy@next\endgroup
        \let\HyPL@found\@undefined
        \def\arabic{\HyPL@Format{arabic}}%
        \def\Roman{\HyPL@Format{Roman}}%
        \def\roman{\HyPL@Format{roman}}%
        \def\Alph{\HyPL@Format{Alph}}%
        \def\alph{\HyPL@Format{alph}}%
        \protected@edef\Hy@temp{#1}%
        \ifx\HyPL@found\relax
          \toks@\expandafter{\Hy@temp}%
          \edef\Hy@next{\endgroup
            \noexpand\HyPL@@@CheckThePage\the\toks@
               \noexpand\HyPL@found\relax\noexpand\@nil
          }%
        \fi
      \Hy@next
    \fi
  }
%    \end{macrocode}
%    \end{macro}
%
%    \begin{macro}{\HyPL@Format}
%    The help macro \cmd{\HyPL@Format} is executed while
%    a \cmd{\protected@edef} in the second check
%    method of \cmd{\HyPL@@CheckPage}.
%    The first occurences of, for example, |\arabic{page}| is
%    marked by \cmd{\HyPL@found} that is also defined by
%    \cmd{\csname}.
%    \begin{macrocode}
  \def\HyPL@Format#1#2{%
    \ifx\HyPL@found\@undefined
      \expandafter\ifx\csname c@#2\endcsname\c@page
        \expandafter\noexpand\csname HyPL@found\endcsname{#1}%
      \else
        \expandafter\noexpand\csname#1\endcsname{#2}%
      \fi
    \else
      \expandafter\noexpand\csname#1\endcsname{#2}%
    \fi
  }
%    \end{macrocode}
%    \end{macro}
%
%    \begin{macro}{\HyPL@@@CheckThePage}
%    If the second check method is successful,
%    \cmd{\HyPL@@@CheckThePage} scans the result of
%    \cmd{\HyPL@Format} and stores the found values.
%    \begin{macrocode}
  \def\HyPL@@@CheckThePage#1\HyPL@found#2#3\@nil{%
    \def\Hy@tempa{#3}%
    \def\Hy@tempb{\HyPL@found\relax}%
    \ifx\Hy@tempa\Hy@tempb
      \def\HyPL@Type{@#2}%
      \def\HyPL@Prefix{#1}%
    \fi
  }
%    \end{macrocode}
%    \end{macro}
%
%    \begin{macro}{\HyPL@StorePageLabel}
%    Dummy for drivers that does not support /PageLabel.
%    \begin{macrocode}
  \providecommand*{\HyPL@StorePageLabel}[1]{}
%    \end{macrocode}
%    \end{macro}
%
%   \begin{macro}{\HyPL@Useless}
%   The |/PageLabels| entry does not make sense,
%   if the absolute page numbers and the page labels are the
%   same. Then \cmd{\HyPL@Labels} has the meaning of \cmd{\HyPL@Useless}.
%   \begin{macrocode}
  \def\HyPL@Useless{0 << /S /D >> }%
%    \end{macrocode}
%    \end{macro}
%
%    \begin{macro}{\HyPL@SetPageLabels}
%    The page labels are written to the PDF cataloge.
%    The command \cmd{\Hy@PutCatalog} is defined in the
%    driver files.
%    \begin{macrocode}
  \def\HyPL@SetPageLabels{%
    \ifx\HyPL@Labels\@empty
    \else
      \ifx\HyPL@Labels\HyPL@Useless
      \else
        \Hy@PutCatalog{/PageLabels << /Nums [\HyPL@Labels] >>}%
      \fi
    \fi
  }
%   \end{macrocode}
%   \end{macro}
%
%    \begin{macro}{\HyPL@EveryPage}
%    Without option `pdfpagelabels' we need a dummy for
%    \cmd{\HyPL@EveryPage}.
%    \begin{macrocode}
\else
  \let\HyPL@EveryPage\@empty
\fi
%    \end{macrocode}
%    \end{macro}
%
%    Option `pdfpagelabels' has been used and is now disabled.
%    \begin{macrocode}
\Hy@DisableOption{pdfpagelabels}
%    \end{macrocode}
%
%    \begin{macrocode}
%</package>
%    \end{macrocode}
%
% \subsubsection{pdfTeX and VTeX}
%
%    Because of pdfTeX's \cmd{\pdfcatalog} command
%    the /PageLabels entry can set at end of document
%    in the first run.
%
%    \begin{macro}{\Hy@PutCatalog}
%    \begin{macrocode}
%<pdftex>\let\Hy@PutCatalog\pdfcatalog
%    \end{macrocode}
%    The code for VTeX is more complicate, because it does
%    not allow the direct access to the /Catalog object.
%    The command scans its argument and looks
%    for a /PageLabels entry.
%
%    VTeX 6.59g is the first version, that
%    implements \verb|\special{!pdfpagelabels...}|.
%    For this version \cmd{\VTeXversion} reports 660.
%    \begin{macrocode}
%<*vtex>
\edef\Hy@VTeXversion{%
  \ifx\VTeXversion\@undefined
    \z@
  \else
    \ifx\VTeXversion\relax
      \z@
    \else
      \VTeXversion
    \fi
  \fi
}
\begingroup
  \ifnum\Hy@VTeXversion<660 %
    \gdef\Hy@PutCatalog#1{%
      \Hy@WarningNoLine{%
        VTeX 6.59g or above required for pdfpagelabels%
      }%
    }
  \else
    \gdef\Hy@PutCatalog#1{%
      \Hy@vt@PutCatalog#1/PageLabels <<>>\@nil
    }
    \gdef\Hy@vt@PutCatalog#1/PageLabels <<#2>>#3\@nil{%
      \ifx\\#2\\%
      \else
        \immediate\special{!pdfpagelabels #2}%
      \fi
    }
  \fi
\endgroup
%</vtex>
%    \end{macrocode}
%    \end{macro}
%
%    \begin{macrocode}
%<*pdftex|vtex>
%    \end{macrocode}
%
%    \begin{macro}{\HyPL@StorePageLabel}
%    This macro adds the entry |#1| to \cmd{\HyPL@Labels}.
%    \begin{macrocode}
\ifHy@pagelabels
  \def\HyPL@StorePageLabel#1{%
    \toks@\expandafter{\HyPL@Labels}%
    \xdef\HyPL@Labels{%
      \the\toks@
      \the\Hy@abspage\space<< #1 >> %
    }%
  }
%    \end{macrocode}
%    \end{macro}
%
%    At the end of the document, \cmd{\clearpage} tries
%    to make sure, that no further pages will follow.
%    Then the PDF catalog entry for |\PageLabels| is set.
%    \begin{macrocode}
  \AtEndDocument{\clearpage\HyPL@SetPageLabels}
%    \end{macrocode}
%
%    \begin{macrocode}
\fi
%</pdftex|vtex>
%    \end{macrocode}
%
% \subsubsection{pdfmarkbase, dvipdfm}
%
%    \begin{macro}{\Hy@PutCatalog}
%    \begin{macrocode}
%<dvipdfm>\def\Hy@PutCatalog#1{\@pdfm@mark{docview << #1 >>}}
%<*pdfmarkbase>
\def\Hy@PutCatalog#1{%
  \pdfmark{pdfmark=/PUT,Raw={\string{Catalog\string} << #1 >>}}%
}
%</pdfmarkbase>
%    \end{macrocode}
%    \end{macro}
%
%    \begin{macrocode}
%<*pdfmarkbase|dvipdfm>
\ifHy@pagelabels
%    \end{macrocode}
%
%    \begin{macro}{\HyPL@StorePageLabel}
%    This macro writes a string to the .aux file.
%    \begin{macrocode}
  \def\HyPL@StorePageLabel#1{%
    \if@filesw
      \begingroup
        \edef\Hy@tempa{\the\Hy@abspage\space<< #1 >> }%
        \immediate\write\@mainaux{%
          \string\HyPL@Entry{\Hy@tempa}%
        }%
      \endgroup
    \fi
  }
%    \end{macrocode}
%    \end{macro}
%
%    Write a dummy definition of \cmd{\HyPL@Entry} for the case,
%    that the next run is done without hyperref.
%    A marker for the rerun warning is set and the /PageLabels
%    is written.
%    \begin{macrocode}
  \AfterBeginDocument{%
    \if@filesw
      \immediate\write\@mainaux{%
        \string\providecommand\string*\string\HyPL@Entry[1]{}%
      }%
    \fi
    \ifx\HyPL@Labels\@empty
      \Hy@WarningNoLine{Rerun to get /PageLabels entry}%
    \else
      \HyPL@SetPageLabels
    \fi
    \let\HyPL@Entry\@gobble
  }%
%    \end{macrocode}
%
%    \begin{macro}{\HyPL@Entry}
%    \begin{macrocode}
  \def\HyPL@Entry#1{%
    \expandafter\gdef\expandafter\HyPL@Labels\expandafter{%
      \HyPL@Labels
      #1%
    }%
  }
%    \end{macrocode}
%    \end{macro}
%
%    \begin{macrocode}
\fi
%</pdfmarkbase|dvipdfm>
%    \end{macrocode}
%
%    \begin{macrocode}
%<*package>
%    \end{macrocode}
%
%    \begin{macrocode}
\MaybeStopEarly
%    \end{macrocode}
%
% \section{Automated \LaTeX\ hypertext cross-references}\label{latexxref}
% Anything which can be referenced advances some counter; we overload
% this to put in a hypertext starting point (with no visible anchor),
% and make a note of that for later use in |\label|.
% This will fail badly if |\theH<name>|
% does not expand to a sensible reference. This means that classes
% or package which introduce new elements need to define
% an equivalent  |\theH<name>|  for every  |\the<name>|. We do make
% a trap to make |\theH<name>| be the same as |\arabic{<name>}|,
% if |\theH<name>| is not defined, but this is not necessarily a good idea.
% Alternatively, the `naturalnames' option uses whatever \LaTeX\
% provides, which may be useable. But then its up to you to make
% sure these are legal PDF and HTML names. The `hypertexnames=false' option
% just makes up arbitrary names.
%
% All the shenanigans is to make sure section numbers etc
% are always arabic, separated by dots. Who knows how people
% will set up |\@currentlabel|? If they put spaces in, or brackets
% (quite legal) then the hypertext processors will get upset.
%
% But this is flaky, and open to abuse. Styles like
% |subeqn| will mess it up, for starters. Appendices are an issue, too.
% We just hope to cover most situations. We can at least cope
% with the standard sectioning structure, allowing for |\part|
% and |\chapter|.
%
% Start with a fallback for equations
%    \begin{macrocode}
\newcommand\theHequation{\theHsection.\arabic{equation}}
\@ifundefined{thepart}{}{\newcommand\theHpart{\arabic{part}}}
\@ifundefined{thechapter}{%
  \newcommand\theHsection    {\arabic{section}}
  \newcommand\theHfigure     {\arabic{figure}}
  \newcommand\theHtable      {\arabic{table}}
}{%
  \newcommand\theHchapter    {\arabic{chapter}}
  \newcommand\theHfigure     {\theHchapter.\arabic{figure}}
  \newcommand\theHtable      {\theHchapter.\arabic{table}}
  \newcommand\theHsection    {\theHchapter.\arabic{section}}
}
\newcommand\theHsubsection   {\theHsection.\arabic{subsection}}
\newcommand\theHsubsubsection{\theHsubsection.\arabic{subsubsection}}
\newcommand\theHparagraph    {\theHsubsubsection.\arabic{paragraph}}
\newcommand\theHsubparagraph {\theHparagraph.\arabic{subparagraph}}
\newcommand\theHtheorem      {\theHsection.\arabic{theorem}}
\newcommand\theHthm          {\theHsection.\arabic{thm}}
%    \end{macrocode}
% Thanks to Greta Meyer (gbd@pop.cwru.edu) for making me realize
% that enumeration starts at 0 for every list! But |\item|
% occurs inside |\trivlist|, so check if its a real |\item| before
% incrementing counters.
%    \begin{macrocode}
\let\H@item\item
\newcounter{Item}
\def\theHItem{\arabic{Item}}
\def\item{%
  \@hyper@itemfalse
  \if@nmbrlist\@hyper@itemtrue\fi
  \H@item
}
%    \end{macrocode}
%
%    \begin{macrocode}
\newcommand\theHenumi     {\theHItem}
\newcommand\theHenumii    {\theHItem}
\newcommand\theHenumiii   {\theHItem}
\newcommand\theHenumiv    {\theHItem}
\newcommand\theHHfootnote {\arabic{Hfootnote}}
\newcommand\theHmpfootnote{\arabic{mpfootnote}}
\let\theHHmpfootnote\theHHfootnote
%    \end{macrocode}
% Tanmoy asked for this default handling of undefined |\theH<name>|
% situations. It really isn't clear what would be ideal, whether to
% turn off hyperizing of unknown elements, to pick up the textual
% definition of the counter, or to default it to something like
% |\arabic{name}|. We take the latter course, slightly worriedly.
%    \begin{macrocode}
\let\H@refstepcounter\refstepcounter
\edef\name@of@eq{equation}%
\edef\name@of@slide{slide}%
%    \end{macrocode}
% We do not want the handler for |\refstepcounter| to cut in
% during the processing of |\item| (we handle that separately),
% so we provide a bypass conditional.
%    \begin{macrocode}
\newif\if@hyper@item
\newif\if@skiphyperref
\@hyper@itemfalse
\@skiphyperreffalse
\def\refstepcounter#1{%
  \H@refstepcounter{#1}%
  \edef\This@name{#1}%
  \ifx\This@name\name@of@slide
  \else
    \if@skiphyperref
    \else
      \if@hyper@item
        \stepcounter{Item}%
        \hyper@refstepcounter{Item}%
      \else
        \hyper@refstepcounter{#1}%
      \fi
    \fi
  \fi
}
%    \end{macrocode}
% AMS\LaTeX\ processes all equations twice; we want to make sure
% that the hyper stuff is not executed twice, so we use the AMS
% |\ifmeasuring@|, initialized if AMS math is not used.
%    \begin{macrocode}
\@ifpackageloaded{amsmath}{}{\newif\ifmeasuring@\measuring@false}
%    \end{macrocode}
%
%    \begin{macro}{\hyper@refstepcounter}
%    \begin{macrocode}
\def\hyper@refstepcounter#1{%
  \edef\This@name{#1}%
  \ifx\This@name\name@of@eq
    \make@stripped@name{\theequation}%
    \let\theHequation\newname
  \fi
  \@ifundefined{theH#1}{%
    \expandafter\def\csname theH#1\endcsname{\arabic{#1}}%
  }{}%
  \hyper@makecurrent{#1}%
  \ifmeasuring@
  \else
    \Hy@raisedlink{%
      \hyper@anchorstart{\@currentHref}\hyper@anchorend
    }%
  \fi
}
%    \end{macrocode}
%    \end{macro}
%
%    After \cmd{\appendix} ``chapter'' should be replaced
%    by ``appendix'' to get \cmd{\autoref} work.
%    Macro \cmd{\Hy@chapapp} contains the current valid
%    name like \cmd{\@chapapp}, which cannot be used,
%    because this string depends on the current language.
%
%    The ``french'' package defines counter \cmd{\thechapter}
%    by \cmd{\newcounter{chapter}}, if \cmd{\@ifundefined{chapter}}.
%    \begin{macrocode}
\def\Hy@chapterstring{chapter}
\def\Hy@appendixstring{appendix}
\def\Hy@chapapp{\Hy@chapterstring}
\let\HyOrg@appendix\appendix
\def\appendix{%
  \@ifundefined{chapter}%
    {\gdef\theHsection{\Alph{section}}}%
    {\gdef\theHchapter{\Alph{chapter}}}%
  \xdef\Hy@chapapp{\Hy@appendixstring}%
  \HyOrg@appendix
}
%    \end{macrocode}
%    \begin{macro}{\hyper@makecurrent}
% Because of Babel mucking around, nullify |\textlatin| when making names.
% And |\@number| because of babel's lrbabel.def.
%    \begin{macrocode}
\def\hyper@makecurrent#1{%
  \begingroup
    \edef\Hy@param{#1}%
    \ifx\Hy@param\Hy@chapterstring
      \let\Hy@param\Hy@chapapp
    \fi
    \ifHy@hypertexnames
      \let\@number\@firstofone
      \ifHy@naturalnames
        \let\textlatin\@firstofone
        \xdef\@currentHlabel{\csname the#1\endcsname}%
      \else
        \xdef\@currentHlabel{\csname theH#1\endcsname}%
      \fi
      \xdef\@currentHref{%
        \Hy@param.\expandafter\strip@prefix\meaning\@currentHlabel
      }%
    \else
      \Hy@GlobalStepCount\Hy@linkcounter
      \xdef\@currentHref{\Hy@param.\the\Hy@linkcounter}%
    \fi
  \endgroup
}
%    \end{macrocode}
%    \end{macro}
%
%    \begin{macrocode}
\@ifpackageloaded{fancyvrb}{%
  \def\FV@StepLineNo{%
    \FV@SetLineNo
    \def\FV@StepLineNo{\H@refstepcounter{FancyVerbLine}}%
    \FV@StepLineNo
  }%
}{}
%    \end{macrocode}
%
% \section{Package lastpage support}
%    Package lastpage directly writes the |\newlabel| command to the
%    aux file. Because package hyperref requires additional arguments,
%    the internal command |\lastpage@putlabel| is redefined.
%    The patch is deferred by |\AtBeginDocument|, because it is possible
%    that package lastpage is loaded after package hyperref.
%    The same algorithm (options hypertexnames and plainpages)
%    is used to get the page anchor name as
%    in |\@hyperfixhead| (see sec. \ref{pagenum}).
%    The link will not work if option pageanchor is set to false.
%    \begin{macro}{\lastpage@putlabel}
%    \begin{macrocode}
\AtBeginDocument{%
  \@ifpackageloaded{lastpage}{%
    \ifHy@pageanchor
    \else
      \Hy@WarningNoLine{%
        The \string\pageref{LastPage} link doesn't work\MessageBreak
        with disabled option `pageanchor'%
      }%
    \fi
    \def\lastpage@putlabel{%
      \addtocounter{page}{-1}%
      \if@filesw
        \begingroup
          \let\@number\@firstofone
          \ifHy@pageanchor
            \ifHy@hypertexnames
              \ifHy@plainpages
                \def\Hy@temp{\arabic{page}}%
              \else
                \let\textlatin\@firstofone
                \edef\Hy@temp{\thepage}%
              \fi
            \else
              \def\Hy@temp{\the\Hy@pagecounter}%
            \fi
          \fi
          \immediate\write\@auxout{%
            \string\newlabel
              {LastPage}{{}{\thepage}{}{%
                \ifHy@pageanchor page.\Hy@temp\fi}{}}%
          }%
        \endgroup
      \fi
      \addtocounter{page}{1}%
    }%
  }{}%
}
%</package>
%<*check>
\checkpackage{lastpage}[1994/06/25]
\checkcommand\def\lastpage@putlabel{%
  \addtocounter{page}{-1}%
  \immediate\write\@auxout{%
    \string\newlabel{LastPage}{{}{\thepage}}%
  }%
  \addtocounter{page}{1}%
}
%</check>
%<*package>
%    \end{macrocode}
%    \end{macro}
%
% \section{Package ifthen support}
%
%    \begin{macro}{\hypergetpageref}
%    Package ifthen's \cmd{\isodd} does not work with
%    \cmd{\pageref} because of the extra link and the
%    star form. Therefore we need an expandable variant:
%    \begin{macrocode}
\newcommand*{\hypergetpageref}[1]{%
  \expandafter\Hy@getpageref\csname r@#1\endcsname{#1}%
}
\def\Hy@getpageref#1#2{%
  \ifx#1\relax
    0%
    \protect\G@refundefinedtrue
    \@latex@warning{Reference `#2' on page \thepage\space
      undefined%
    }%
  \else
    \expandafter\Hy@GetSecondArg#1\@nil
  \fi
}
\long\def\Hy@GetSecondArg#1#2#3\@nil{#2}
%    \end{macrocode}
%    \end{macro}
%
%    \begin{macro}{\hypergetref}
%    The corresponding macro of \cmd{\hypergetpageref} is
%    \cmd{\hypergetref}:
%    \begin{macrocode}
\newcommand*{\hypergetref}[1]{%
  \expandafter\Hy@getref\csname r@#1\endcsname{#1}%
}
\def\Hy@getref#1#2{%
  \ifx#1\relax
    ??%
    \protect\G@refundefinedtrue
    \@latex@warning{Reference `#2' on page \thepage\space
      undefined%
    }%
  \else
    \expandafter\@car#1\@nil
  \fi
}
%    \end{macrocode}
%    \end{macro}
%
%    \begin{macro}{\Hy@TE@begingroup}
%    Unfortunately there is no hook in \cmd{\ifthenelse} in order
%    to provide expandable versions without link of \cmd{\ref}
%    and \cmd{\pageref}. As possible target I could only find
%    \cmd{\begingroup}:
%    \begin{macrocode}
\@ifpackageloaded{ifthen}{%
  \let\HyOrg@ifthenelse\ifthenelse
  \def\ifthenelse{%
    \let\begingroup\Hy@TE@begingroup
    \HyOrg@ifthenelse
  }%
  \let\HyOrg@begingroup\begingroup
  \def\Hy@TE@begingroup{%
    \let\begingroup\HyOrg@begingroup
    \begingroup
    \let\ref\hypergetref
    \let\pageref\hypergetpageref
  }%
}{}
%    \end{macrocode}
%    \end{macro}
%
% \section{Package titlesec and titletoc support}
%
%    This code is contributed by Javier Bezos
%    (Email: \Email{jbezos@arrakis.es}).
%
%    Package titlesec support:
%    \begin{macrocode}
\@ifpackageloaded{titlesec}{%
  \def\ttl@Hy@steplink#1{%
    \Hy@GlobalStepCount\Hy@linkcounter
    \xdef\@currentHref{#1*.\the\Hy@linkcounter}%
    \def\ttl@Hy@saveanchor{%
      \Hy@raisedlink{\hyper@anchorstart{\@currentHref}\hyper@anchorend}%
    }%
  }%
  \def\ttl@Hy@refstepcounter#1{%
    \let\ttl@b\Hy@raisedlink
    \def\Hy@raisedlink##1{\def\ttl@Hy@saveanchor{\Hy@raisedlink{##1}}}%
    \refstepcounter{#1}%
    \let\Hy@raisedlink\ttl@b
  }%
}{}
%    \end{macrocode}
%
%    Package titletoc support:
%    \begin{macrocode}
\@ifpackageloaded{titletoc}{%
  \def\ttl@gobblecontents#1#2#3#4{\ignorespaces}%
}{}
%    \end{macrocode}
%
% \section{Package varioref support}
%
%    Package nameref uses five arguments for the ref system.
%    Fix provided by Felix Neubauer (\Email{felix.neubauer@gmx.net}).
%    \begin{macrocode}
\@ifpackageloaded{varioref}{%
  \def\vref@pagenum#1#2{%
    \@ifundefined{r@#2}{%
      \@namedef{r@#2}{{??}{??}{}{}{}}%
    }{}%
    \edef#1{\hypergetpageref{#2}}%
  }%
}{}
%    \end{macrocode}
%
% \section{Equations}\label{equations}
% We want to  make the whole equation a target anchor.
% Overload equation, temporarily reverting to original
% |\refstepcounter|. If, however, it is in AMS math, we do not
% do anything, as the tag mechanism is used there (see section \ref{ams}).
%    \begin{macrocode}
\let\new@refstepcounter\refstepcounter
\let\H@equation\equation
\let\H@endequation\endequation
%    \end{macrocode}
%
%    \begin{macrocode}
\@ifpackageloaded{amsmath}{}{%
  \def\equation{%
  \let\refstepcounter\H@refstepcounter
  \H@equation
  \make@stripped@name{\theequation}%
  \let\theHequation\newname
  \hyper@makecurrent{equation}%
  \Hy@raisedlink{\hyper@anchorstart{\@currentHref}}%
  \let\refstepcounter\new@refstepcounter}%
  \def\endequation{\Hy@raisedlink{\hyper@anchorend}\H@endequation}%
}
%    \end{macrocode}
% My goodness, why can't \LaTeX{} be consistent? Why is |\eqnarray|
% set up differently from other objects?
%
% People (you know who you are, Thomas Beuth) sometimes make
% an eqnarray where \emph{all} the lines end with  |\notag|,
% so there is no suitable anchor at all. In this case, pass by
% on the other side.
%    \begin{macrocode}
\newif\if@eqnstar
\@eqnstarfalse
\let\H@eqnarray\eqnarray
\let\H@endeqnarray\endeqnarray
\def\eqnarray{%
  \let\Hy@reserved@a\relax
  \def\@currentHref{}%
  \H@eqnarray
  \if@eqnstar
  \else
    \ifx\\\@currentHref\\%
    \else
      \make@stripped@name{\theequation}%
      \let\theHequation\newname
      \hyper@makecurrent{equation}%
      \hyper@anchorstart{\@currentHref}{}\hyper@anchorend
    \fi
  \fi
}
\def\endeqnarray{%
  \H@endeqnarray
}
%    \end{macrocode}
% This is quite heavy-handed, but it works for now. If its an |eqnarray*|
% we need to disable the hyperref actions. There may well be a cleaner
% way to trap this. Bill Moss found this.
%    \begin{macrocode}
\@namedef{eqnarray*}{%
  \def\@eqncr{\nonumber\@seqncr}\@eqnstartrue\eqnarray
}
\@namedef{endeqnarray*}{%
  \nonumber\endeqnarray\@eqnstarfalse
}
%    \end{macrocode}
% Then again, we have the \emph{subeqnarray}
% package. Tanmoy provided some code for this:
%    \begin{macrocode}
\@ifundefined{subeqnarray}{}{%
  \let\H@subeqnarray\subeqnarray
  \let\H@endsubeqnarray\endsubeqnarray
  \def\subeqnarray{%
    \let\Hy@reserved@a\relax
    \H@subeqnarray
    \make@stripped@name{\theequation}%
    \let\theHequation\newname
    \hyper@makecurrent{equation}%
    \hyper@anchorstart{\@currentHref}{}\hyper@anchorend
  }%
  \def\endsubeqnarray{%
    \H@endsubeqnarray
  }%
  \newcommand\theHsubequation{\theHequation\alph{subequation}}%
}
%    \end{macrocode}
% The aim of this macro is to produce a sanitized version of
% its argument, to make it a safe label.
%    \begin{macrocode}
\def\make@stripped@name#1{%
  \begingroup
    \escapechar\m@ne
    \global\let\newname\@empty
    \protected@edef\Hy@tempa{#1}%
    \edef\@tempb{%
      \noexpand\@tfor\noexpand\Hy@tempa:=%
        \expandafter\strip@prefix\meaning\Hy@tempa
    }%
    \@tempb\do{%
      \if{\Hy@tempa\else
        \if}\Hy@tempa\else
          \xdef\newname{\newname\Hy@tempa}%
        \fi
      \fi
    }%
  \endgroup
}
%    \end{macrocode}
%
% \section{Footnotes}\label{footnotes}
% The footnote mark is a hypertext link, and the text is a target.
% We separately number the footnotes sequentially through the
% text, separately from whatever labels the text assigns. Too hard
% to keep track of markers otherwise. If the raw forms |\footnotemark|
% and |\footnotetext| are used, force them to use un-hyper original.
%
% Tabularx causes footnote problems, disable the linking if that is loaded.
%    \begin{macrocode}
\@ifpackageloaded{tabularx}{\Hy@hyperfootnotesfalse}{}
\ifHy@hyperfootnotes
  \newcounter{Hfootnote}
  \let\H@@footnotetext\@footnotetext
  \let\H@@footnotemark\@footnotemark
  \def\@xfootnotenext[#1]{%
    \begingroup
      \csname c@\@mpfn\endcsname #1\relax
      \unrestored@protected@xdef\@thefnmark{\thempfn}%
    \endgroup
    \ifx\@footnotetext\@mpfootnotetext
      \expandafter\H@@mpfootnotetext
    \else
      \expandafter\H@@footnotetext
    \fi
  }%
  \def\@xfootnotemark[#1]{%
    \begingroup
      \c@footnote #1\relax
      \unrestored@protected@xdef\@thefnmark{\thefootnote}%
    \endgroup
    \H@@footnotemark
  }%
  \let\H@@mpfootnotetext\@mpfootnotetext
  \long\def\@mpfootnotetext#1{%
    \H@@mpfootnotetext{%
      \ifHy@nesting
        \hyper@@anchor{\@currentHref}{#1}%
      \else
        \Hy@raisedlink{\hyper@@anchor{\@currentHref}{\relax}}#1%
      \fi
    }%
  }%
  \long\def\@footnotetext#1{%
    \H@@footnotetext{%
      \ifHy@nesting
        \hyper@@anchor{\@currentHref}{#1}%
      \else
        \Hy@raisedlink{\hyper@@anchor{\@currentHref}{\relax}}#1%
      \fi
    }%
  }%
%    \end{macrocode}
%    Redefine \verb+\@footnotemark+, borrowing its code (at the
%    cost of getting out of sync with latex.ltx), to take
%    advantage of its white space and hyphenation fudges. If we just
%    overload it, we can get variant documents (the word before the
%    footnote is treated differently). Thanks to David Carlisle and
%    Brian Ripley for confusing and helping me on this.
%    \begin{macrocode}
  \def\@footnotemark{%
    \leavevmode
    \ifhmode\edef\@x@sf{\the\spacefactor}\nobreak\fi
    \H@refstepcounter{Hfootnote}%
    \hyper@makecurrent{Hfootnote}%
    \hyper@linkstart{link}{\@currentHref}%
    \@makefnmark
    \hyper@linkend
    \ifhmode\spacefactor\@x@sf\fi
    \relax
  }%
%    \end{macrocode}
%
%    Support for footnotes in p columns of longtable.
%    Here \verb+\footnote+ commands are splitted into
%    \verb+\footnotemark+ and a call of \verb+\footnotetext+
%    with the optional argument, that is not supported
%    by hyperref. The result is a link by \verb+\footnotemark+
%    without valid anchor.
%
%    \begin{macrocode}
  \@ifpackageloaded{longtable}{%
    \CheckCommand*{\LT@p@ftntext}[1]{%
      \edef\@tempa{%
        \the\LT@p@ftn
        \noexpand\footnotetext[\the\c@footnote]%
      }%
      \global\LT@p@ftn\expandafter{\@tempa{#1}}%
    }%
    \long\def\LT@p@ftntext#1{%
      \edef\@tempa{%
        \the\LT@p@ftn
        \begingroup
          \noexpand\c@footnote=\the\c@footnote\relax
          \noexpand\protected@xdef
              \noexpand\@thefnmark{\noexpand\thempfn}%
          \noexpand\footnotetext
      }%
      \global\LT@p@ftn\expandafter{%
          \@tempa{#1}%
        \endgroup
      }%
    }%
  }{}%
%    \end{macrocode}
%
%    But the special footnotes
%    in |\maketitle| are much too hard to deal with
%    properly. Let them revert to plain behaviour.
%    The koma classes add an optional argument.
%    \begin{macrocode}
  \let\HyOrg@maketitle\maketitle
  \def\maketitle{%
    \let\Hy@saved@footnotemark\@footnotemark
    \let\Hy@saved@footnotetext\@footnotetext
    \let\@footnotemark\H@@footnotemark
    \let\@footnotetext\H@@footnotetext
    \@ifnextchar[\Hy@maketitle@optarg{% ]
      \HyOrg@maketitle
      \Hy@maketitle@end
    }%
  }%
  \def\Hy@maketitle@optarg[#1]{%
    \HyOrg@maketitle[{#1}]%
    \Hy@maketitle@end
  }%
  \def\Hy@maketitle@end{%
    \ifx\@footnotemark\H@@footnotemark
      \let\@footnotemark\Hy@saved@footnotemark
    \fi
    \ifx\@footnotetext\H@@footnotetext
      \let\@footnotetext\Hy@saved@footnotetext
    \fi
  }%
%    \end{macrocode}
%    \begin{macro}{\realfootnote}
%    Does anyone remember the function and purpose of \cmd{\realfootnote}?
%    \begin{macrocode}
  \def\realfootnote{%
    \@ifnextchar[\@xfootnote{%
      \stepcounter{\@mpfn}%
      \protected@xdef\@thefnmark{\thempfn}%
      \H@@footnotemark\H@@footnotetext
    }%
  }%
%    \end{macrocode}
%    \begin{macrocode}
\fi
\Hy@DisableOption{hyperfootnotes}
%    \end{macrocode}
%    \end{macro}
%
%    \begin{macrocode}
%</package>
%<*check>
\checklatex
\checkcommand\def\@xfootnotenext[#1]{%
  \begingroup
    \csname c@\@mpfn\endcsname #1\relax
    \unrestored@protected@xdef\@thefnmark{\thempfn}%
  \endgroup
  \@footnotetext
}
\checkcommand\def\@xfootnotemark[#1]{%
  \begingroup
    \c@footnote #1\relax
    \unrestored@protected@xdef\@thefnmark{\thefootnote}%
  \endgroup
  \@footnotemark
}
\checkcommand\def\@footnotemark{%
  \leavevmode
  \ifhmode\edef\@x@sf{\the\spacefactor}\nobreak\fi
  \@makefnmark
  \ifhmode\spacefactor\@x@sf\fi
  \relax
}
%</check>
%<*package>
%    \end{macrocode}
%
% \section{Float captions}\label{captions}
% Make the float caption the hypertext anchor; curiously enough,
% we can't just copy the definition of |\@caption|. Its all to do
% with expansion. It screws up. Sigh.
%    \begin{macrocode}
\def\caption{%
  \ifx\@captype\@undefined
    \@latex@error{\noexpand\caption outside float}\@ehd
    \expandafter\@gobble
  \else
    \H@refstepcounter\@captype
    \expandafter\@firstofone
  \fi
  {\@dblarg{\@caption\@captype}}%
}
\long\def\@caption#1[#2]#3{%
  \hyper@makecurrent{\@captype}%
  \par\addcontentsline{\csname ext@#1\endcsname}{#1}{%
    \protect\numberline{\csname the#1\endcsname}{\ignorespaces #2}%
  }%
  \begingroup
    \@parboxrestore
    \if@minipage
      \@setminipage
    \fi
    \normalsize
    \@makecaption{\csname fnum@#1\endcsname}{%
      \ignorespaces
%    \end{macrocode}
% If we cannot have nesting, the anchor is empty.
%    \begin{macrocode}
      \ifHy@nesting
        \hyper@@anchor{\@currentHref}{#3}%
      \else
        \hyper@@anchor{\@currentHref}{\relax}#3%
      \fi
    }%
    \par
  \endgroup
}
%    \end{macrocode}
%
%    \begin{macrocode}
%</package>
%<*check>
\checklatex[1999/06/01 - 2000/06/01]
\checkcommand\def\caption{%
  \ifx\@captype\@undefined
    \@latex@error{\noexpand\caption outside float}\@ehd
    \expandafter\@gobble
  \else
    \refstepcounter\@captype
    \expandafter\@firstofone
  \fi
  {\@dblarg{\@caption\@captype}}%
}
\checkcommand\long\def\@caption#1[#2]#3{%
  \par
  \addcontentsline{\csname ext@#1\endcsname}{#1}%
    {\protect\numberline{\csname the#1\endcsname}{\ignorespaces #2}}%
  \begingroup
    \@parboxrestore
    \if@minipage
      \@setminipage
    \fi
    \normalsize
    \@makecaption{\csname fnum@#1\endcsname}{\ignorespaces #3}\par
  \endgroup
}
%</check>
%<*package>
%    \end{macrocode}
%
% \section{Bibliographic references}\label{bib}
% This is not very robust, since many styles redefine these things.
% The package used to redefine |\@citex| and the like; then we tried
% adding the hyperref call explicitly into the .aux file.
% Now we redefine |\bibcite|; this still breaks some citation packages
% so we have to work around them. But this remains extremely dangerous.
% Any or all of \emph{achemso}, \emph{chapterbib},
% and \emph{drftcite} may break.
%
% However, lets make an attempt to get \emph{natbib} right, because
% thats a powerful, important package.
% Patrick Daly (\Email{daly@linmpi.mpg.de})  has
% provided hooks for us, so all we need to do is activate them.
%    \begin{macrocode}
\def\hyper@natlinkstart#1{%
  \Hy@backout{#1}%
  \hyper@linkstart{cite}{cite.#1}%
  \def\hyper@nat@current{#1}%
}
\def\hyper@natlinkend{%
  \hyper@linkend
}
\def\hyper@natlinkbreak#1#2{%
  \hyper@linkend#1\hyper@linkstart{cite}{cite.#2}%
}
\def\hyper@natanchorstart#1{%
  \Hy@raisedlink{\hyper@anchorstart{cite.#1}}%
}
\def\hyper@natanchorend{\hyper@anchorend}
%    \end{macrocode}
% Do not play games if we have natbib support.
% Macro \@extra@binfo added for chapterbib support.
%    \begin{macrocode}
\@ifundefined{NAT@parse}{%
  \def\bibcite#1#2{%
    \@newl@bel{b}{#1\@extra@binfo}{\hyper@@link[cite]{}{cite.#1}{#2}}%
  }%
  \gdef\@extra@binfo{}%
%    \end{macrocode}
%    Package |babel| redefines \cmd{\bibcite} with
%    macro \cmd{\bbl@cite@choice}. It needs to be overwritten
%    to avoid the warning ``Label(s) may have changed.''.
%    \begin{macrocode}
  \let\Hy@bibcite\bibcite
  \begingroup
    \@ifundefined{bbl@cite@choice}{}{%
      \g@addto@macro\bbl@cite@choice{%
        \let\bibcite\Hy@bibcite
      }%
    }%
  \endgroup
%    \end{macrocode}
%    \begin{macrocode}
% |\@BIBLABEL| is working around a `feature' of Rev\TeX.
%    \begin{macrocode}
  \providecommand*{\@BIBLABEL}{\@biblabel}%
  \def\@lbibitem[#1]#2{%
    \@skiphyperreftrue
    \H@item[%
      \ifx\Hy@raisedlink\@empty
        \hyper@anchorstart{cite.#2}\@BIBLABEL{#1}\hyper@anchorend
      \else
        \Hy@raisedlink{\hyper@anchorstart{cite.#2}\hyper@anchorend}%
        \@BIBLABEL{#1}%
      \fi
      \hfill
    ]%
    \@skiphyperreffalse
    \if@filesw
      \begingroup
        \let\protect\noexpand
        \immediate\write\@auxout{%
          \string\bibcite{#2}{#1}%
        }%
      \endgroup
    \fi
    \ignorespaces
  }%
%    \end{macrocode}
% Since |\bibitem| is doing its own labelling, call the raw
% version of |\item|, to avoid extra spurious labels
%    \begin{macrocode}
  \def\@bibitem#1{%
    \@skiphyperreftrue\H@item\@skiphyperreffalse
    \Hy@raisedlink{\hyper@anchorstart{cite.#1}\relax\hyper@anchorend}%
    \if@filesw
      \begingroup
        \let\protect\noexpand
        \immediate\write\@auxout{%
          \string\bibcite{#1}{\the\value{\@listctr}}%
        }%
      \endgroup
    \fi
    \ignorespaces
  }%
}{}
%    \end{macrocode}
%
%    \begin{macrocode}
%</package>
%<*check>
\checklatex
\checkcommand\def\@lbibitem[#1]#2{%
  \item[\@biblabel{#1}\hfill]%
  \if@filesw
    {%
      \let\protect\noexpand
      \immediate\write\@auxout{%
        \string\bibcite{#2}{#1}%
      }%
    }%
  \fi
  \ignorespaces
}
\checkcommand\def\@bibitem#1{%
  \item
  \if@filesw
    \immediate\write\@auxout{%
      \string\bibcite{#1}{\the\value{\@listctr}}%
    }%
  \fi
  \ignorespaces
}
%</check>
%<*package>
%    \end{macrocode}
%
% Revtex (bless its little heart) takes over |\bibcite| and looks
% at the result to measure something. Make this a hypertext link
% and it goes ape. Therefore, make an anodyne result first, call
% its business, then go back to the real thing.
%    \begin{macrocode}
\@ifclassloaded{revtex}{%
  \Hy@Info{*** compatibility with revtex **** }%
  \def\revtex@checking#1#2{%
    \expandafter\let\expandafter\T@temp\csname b@#1\endcsname
    \expandafter\def\csname b@#1\endcsname{#2}%
    \@SetMaxRnhefLabel{#1}%
    \expandafter\let\csname b@#1\endcsname\T@temp
  }%
%    \end{macrocode}
% Tanmoy provided this replacement for CITEX. Lord knows what it does.
% For chapterbib added: \@extra@b@citeb
%    \begin{macrocode}
  \@ifundefined{@CITE}{\def\@CITE{\@cite}}{}%
  \providecommand*{\@extra@b@citeb}{}%
  \def\@CITEX[#1]#2{%
    \let\@citea\@empty
    \leavevmode
    \unskip
    $^{%
      \scriptstyle
      \@CITE{%
        \@for\@citeb:=#2\do{%
          \@citea
          \def\@citea{,\penalty\@m\ }%
          \edef\@citeb{\expandafter\@firstofone\@citeb}%
          \if@filesw
            \immediate\write\@auxout{\string\citation{\@citeb}}%
          \fi
          \@ifundefined{b@\@citeb\extra@b@citeb}{%
            \mbox{\reset@font\bfseries ?}%
            \G@refundefinedtrue
            \@latex@warning{%
              Citation `\@citeb' on page \thepage \space undefined%
            }%
          }{%
            {\csname b@\@citeb\@extra@b@citeb\endcsname}%
          }%
        }%
      }{#1}%
    }$%
  }%
%    \end{macrocode}
% No, life is too short. I am not going to understand the
% Revtex |\@collapse| macro, I shall
% just restore the original behaviour of |\@citex|;
% sigh. This is SO vile.
%    \begin{macrocode}
  \def\@citex[#1]#2{%
    \let\@citea\@empty
    \@cite{%
      \@for\@citeb:=#2\do{%
        \@citea
        \def\@citea{,\penalty\@m\ }%
        \edef\@citeb{\expandafter\@firstofone\@citeb}%
        \if@filesw
          \immediate\write\@auxout{\string\citation{\@citeb}}%
        \fi
        \@ifundefined{b@\@citeb\@extra@b@citeb}{%
          \mbox{\reset@font\bfseries ?}%
          \G@refundefinedtrue
          \@latex@warning{%
            Citation `\@citeb' on page \thepage \space undefined%
          }%
        }{%
          \hbox{\csname b@\@citeb\@extra@b@citeb\endcsname}%
        }%
      }%
    }{#1}%
  }%
}{}
%    \end{macrocode}
%
% \subsection{Package harvard}
%
% Override Peter Williams' Harvard package; we have to
% a) make each of the citation types into a link; b) make
% each citation write a backref entry, and c) kick off a backreference
% section for each bibliography entry.
%
% The redefinitions have to be deferred to |\begin{document}|,
% because if harvard.sty is loaded and html.sty is present and
% detects pdf\TeX, then hyperref is already loaded at the begin
% of harvard.sty, and the |\newcommand| macros causes error
% messages.
%    \begin{macrocode}
\@ifpackageloaded{harvard}{%
  \AtBeginDocument{%
    \Hy@Info{*** compatibility with harvard **** }%
    \Hy@raiselinksfalse
    \def\harvardcite#1#2#3#4{%
      \global\@namedef{HAR@fn@#1}{\hyper@@link[cite]{}{cite.#1}{#2}}%
      \global\@namedef{HAR@an@#1}{\hyper@@link[cite]{}{cite.#1}{#3}}%
      \global\@namedef{HAR@yr@#1}{\hyper@@link[cite]{}{cite.#1}{#4}}%
      \global\@namedef{HAR@df@#1}{\csname HAR@fn@#1\endcsname}%
    }%
    \def\HAR@citetoaux#1{%
      \if@filesw\immediate\write\@auxout{\string\citation{#1}}\fi%
      \ifHy@backref
        \ifx\@empty\@currentlabel
        \else
          \@bsphack
          \protected@write\@auxout{}{%
            \string\@writefile{brf}{%
              \string\backcite{#1}{%
                {\thepage}{\@currentlabel}{\@currentHref}%
              }%
            }%
          }%
          \@esphack
        \fi
      \fi
    }%
    \def\harvarditem{%
      \@ifnextchar[{\@harvarditem}{\@harvarditem[\null]}%
    }%
    \def\@harvarditem[#1]#2#3#4#5\par{%
      \item[]%
      \hyper@anchorstart{cite.#4}\relax\hyper@anchorend
      \if@filesw
        \begingroup
          \def\protect##1{\string ##1\space}%
          \ifthenelse{\equal{#1}{\null}}%
            {\def\next{{#4}{#2}{#2}{#3}}}%
            {\def\next{{#4}{#2}{#1}{#3}}}%
          \immediate\write\@auxout{\string\harvardcite\codeof\next}%
       \endgroup
      \fi
      \protect\hspace*{-\labelwidth}%
      \protect\hspace*{-\labelsep}%
      \ignorespaces
      #5%
      \ifHy@backref
        \newblock
        \backref{\csname br@#4\endcsname}%
      \fi
      \par
    }%
%    \end{macrocode}
%    \begin{macro}{\HAR@checkcitations}
%    Package hyperref has added \cmd{\hyper@@link}, so
%    the original test \cmd{\HAR@checkcitations} will
%    fail every time and always will appear the ``Changed
%    labels'' warning. So we have to redefine
%    \cmd{\Har@checkcitations}:
%    \begin{macrocode}
    \long\def\HAR@checkcitations#1#2#3#4{%
      \def\HAR@tempa{\hyper@@link[cite]{}{cite.#1}{#2}}%
      \expandafter\ifx\csname HAR@fn@#1\endcsname\HAR@tempa
        \def\HAR@tempa{\hyper@@link[cite]{}{cite.#1}{#3}}%
        \expandafter\ifx\csname HAR@an@#1\endcsname\HAR@tempa
          \def\HAR@tempa{\hyper@@link[cite]{}{cite.#1}{#4}}%
          \expandafter\ifx\csname HAR@yr@#1\endcsname\HAR@tempa
          \else
            \@tempswatrue
          \fi
        \else
          \@tempswatrue
        \fi
      \else
        \@tempswatrue
      \fi
    }%
  }%
%    \end{macrocode}
%    \end{macro}
%    \begin{macrocode}
}{}
%    \end{macrocode}
%
% \subsection{Package chicago}
%    The links by \cmd{\citeN} and \cmd{\shortciteN} should
%    include the closing parentheses.
%
%    \begin{macrocode}
\@ifpackageloaded{chicago}{%
%    \end{macrocode}
%    \begin{macro}{\citeN}
%    \begin{macrocode}
  \def\citeN{%
    \def\@citeseppen{-1000}%
    \def\@cite##1##2{##1}%
    \def\citeauthoryear##1##2##3{##1 (##3\@cite@opt)}%
    \@citedata@opt
  }%
%    \end{macrocode}
%    \end{macro}
%    \begin{macro}{\shortciteN}
%    \begin{macrocode}
  \def\shortciteN{%
    \def\@citeseppen{-1000}%
    \def\@cite##1##2{##1}%
    \def\citeauthoryear##1##2##3{##2 (##3\@cite@opt)}%
    \@citedata@opt
  }%
%    \end{macrocode}
%    \end{macro}
%    \begin{macro}{\@citedata@opt}
%    \begin{macrocode}
  \def\@citedata@opt{%
    \let\@cite@opt\@empty
    \@ifnextchar [{%
      \@tempswatrue
      \@citedatax@opt
    }{%
      \@tempswafalse
      \@citedatax[]%
    }%
  }%
%    \end{macrocode}
%    \end{macro}
%    \begin{macro}{\@citedatax@opt}
%    \begin{macrocode}
  \def\@citedatax@opt[#1]{%
    \def\@cite@opt{, #1}%
    \@citedatax[{#1}]%
  }
%    \end{macrocode}
%    \end{macro}
%    \begin{macrocode}
}{}
%    \end{macrocode}
%
% \section{Page numbers}\label{pagenum}
% Give every page an automatic number anchor. This involves, sigh,
% overloading \LaTeX's output bits and pieces, which must be dangerous.
% This used to be |\@shipoutsetup|, now |\@begindvi|. We cannot even
% overload this, as it sets itself to null. SIGH.
%    \begin{macrocode}
\def\@begindvi{%
  \unvbox \@begindvibox
  \Hy@begindvi
  \global\let\@begindvi\Hy@begindvi
}
\def\Hy@begindvi{%
  \ifHy@pageanchor
    \@hyperfixhead
  \fi
  \HyPL@EveryPage
}
\def\pagenumbering#1{%
  \global\c@page \@ne
  \gdef\thepage{\csname @#1\endcsname\c@page}%
}
%    \end{macrocode}
% This is needed for some unremembered reason\ldots
%    \begin{macrocode}
\let\HYPERPAGEANCHOR\hyperpageanchor
%    \end{macrocode}
%
%    Macro \cmd{\@hyperfixhead} calls \cmd{\hyper@pagetransition}
%    and \cmd{\hyper@pageduration}. Therefore empty definitions
%    are provided for drivers that do not define these macros.
%
%    The last page should not contain a /Dur key, because there
%    is no page after the last page. Therefore at the last page
%    there should be a command |\hypersetup{pdfpageduration={}}|.
%    This can be set with \cmd{\AtEndDocument}, but it can
%    be too late, if the last page is already finished, or too
%    early, if lots of float pages will follow.
%    Therefore currently nothing is done by hyperref.
%    \begin{macrocode}
\providecommand\hyper@pagetransition{}
\providecommand\hyper@pageduration{}
\providecommand\hyper@pagehidden{}
%    \end{macrocode}
%
% This where we supply a destination for each page. Test to see
% if there is some sort of header. The test used to be
% |\expandafter\ifx\expandafter\@empty\H@old@thehead|.
%    \begin{macro}{\@hyperfixhead}
%    \begin{macrocode}
\def\@hyperfixhead{%
  \Hy@DistillerDestFix
  \ifHy@hypertexnames
    \ifHy@plainpages
      \gdef\Hy@TempPageAnchor{\hyper@@anchor{page.\the\c@page}}%
    \else
      \begingroup
        \let\@number\@firstofone
        \let\textlatin\@firstofone
        \xdef\@the@H@page{\thepage}%
      \endgroup
      \gdef\Hy@TempPageAnchor{\hyper@@anchor{page.\@the@H@page}}%
    \fi
  \else
    \Hy@GlobalStepCount\Hy@pagecounter
    \gdef\Hy@TempPageAnchor{\hyper@@anchor{page.\the\Hy@pagecounter}}%
  \fi
  \let\H@old@thehead\@thehead
  \if^\@thehead^%
    \def\H@old@thehead{\hfil}%
  \fi
  \ifHy@texht
    \def\@thehead{}%
  \else
    \ifHy@seminarslides
      \begingroup
        \let\leavevmode\relax
        \Hy@TempPageAnchor\relax
        \hyper@pagetransition
        \hyper@pageduration
        \hyper@pagehidden
      \endgroup
    \else
      \def\@thehead{%
        \Hy@TempPageAnchor\relax
        \hyper@pagetransition
        \hyper@pageduration
        \hyper@pagehidden
        \H@old@thehead
      }%
    \fi
  \fi
  \ifx\PDF@FinishDoc\@empty
  \else
    \PDF@FinishDoc
    \gdef\PDF@FinishDoc{}%
  \fi
}
%    \end{macrocode}
%    \end{macro}
%
%\section{Table of contents}\label{toc}
% TV Raman noticed that people who add arbitrary material into the TOC
% generate a bad or null link. We avoid that by checking if the current
% destination is empty. But if `the most recent destination' is not
% what you expect, you will be in trouble.
%    \begin{macrocode}
\begingroup\expandafter\expandafter\expandafter\endgroup
\expandafter\ifx\csname chapter\endcsname\relax
  \def\toclevel@part{0}
\else
  \def\toclevel@part{-1}
\fi
\def\toclevel@chapter{0}
\def\toclevel@section{1}
\def\toclevel@subsection{2}
\def\toclevel@subsubsection{3}
\def\toclevel@paragraph{4}
\def\toclevel@subparagraph{5}
\def\toclevel@figure{0}
\def\toclevel@table{0}
\@ifpackageloaded{listings}{%
  \def\theHlstlisting{\thelstlisting}%
  \def\toclevel@lstlisting{0}%
}{}
\@ifpackageloaded{listing}{%
  \def\theHlisting{\thelisting}%
  \def\toclevel@listing{0}%
}{}
\def\addcontentsline#1#2#3{% toc extension, type, tag
  \begingroup
    \let\label\@gobble
    \let\textlatin\@firstofone
    \ifx\@currentHref\@empty
      \Hy@Warning{%
        No destination for bookmark of \string\addcontentsline,%
        \MessageBreak destination is added%
      }%
      \phantomsection
    \fi
    \expandafter\ifx\csname toclevel@#2\endcsname\relax
      \Hy@WarningNoLine{bookmark level for unknown #2 defaults to 0}%
      \def\Hy@toclevel{0}%
    \else
      \edef\Hy@toclevel{\csname toclevel@#2\endcsname}%
    \fi
    \Hy@writebookmark{\csname the#2\endcsname}%
      {#3}%
      {\@currentHref}%
      {\Hy@toclevel}%
      {#1}%
    \ifHy@verbose
      \typeout{pdftex: bookmark at \the\inputlineno:
        {\csname the#2\endcsname}
        {#3}
        {\@currentHref}%
        {\Hy@toclevel}%
        {#1}%
      }%
    \fi
    \addtocontents{#1}{%
      \protect\contentsline{#2}{#3}{\thepage}{\@currentHref}%
    }%
  \endgroup
}
\def\contentsline#1#2#3#4{%
  \ifx\\#4\\%
    \csname l@#1\endcsname{#2}{#3}%
  \else
    \ifHy@linktocpage
      \csname l@#1\endcsname{{#2}}{%
        \hyper@linkstart{link}{#4}{#3}\hyper@linkend
      }%
    \else
      \csname l@#1\endcsname{%
        \hyper@linkstart{link}{#4}{#2}\hyper@linkend
      }{#3}%
    \fi
  \fi
}
%    \end{macrocode}
%
%    \begin{macrocode}
%</package>
%<*check>
\checklatex
\checkcommand\def\addcontentsline#1#2#3{%
  \addtocontents{#1}{\protect\contentsline{#2}{#3}{\thepage}}%
}
\checkcommand\def\contentsline#1{\csname l@#1\endcsname}
%</check>
%<*package>
%    \end{macrocode}
%
% \section{New counters}\label{counters}
% The whole theorem business makes up new counters on the fly;
% we are going to intercept this. Sigh. Do it at the level where
% new counters are defined.
%    \begin{macrocode}
\let\H@definecounter\@definecounter
\def\@definecounter#1{%
  \H@definecounter{#1}%
  \expandafter\def\csname theH#1\endcsname{\arabic{#1}}%
}
%    \end{macrocode}
% But what if they have used the optional argument to e.g. |\newtheorem|
% to determine when the numbering is reset? OK, we'll trap that too.
%    \begin{macrocode}
\let\H@newctr\@newctr
\def\@newctr#1[#2]{%
  \H@newctr#1[{#2}]%
  \expandafter\def\csname theH#1\endcsname
    {\csname the#2\endcsname.\arabic{#1}}%
}
%    \end{macrocode}
% \section{AMS\LaTeX\ compatibility}\label{ams}
% Oh, no, they don't use anything as simple as |\refstepcounter|
% in the AMS! We need to intercept some low-level operations
% of theirs. Damned if we are going to try and work out what
% they get up to. Just stick a label of `AMS' on the front, and use the
% label \emph{they} worked out. If that produces something invalid, I give
% up. They'll change all the code again anyway, I expect.
%    \begin{macrocode}
\let\Hmake@df@tag@@\make@df@tag@@
\def\make@df@tag@@#1{%
  \Hmake@df@tag@@{#1}%
  \Hy@GlobalStepCount\Hy@linkcounter
  \xdef\@currentHref{AMS.\the\Hy@linkcounter}%
  \Hy@raisedlink{\hyper@anchorstart{\@currentHref}\hyper@anchorend}%
}
\let\Hmake@df@tag@@@\make@df@tag@@@
\def\make@df@tag@@@#1{%
  \Hmake@df@tag@@@{#1}%
  \Hy@GlobalStepCount\Hy@linkcounter
  \xdef\@currentHref{AMS.\the\Hy@linkcounter}%
  \Hy@raisedlink{\hyper@anchorstart{\@currentHref}\hyper@anchorend}%
}
%    \end{macrocode}
% Only play with |\seteqlebal| if we are using pdftex. Other drivers
% cause problems; requested by Michael Downes (AMS).
%    \begin{macrocode}
\@ifpackagewith{hyperref}{pdftex}{%
   \let\H@seteqlabel\@seteqlabel
   \def\@seteqlabel#1{%
     \H@seteqlabel{#1}%
     \xdef\@currentHref{AMS.\the\Hy@linkcounter}%
     \Hy@raisedlink{\hyper@anchorstart{\@currentHref}\hyper@anchorend}%
   }%
}{}
%    \end{macrocode}
% This code I simply cannot remember what I was trying to achieve.
% The final result seems to do nothing anyway.
%\begin{verbatim}
%\let\H@tagform@\tagform@
%\def\tagform@#1{%
%  \maketag@@@{\hyper@@anchor{\@currentHref}%
%  {(\ignorespaces#1\unskip)}}%
%}
%\def\eqref#1{\textup{\H@tagform@{\ref{#1}}}}
%\end{verbatim}
%
% \subsection{\texorpdfstring{\cs{numberwithin}}{\\numberwithin} patch}
%
%    \begin{macro}{\numberwithin}
%    A appropiate definition of hyperref's companion counter
%    (\cmd{\theH...}) is added for correct link names.
%    \begin{macrocode}
%</package>
%<*check>
\checkpackage{amsmath}[1999/12/14 - 2000/06/06]
\checkcommand\newcommand{\numberwithin}[3][\arabic]{%
  \@ifundefined{c@#2}{\@nocounterr{#2}}{%
    \@ifundefined{c@#3}{\@nocnterr{#3}}{%
      \@addtoreset{#2}{#3}%
      \@xp\xdef\csname the#2\endcsname{%
        \@xp\@nx\csname the#3\endcsname .\@nx#1{#2}%
      }%
    }%
  }%
}%
%</check>
%<*package>
\@ifpackageloaded{amsmath}{%
  \@ifpackagelater{amsmath}{1999/12/14}{%
    \renewcommand*{\numberwithin}[3][\arabic]{%
      \@ifundefined{c@#2}{\@nocounterr{#2}}{%
        \@ifundefined{c@#3}{\@nocnterr{#3}}{%
          \@addtoreset{#2}{#3}%
          \@xp\xdef\csname the#2\endcsname{%
            \@xp\@nx\csname the#3\endcsname .\@nx#1{#2}%
          }%
          \@xp\xdef\csname theH#2\endcsname{%
            \@xp\@nx
            \csname the\@ifundefined{theH#3}{}H#3\endcsname
            .\@nx#1{#2}%
          }%
        }%
      }%
    }%
  }{%
    \Hy@WarningNoLine{%
      \string\numberwithin\space of package `amsmath'
      only fixed\MessageBreak
      for version 2000/06/06 v2.12 or newer%
    }%
  }
}{}
%    \end{macrocode}
%    \end{macro}
%
% \section{Included figures}
% Simply intercept the low level graphics package macro.
%    \begin{macrocode}
\ifHy@figures
  \let\Hy@Gin@setfile\Gin@setfile
  \def\Gin@setfile#1#2#3{%
    \hyperimage{#3}{\Hy@Gin@setfile{#1}{#2}{#3}}%
  }
\fi
\Hy@DisableOption{hyperfigures}
%    \end{macrocode}
%
% \section{hyperindex entries}\label{hyperindex}
% Internal command names are prefixed with \cmd{\HyInd@}.
%
% Hyper-indexing works crudely, by forcing code onto the end of the index
% entry with the \verb+|+ feature; this puts a hyperlink around
% the printed page numbers. It will not proceed if the author has already
% used the \verb+|+ specifier for something like emboldening entries.
% That would make Makeindex fail (cannot have two \verb+|+ specifiers).
% The solution is for the author to use generic coding, and put in
% the requisite |\hyperpage| in his/her own macros along with the boldness.
%
% This section is poor stuff; it's open to all sorts of abuse. Sensible
% large projects will design their own indexing macros any bypass this.
%    \begin{macrocode}
\ifHy@hyperindex
  \def\HyInd@ParenLeft{(}%
  \Hy@nextfalse
  \@ifpackageloaded{multind}{\Hy@nexttrue}{}%
  \@ifpackageloaded{index}{\Hy@nexttrue}{}%
  \ifHy@next
    \let\HyInd@org@wrindex\@wrindex
    \def\@wrindex#1#2{\HyInd@@wrindex{#1}#2||\\}%
    \def\HyInd@@wrindex#1#2|#3|#4\\{%
      \ifx\\#3\\%
        \HyInd@org@wrindex{#1}{#2|hyperpage}%
      \else
        \def\Hy@temp@A{#3}%
        \ifx\Hy@temp@A\HyInd@ParenLeft
          \HyInd@org@wrindex{#1}{#2|#3hyperpage}%
        \else
          \HyInd@org@wrindex{#1}{#2|#3}%
        \fi
      \fi
    }%
  \else
    \def\@wrindex#1{\@@wrindex#1||\\}
    \def\@@wrindex#1|#2|#3\\{%
      \ifx\\#2\\%
        \protected@write\@indexfile{}{%
          \string\indexentry{#1|hyperpage}{\thepage}%
        }%
      \else
        \def\Hy@temp@A{#2}%
        \ifx\Hy@temp@A\HyInd@ParenLeft
          \protected@write\@indexfile{}{%
             \string\indexentry{#1|#2hyperpage}{\thepage}%
          }%
        \else
          \protected@write\@indexfile{}{%
            \string\indexentry{#1|#2}{\thepage}%
          }%
        \fi
      \fi
      \endgroup
      \@esphack
    }%
  \fi
\fi
\Hy@DisableOption{hyperindex}
%    \end{macrocode}
% This again is quite flaky, but allow for the common situation of a
% page range separated by en-rule. We split this into two different
% hyperlinked pages.
%    \begin{macrocode}
\def\hyperpage#1{\@hyperpage#1----\\}
\def\@hyperpage#1--#2--#3\\{%
  \ifx\\#2\\%
    \@commahyperpage{#1}%
  \else
    \hyperlink{page.#1}{#1}--\hyperlink{page.#2}{#2}%
  \fi
}
\def\@commahyperpage#1{\@@commahyperpage#1, ,\\}
\def\@@commahyperpage#1, #2,#3\\{%
  \ifx\\#2\\%
    \hyperlink{page.#1}{#1}%
  \else
    \hyperlink{page.#1}{#1}, \hyperlink{page.#2}{#2}%
  \fi
}
%    \end{macrocode}
%
% \section{Compatibility with foiltex}
%
%    \begin{macrocode}
\@ifclassloaded{foils}{%
  \providecommand*\ext@table{lot}%
  \providecommand*\ext@figure{lof}%
}{}
%    \end{macrocode}
%
% \section{Compatibility with seminar slide package}\label{seminar}
%    This requires \texttt{seminar.bg2}, version 1.6 or later.
%    Contributions by Denis Girou (\Email{denis.girou@idris.fr}).
%    \begin{macrocode}
\@ifclassloaded{seminar}{%
  \Hy@seminarslidestrue\newcommand\theHslide{\arabic{slide}}%
}{%
  \Hy@seminarslidesfalse
}
\@ifpackageloaded{slidesec}{%
  \newcommand\theHslidesection   {\arabic{slidesection}}%
  \newcommand\theHslidesubsection{%
    \theHslidesection.\arabic{slidesubsection}%
  }%
  \def\slide@heading[#1]#2{%
    \H@refstepcounter{slidesection}%
    \@addtoreset{slidesubsection}{slidesection}%
    \addtocontents{los}{%
      \protect\l@slide{\the\c@slidesection}{\ignorespaces#1}%
        {\@SCTR}{slideheading.\theslidesection}%
    }%
    \def\Hy@tempa{#2}%
    \ifx\Hy@tempa\@empty
    \else
      {%
        \edef\@currentlabel{%
          \csname p@slidesection\endcsname\theslidesection
        }%
        \makeslideheading{#2}%
      }%
    \fi
    \gdef\theslideheading{#1}%
    \gdef\theslidesubheading{}%
    \ifHy@bookmarksnumbered
      \def\Hy@slidetitle{\theslidesection\space #1}%
    \else
      \def\Hy@slidetitle{#1}%
    \fi
    \ifHy@hypertexnames
       \ifHy@naturalnames
         \hyper@@anchor{slideheading.\theslidesection}{\relax}%
         \Hy@writebookmark
           {\theslidesection}%
           {\Hy@slidetitle}%
           {slideheading.\theslidesection}%
           {1}%
           {toc}%
       \else
         \hyper@@anchor{slideheading.\theHslidesection}{\relax}%
         \Hy@writebookmark
           {\theslidesection}%
           {\Hy@slidetitle}%
           {slideheading.\theHslidesection}%
           {1}%
           {toc}%
       \fi
    \else
      \Hy@GlobalStepCount\Hy@linkcounter
      \hyper@@anchor{slideheading.\the\Hy@linkcounter}{\relax}%
      \Hy@writebookmark
        {\theslidesection}%
        {\Hy@slidetitle}%
        {slideheading.\the\Hy@linkcounter}%
        {1}%
        {toc}%
    \fi
  }%
  \def\slide@subheading[#1]#2{%
    \H@refstepcounter{slidesubsection}%
    \addtocontents{los}{%
      \protect\l@subslide{\the\c@slidesubsection}{\ignorespaces#1}%
        {\@SCTR}{slideheading.\theslidesubsection}%
    }%
    \def\Hy@tempa{#2}%
    \ifx\Hy@tempa\@empty
    \else
      {%
        \edef\@currentlabel{%
          \csname p@slidesubsection\endcsname\theslidesubsection
        }%
        \makeslidesubheading{#2}%
      }%
    \fi
    \gdef\theslidesubheading{#1}%
    \ifHy@bookmarksnumbered
      \def\Hy@slidetitle{\theslidesubsection\space #1}%
    \else
      \def\Hy@slidetitle{#1}%
    \fi
    \ifHy@hypertexnames
      \ifHy@naturalnames
        \hyper@@anchor{slideheading.\theslidesubsection}{\relax}%
        \Hy@writebookmark
          {\theslidesubsection}%
          {\Hy@slidetitle}%
          {slideheading.\theslidesubsection}%
          {2}%
          {toc}%
      \else
        \hyper@@anchor{slideheading.\theHslidesubsection}{\relax}%
        \Hy@writebookmark
          {\theslidesubsection}%
          {\Hy@slidetitle}%
          {slideheading.\theHslidesubsection}%
          {2}%
          {toc}%
      \fi
    \else
      \Hy@GlobalStepCount\Hy@linkcounter
      \hyper@@anchor{slideheading.\the\Hy@linkcounter}{\relax}%
      \Hy@writebookmark
        {\theslidesubsection}%
        {\Hy@slidetitle}%
        {slideheading.\the\Hy@linkcounter}%
        {1}%
        {toc}%
    \fi
  }%
  \providecommand*{\listslidename}{List of Slides}%
  \def\listofslides{%
    \section*{%
      \listslidename
      \@mkboth{%
        \expandafter\MakeUppercase\listslidename
      }{%
        \expandafter\MakeUppercase\listslidename
      }%
    }%
    \def\l@slide##1##2##3##4{%
      \slide@undottedcline{%
        \slidenumberline{##3}{\hyperlink{##4}{##2}}%
      }{}%
    }%
    \let\l@subslide\l@slide
    \@startlos
  }%
  \def\slide@contents{%
    \def\l@slide##1##2##3##4{%
      \slide@cline{\slidenumberline{##3}{\hyperlink{##4}{##2}}}{##3}%
    }%
    \let\l@subslide\@gobblefour
    \@startlos
  }%
  \def\Slide@contents{%
    \def\l@slide##1##2##3##4{%
      \ifcase\lslide@flag
        \message{##1 ** \the\c@slidesection}%
        \ifnum##1>\c@slidesection
          \def\lslide@flag{1}%
          {%
            \large
            \slide@cline{%
              \slidenumberline{$\Rightarrow\bullet$}%
                {\hyperlink{##4}{##2}}%
            }{##3}%
          }%
        \else
          {%
            \large
            \slide@cline{%
              \slidenumberline{$\surd\;\bullet$}%
                {\hyperlink{##4}{##2}}%
            }{##3}%
          }%
        \fi
      \or
        \def\lslide@flag{2}%
        {%
          \large
          \slide@cline{%
            \slidenumberline{$\bullet$}%
              {\hyperlink{##4}{##2}}%
          }{##3}%
        }%
      \or
        {%
          \large
          \slide@cline{%
            \slidenumberline{$\bullet$}%
             {\hyperlink{##4}{##2}}%
          }{##3}%
        }%
      \fi
    }%
    \def\l@subslide##1##2##3##4{%
      \ifnum\lslide@flag=1
        \@undottedtocline{2}{3.8em}{3.2em}{\hyperlink{##4}{##2}}{}%
      \fi
    }%
    \def\lslide@flag{0}%
    \@startlos
  }%
}{}
%    \end{macrocode}
% This breaks TeX4ht, so leave it to last.
% Emend |\@setref| to put out a hypertext link as well as its
% normal text (which is used as an anchor).
% (|\endinput| have to be on the same line like |\fi|, or you
% have to use |\expandafter| before.)
%    \begin{macrocode}
\ifHy@texht\endinput\fi
\let\real@setref\@setref
\def\@setref#1#2#3{% csname, extract group, refname
  \ifx#1\relax
    \protect\G@refundefinedtrue
    \nfss@text{\reset@font\bfseries ??}%
    \@latex@warning{%
      Reference `#3' on page \thepage \space undefined%
    }%
  \else
    \hyper@@link
      {\expandafter\@fifthoffive#1}%
      {\expandafter\@fourthoffive#1\@empty\@empty}%
      {\expandafter#2#1\@empty\@empty\null}%
  \fi
}
\def\@pagesetref#1#2#3{% csname, extract macro, ref
  \ifx#1\relax
    \protect\G@refundefinedtrue
    \nfss@text{\reset@font\bfseries ??}%
    \@latex@warning{%
      Reference `#3' on page \thepage \space undefined%
    }%
  \else
    \protect\hyper@@link
      {\expandafter\@fifthoffive#1}%
      {page.\expandafter\@secondoffive#1}%
      {\expandafter\@secondoffive#1}%
  \fi
}
%    \end{macrocode}
%
%    \begin{macrocode}
%</package>
%<*check>
\checklatex
\checkcommand\def\@setref#1#2#3{%
  \ifx#1\relax
    \protect\G@refundefinedtrue
    \nfss@text{\reset@font\bfseries ??}%
    \@latex@warning{%
      Reference `#3' on page \thepage\space undefined%
    }%
  \else
    \expandafter#2#1\null
  \fi
}
%</check>
%<*package>
%    \end{macrocode}
%
% Now some extended referencing. |\ref*| and |\pageref*| are not linked,
% and |\autoref| prefixes with a tag based on the type.
%    \begin{macrocode}
\def\@refstar#1{%
  \@safe@activestrue
  \expandafter\real@setref\csname r@#1\endcsname\@firstoffive{#1}%
  \@safe@activesfalse
}
\def\@pagerefstar#1{%
  \@safe@activestrue
  \expandafter\real@setref\csname r@#1\endcsname\@secondoffive{#1}%
  \@safe@activesfalse
}
\DeclareRobustCommand\autoref[1]{%
  \expandafter\auto@setref\csname r@#1\endcsname\@firstoffive{#1}%
}
\def\auto@setref#1#2#3{% csname, extract group, refname
  \@safe@activestrue
  \ifx#1\relax
    \protect\G@refundefinedtrue
    \nfss@text{\reset@font\bfseries ??}%
    \@latex@warning{%
      Reference `#3' on page \thepage \space undefined%
    }%
  \else
    \edef\@thisref{\expandafter\@fourthoffive#1\@empty\@empty}%
    \expandafter\test@reftype\@thisref\\%
    \hyper@@link
      {\expandafter\@fifthoffive#1}%
      {\expandafter\@fourthoffive#1\@empty\@empty}%
      {\@currentHtag\expandafter#2#1\@empty\@empty\null}%
  \fi
  \@safe@activesfalse
}
\def\test@reftype#1.#2\\{%
  \@ifundefined{#1autorefname}{%
    \@ifundefined{#1name}{%
      \def\@currentHtag{}%
      \@latex@warning{no tag name for #1 at \the\inputlineno}%
    }{%
      \def\@currentHtag{\csname#1name\endcsname~}%
    }%
  }{%
    \def\@currentHtag{\csname#1autorefname\endcsname~}%
  }%
}
\def\@currentHtag{}
%    \end{macrocode}
%
%    Defaults for the names that \cmd{\autoref} uses.
%    \begin{macrocode}
\providecommand\AMSautorefname{\equationautorefname}
\providecommand\Hfootnoteautorefname{\footnoteautorefname}
\providecommand\Itemautorefname{\itemautorefname}
\providecommand\equationautorefname{Equation}
\providecommand\footnoteautorefname{footnote}
\providecommand\itemautorefname{item}
\providecommand\figureautorefname{Figure}
\providecommand\tableautorefname{Table}
\providecommand\partautorefname{Part}
\providecommand\appendixautorefname{Appendix}
\providecommand\chapterautorefname{chapter}
\providecommand\sectionautorefname{section}
\providecommand\subsectionautorefname{subsection}
\providecommand\subsubsectionautorefname{subsubsection}
\providecommand\paragraphautorefname{paragraph}
\providecommand\subparagraphautorefname{subparagraph}
\providecommand\FancyVerbLineautorefname{line}
\providecommand\theoremautorefname{Theorem}
%    \end{macrocode}
%
%    \begin{macrocode}
%</package>
%<*pdftex>
%    \end{macrocode}
% \section{Configuration files}
% \subsection{pdftex}
%
%    \begin{macrocode}
%    \end{macrocode}
% This driver is for Han The Thanh's \TeX{} variant
% which produces PDF directly. This has new primitives
% to do PDF things, which usually translate almost directly to
% PDF code, so there is a lot of flexibility which we do not at
% present harness.
%
% First, allow for some changes and additions to pdftex  syntax:
%    \begin{macrocode}
\def\setpdflinkmargin#1{\pdflinkmargin#1}
\ifx\pdfstartlink\@undefined% less than version 14
  \let\pdfstartlink\pdfannotlink
  \let\pdflinkmargin\@tempdima
  \let\pdfxform\pdfform
  \let\pdflastxform\pdflastform
  \let\pdfrefxform\pdfrefform
\else
  \pdflinkmargin1pt
\fi
%    \end{macrocode}
% First set up the default linking
%    \begin{macrocode}
\providecommand\@pdfview{XYZ}
%    \end{macrocode}
% First define the anchors:
%    \begin{macrocode}
\def\new@pdflink#1{%
  \ifHy@verbose
    \typeout{pdftex: define anchor at line \the\inputlineno: #1}%
  \fi
  \Hy@SaveLastskip
  \pdfdest name {#1}\@pdfview
  \Hy@RestoreLastskip
}
\let\pdf@endanchor\@empty
%    \end{macrocode}
%
% Now the links; the interesting part here is the set of attributes
% which define how the link looks. We probably want to add a border
% and color it, but there are other choices. This directly translates
% to PDF code, so consult the manual for how to change this. We will
% add an interface at some point.
%    \begin{macrocode}
\providecommand\@pdfborder{0 0 1}
\def\Hy@undefinedname{UNDEFINED}
\def\find@pdflink#1#2{%
  \leavevmode
  \protected@edef\Hy@testname{#2}%
  \ifx\Hy@testname\@empty
    \let\Hy@testname\Hy@undefinedname
  \fi
  \pdfstartlink
    attr{%
      /Border[\@pdfborder]%
      /H\@pdfhighlight
      /C[\CurrentBorderColor]%
    }%
    goto name {\Hy@testname}%
  \Hy@colorlink{\csname @#1color\endcsname}%
}
\def\close@pdflink{\Hy@endcolorlink\pdfendlink}
\def\hyper@anchor#1{\new@pdflink{#1}\anchor@spot\pdf@endanchor}
\def\hyper@anchorstart#1{\new@pdflink{#1}\Hy@activeanchortrue}
\def\hyper@anchorend{\pdf@endanchor\Hy@activeanchorfalse}
\def\hyper@linkstart#1#2{%
  \edef\CurrentBorderColor{\csname @#1bordercolor\endcsname}%
  \find@pdflink{#1}{#2}}
\def\hyper@linkend{\close@pdflink}
\def\hyper@link#1#2#3{%
  \edef\CurrentBorderColor{\csname @#1bordercolor\endcsname}%
  \find@pdflink{#1}{#2}#3\close@pdflink
}
\def\CurrentBorderColor{\@linkbordercolor}
\def\hyper@linkurl#1#2{%
  \bgroup
    \hyper@chars
    \leavevmode
    \pdfstartlink
      attr{%
        /Border[\@pdfborder]%
        /H\@pdfhighlight
        /C[\@urlbordercolor]%
      }%
      user{%
       /Subtype/Link%
       /A<<%
         /Type/Action%
         /S/URI%
         /URI(#2)%
       >>%
      }%
    \Hy@colorlink{\@urlcolor}#1%
    \close@pdflink
  \egroup
}
\def\hyper@linkfile#1#2#3{% anchor text, filename, linkname
  \bgroup
    \leavevmode
    \pdfstartlink
      attr{%
        /Border[\@pdfborder]%
        /H\@pdfhighlight
        /C[\@filebordercolor]%
      }%
      user {%
        /Subtype/Link%
        /A<<%
          /F(#2)%
          /S/GoToR%
          \ifHy@newwindow /NewWindow true \fi
%    \end{macrocode}
% If |#3| is empty, page 0; if its a number, Page number, otherwise
% a named destination.
% \begin{verbatim}
% \afterassignment\xxx\count@=0\foo!%
%
%\def\xxx#1!{%
%  \ifx\xxx#1\xxx
%     foo was an integer
%  \else
%     it wasnt
%  \fi}
% \end{verbatim}
%    \begin{macrocode}
          \ifx\\#3\\%
            /D[0 \@pdfstartview]%
          \else
            /D(#3)%
          \fi
        >>%
      }%
    \Hy@colorlink{\@filecolor}#1%
    \close@pdflink
  \egroup
}
\def\@hyper@launch run:#1\\#2#3{% filename, anchor text linkname
  \bgroup
    \leavevmode
    \pdfstartlink
      attr{%
        /Border[\@pdfborder]%
        /H\@pdfhighlight
        /C[\@runbordercolor]%
      }%
      user {%
        /Subtype/Link%
        /A<<%
          /F(#1)%
          /S/Launch%
          \ifHy@newwindow /NewWindow true \fi
          \ifx\\#3\\%
          \else
            /Win<</P(#3)/F(#1)>>%
          \fi
        >>%
      }%
    \Hy@colorlink{\@filecolor}#2%
    \close@pdflink
  \egroup
}
%    \end{macrocode}
%    \begin{macro}{\@pdfproducer}
%    \begin{macrocode}
\def\@pdfproducer{pdfTeX}
\ifx\eTeXversion\@undefined
\else
  \ifx\eTeXversion\relax
  \else
    \ifnum\eTeXversion>0 %
      \def\@pdfproducer{pdfeTeX}
    \fi
  \fi
\fi
\ifx\pdftexversion\@undefined
\else
  \ifnum\pdftexversion<100 %
    \edef\@pdfproducer{%
      \@pdfproducer
      \the\pdftexversion.\pdftexrevision
    }
  \else
    \edef\@pdfproducer{%
      \@pdfproducer-%
      \expandafter\@car\the\pdftexversion\@empty\@nil.%
      \expandafter\@cdr\the\pdftexversion\@empty\@nil
      \pdftexrevision
    }
  \fi
\fi
%    \end{macrocode}
%    \end{macro}
%    \begin{macro}{\PDF@SetupDox}
%    \begin{macrocode}
\def\PDF@SetupDoc{%
  \ifx\@pdfpagescrop\@empty
  \else
    \edef\process@me{%
      \pdfpagesattr={%
        /CropBox[\@pdfpagescrop]%
        \expandafter\ifx\expandafter\\\the\pdfpagesattr\\%
        \else
          ^^J\the\pdfpagesattr
        \fi
      }%
    }%
    \process@me
  \fi
  \pdfcatalog{%
    /PageMode \@pdfpagemode
    /URI<</Base(\@baseurl)>>
  }
  \ifx\@pdfstartview\@empty
  \else
    openaction goto page \@pdfstartpage {\@pdfstartview}%
  \fi
  \pdfcatalog{
    /ViewerPreferences<<%
      \ifHy@toolbar\else /HideToolbar true \fi
      \ifHy@menubar\else /HideMenubar true \fi
      \ifHy@windowui\else /HideWindowUI true \fi
      \ifHy@fitwindow /FitWindow true \fi
      \ifHy@centerwindow /CenterWindow true \fi
    >>
    \ifx\pdf@pagelayout\@empty
    \else
      /PageLayout/\pdf@pagelayout\space
    \fi
  }%
}
%    \end{macrocode}
%    \end{macro}
%    \begin{macro}{\PDF@FinishDoc}
%    \begin{macrocode}
\def\PDF@FinishDoc{%
  \Hy@UseMaketitleInfos
  \pdfinfo{%
    /Author(\@pdfauthor)%
    /Title(\@pdftitle)%
    /Subject(\@pdfsubject)%
    /Creator(\@pdfcreator)%
    /Producer(\@pdfproducer)%
    /Keywords(\@pdfkeywords)%
  }%
  \Hy@DisableOption{pdfauthor}%
  \Hy@DisableOption{pdftitle}%
  \Hy@DisableOption{pdfsubject}%
  \Hy@DisableOption{pdfcreator}%
  \Hy@DisableOption{pdfproducer}%
  \Hy@DisableOption{pdfkeywords}%
}
%    \end{macrocode}
%    \end{macro}
%    \begin{macro}{\hyper@pagetransition}
%    \cmd{\@pdfpagetransition} is initialized with \cmd{\relax}. So
%    it indicates, if option pdfpagetransition is used. First previous
%    |/Trans| entries are removed. If a new |/Trans| key exists, it is
%    appended to \cmd{\pdfpageattr}.
%    \begin{macrocode}
\def\hyper@pagetransition{%
  \ifx\@pdfpagetransition\relax
  \else
    \expandafter\Hy@RemoveTransPageAttr\the\pdfpageattr^^J/Trans{}>>\END
    \ifx\@pdfpagetransition\@empty
    \else
      \edef\@processme{%
        \global\pdfpageattr{%
          \the\pdfpageattr
          ^^J/Trans << /S /\@pdfpagetransition\space >>%
        }%
      }%
      \@processme
    \fi
  \fi
}
%    \end{macrocode}
%    \end{macro}
%    \begin{macro}{\Hy@RemoveTransPageAttr}
%    Macro \cmd{\Hy@RemoveTransPageAttr} removes a |/Trans|
%    entry from \cmd{\pdfpageattr}. It is called with
%    the end marker |^^J/Trans{}>>\END|. The trick is the
%    empty group that does not appear in legal
%    \cmd{\pdfpageattr} code. It appears in argument
%    |#2| and shows, whether the parameter text
%    catches a really |/Trans| object or the end marker.
%    \begin{macrocode}
\gdef\Hy@RemoveTransPageAttr#1^^J/Trans#2#3>>#4\END{%
  \ifx\\#2\\%
    \global\pdfpageattr{#1}%
  \else
    \Hy@RemoveTransPageAttr#1#4\END
  \fi
}
%    \end{macrocode}
%    \end{macro}
%
%    \begin{macro}{\hyper@pageduration}
%    \cmd{\@pdfpageduration} is initialized with \cmd{\relax}. So
%    it indicates, if option pdfpageduration is used. First previous
%    |/Dur| entries are removed. If a new |/Dur| key exists, it is
%    appended to \cmd{\pdfpageattr}.
%    \begin{macrocode}
\def\hyper@pageduration{%
  \ifx\@pdfpageduration\relax
  \else
    \expandafter\Hy@RemoveDurPageAttr\the\pdfpageattr^^J/Dur{} \END
    \ifx\@pdfpageduration\@empty
    \else
      \edef\@processme{%
        \global\pdfpageattr{%
          \the\pdfpageattr
          ^^J/Dur \@pdfpageduration\space
        }%
      }%
      \@processme
    \fi
  \fi
}
%    \end{macrocode}
%    \end{macro}
%    \begin{macro}{\Hy@RemoveDurPageAttr}
%    Macro \cmd{\Hy@RemoveDurPageAttr} removes a |/Dur|
%    entry from \cmd{\pdfpageattr}. It is called with
%    the end marker |^^J/Dur{} \END|. The trick is the
%    empty group that does not appear in legal
%    \cmd{\pdfpageattr} code. It appears in argument
%    |#2| and shows, whether the parameter text
%    catches a really |/Dur| object or the end marker.
%    \begin{macrocode}
\gdef\Hy@RemoveDurPageAttr#1^^J/Dur#2#3 #4\END{%
  \ifx\\#2\\%
    \global\pdfpageattr{#1}%
  \else
    \Hy@RemoveDurPageAttr#1#4\END
  \fi
}
%    \end{macrocode}
%    \end{macro}
%
%    \begin{macro}{\hyper@pagehidden}
%    The boolean value of the key |/Hid| is stored in switch
%    \cmd{\ifHy@pdfpagehidden}.
%    First previous |/Hid| entries are removed, then the new
%    one is appended, if the value is true (the PDF default
%    is false).
%    \begin{macrocode}
\def\hyper@pagehidden{%
  \ifHy@useHidKey
    \expandafter\Hy@RemoveHidPageAttr\the\pdfpageattr^^J/Hid{} \END
    \ifHy@pdfpagehidden
      \edef\@processme{%
        \global\pdfpageattr{%
          \the\pdfpageattr
          ^^J/Hid true % SPACE
        }%
      }%
      \@processme
    \fi
  \fi
}
%    \end{macrocode}
%    \end{macro}
%    \begin{macro}{\Hy@RemoveHidPageAttr}
%    Macro \cmd{\Hy@RemoveHidPageAttr} removes a |/Hid|
%    entry from \cmd{\pdfpageattr}. It is called with
%    the end marker |^^J/Hid{} \END|. The trick is the
%    empty group that does not appear in legal
%    \cmd{\pdfpageattr} code. It appears in argument
%    |#2| and shows, whether the parameter text
%    catches a really |/Hid| object or the end marker.
%    \begin{macrocode}
\gdef\Hy@RemoveHidPageAttr#1^^J/Hid#2#3 #4\END{%
  \ifx\\#2\\%
    \global\pdfpageattr{#1}%
  \else
    \Hy@RemoveHidPageAttr#1#4\END
  \fi
}
%    \end{macrocode}
%    \end{macro}
%
% Let us explicitly turn on PDF generation; they can reverse
% this decision in the document, but since we are emitting PDF
% links anyway, we \emph{must} be in PDF mode.
%    \begin{macrocode}
\pdfoutput=1
\pdfcompresslevel=9
\AtBeginDocument{%
  \@ifclassloaded{seminar}{%
    \setlength{\pdfhorigin}{1truein}%
    \setlength{\pdfvorigin}{1truein}%
    \ifportrait
      \ifdim\paperwidth=\z@
      \else
        \setlength{\pdfpagewidth}{\strip@pt\paperwidth truept}%
      \fi
      \ifdim\paperheight=\z@
      \else
        \setlength{\pdfpageheight}{\strip@pt\paperheight truept}%
      \fi
    \else
      \ifdim\paperheight=\z@
      \else
        \setlength{\pdfpagewidth}{\strip@pt\paperheight truept}%
      \fi
      \ifdim\paperwidth=\z@
      \else
        \setlength{\pdfpageheight}{\strip@pt\paperwidth truept}%
      \fi
    \fi
  }{%
    \@ifundefined{stockwidth}{%
      \ifdim\paperwidth=\z@
      \else
        \setlength{\pdfpagewidth}{\paperwidth}%
      \fi
      \ifdim\paperheight=\z@
      \else
        \setlength{\pdfpageheight}{\paperheight}%
      \fi
    }{%
      \ifdim\stockwidth=\z@
      \else
        \setlength{\pdfpagewidth}{\stockwidth}%
      \fi
      \ifdim\stockheight=\z@
      \else
        \setlength{\pdfpageheight}{\stockheight}%
      \fi
    }%
  }%
}
\def\Acrobatmenu#1#2{%
  \leavevmode
  \pdfstartlink
    attr{%
      /Border [\@pdfborder]
      /H \@pdfhighlight\space
      /C [\@menubordercolor]%
    }%
    user{
      /Subtype /Link
      /A <<
        /S /Named /N /#1
      >>
    }%
  \Hy@colorlink{\@menucolor}#2\close@pdflink
}
%</pdftex>
%<*hypertex>
%    \end{macrocode}
% \subsection{hypertex}
% The Hyper\TeX\ specification (this is
% borrowed from an article by Arthur Smith)
% says that conformant viewers/translators
% must recognize the following set of |\special| commands:
% \begin{description}
% \item[href:] |html:<a href = "href_string">|
% \item[name:] |html:<a name = "name_string">|
% \item[end:] |html:</a>|
% \item[image:] |html:<img src = "href_string">|
% \item[base\_name:] |html:<base href = "href_string">|
% \end{description}
%
% The \emph{href}, \emph{name} and \emph{end} commands are used to do
% the basic hypertext operations of establishing links between sections
% of documents. The \emph{image} command is intended (as with current
% html viewers) to place an image of arbitrary graphical
% format on the page in the current location.  The \emph{base\_name}
% command is be used to communicate to the \emph{dvi} viewer the full (URL)
% location of the current document so that
% files specified by relative URL's may be retrieved correctly.
%
% The \emph{href} and \emph{name} commands must be paired with an
% \emph{end} command later in
% the \TeX{} file --- the \TeX{} commands between the two ends of a pair
% form an \emph{anchor} in the document. In the case of an \emph{href}
% command, the \emph{anchor} is to be highlighted in the
% \emph{dvi} viewer, and
% when clicked on will cause the scene to shift to the destination
% specified by \emph{href\_string}. The \emph{anchor} associated with a
% name command represents a possible location to which other hypertext
% links may refer, either as local references (of the form
% \texttt{href="\#name\_string"} with the \emph{name\_string}
% identical to the one in the name command) or as part of a URL (of the
% form \emph{URL\#name\_string}). Here \emph{href\_string} is a valid
% URL or local identifier, while name\_string could be any string at
% all: the only caveat is that `|"|' characters should be escaped with a
% backslash (|\|), and if it looks like a URL name it may cause
% problems.
%
%    \begin{macrocode}
\def\PDF@FinishDoc{}
\def\PDF@SetupDoc{%
  \ifx\@baseurl\@empty\else
    \special{html:<base href="\@baseurl">}%
  \fi
}
\def\hyper@anchor#1{%
  \Hy@SaveLastskip
  \begingroup
    \let\protect=\string
    \hyper@chars
    \special{html:<a name=\hyper@quote #1\hyper@quote>}%
  \endgroup
  \Hy@activeanchortrue
  \Hy@colorlink{\@anchorcolor}\anchor@spot\Hy@endcolorlink
  \special{html:</a>}%
  \Hy@activeanchorfalse
  \Hy@RestoreLastskip
}
\def\hyper@anchorstart#1{%
  \Hy@SaveLastskip
  \begingroup
    \hyper@chars
    \special{html:<a name=\hyper@quote#1\hyper@quote>}%
  \endgroup
  \Hy@activeanchortrue
}
\def\hyper@anchorend{%
  \special{html:</a>}%
  \Hy@activeanchorfalse
  \Hy@RestoreLastskip
}
\def\@urltype{url}
\def\hyper@linkstart#1#2{%
  \Hy@colorlink{\csname @#1color\endcsname}%
  \def\Hy@tempa{#1}%
  \ifx\Hy@tempa\@urltype
    \special{html:<a href=\hyper@quote#2\hyper@quote>}%
  \else
    \begingroup
      \hyper@chars
      \special{html:<a href=\hyper@quote\##2\hyper@quote>}%
    \endgroup
  \fi
}
\def\hyper@linkend{%
  \special{html:</a>}%
  \Hy@endcolorlink
}
\def\hyper@linkfile#1#2#3{%
  \hyper@linkurl{#1}{file:#2\ifx\\#3\\\else\##3\fi}%
}
\def\hyper@linkurl#1#2{%
%    \end{macrocode}
% If we want to raise up the final link |\special|, we need to
% get its height; ask me why \LaTeX\ constructs make this totally
% foul up, and make us revert to basic \TeX. I do not know.
%    \begin{macrocode}
  \leavevmode
  \ifHy@raiselinks
    \setbox\@tempboxa=\color@hbox #1\color@endbox
    \@linkdim\dp\@tempboxa
    \lower\@linkdim\hbox{%
      \hyper@chars
      \special{html:<a href=\hyper@quote#2\hyper@quote>}%
    }%
    \Hy@colorlink{\@urlcolor}#1%
    \@linkdim\ht\@tempboxa
%    \end{macrocode}
% Because of the interaction with the dvihps processor, we have to subtract a
% little from the height. This is not clean, or checked. Check with Mark
% Doyle about what gives here. It may not be needed with
% the new dvips (Jan 1997).
%    \begin{macrocode}
    \advance\@linkdim by -6.5\p@
    \raise\@linkdim\hbox{\special{html:</a>}}%
    \Hy@endcolorlink
  \else
    \begingroup
      \hyper@chars
      \special{html:<a href=\hyper@quote#2\hyper@quote>}%
      \Hy@colorlink{\@urlcolor}#1%
      \special{html:</a>}%
      \Hy@endcolorlink
    \endgroup
  \fi
}
\def\hyper@link#1#2#3{%
  \hyper@linkurl{#3}{\##2}%
}
%    \end{macrocode}
%
%    \begin{macrocode}
\def\hyper@image#1#2{%
  \begingroup
    \hyper@chars
    \special{html:<img src=\hyper@quote#1\hyper@quote>}%
  \endgroup
}
%</hypertex>
%<*dviwindo>
%    \end{macrocode}
% \subsection{dviwindo}
% [This was developed by David Carlisle].
% Within a file dviwindo hyperlinking is used, for external
% URL's a call to |\wwwbrowser| is made. (You can define
% this command before or after loading the hyperref package
% if the default |c:/netscape/netscape| is not suitable)
% Dviwindo could in fact handle external links to dvi files on
% the same machine without calling a web browser, but that would
% mean parsing the URL to recognise such, and this is currently
% not done.
%
% This was more or less blindly copied from the hypertex cfg.
% For dviwindo,  \LaTeX{} must specify the size of the active area
% for links. For some hooks this information is available
% but for some, the start and end of the link are
% specified separately in which case a fixed size area
% of 10000000sp wide by |\baselineskip| high is used.
%    \begin{macrocode}
\providecommand\wwwbrowser{c:\string\netscape\string\netscape}
\def\hyper@anchor#1{%
  \Hy@SaveLastskip
  \begingroup
    \let\protect=\string
    \special{mark: #1}%
  \endgroup
  \Hy@activeanchortrue
  \Hy@colorlink{\@anchorcolor}\anchor@spot\Hy@endcolorlink
  \Hy@activeanchorfalse
  \Hy@RestoreLastskip
}
\def\hyper@anchorstart#1{%
  \Hy@SaveLastskip
  \special{mark: #1}%
  \Hy@activeanchortrue
}
\def\hyper@anchorend{%
  \Hy@activeanchorfalse
  \Hy@RestoreLastskip
}
\def\hyper@linkstart#1#2{%
  \Hy@colorlink{\csname @#1color\endcsname}%
  \special{button:
    10000000
    \number\baselineskip\space
    #2%
  }%
}
\def\hyper@linkend{%
  \Hy@endcolorlink
}
\def\hyper@link#1#2#3{%
  \setbox\@tempboxa=\color@hbox #3\color@endbox
  \leavevmode
  \ifHy@raiselinks
    \@linkdim\dp\@tempboxa
    \lower\@linkdim\hbox{%
      \special{button:
        \number\wd\@tempboxa\space
        \number\ht\@tempboxa\space
        #2%
      }%
      \Hy@colorlink{\csname @#1color\endcsname}#3%
      \Hy@endcolorlink
    }%
    \@linkdim\ht\@tempboxa
    \advance\@linkdim by -6.5\p@
    \raise\@linkdim\hbox{}%
  \else
    \special{button:
      \number\wd\@tempboxa\space
      \number\ht\@tempboxa\space
      #2%
    }%
    \Hy@colorlink{\csname @#1color\endcsname}#3\Hy@endcolorlink
  \fi
}
\def\hyper@linkurl#1#2{%
  \begingroup
    \hyper@chars
    \leavevmode
    \setbox\@tempboxa=\color@hbox #1\color@endbox
    \ifHy@raiselinks
      \@linkdim\dp\@tempboxa
      \lower\@linkdim\hbox{%
        \special{button:
          \number\wd\@tempboxa\space
          \number\ht\@tempboxa\space
          launch: \wwwbrowser\space
          #2%
        }%
        \Hy@colorlink{\@urlcolor}#1\Hy@endcolorlink
      }%
      \@linkdim\ht\@tempboxa
      \advance\@linkdim by -6.5\p@
      \raise\@linkdim\hbox{}%
    \else
      \special{button:
        \number\wd\@tempboxa\space
        \number\ht\@tempboxa\space
        launch: \wwwbrowser\space
        #2%
      }%
      \Hy@colorlink{\@urlcolor}#1\Hy@endcolorlink
    \fi
  \endgroup
}
\def\hyper@linkfile#1#2#3{%
  \begingroup
    \hyper@chars
    \leavevmode
    \setbox\@tempboxa=\color@hbox #1\color@endbox
    \ifHy@raiselinks
      \@linkdim\dp\@tempboxa
      \lower\@linkdim\hbox{%
        \special{button:
          \number\wd\@tempboxa\space
          \number\ht\@tempboxa\space
          #3,
          file: #2%
        }%
        \Hy@colorlink{\@filecolor}#1\Hy@endcolorlink
      }%
      \@linkdim\ht\@tempboxa
      \advance\@linkdim by -6.5\p@
      \raise\@linkdim\hbox{}%
    \else
      \special{button:
        \number\wd\@tempboxa\space
        \number\ht\@tempboxa\space
        #3,
        file: #2
      }%
      \Hy@colorlink{\@filecolor}#1\Hy@endcolorlink
    \fi
  \endgroup
}
\def\@pdfproducer{dviwindo + Distiller}
\def\PDF@FinishDoc{%
  \Hy@UseMaketitleInfos
  \special{PDF: Keywords \@pdfkeywords}%
  \special{PDF: Title \@pdftitle}%
  \special{PDF: Creator \@pdfcreator}%
  \special{PDF: Author \@pdfauthor}%
  \special{PDF: Producer \@pdfproducer}%
  \special{PDF: Subject \@pdfsubject}%
  \Hy@DisableOption{pdfauthor}%
  \Hy@DisableOption{pdftitle}%
  \Hy@DisableOption{pdfsubject}%
  \Hy@DisableOption{pdfcreator}%
  \Hy@DisableOption{pdfproducer}%
  \Hy@DisableOption{pdfkeywords}%
}
\def\PDF@SetupDoc{%
  \ifx\@baseurl\@empty\else
    \special{PDF: Base \@baseurl}%
  \fi
  \ifx\@pdfpagescrop\@empty\else
    \special{PDF: BBox \@pdfpagescrop}%
  \fi
  \pdfmark{pdfmark=/DOCVIEW,
    Page=\@pdfstartpage,
    View=\@pdfstartview,
    PageMode=\@pdfpagemode
  }%
  \ifx\@pdfpagescrop\@empty
  \else
    \pdfmark{pdfmark=/PAGES,CropBox=\@pdfpagescrop}%
  \fi
  \pdfmark{pdfmark=/PUT,
    Raw={%
      \string{Catalog\string} <<
        /ViewerPreferences <<
          \ifHy@toolbar\else /HideToolbar true \fi
          \ifHy@menubar\else /HideMenubar true \fi
          \ifHy@windowui\else /HideWindowUI true \fi
          \ifHy@fitwindow /FitWindow true \fi
          \ifHy@centerwindow /CenterWindow true \fi
        >>
        \ifx\pdf@pagelayout\@empty
        \else
          /PageLayout /\pdf@pagelayout\space
        \fi
      >>%
    }%
  }%
}
%</dviwindo>
%<*dvipdfm>
%    \end{macrocode}
% \subsection{dvipdfm dvi to PDF converter}
% Provided by Mark Wicks (mwicks@kettering.edu)
%    \begin{macrocode}
\newsavebox{\pdfm@box}
\def\@pdfm@mark#1{\special{pdf:#1}}
\def\@pdfm@dest#1{%
  \Hy@SaveLastskip
  \@pdfm@mark{dest (#1) [@thispage /\@pdfview\space @xpos @ypos]}%
  \Hy@RestoreLastskip
}
\providecommand\@pdfview{XYZ}
\providecommand\@pdfborder{0 0 1}
\def\hyper@anchor#1{%
  \@pdfm@dest{#1}%
}
\def\hyper@anchorstart#1{%
  \Hy@activeanchortrue
  \@pdfm@dest{#1}%
}
\def\hyper@anchorend{%
  \Hy@activeanchorfalse
}
\def\Hy@undefinedname{UNDEFINED}
\def\hyper@linkstart#1#2{%
  \protected@edef\Hy@testname{#2}%
  \ifx\Hy@testname\@empty
    \let\Hy@testname\Hy@undefinedname
  \fi
  \@pdfm@mark{%
    bann <<
      /Type /Annot
      /Subtype /Link
      /Border [\@pdfborder]
      /C [\csname @#1bordercolor\endcsname]
      /A <<
        /S /GoTo
        /D (\Hy@testname)
      >>
    >>
  }%
  \Hy@colorlink{\csname @#1color\endcsname}%
}
\def\hyper@linkend{%
  \Hy@endcolorlink
  \@pdfm@mark{eann}%
}
\def\hyper@link#1#2#3{%
  \hyper@linkstart{#1}{#2}#3\hyper@linkend
}
\def\hyper@linkfile#1#2#3{%
  \@pdfm@mark{%
    bann <<
      /Type /Annot
      /Subtype /Link
      /Border [\@pdfborder]
      /C [\@filebordercolor]
      /A <<
        /S /GoToR
        /F (#2)
        /D \ifx\\#3\\[0 \@pdfstartview]\else(#3)\fi\space
        \ifHy@newwindow /NewWindow true \fi
      >>
    >>%
  }%
  \Hy@colorlink{\@filecolor}#1\Hy@endcolorlink
  \@pdfm@mark{eann}%
}
\def\@hyper@launch run:#1\\#2#3{% filename, anchor text linkname
  \@pdfm@mark{%
    bann <<
      /Type /Annot
      /Subtype /Link
      /Border [\@pdfborder]
      /C [\@filebordercolor]
      /A <<
        /S /Launch
        /F (#1)
        \ifHy@newwindow /NewWindow true \fi
        \ifx\\#3\\%
        \else
          /Win << /P (#3) /F (#1) >>
        \fi
      >>
    >>%
  }%
  \Hy@colorlink{\@filecolor}#2\Hy@endcolorlink
  \@pdfm@mark{eann}%
}
\def\hyper@linkurl#1#2{%
  \@pdfm@mark{%
    bann <<
      /Type /Annot
      /Subtype /Link
      /Border [\@pdfborder]
      /C [\@urlbordercolor]
      /A <<
        /S /URI
        /URI (#2)
      >>
    >>%
  }%
  \Hy@colorlink{\@urlcolor}#1\Hy@endcolorlink
  \@pdfm@mark{eann}%
}
\def\Acrobatmenu#1#2{%
  \@pdfm@mark{%
    bann <<
      /Type /Annot
      /Subtype /Link
      /A <<
        /S /Named
        /N /#1
      >>
      /Border [\@pdfborder]
      /C [\@menubordercolor]
    >>%
  }%
  \Hy@colorlink{\@menucolor}#2\Hy@endcolorlink
  \@pdfm@mark{eann}%
}
\def\@pdfproducer{dvipdfm}
\def\PDF@FinishDoc{%
  \Hy@UseMaketitleInfos
  \@pdfm@mark{%
    docinfo <<
      /Title (\@pdftitle)
      /Subject (\@pdfsubject)
      /Creator (\@pdfcreator)
      /Author (\@pdfauthor)
      /Producer (\@pdfproducer)
      /Keywords (\@pdfkeywords)
    >>%
  }%
  \Hy@DisableOption{pdfauthor}%
  \Hy@DisableOption{pdftitle}%
  \Hy@DisableOption{pdfsubject}%
  \Hy@DisableOption{pdfcreator}%
  \Hy@DisableOption{pdfproducer}%
  \Hy@DisableOption{pdfkeywords}%
}
%    \end{macrocode}
%    \begin{macrocode}
\def\PDF@SetupDoc{%
  \@pdfm@mark{%
    docview <<%
      \ifx\@pdfstartview\@empty
      \else
        /OpenAction[@page\@pdfstartpage\@pdfstartview]%
      \fi
      \ifx\@baseurl\@empty
      \else
        /URI<</Base(\@baseurl)>>%
      \fi
      /PageMode \@pdfpagemode
      /ViewerPreferences<<%
        \ifHy@toolbar\else /HideToolbar true\fi
        \ifHy@menubar\else /HideMenubar true\fi
        \ifHy@windowui\else /HideWindowUI true\fi
        \ifHy@fitwindow /FitWindow true\fi
        \ifHy@centerwindow /CenterWindow true\fi
      >>
      \ifx\pdf@pagelayout\@empty
      \else
        /PageLayout/\pdf@pagelayout\space
      \fi
    >>%
  }%
  \ifx\@pdfpagescrop\@empty
  \else
    \@pdfm@mark{put @pages <</CropBox[\@pdfpagescrop]>>}
  \fi
}
%</dvipdfm>
%    \end{macrocode}
%
% \subsection{VTeX typesetting system}
% Provided by MicroPress, May 1998.
% They require VTeX version 6.02 or newer;
% see \url{http://www.micropress-inc.com/} for details.
%    \begin{macrocode}
%<*vtexhtml>
\RequirePackage{vtexhtml}
\newif\if@Localurl
\def\PDF@FinishDoc{}
\def\PDF@SetupDoc{%
  \ifx\@baseurl\@empty\else
    \special{!direct <base href="\@baseurl">}%
  \fi
}
\def\@urltype{url}
\def\hyper@link#1#2#3{%
  \leavevmode
  \special{!direct <a href=\hyper@quote\hyper@hash#2\hyper@quote>}%
  #3%
  \special{!direct </a>}%
}
\def\hyper@linkurl#1#2{%
  \begingroup
    \hyper@chars
    \leavevmode
    \MathBSuppress=1\relax
    \special{!direct <a href=\hyper@quote#2\hyper@quote>}%
    #1%
    \MathBSuppress=0\relax
    \special{!direct </a>}%
  \endgroup
}
\def\hyper@linkfile#1#2#3{%
  \hyper@linkurl{#1}{file:#2\ifx\\#3\\\else\##3\fi}%
}
\def\hyper@linkstart#1#2{%
  \def\Hy@tempa{#1}\ifx\Hy@tempa\@urltype
    \@Localurltrue
    \special{!direct <a href=\hyper@quote#2\hyper@quote>}%
  \else
    \@Localurlfalse
    \begingroup
      \hyper@chars
      \special{!aref #2}%
    \endgroup
  \fi
}
\def\hyper@linkend{%
  \if@Localurl
    \special{!endaref}%
  \else
    \special{!direct </a>}%
  \fi
}
\def\hyper@anchorstart#1{%
  \Hy@SaveLastskip
  \begingroup
    \hyper@chars
    \special{!aname #1}%
    \special{!direct <a name=\hyper@quote#1\hyper@quote>}%
  \endgroup
  \Hy@activeanchortrue
}
\def\hyper@anchorend{%
  \special{!direct </a>}%
  \Hy@activeanchorfalse
  \Hy@RestoreLastskip
}
\def\hyper@anchor#1{%
  \Hy@SaveLastskip
  \begingroup
    \let\protect=\string
    \hyper@chars
    \leavevmode
    \special{!aname #1}%
    \special{!direct <a name=\hyper@quote #1\hyper@quote>}%
  \endgroup
  \Hy@activeanchortrue
  \bgroup\anchor@spot\egroup
  \special{!direct </a>}%
  \Hy@activeanchorfalse
  \Hy@RestoreLastskip
}
\def\@Form[#1]{\typeout{Sorry, TeXpider does not yet support FORMs}}
\def\@endForm{}
\def\@Gauge[#1]#2#3#4{% parameters, label, minimum, maximum
  \typeout{Sorry, TeXpider does not yet support FORM gauges}%
}
\def\@TextField[#1]#2{% parameters, label
  \typeout{Sorry, TeXpider does not yet support FORM text fields}%
  }
\def\@CheckBox[#1]#2{% parameters, label
  \typeout{Sorry, TeXpider does not yet support FORM checkboxes}%
  }
\def\@ChoiceMenu[#1]#2#3{% parameters, label, choices
  \typeout{Sorry, TeXpider does not yet support FORM choice menus}%
}
\def\@PushButton[#1]#2{% parameters, label
  \typeout{Sorry, TeXpider does not yet support FORM pushbuttons}%
}
\def\@Reset[#1]#2{\typeout{Sorry, TeXpider does not yet support FORMs}}
\def\@Submit[#1]#2{\typeout{Sorry, TeXpider does not yet support FORMs}}
%</vtexhtml>
%    \end{macrocode}
%    \begin{macrocode}
%<*vtex>
%    \end{macrocode}
%    VTeX version 6.68 supports \cs{mediawidth} and \cs{mediaheight}.
%    The \cs{ifx} construct is better than a \cs{csname}, because
%    it avoids the definition and the hash table entry of a
%    previous undefined macro.
%    \begin{macrocode}
\ifx\mediaheight\@undefined
\else
   \ifx\mediaheight\relax
   \else
      \providecommand*{\VTeXInitMediaSize}{%
        \@ifundefined{stockwidth}{%
          \setlength\mediaheight\paperheight
          \setlength\mediawidth\paperwidth
        }{%
          \setlength\mediaheight\stockheight
          \setlength\mediawidth\stockwidth
        }%
      }%
      \AtBeginDocument{\VTeXInitMediaSize}%
   \fi
\fi
%    \end{macrocode}
%    Older versions of VTeX require |xyz| in lower case.
%    \begin{macrocode}
\providecommand\@pdfview{xyz}
\providecommand\@pdfborder{0 0 1}%
\def\CurrentBorderColor{\@linkbordercolor}
\def\hyper@anchor#1{%
  \Hy@SaveLastskip
  \begingroup
    \let\protect=\string
    \hyper@chars
    \special{!aname #1;\@pdfview}%
  \endgroup
  \Hy@activeanchortrue
  \Hy@colorlink{\@anchorcolor}\anchor@spot\Hy@endcolorlink
  \Hy@activeanchorfalse
  \Hy@RestoreLastskip
}
\def\hyper@anchorstart#1{%
  \Hy@SaveLastskip
  \begingroup
    \hyper@chars
    \special{!aname #1;\@pdfview}%
  \endgroup
  \Hy@activeanchortrue
}
\def\hyper@anchorend{%
  \Hy@activeanchorfalse
  \Hy@RestoreLastskip
}
\def\@urltype{url}
\def\Hy@undefinedname{UNDEFINED}
\def\hyper@linkstart#1#2{%
  \Hy@colorlink{\csname @#1color\endcsname}%
  \edef\CurrentBorderColor{\csname @#1bordercolor\endcsname}%
  \def\Hy@tempa{#1}%
  \ifx\Hy@tempa\@urltype
    \special{!%
      aref <u=/Type /Action /S /URI /URI (#2)>;%
      a=</Border [\@pdfborder] /C [\CurrentBorderColor]>%
    }%
  \else
    \protected@edef\Hy@testname{#2}%
    \ifx\Hy@testname\@empty
      \let\Hy@testname\Hy@undefinedname
    \fi
    \special{!%
      aref \Hy@testname;%
      a=</Border [\@pdfborder] /C [\CurrentBorderColor]>%
    }%
  \fi
}
\def\hyper@linkend{%
  \special{!endaref}%
  \Hy@endcolorlink
}
\def\hyper@linkfile#1#2#3{%
  \leavevmode
  \special{!%
    aref <%
    \ifnum\Hy@VTeXversion>753 \ifHy@newwindow n\fi\fi
    f=#2>#3;%
    a=</Border [\@pdfborder] /C [\@filebordercolor]>%
  }%
  \Hy@colorlink{\@filecolor}#1\Hy@endcolorlink
  \special{!endaref}%
}
\def\hyper@linkurl#1#2{%
  \begingroup
    \hyper@chars
    \leavevmode
    \special{!%
      aref <u=/Type /Action /S /URI /URI (#2)>;%
      a=</Border [\@pdfborder] /C [\@urlbordercolor]>%
    }%
    \Hy@colorlink{\@urlcolor}#1\Hy@endcolorlink
    \special{!endaref}%
  \endgroup
}
\def\hyper@link#1#2#3{%
  \edef\CurrentBorderColor{\csname @#1bordercolor\endcsname}%
  \leavevmode
  \protected@edef\Hy@testname{#2}%
  \ifx\Hy@testname\@empty
    \let\Hy@testname\Hy@undefinedname
  \fi
  \special{!%
    aref \Hy@testname;%
    a=</Border [\@pdfborder] /C [\CurrentBorderColor]>%
  }%
  \Hy@colorlink{\csname @#1color\endcsname}#3\Hy@endcolorlink
  \special{!endaref}%
}
\def\hyper@image#1#2{%
  \hyper@linkurl{#2}{#1}%
}
\def\@hyper@launch run:#1\\#2#3{%
  \leavevmode
  \special{!aref
    <u=%
      /Type /Action
      /S /Launch
      /F (#1)
      \ifHy@newwindow /NewWindow true \fi
      \ifx\\#3\\\else /Win << /F (#1) /P (#3) >> \fi%
    >;%
    a=</Border [\@pdfborder] /C [\@runbordercolor]>%
  }%
  \Hy@colorlink{\@filecolor}#2\Hy@endcolorlink
  \special{!endaref}%
}
\def\Acrobatmenu#1#2{%
  \leavevmode
  \special{!%
    aref <u=/S /Named /N /#1>;%
    a=</Border [\@pdfborder] /C [\@menubordercolor]>%
  }%
  \Hy@colorlink{\@menucolor}#2\Hy@endcolorlink
  \special{!endaref}%
}
%    \end{macrocode}
%
%    The following code (transition effects) is
%    made by Alex Kostin.
%
%    The code below makes sense for V\TeX\ 7.02 or later.
%
%    Please never use |\@ifundefined{VTeXversion}{..}{..}| \emph{globally}.
%    \begin{macrocode}
\ifnum\Hy@VTeXversion<702 %
\else
  \def\hyper@pagetransition{%
    \ifx\@pdfpagetransition\relax
    \else
      \ifx\@pdfpagetransition\@empty
%    \end{macrocode}
%
%    Standard incantation.
%
%    1. Does an old entry have to be deleted?
%    2. If 1=yes, how to delete?
%    \begin{macrocode}
      \else
        \hvtex@parse@trans\@pdfpagetransition
      \fi
    \fi
  }
%    \end{macrocode}
%
%    I have to write an ``honest'' parser to convert raw PDF code
%    into V\TeX\ |\special|. (AVK)
%
%    Syntax of V\TeX\ |\special{!trans <transition_effect>}|:
%\begin{verbatim}
%<transition_effect> ::= <transition_style>[,<transition_duration>]
%<transition_style> ::= <Blinds_effect> | <Box_effect> |
%                       <Dissolve_effect> | <Glitter_effect> |
%                       <Split_effect> | <Wipe_effect>
%<Blinds_effect> ::= B[<effect_dimension>]
%<Box_effect> ::= X[<effect_motion>]
%<Dissolve_effect> ::= D
%<Glitter_effect> ::= G[<effect_direction>]
%<Split_effect> ::= S[<effect_motion>][<effect_dimension>]
%<Wipe_effect> ::= W[<effect_direction>]
%<Replace_effect> ::= R
%<effect_direction> ::= <number>
%<effect_dimension> ::= H | V
%<effect_motion> ::= I | O
%<transition_duration> ::= <number>
%\end{verbatim}
%
%    Transition codes:
%    \begin{macrocode}
  \def\hvtex@trans@effect@Blinds{\def\hvtex@trans@code{B}}
  \def\hvtex@trans@effect@Box{\def\hvtex@trans@code{X}}
  \def\hvtex@trans@effect@Dissolve{\def\hvtex@trans@code{D}}
  \def\hvtex@trans@effect@Glitter{\def\hvtex@trans@code{G}}
  \def\hvtex@trans@effect@Split{\def\hvtex@trans@code{S}}
  \def\hvtex@trans@effect@Wipe{\def\hvtex@trans@code{W}}
  \def\hvtex@trans@effect@R{\def\hvtex@trans@code{R}}
%    \end{macrocode}
%
%    Optional parameters:
%    \begin{macrocode}
  \def\hvtex@par@dimension{/Dm}
  \def\hvtex@par@direction{/Di}
  \def\hvtex@par@duration{/D}
  \def\hvtex@par@motion{/M}
%    \end{macrocode}
%
%    Tokenizer:
%    \begin{macrocode}
  \def\hvtex@gettoken{\expandafter\hvtex@gettoken@\hvtex@buffer\@nil}
%    \end{macrocode}
%
%    Notice that tokens in the input buffer must be space delimited.
%    \begin{macrocode}
  \def\hvtex@gettoken@#1 #2\@nil{%
    \edef\hvtex@token{#1}%
    \edef\hvtex@buffer{#2}%
  }
  \def\hvtex@parse@trans#1{%
%    \end{macrocode}
%
%    Initializing code:
%    \begin{macrocode}
    \let\hvtex@trans@code\@empty
    \let\hvtex@param@dimension\@empty
    \let\hvtex@param@direction\@empty
    \let\hvtex@param@duration\@empty
    \let\hvtex@param@motion\@empty
    \edef\hvtex@buffer{#1\space}%
%    \end{macrocode}
%    First token is the PDF transition name without escape.
%    \begin{macrocode}
    \hvtex@gettoken
    \ifx\hvtex@token\@empty
%    \end{macrocode}
%    Leading space(s)?
%    \begin{macrocode}
      \ifx\hvtex@buffer\@empty
%    \end{macrocode}
%    The buffer is empty, nothing to do.
%    \begin{macrocode}
      \else
        \hvtex@gettoken
      \fi
    \fi
    \csname hvtex@trans@effect@\hvtex@token\endcsname
%    \end{macrocode}
%    Now is time to parse optional parameters.
%    \begin{macrocode}
    \hvtex@trans@params
  }
%    \end{macrocode}
%
%    Reentrable macro to parse optional parameters.
%    \begin{macrocode}
  \def\hvtex@trans@params{%
    \ifx\hvtex@buffer\@empty
    \else
      \hvtex@gettoken
      \let\hvtex@trans@par\hvtex@token
      \ifx\hvtex@buffer\@empty
      \else
        \hvtex@gettoken
        \ifx\hvtex@trans@par\hvtex@par@duration
%    \end{macrocode}
%    /D is the effect duration in seconds. V\TeX\ special
%    takes it in milliseconds.
%    \begin{macrocode}
          \let\hvtex@param@duration\hvtex@token
        \else \ifx\hvtex@trans@par\hvtex@par@motion
%    \end{macrocode}
%    /M can be either /I or /O
%    \begin{macrocode}
          \expandafter\edef\expandafter\hvtex@param@motion
            \expandafter{\expandafter\@gobble\hvtex@token}%
        \else \ifx\hvtex@trans@par\hvtex@par@dimension
%    \end{macrocode}
%    /Dm can be either /H or /V
%    \begin{macrocode}
          \expandafter\edef\expandafter\hvtex@param@dimension
            \expandafter{\expandafter\@gobble\hvtex@token}%
        \else \ifx\hvtex@trans@par\hvtex@par@direction
%    \end{macrocode}
%
%    Valid values for /Di are 0, 270, 315 (the Glitter effect) or
%    0, 90, 180, 270 (the Wipe effect).
%    \begin{macrocode}
          \let\hvtex@param@direction\hvtex@token
        \fi\fi\fi\fi
      \fi
    \fi
    \ifx\hvtex@buffer\@empty
      \let\next\hvtex@produce@trans
    \else
      \let\next\hvtex@trans@params
    \fi
    \next
  }
%    \end{macrocode}
%
%    Merge |<transition_effect>| and issue the special when possible.
%    Too lazy to validate optional parameters.
%    \begin{macrocode}
  \def\hvtex@produce@trans{%
    \let\vtex@trans@special\@empty
    \if S\hvtex@trans@code
      \edef\vtex@trans@special{\hvtex@trans@code
        \hvtex@param@dimension\hvtex@param@motion}%
    \else \if B\hvtex@trans@code
      \edef\vtex@trans@special{\hvtex@trans@code\hvtex@param@dimension}%
    \else \if X\hvtex@trans@code
      \edef\vtex@trans@special{\hvtex@trans@code\hvtex@param@motion}%
    \else \if W\hvtex@trans@code
      \edef\vtex@trans@special{\hvtex@trans@code\hvtex@param@direction}%
    \else \if D\hvtex@trans@code
      \let\vtex@trans@special\hvtex@trans@code
    \else \if R\hvtex@trans@code
      \let\vtex@trans@special\hvtex@trans@code
    \else \if G\hvtex@trans@code
      \edef\vtex@trans@special{\hvtex@trans@code\hvtex@param@direction}%
    \fi\fi\fi\fi\fi\fi\fi
    \ifx\vtex@trans@special\@empty
    \else
      \ifx\hvtex@param@duration\@empty
      \else
        \setlength{\dimen@}{\hvtex@param@duration\p@}%
%    \end{macrocode}
%    I'm not guilty of possible overflow.
%    \begin{macrocode}
        \multiply\dimen@\@m
        \edef\vtex@trans@special{\vtex@trans@special,\strip@pt\dimen@}%
      \fi
%    \end{macrocode}
%
%    And all the mess is just for this.
%
%    \begin{macrocode}
      \special{!trans \vtex@trans@special}%
    \fi
  }
%    \end{macrocode}
%    \begin{macrocode}
  \def\hyper@pageduration{%
    \ifx\@pdfpageduration\relax
    \else
      \ifx\@pdfpageduration\@empty
        \special{!duration-}%
      \else
        \special{!duration \@pdfpageduration}%
      \fi
    \fi
  }
  \def\hyper@pagehidden{%
    \ifHy@useHidKey
      \special{!hidden\ifHy@pdfpagehidden +\else -\fi}%
    \fi
  }
\fi
%    \end{macrocode}
%
%    Caution: In opposite to the other drivers,
%    the argument of |\special{!onopen #1}| is
%    a reference name. The VTeX's postscript
%    mode will work with a version higher than
%    7.0x.
%
%    The command \verb|\VTeXOS| is defined since version 7.45.
%    Magic values encode the operating system:\\
%    \begin{tabular}{@{}l@{: }l@{}}
%      1 & WinTel\\
%      2 & Linux\\
%      3 & OS/2\\
%      4 & MacOS\\
%      5 & MacOS/X\\
%    \end{tabular}
%    \begin{macrocode}
\def\@pdfproducer{VTeX}
\ifnum\Hy@VTeXversion>\z@
  \count@\VTeXversion
  \divide\count@ 100
  \edef\@pdfproducer{\@pdfproducer\space v\the\count@}
  \multiply\count@ -100
  \advance\count@\VTeXversion
  \edef\@pdfproducer{%
    \@pdfproducer
    .\ifnum\count@<10 0\fi\the\count@
    \ifx\VTeXOS\@undefined\else
      \ifnum\VTeXOS>0 %
        \ifnum\VTeXOS<6 %
          \space(%
          \ifcase\VTeXOS
          \or Windows\or Linux\or OS/2\or MacOS\or MacOS/X%
          \fi
          )%
        \fi
      \fi
    \fi
    ,\space
    \ifnum\OpMode=\@ne PDF\else PS\fi
    \space backend%
    \ifx\gexmode\@undefined\else
      \ifnum\gexmode>\z@\space with GeX\fi
    \fi
  }
\fi
%    \end{macrocode}
%
%    Current |!pdfinfo| key syntax:
%
%    \begin{tabular}{lll}
%     \hline
%      Key        & Field                    & Type   \\
%     \hline
%     \texttt{a} & \textbf{A}uthor          & String \\
%     \texttt{b} & Crop\textbf{B}ox         & String \\
%     \texttt{c} & \textbf{C}reator         & String \\
%     \texttt{k} & \textbf{K}eywords        & String \\
%     \texttt{l} & Page\textbf{L}ayout      & PS     \\
%     \texttt{p} & \textbf{P}ageMode        & PS     \\
%     \texttt{r} & P\textbf{r}oducer        & String \\
%     \texttt{s} & \textbf{S}ubject         & String \\
%     \texttt{t} & \textbf{T}itle           & String \\
%     \texttt{u} & \textbf{U}RI             & PS     \\
%     \texttt{v} & \textbf{V}iewPreferences & PS     \\
%    \hline
%    \end{tabular}
%
%    Note: PS objects that are dicts are in |<<<..>>>| (yuck; no choice).
%
%    \begin{macrocode}
\def\PDF@SetupDoc{%
  \ifx\@pdfpagescrop\@empty
  \else
    \special{!pdfinfo b=<\@pdfpagescrop>}%
  \fi
  \special{!onopen Page\@pdfstartpage}%
  \special{!pdfinfo p=<\@pdfpagemode>}%
  \ifx\@baseurl\@empty
  \else
    \special{!pdfinfo u=<<</Base (\@baseurl)>>>}%
  \fi
  \special{!pdfinfo v=<<<%
    \ifHy@toolbar\else /HideToolbar true \fi
    \ifHy@menubar\else /HideMenubar true \fi
    \ifHy@windowui\else /HideWindowUI true \fi
    \ifHy@fitwindow /FitWindow true \fi
    \ifHy@centerwindow /CenterWindow true \fi
  >>>}%
  \ifx\pdf@pagelayout\@empty
  \else
    \special{!pdfinfo l=</\pdf@pagelayout\space>}%
  \fi
}%
\def\PDF@FinishDoc{%
  \Hy@UseMaketitleInfos
  \special{!pdfinfo a=<\@pdfauthor>}%
  \special{!pdfinfo t=<\@pdftitle>}%
  \special{!pdfinfo s=<\@pdfsubject>}%
  \special{!pdfinfo c=<\@pdfcreator>}%
  \special{!pdfinfo r=<\@pdfproducer>}%
  \special{!pdfinfo k=<\@pdfkeywords>}%
  \Hy@DisableOption{pdfauthor}%
  \Hy@DisableOption{pdftitle}%
  \Hy@DisableOption{pdfsubject}%
  \Hy@DisableOption{pdfcreator}%
  \Hy@DisableOption{pdfproducer}%
  \Hy@DisableOption{pdfkeywords}%
}
%</vtex>
%    \end{macrocode}
%
% \subsection{Fix for Adobe bug number 466320}
%    If a destination occurs at the very begin of a page,
%    the destination is moved to the previous page by
%    Adobe Distiller 5.
%    As workaround Adobe suggests:
%\begin{verbatim}
%/showpage {
%  //showpage
%  clippath stroke erasepage
%} bind def
%\end{verbatim}
%
%    But unfortunately this fix generates an empty page
%    at the end of the document. Therefore another fix
%    is used by writing some clipped text.
%    \begin{macrocode}
%<dviwindo>\def\literalps@out#1{\special{ps:#1}}%
%<package>\providecommand*{\Hy@DistillerDestFix}{}
%<*pdfmark|dviwindo>
\def\Hy@DistillerDestFix{%
  \begingroup
    \let\x\literalps@out
%    \end{macrocode}
%    The fix has to be passed unchanged through GeX, if
%    VTeX in PostScript mode with GeX is used.
%    \begin{macrocode}
    \ifnum \@ifundefined{OpMode}{0}{%
           \@ifundefined{gexmode}{0}{%
           \ifnum\gexmode>0 \OpMode\else 0\fi
           }}>1 %
      \def\x##1{%
        \immediate\special{!=##1}%
      }%
    \fi
    \x{%
      /product where{%
        pop %
        product(Distiller)search{%
          pop pop pop %
          version(.)search{%
            exch pop exch pop%
            (3011)eq{%
              gsave %
              newpath 0 0 moveto closepath clip%
              /Courier findfont 10 scalefont setfont %
              72 72 moveto(.)show %
              grestore%
            }if%
          }{pop}ifelse%
        }{pop}ifelse%
      }if%
    }%
  \endgroup
}
%</pdfmark|dviwindo>
%    \end{macrocode}
%
% \subsection{Direct pdfmark support (dvipdf and pdfmark)}
%    Drivers that load |pdfmark.def| have to provide the
%    correct macro definitions of
%    \begin{center}
%      \begin{tabular}{@{}ll@{}}
%        |\@pdfproducer|& for document information\\
%        |\literalps@out|& PostScript output\\
%        |\headerps@out|& PostScript output that goes in the header area\\
%      \end{tabular}
%    \end{center}
%    and the correct definitions of the following PostScript procedures:
%    \begin{center}
%      \begin{tabular}{@{}ll@{}}
%        |H.S|& start of anchor, link or rect\\
%        |#1 H.A|& end of anchor, argument=baselineskip in pt\\
%        |#1 H.L|& end of link, argument=baselineskip in pt\\
%        |H.R|& end of rect\\
%        |H.B|& raw rect code\\
%      \end{tabular}
%    \end{center}
%
%    \begin{macrocode}
%<*pdfmark|dvipdf>
\def\hyper@anchor#1{%
  \Hy@SaveLastskip
  \begingroup
    \pdfmark[\anchor@spot]{%
      pdfmark=/DEST,%
      linktype=anchor,%
      View=/\@pdfview \@pdfviewparams,%
      DestAnchor={#1}%
    }%
  \endgroup
  \Hy@RestoreLastskip
}
%<*dvipdf>
\def\hyper@anchorstart#1{\Hy@activeanchortrue}
\def\hyper@anchorend{\Hy@activeanchorfalse}
\def\hyper@linkstart#1#2{%
  \Hy@colorlink{\csname @#1color\endcsname}%
  \xdef\hyper@currentanchor{#2}%
}
\def\hyper@linkend{%
  \Hy@endcolorlink
}
%</dvipdf>
%<*pdfmark>
\@ifundefined{hyper@anchorstart}{}{\endinput}
\def\hyper@anchorstart#1{%
  \Hy@SaveLastskip
  \literalps@out{H.S}%
  \xdef\hyper@currentanchor{#1}%
  \Hy@activeanchortrue
}
\def\hyper@anchorend{%
  \literalps@out{\strip@pt@and@otherjunk\baselineskip\space H.A}%
  \pdfmark{%
    pdfmark=/DEST,%
    linktype=anchor,%
    View=/\@pdfview \@pdfviewparams,%
    DestAnchor=\hyper@currentanchor,%
    Raw=H.B%
  }%
  \Hy@activeanchorfalse
  \Hy@RestoreLastskip
}
\def\hyper@linkstart#1#2{%
  \ifHy@breaklinks
  \else
    \leavevmode\hbox\bgroup
  \fi
  \Hy@colorlink{\csname @#1color\endcsname}%
  \literalps@out{H.S}%
  \xdef\hyper@currentanchor{#2}%
  \gdef\hyper@currentlinktype{#1}%
}
\def\hyper@linkend{%
  \literalps@out{\strip@pt@and@otherjunk\baselineskip\space H.L}%
  \edef\Hy@temp{\csname @\hyper@currentlinktype bordercolor\endcsname}%
  \pdfmark{%
    pdfmark=/ANN,%
    linktype=link,%
    Subtype=/Link,%
    Dest=\hyper@currentanchor,%
    AcroHighlight=\@pdfhighlight,%
    Border=\@pdfborder,%
    Color=\Hy@temp,%
    Raw=H.B%
  }%
  \Hy@endcolorlink
  \ifHy@breaklinks
  \else
    \egroup
  \fi
}
%    \end{macrocode}
%    \begin{macro}{\hyper@pagetransition}
%    \begin{macrocode}
\def\hyper@pagetransition{%
  \ifx\@pdfpagetransition\relax
  \else
    \ifx\@pdfpagetransition\@empty
      % 1. Does an old entry have to be deleted?
      % 2. If 1=yes, how to delete?
    \else
      \pdfmark{%
        pdfmark=/PUT,%
        Raw={%
          \string{ThisPage\string}%
          <</Trans << /S /\@pdfpagetransition\space >> >>%
        }%
      }%
    \fi
  \fi
}
%    \end{macrocode}
%    \end{macro}
%    \begin{macro}{\hyper@pageduration}
%    \begin{macrocode}
\def\hyper@pageduration{%
  \ifx\@pdfpageduration\relax
  \else
    \ifx\@pdfpageduration\@empty
      % 1. Does an old entry have to be deleted?
      % 2. If 1=yes, how to delete?
    \else
      \pdfmark{%
        pdfmark=/PUT,%
        Raw={%
          \string{ThisPage\string}%
          <</Dur \@pdfpageduration>>%
        }%
      }%
    \fi
  \fi
}
%    \end{macrocode}
%    \end{macro}
%    \begin{macro}{\hyper@pagehidden}
%    \begin{macrocode}
\def\hyper@pagehidden{%
  \ifHy@useHidKey
    \pdfmark{%
      pdfmark=/PUT,%
      Raw={%
        \string{ThisPage\string}%
        <</Hid \ifHy@pdfpagehidden true\else false\fi>>%
      }%
    }%
  \fi
}
%    \end{macrocode}
%    \end{macro}
%    \begin{macrocode}
%</pdfmark>
\def\hyper@image#1#2{%
  \hyper@linkurl{#2}{#1}}
\def\Hy@undefinedname{UNDEFINED}
\def\hyper@link#1#2#3{%
  \edef\Hy@temp{\csname @#1bordercolor\endcsname}%
  \begingroup
    \protected@edef\Hy@testname{#2}%
    \ifx\Hy@testname\@empty
      \let\Hy@testname\Hy@undefinedname
    \fi
%<*dvipdf>
    \pdfmark[{#3}]{%
      pdfmark=/LNK,%
      {},%
      linktype=#1,%
      AcroHighlight=\@pdfhighlight,%
      Border=\@pdfborder,%
      Color=\Hy@temp,%
      Dest=\Hy@testname
    }%
%</dvipdf>
%<*pdfmarkbase>
    \pdfmark[{#3}]{%
      Color=\Hy@temp,%
      linktype=#1,%
      AcroHighlight=\@pdfhighlight,%
      Border=\@pdfborder,%
      pdfmark=/ANN,%
      Subtype=/Link,%
      Dest=\Hy@testname
    }%
%</pdfmarkbase>
  \endgroup
}
\newtoks\pdf@docset
\def\PDF@FinishDoc{%
  \Hy@UseMaketitleInfos
  \pdfmark{%
    pdfmark=/DOCINFO,%
    Title=\@pdftitle,%
    Subject=\@pdfsubject,%
    Creator=\@pdfcreator,%
    Author=\@pdfauthor,%
    Producer=\@pdfproducer,%
    Keywords=\@pdfkeywords
  }%
  \Hy@DisableOption{pdfauthor}%
  \Hy@DisableOption{pdftitle}%
  \Hy@DisableOption{pdfsubject}%
  \Hy@DisableOption{pdfcreator}%
  \Hy@DisableOption{pdfproducer}%
  \Hy@DisableOption{pdfkeywords}%
}
\def\PDF@SetupDoc{%
  \pdfmark{%
    pdfmark=/DOCVIEW,%
    Page=\@pdfstartpage,%
    View=\@pdfstartview,%
    PageMode=\@pdfpagemode
  }%
  \ifx\@pdfpagescrop\@empty
  \else
    \pdfmark{pdfmark=/PAGES,CropBox=\@pdfpagescrop}%
  \fi
  \pdfmark{%
    pdfmark=/PUT,%
    Raw={%
      \string{Catalog\string} <<
        /ViewerPreferences <<
          \ifHy@toolbar\else /HideToolbar true \fi
          \ifHy@menubar\else /HideMenubar true \fi
          \ifHy@windowui\else /HideWindowUI true \fi
          \ifHy@fitwindow /FitWindow true \fi
          \ifHy@centerwindow /CenterWindow true \fi
        >>
        \ifx\pdf@pagelayout\@empty
        \else
          /PageLayout /\pdf@pagelayout\space
        \fi
        \ifx\@baseurl\@empty
        \else
          /URI << /Base (\@baseurl) >>%
        \fi
      >>%
    }%
  }%
}
%</pdfmark|dvipdf>
%<*pdfmarkbase|dvipdf>
%    \end{macrocode}
% We define a single macro, pdfmark, which uses the `keyval' system
% to define the various allowable keys; these are \emph{exactly}
% as listed in the pdfmark reference for Acrobat 3.0. The only addition
% is \texttt{pdfmark} which specifies the type of pdfmark to create
% (like ANN, LINK etc). The
% surrounding round and square brackets in the pdfmark commands
% are supplied, but you have to put in / characters as needed for the
% values.
%
%    \begin{macrocode}
\def\pdfmark{\@ifnextchar[{\pdfmark@}{\pdfmark@[]}}
\def\pdfmark@[#1]#2{%
    \edef\@processme{\noexpand\pdf@toks={\the\pdf@defaulttoks}}%
    \@processme
    \let\pdf@type\relax
    \setkeys{PDF}{#2}%
    \ifx\pdf@type\relax
       \Hy@WarningNoLine{no pdfmark type specified in #2!!}%
       \ifx\\#1\\\relax\else\pdf@rect{#1}\fi
    \else
       \ifx\\#1\\\relax
%<pdfmarkbase>  \literalps@out{[\the\pdf@toks\space\pdf@type\space pdfmark}%
%<dvipdf>  \literalps@out{/ANN >>}%
       \else
         \Hy@colorlink{\@ifundefined{@\pdf@linktype color}%
                      {\@linkcolor}%
                      {\csname @\pdf@linktype color\endcsname}}%
         \pdf@rect{#1}%
%<pdfmarkbase>  \literalps@out{[\the\pdf@toks\space\pdf@type\space pdfmark}%
%<dvipdf>  \literalps@out{/ANN >>}%
         \Hy@endcolorlink
       \fi
    \fi
}
%    \end{macrocode}
% The complicated bit is working out the right enclosing rectangle of
% some piece of \TeX\ text, needed by the /Rect key. This solution originates
% with  Toby Thain (\texttt{tobyt@netspace.net.au}).
%    \begin{macrocode}
\newsavebox{\pdf@box}
\def\pdf@rect#1{%
%<dvipdf>   \literalps@out{/ANN \pdf@type\space\the\pdf@toks\space <<}#1%
  \leavevmode
  \sbox\pdf@box{#1}%
  \dimen@\ht\pdf@box
  \leavevmode
  \ifdim\dp\pdf@box=\z@
    \literalps@out{H.S}%
  \else
    \lower\dp\pdf@box\hbox{\literalps@out{H.S}}%
  \fi
%    \end{macrocode}
% If the text has to be horizontal mode stuff then just unbox
% the saved box like this, which saves executing it twice, which can
% mess up counters etc (thanks DPC\ldots).
%    \begin{macrocode}
  \ifHy@breaklinks\unhbox\else\box\fi\pdf@box
%    \end{macrocode}
% but if it can have multiple paragraphs you'd need one of these,
% but in that case the measured box size would be wrong anyway.
%    \begin{quote}
%   |\ifHy@breaklinks#1\else\box\pdf@box\fi|\\
%   |\ifHy@breaklinks{#1}\else\box\pdf@box\fi|
%    \end{quote}
%    \begin{macrocode}
  \ifdim\dimen@=\z@
    \literalps@out{H.R}%
  \else
    \raise\dimen@\hbox{\literalps@out{H.R}}%
  \fi
  \pdf@addtoksx{H.B}%
}
%    \end{macrocode}
% All the supplied material is stored in a token list; since I do not
% feel sure I quite understand these, things may not work as expected
% with expansion. We'll have to experiment.
%    \begin{macrocode}
\newtoks\pdf@toks
\newtoks\pdf@defaulttoks
\pdf@defaulttoks={}%
\def\pdf@addtoks#1#2{%
  \edef\@processme{\pdf@toks{\the\pdf@toks\space /#2 #1}}%
  \@processme
}
\def\pdf@addtoksx#1{%
  \edef\@processme{\pdf@toks{\the\pdf@toks\space #1}}%
  \@processme
}
\def\PDFdefaults#1{%
  \pdf@defaulttoks={#1}%
}
%    \end{macrocode}
% This is the list of allowed keys. See the Acrobat manual for an
% explanation.
%    \begin{macrocode}
% what is the type of pdfmark?
\define@key{PDF}{pdfmark}{\def\pdf@type{#1}}
% what is the link type?
\define@key{PDF}{linktype}{\def\pdf@linktype{#1}}
\def\pdf@linktype{link}
% parameter is a stream of PDF
\define@key{PDF}{Raw}{\pdf@addtoksx{#1}}
% parameter is a name
\define@key{PDF}{Action}{\pdf@addtoks{#1}{Action}}
% parameter is a array
\define@key{PDF}{Border}{\pdf@addtoks{[#1]}{Border}}
% parameter is a array
\define@key{PDF}{Color}{\pdf@addtoks{[#1]}{Color}}
% parameter is a string
\define@key{PDF}{Contents}{\pdf@addtoks{(#1)}{Contents}}
% parameter is a integer
\define@key{PDF}{Count}{\pdf@addtoks{#1}{Count}}
% parameter is a array
\define@key{PDF}{CropBox}{\pdf@addtoks{[#1]}{CropBox}}
% parameter is a string
\define@key{PDF}{DOSFile}{\pdf@addtoks{(#1)}{DOSFile}}
% parameter is a string or file
\define@key{PDF}{DataSource}{\pdf@addtoks{(#1)}{DataSource}}
% parameter is a destination
\define@key{PDF}{Dest}{%
  \begingroup
    \edef\x{#1}%
  \expandafter\endgroup
  \ifx\x\@empty\else\pdf@addtoks{(#1) cvn}{Dest}\fi
}
\define@key{PDF}{DestAnchor}{%
  \begingroup
    \edef\x{#1}%
  \expandafter\endgroup
  \ifx\x\@empty\else\pdf@addtoks{(#1) cvn}{Dest}\fi
}
% parameter is a string
\define@key{PDF}{Dir}{\pdf@addtoks{(#1)}{Dir}}
% parameter is a string
\define@key{PDF}{File}{\pdf@addtoks{(#1)}{File}}
% parameter is a int
\define@key{PDF}{Flags}{\pdf@addtoks{#1}{Flags}}
% parameter is a name
\define@key{PDF}{AcroHighlight}{\pdf@addtoks{#1}{H}}
% parameter is a string
\define@key{PDF}{ID}{\pdf@addtoks{[#1]}{ID}}
% parameter is a string
\define@key{PDF}{MacFile}{\pdf@addtoks{(#1)}{MacFile}}
% parameter is a string
\define@key{PDF}{ModDate}{\pdf@addtoks{(#1)}{ModDate}}
% parameter is a string
\define@key{PDF}{Op}{\pdf@addtoks{(#1)}{Op}}
% parameter is a Boolean
\define@key{PDF}{Open}{\pdf@addtoks{#1}{Open}}
% parameter is a integer or name
\define@key{PDF}{Page}{\pdf@addtoks{#1}{Page}}
% parameter is a name
\define@key{PDF}{PageMode}{\pdf@addtoks{#1}{PageMode}}
% parameter is a string
\define@key{PDF}{Params}{\pdf@addtoks{(#1)}{Params}}
% parameter is a array
\define@key{PDF}{Rect}{\pdf@addtoks{[#1]}{Rect}}
% parameter is a integer
\define@key{PDF}{SrcPg}{\pdf@addtoks{#1}{SrcPg}}
% parameter is a name
%<pdfmarkbase>\define@key{PDF}{Subtype}{\pdf@addtoks{#1}{Subtype}}
%<dvipdf>\define@key{PDF}{Subtype}{\pdf@addtoks{#1}{}}
% parameter is a string
\define@key{PDF}{Title}{\pdf@addtoks{(#1)}{Title}}
% parameter is a string
\define@key{PDF}{Unix}{\pdf@addtoks{(#1)}{Unix}}
% parameter is a string
\define@key{PDF}{UnixFile}{\pdf@addtoks{(#1)}{UnixFile}}
% parameter is an array
\define@key{PDF}{View}{\pdf@addtoks{[#1]}{View}}
% parameter is a string
\define@key{PDF}{WinFile}{\pdf@addtoks{(#1)}{WinFile}}
%    \end{macrocode}
% These are the keys used in the DOCINFO section.
%    \begin{macrocode}
\define@key{PDF}{Author}{\pdf@addtoks{(#1)}{Author}}
\define@key{PDF}{CreationDate}{\pdf@addtoks{(#1)}{CreationDate}}
\define@key{PDF}{Creator}{\pdf@addtoks{(#1)}{Creator}}
\define@key{PDF}{Producer}{\pdf@addtoks{(#1)}{Producer}}
\define@key{PDF}{Subject}{\pdf@addtoks{(#1)}{Subject}}
\define@key{PDF}{Keywords}{\pdf@addtoks{(#1)}{Keywords}}
\define@key{PDF}{ModDate}{\pdf@addtoks{(#1)}{ModDate}}
\define@key{PDF}{Base}{\pdf@addtoks{(#1)}{Base}}
\define@key{PDF}{URI}{\pdf@addtoks{#1}{URI}}
%</pdfmarkbase|dvipdf>
%<*pdfmark|dvipdf>
\def\Acrobatmenu#1#2{%
  \pdfmark[{#2}]{%
    linktype=menu,%
    pdfmark=/ANN,%
    AcroHighlight=\@pdfhighlight,%
    Border=\@pdfborder,%
    Action=<< /Subtype /Named /N /#1 >>,%
    Subtype=/Link%
  }%
}
%    \end{macrocode}
% And now for some useful examples:
%    \begin{macrocode}
\def\PDFNextPage{\@ifnextchar[{\PDFNextPage@}{\PDFNextPage@[]}}
\def\PDFNextPage@[#1]#2{%
  \pdfmark[{#2}]{#1,Border=\@pdfborder,Color=.2 .1 .5,
  pdfmark=/ANN,Subtype=/Link,Page=/Next}}
\def\PDFPreviousPage{%
  \@ifnextchar[{\PDFPreviousPage@}{\PDFPreviousPage@[]}%
}
\def\PDFPreviousPage@[#1]#2{%
  \pdfmark[{#2}]{#1,Border=\@pdfborder,Color=.4 .4 .1,
  pdfmark=/ANN,Subtype=/Link,Page=/Prev}}
\def\PDFOpen#1{%
  \pdfmark{#1,pdfmark=/DOCVIEW}%
}
%    \end{macrocode}
% This is not as simple as it looks; if we make the argument of
% this macro eg |\pageref{foo}| and expect it to expand to `3',
% we need a special version of |\pageref|
% which does \emph{not} produce
% `3 \hbox{}'\ldots. David Carlisle looked at this bit and provided
% the solution, as ever!
%    \begin{macrocode}
\def\PDFPage{\@ifnextchar[{\PDFPage@}{\PDFPage@[]}}
\def\PDFPage@[#1]#2#3{%
  \let\pageref\simple@pageref
  \pdfmark[{#3}]{%
    #1,%
    Page=#2,%
    AcroHighlight=\@pdfhighlight,%
    Border=\@pdfborder,%
    Color=\@pagebordercolor,%
    pdfmark=/ANN,%
    Subtype=/Link%
  }%
}
\def\simple@pageref#1{%
  \expandafter\ifx\csname r@#1\endcsname\relax
   0%
  \else
    \expandafter\expandafter\expandafter
          \@secondoffive\csname r@#1\endcsname
  \fi}
%    \end{macrocode}
% This will only work if you use Distiller 2.1 or higher.
%    \begin{macrocode}
\def\hyper@linkurl#1#2{%
  \begingroup
    \hyper@chars
    \leavevmode
%<*pdfmarkbase>
    \pdfmark[{#1}]{%
      pdfmark=/ANN,%
      linktype=url,%
      AcroHighlight=\@pdfhighlight,%
      Border=\@pdfborder,%
      Color=\@urlbordercolor,%
      Action={<< /Subtype /URI /URI (#2) >>},%
      Subtype=/Link%
    }%
%</pdfmarkbase>
%<*dvipdf>
    \pdfmark[{#1}]{%
      pdfmark=/LNK,%
      linktype=url,%
      AcroHighlight=\@pdfhighlight,%
      Border=\@pdfborder,%
      Color=\@urlbordercolor,%
      Action=URI /URI (#2)%
    }%
%</dvipdf>
  \endgroup
}
\def\hyper@linkfile#1#2#3{%
  \begingroup
    \leavevmode
%<*pdfmark>
    \pdfmark[{#1}]{%
      pdfmark=/ANN,%
      Subtype=/Link,
      AcroHighlight=\@pdfhighlight,%
      Border=\@pdfborder,%
      linktype=file,%
      Color=\@filebordercolor,%
      Action=<<
        /S /GoToR
        \ifHy@newwindow /NewWindow true \fi
        /F (#2)
        /D \ifx\\#3\\[0 \@pdfstartview]\else(#3)\fi
      >>%
    }%
%</pdfmark>
%<*dvipdf>
    \pdfmark[{#1}]{%
      pdfmark=/LNK,%
      linktype=file,
      AcroHighlight=\@pdfhighlight,%
      Border=\@pdfborder,%
      Color=\@filebordercolor,%
      Action=<<
        /S /GoToR
        \ifHy@newwindow /NewWindow true \fi
        /F (#2)
        /D \ifx\\#3\\[0 \@pdfstartview]\else(#3)\fi
      >>%
    }%
%</dvipdf>
  \endgroup
}
\def\@hyper@launch run:#1\\#2#3{%
  \begingroup
    \leavevmode
%<*pdfmark>
    \pdfmark[{#2}]{%
      pdfmark=/ANN,%
      Subtype=/Link,%
      AcroHighlight=\@pdfhighlight,%
      Border=\@pdfborder,%
      linktype=file,%
      Color=\@filebordercolor,%
      Action=<<
        /S /Launch
        \ifHy@newwindow /NewWindow true \fi
        /F (#1)
        \ifx\\#3\\\else /Win << /P (#3) /F (#1) >> \fi
      >>%
    }%
%</pdfmark>
%<*dvipdf>
    \pdfmark[{#2}]{%
      pdfmark=/LNK,%
      linktype=file,%
      AcroHighlight=\@pdfhighlight,%
      Border=\@pdfborder,%
      Color=\@filebordercolor,%
      Action=<<
        /S /GoToR
        \ifHy@newwindow /NewWindow true \fi
        /F (#1)
        /D \ifx\\#3\\[0 \@pdfstartview]\else(#3)\fi
      >>%
    }%
%</dvipdf>
  \endgroup
}
%</pdfmark|dvipdf>
%    \end{macrocode}
% Unfortunately, some parts of the |pdfmark|
% PostScript code depend on vagaries
% of the dvi driver. We isolate here all the problems.
%
% \subsection{Rokicki's dvips}
% dvips thinks in 10ths of a big point, its
% coordinate space is resolution dependent,
% and its $y$ axis starts at the top of the
% page. Other drivers can and will be different!
%
% The work is done in |SDict|, because we add in some header
% definitions in a moment.
%    \begin{macrocode}
%<*dvips>
\input{pdfmark.def}%
\def\@pdfproducer{dvips + Distiller}
\def\literalps@out#1{\special{ps:SDict begin #1 end}}%
\def\headerps@out#1{\special{! #1}}%
\providecommand\@pdfborder{0 0 12}
\providecommand\@pdfview{XYZ}
\providecommand\@pdfviewparams{ H.V}
%    \end{macrocode}
%
%    \begin{macrocode}
\AtBeginDvi{%
  \headerps@out{%
%    \end{macrocode}
% Unless I am going mad, this \emph{appears} to be the relationship
% between the default coordinate system (PDF), and dvips;
% \begin{verbatim}
% /DvipsToPDF { .01383701 div Resolution div } def
% /PDFToDvips { .01383701 mul Resolution mul } def
% \end{verbatim}
% the latter's coordinates are resolution dependent, but what that
% .01383701 is, who knows? well, almost everyone except me, I expect\ldots
% And yes, Maarten Gelderman \texttt{<mgelderman@econ.vu.nl>}
% points out that its 1/72.27 (the number of points to an inch, big points
% to inch is 1/72). This also suggests that the code would be more
% understandable (and exact) if 0.013 div would be replaced by 72.27 mul,
% so here we go. If this isn't right, I'll revert it.
%    \begin{macrocode}
    /DvipsToPDF { 72.27 mul Resolution div } def
    /PDFToDvips { 72.27 div Resolution mul } def
%    \end{macrocode}
% The rectangle around the links starts off
% \emph{exactly} the size of the box;
% we will to make it slightly bigger, 1 point on all sides.
%    \begin{macrocode}
    /HyperBorder  { 1 PDFToDvips } def
    /H.V {pdf@hoff pdf@voff null} def
    /H.B {/Rect[pdf@llx pdf@lly pdf@urx pdf@ury]} def
%    \end{macrocode}
% |H.S| (start of anchor, link, or rect) stores
% the $x$ and $y$ coordinates of the current point, in PDF coordinates
%    \begin{macrocode}
    /H.S {
      currentpoint
      HyperBorder add /pdf@lly exch def
      dup DvipsToPDF  /pdf@hoff exch def
      HyperBorder sub /pdf@llx exch def
    } def
%    \end{macrocode}
%
% The calculation of upper left $y$ is done without
% raising the point in \TeX,
% by simply adding on the current |\baselineskip| to the current $y$.
% This is usually too much, so we remove a notional 2 points.
%
% We have to see what the current baselineskip is, and convert it
% to the dvips coordinate system.
%
% Argument: baselineskip in pt.
% The $x$ and $y$ coordinates of the current point, minus the baselineskip
%    \begin{macrocode}
    /H.L {
      2 sub dup
      /HyperBasePt exch def
      PDFToDvips /HyperBaseDvips exch def
      currentpoint
      HyperBaseDvips sub /pdf@ury exch def
      /pdf@urx exch def
    } def
    /H.A {
      H.L
% |/pdf@voff| = the distance from the top of the page to a point
% |\baselineskip| above the current point in PDF coordinates
      currentpoint exch pop
      vsize 72 sub exch DvipsToPDF
      HyperBasePt sub % baseline skip
      sub /pdf@voff exch def
    } def
    /H.R {
      currentpoint
      HyperBorder sub /pdf@ury exch def
      HyperBorder add /pdf@urx exch def
% |/pdf@voff| = the distance from the top of the page to the current point, in
% PDF coordinates
      currentpoint exch pop vsize 72 sub
      exch DvipsToPDF sub /pdf@voff exch def
    } def
    systemdict
    /pdfmark known not
    {userdict /pdfmark systemdict /cleartomark get put} if
  }%
}
\AfterBeginDocument{%
  \ifx\special@paper\@empty\else
    \special{papersize=\special@paper}%
  \fi
}
%</dvips>
%    \end{macrocode}
%
% \subsection{VTeX's vtexpdfmark driver}
%
% This part is derived from the dvips
% (many names reflect this).
%
% The origin seems to be the same as TeX's origin,
% 1 in from the left and 1 in downwards from the top.
% The direction of the $y$ axis is downwards,
% the opposite of the dvips case. Units seems
% to be pt or bp.
%
%    \begin{macrocode}
%<*vtexpdfmark>
\input{pdfmark.def}%
\ifnum\OpMode=\@ne
  \def\@pdfproducer{VTeX}
\else
  \def\@pdfproducer{VTeX + Distiller}
\fi
\def\literalps@out#1{\special{pS:#1}}%
\def\headerps@out#1{\immediate\special{pS:#1}}%
\providecommand\@pdfborder{0 0 12}
\providecommand\@pdfview{XYZ}
\providecommand\@pdfviewparams{ H.V}
%    \end{macrocode}
%
%    \begin{macrocode}
\AtBeginDvi{%
  \headerps@out{%
    /vsize {\Hy@pageheight} def
%    \end{macrocode}
% The rectangle around the links starts off
% \emph{exactly} the size of the box;
% we will to make it slightly bigger, 1 point on all sides.
%    \begin{macrocode}
    /HyperBorder {1} def
    /H.V {pdf@hoff pdf@voff null} def
    /H.B {/Rect[pdf@llx pdf@lly pdf@urx pdf@ury]} def
%    \end{macrocode}
%
% |H.S| (start of anchor, link, or rect) stores
% the $x$ and $y$ coordinates of the current point, in PDF coordinates:
% $\mathtt{pdf@lly}  = Y_c - \mathtt{HyperBorder}$,
% $\mathtt{pdf@hoff} = X_c + 72$,
% $\mathtt{pdf@llx}  = X_c - \mathtt{HyperBorder}$
%    \begin{macrocode}
    /H.S {
      currentpoint
      HyperBorder sub
      /pdf@lly exch def
      dup 72 add /pdf@hoff exch def
      HyperBorder sub
      /pdf@llx exch def
    } def
%    \end{macrocode}
% The $x$ and $y$ coordinates of the current point, minus the
% |\baselineskip|:
% $\mathtt{pdf@ury} = Y_c + \mathtt{HyperBasePt} + \mathtt{HyperBorder}$,
% $\mathtt{pdf@urx} = X_c + \mathtt{HyperBorder}$
%    \begin{macrocode}
    /H.L {
      2 sub
      /HyperBasePt exch def
      currentpoint
      HyperBasePt add HyperBorder add
      /pdf@ury exch def
      HyperBorder add
      /pdf@urx exch def
    } def
    /H.A {
      H.L
      currentpoint exch pop
      vsize 72 sub exch
      HyperBasePt add add
      /pdf@voff exch def
    } def
%    \end{macrocode}
% $\mathtt{pdf@ury} = Y_c + \mathtt{HyperBorder}$,
% $\mathtt{pdf@urx} = X_c + \mathtt{HyperBorder}$
%    \begin{macrocode}
    /H.R {
      currentpoint
      HyperBorder add
      /pdf@ury exch def
      HyperBorder add
      /pdf@urx exch def
      currentpoint exch pop vsize 72 sub add
      /pdf@voff exch def
    } def
  }%
}
%</vtexpdfmark>
%    \end{macrocode}
%
% \subsection{Textures}
% At the suggestion of Jacques Distler (distler@golem.ph.utexas.edu), try
% to derive a suitable driver for Textures. This was initially a copy of
% dvips, with some guesses about Textures behaviour.
% Ross Moore (\Email{ross@maths.mq.edu.au}) has added modifications
% for better compatibility, and to support use of pdfmark.
%
% Start by defining a macro that expands to the end-of-line character.
% This will be used to format the appearance of PostScript code,
% to enhance readability, and avoid excessively long lines which
% might otherwise become broken to bad places.
%
%    \begin{macro}{\Hy@ps@CR}
%    The macro \verb|\Hy@ps@CR| contains the end-of-line character.
%    \begin{macrocode}
%<*textures>
\begingroup
  \obeylines %
  \gdef\Hy@ps@CR{\noexpand
  }%
\endgroup %
%    \end{macrocode}
%    \end{macro}
%
% Textures has two types of \verb|\special| command for inserting
% PostScript code directly into the dvi output. The `postscript'
% way preserves TeX's idea of where on the page the \verb|\special|
% occurred, but it wraps the contents with a \verb|save|--\verb|restore|
% pair, and adjusts the user-space coordinate system for local drawing
% commands. The `rawpostscript' way simply inserts code, without regard
% for the location on the page.
%
% Thus, to put arbitrary PostScript coding at a fixed location requires
% using \emph{both} \verb|\special| constructions.
% It works by pushing the device-space coordinates onto the operand stack,
% where they can be used to transform back to the correct user-space
% coordinates for the whole page, within a `rawpostscript' \verb|\special|.
%
%    \begin{macrocode}
\def\literalps@out#1{%
  \special{postscript 0 0 transform}%
  \special{rawpostscript itransform moveto\Hy@ps@CR #1}%
}%
%
%    \end{macrocode}
%
% The `prepostscript' is a 3rd kind of \verb|\special|, used for
% inserting definitions into the dictionaries, before page-building
% begins. These are to be available for use on all pages.
%
%    \begin{macrocode}
\def\headerps@out#1{%
  \special{%
    prepostscript TeXdict begin\Hy@ps@CR
      #1\Hy@ps@CR
    end%
  }%
}%
%
%    \end{macrocode}
%
% To correctly support the \verb|pdfmark| method, for embedding
% PDF definitions with \verb|.ps| files in a non-intrusive way,
% an appropriate definition needs to be made \emph{before}
% the file \verb|pdfmark.def| is read. Other parameters are best
% set afterwards.
%
%    \begin{macrocode}
\AtBeginDvi{%
  \headerps@out{%
    /betterpdfmark {%
      systemdict begin
        dup /BP eq
        {cleartomark gsave nulldevice [}
        {dup /EP eq
          {cleartomark cleartomark grestore}
          {cleartomark}
          ifelse
        }ifelse
      end
    }def\Hy@ps@CR
    __pdfmark__ not{/pdfmark /betterpdfmark load def}if
  }% end of \headerps@out
}% end of \AtBeginDvi
%
\input{pdfmark.def}%
%
\def\@pdfproducer{Textures + Distiller}%
\providecommand\@pdfborder{0 0 1}
\providecommand\@pdfview{XYZ}
\providecommand\@pdfviewparams{ H.V}
%
%    \end{macrocode}
% These are called at the start and end of unboxed links;
% their job is to leave available PS variables called
% |pdf@llx pdf@lly pdf@urx pdf@ury|, which are the coordinates
% of the bounding rectangle of the link, and |pdf@hoff pdf@voff|
% which are the PDF page offsets.
% The Rect pair are called at the LL and UR corners of a box
% known to \TeX.
%    \begin{macrocode}
\headerps@out{%
%    \end{macrocode}
% Textures lives in normal points, I think. So conversion from one
% coordinate system to another involves doing nothing.
%
%    \begin{macrocode}
  /vsize {\Hy@pageheight} def
  /DvipsToPDF {} def
  /PDFToDvips {} def
  /HyperBorder  { 1 PDFToDvips } def\Hy@ps@CR
  /H.V {pdf@hoff pdf@voff null} def\Hy@ps@CR
  /H.B {/Rect[pdf@llx pdf@lly pdf@urx pdf@ury]} def\Hy@ps@CR
  /H.S {
    currentpoint
    HyperBorder add /pdf@lly exch def
    dup DvipsToPDF  /pdf@hoff exch def
    HyperBorder sub /pdf@llx exch def
  } def\Hy@ps@CR
  /H.L {
    2 sub
    PDFToDvips /HyperBase exch def
    currentpoint
    HyperBase sub /pdf@ury exch def
    /pdf@urx exch def
  } def\Hy@ps@CR
  /H.A {
    H.L
    currentpoint exch pop
    vsize 72 sub exch DvipsToPDF
    HyperBase sub % baseline skip
    sub /pdf@voff exch def
  } def\Hy@ps@CR
  /H.R {
    currentpoint
    HyperBorder sub /pdf@ury exch def
    HyperBorder add /pdf@urx exch def
    currentpoint exch pop vsize 72 sub
    exch DvipsToPDF sub /pdf@voff exch def
  } def\Hy@ps@CR
}
%    \end{macrocode}
%    \begin{macrocode}
\AfterBeginDocument{%
  \ifHy@colorlinks
    \headerps@out{/PDFBorder{/Border [0 0 0]}def}%
  \fi
}
%    \end{macrocode}
%    Textures provides built-in support for HyperTeX specials
%    so this part combines code from  \verb|hypertex.def| with what
%    is established by loading \verb|pdfmark.def|, or any other driver.
%    \begin{macrocode}
\expandafter\let\expandafter\keepPDF@SetupDoc
  \csname PDF@SetupDoc\endcsname
\def\PDF@SetupDoc{%
  \ifx\@baseurl\@empty\else
    \special{html:<base href="\@baseurl">}%
  \fi
  \keepPDF@SetupDoc
}
\def\hyper@anchor#1{%
  \Hy@SaveLastskip
  \begingroup
    \let\protect=\string
    \special{html:<a name=\hyper@quote #1\hyper@quote>}%
  \endgroup
  \Hy@activeanchortrue
  \Hy@colorlink{\@anchorcolor}\anchor@spot\Hy@endcolorlink
  \special{html:</a>}%
  \Hy@activeanchorfalse
  \Hy@RestoreLastskip
}
\def\hyper@anchorstart#1{%
  \Hy@SaveLastskip
  \begingroup
    \hyper@chars
    \special{html:<a name=\hyper@quote#1\hyper@quote>}%
  \endgroup
  \Hy@activeanchortrue
}
\def\hyper@anchorend{%
  \special{html:</a>}%
  \Hy@activeanchorfalse
  \Hy@RestoreLastskip
}
\def\@urltype{url}
\def\hyper@linkstart#1#2{%
  \Hy@colorlink{\csname @#1color\endcsname}%
  \def\Hy@tempa{#1}%
  \ifx\Hy@tempa\@urltype
    \special{html:<a href=\hyper@quote#2\hyper@quote>}%
  \else
    \begingroup
      \hyper@chars
      \special{html:<a href=\hyper@quote\##2\hyper@quote>}%
    \endgroup
  \fi
}
\def\hyper@linkend{%
  \special{html:</a>}%
  \Hy@endcolorlink
}
\def\hyper@linkfile#1#2#3{%
  \hyper@linkurl{#1}{file:#2\ifx\\#3\\\else\##3\fi}%
}
\def\hyper@linkurl#1#2{%
  \leavevmode
  \ifHy@raiselinks
    \setbox\@tempboxa=\color@hbox #1\color@endbox
    \@linkdim\dp\@tempboxa
    \lower\@linkdim\hbox{%
      \hyper@chars
      \special{html:<a href=\hyper@quote#2\hyper@quote>}%
    }%
    \Hy@colorlink{\@urlcolor}#1%
    \@linkdim\ht\@tempboxa
    \advance\@linkdim by -6.5\p@
    \raise\@linkdim\hbox{\special{html:</a>}}%
    \Hy@endcolorlink
  \else
    \begingroup
      \hyper@chars
      \special{html:<a href=\hyper@quote#2\hyper@quote>}%
      \Hy@colorlink{\@urlcolor}#1%
      \special{html:</a>}%
      \Hy@endcolorlink
    \endgroup
  \fi
}
\def\hyper@link#1#2#3{%
  \hyper@linkurl{#3}{\##2}%
}
\def\hyper@image#1#2{%
  \begingroup
    \hyper@chars
    \special{html:<img src=\hyper@quote#1\hyper@quote>}%
  \endgroup
}
%</textures>
%    \end{macrocode}
% \subsection{dvipsone}
%    \begin{macrocode}
% \subsection{dvipsone driver}
% Over-ride the default setup macro in pdfmark driver to use Y\&Y
% |\special| commands.
%<*dvipsone>
\providecommand\@pdfborder{0 0 65781}
\input{pdfmark.def}%
\def\@pdfproducer{dvipsone + Distiller}
\def\literalps@out#1{\special{ps:#1}}%
\def\headerps@out#1{\special{! #1}}%
\def\PDF@FinishDoc{%
  \Hy@UseMaketitleInfos
  \special{PDF: Keywords \@pdfkeywords}%
  \special{PDF: Title \@pdftitle}%
  \special{PDF: Creator \@pdfcreator}%
  \special{PDF: Author \@pdfauthor}%
  \special{PDF: Producer \@pdfproducer}%
  \special{PDF: Subject \@pdfsubject}%
  \Hy@DisableOption{pdfauthor}%
  \Hy@DisableOption{pdftitle}%
  \Hy@DisableOption{pdfsubject}%
  \Hy@DisableOption{pdfcreator}%
  \Hy@DisableOption{pdfproducer}%
  \Hy@DisableOption{pdfkeywords}%
}
\def\PDF@SetupDoc{%
  \pdfmark{%
    pdfmark=/DOCVIEW,
    Page=\@pdfstartpage,
    View=\@pdfstartview,
    URI={<< /Base (\@baseurl) >>},
    PageMode=\@pdfpagemode
  }%
  \ifx\@pdfpagescrop\@empty
  \else
    \pdfmark{pdfmark=/PAGES,CropBox=\@pdfpagescrop}%
  \fi
  \pdfmark{%
    pdfmark=/PUT,%
    Raw={%
      \string{Catalog\string} <<
        /ViewerPreferences <<
          \ifHy@toolbar\else /HideToolbar true \fi
          \ifHy@menubar\else /HideMenubar true \fi
          \ifHy@windowui\else /HideWindowUI true \fi
          \ifHy@fitwindow /FitWindow true \fi
          \ifHy@centerwindow /CenterWindow true \fi
        >>
        \ifx\pdf@pagelayout\@empty
        \else
          /PageLayout /\pdf@pagelayout\space
        \fi
      >>%
    }%
  }%
}
\providecommand\@pdfview{XYZ}
\providecommand\@pdfviewparams{ %
  gsave revscl currentpoint grestore
  72 add exch pop null exch null
}
%    \end{macrocode}
% These are called at the start and end of unboxed links;
% their job is to leave available PS variables called
% |pdf@llx pdf@lly pdf@urx pdf@ury|, which are the coordinates
% of the bounding rectangle of the link, and |pdf@hoff pdf@voff|
% which are the PDF page offsets. These latter are currently not
% used in the dvipsone setup.
% The Rect pair are called at the LL and UR corners of a box
% known to \TeX.
%    \begin{macrocode}
\special{headertext=
%    \end{macrocode}
% dvipsone lives in scaled points; does this mean 65536 or 65781?
%    \begin{macrocode}
  /DvipsToPDF { 65781 div  } def
  /PDFToDvips { 65781 mul } def
  /HyperBorder  { 1 PDFToDvips } def
  /H.B {/Rect[pdf@llx pdf@lly pdf@urx pdf@ury]} def
%    \end{macrocode}
%
%    \begin{macrocode}
  /H.S {
    currentpoint
    HyperBorder add /pdf@lly exch def
    dup   DvipsToPDF /pdf@hoff exch def
    HyperBorder sub /pdf@llx exch def
  } def
  /H.L {
    2 sub
    PDFToDvips /HyperBase exch def
    currentpoint
    HyperBase sub /pdf@ury exch def
    /pdf@urx exch def
  } def
  /H.A {
    H.L
    currentpoint exch pop
    HyperBase sub % baseline skip
    DvipsToPDF /pdf@voff exch def
  } def
  /H.R {
    currentpoint
    HyperBorder sub /pdf@ury exch def
    HyperBorder add /pdf@urx exch def
    currentpoint exch pop DvipsToPDF /pdf@voff exch def
  } def
  systemdict
  /pdfmark known not
  {userdict /pdfmark systemdict /cleartomark get put} if
}
%</dvipsone>
%<*dvipdf>
\def\literalps@out#1{\special{pdf: #1}}%
\providecommand\@pdfborder{0 0 1}
%</dvipdf>
%    \end{macrocode}
%
% \subsection{TeX4ht}
%    \begin{macrocode}
%<*tex4ht>
\@ifpackageloaded{tex4ht}
  {\typeout{hyperref tex4ht: tex4ht already loaded}}%
  {\RequirePackage[htex4ht]{tex4ht}}
\def\PDF@FinishDoc{}
\def\PDF@SetupDoc{%
  \ifx\@baseurl\@empty\else
    \special{t4ht=<base href="\@baseurl">}%
  \fi
}
\def\hyper@anchor#1{%
  \Hy@SaveLastskip
  \begingroup
    \let\protect=\string
    \special{t4ht=<a name=\hyper@quote #1\hyper@quote>}%
  \endgroup
  \Hy@activeanchortrue
  \Hy@colorlink{\@anchorcolor}\anchor@spot\Hy@endcolorlink
  \special{t4ht=</a>}%
  \Hy@activeanchorfalse
  \Hy@RestoreLastskip
}
\def\hyper@anchorstart#1{%
  \Hy@SaveLastskip
  \begingroup
    \hyper@chars\special{t4ht=<a name=\hyper@quote#1\hyper@quote>}%
  \endgroup
  \Hy@activeanchortrue
}
\def\hyper@anchorend{%
  \special{t4ht=</a>}%
  \Hy@activeanchorfalse
  \Hy@RestoreLastskip
}
\def\@urltype{url}
\def\hyper@linkstart#1#2{%
  \Hy@colorlink{\csname @#1color\endcsname}%
  \def\Hy@tempa{#1}%
  \ifx\Hy@tempa\@urltype
    \special{t4ht=<a href=\hyper@quote#2\hyper@quote>}%
  \else
    {\hyper@chars\special{t4ht=<a href=\hyper@quote\##2\hyper@quote>}}%
  \fi
}
\def\hyper@linkend{%
  \special{t4ht=</a>}%
  \Hy@endcolorlink
}
\def\hyper@linkfile#1#2#3{%
  \hyper@linkurl{#1}{file:#2\ifx\\#3\\\else\##3\fi}%
}
\def\hyper@linkurl#1#2{%
  \leavevmode
  \ifHy@raiselinks
    \setbox\@tempboxa=\color@hbox #1\color@endbox
    \@linkdim\dp\@tempboxa
    \lower\@linkdim\hbox{%
      \begingroup
        \hyper@chars\special{t4ht=<a href=\hyper@quote#2\hyper@quote>}%
      \endgroup
    }%
    \Hy@colorlink{\@urlcolor}#1\Hy@endcolorlink
    \@linkdim\ht\@tempboxa
    \advance\@linkdim by -6.5\p@
    \raise\@linkdim\hbox{\special{t4ht=</a>}}%
 \else
   \begingroup
     \hyper@chars
     \special{t4ht=<a href=\hyper@quote#2\hyper@quote>}%
     \Hy@colorlink{\@urlcolor}#1\Hy@endcolorlink
     \special{t4ht=</a>}%
   \endgroup
 \fi
}
\def\hyper@link#1#2#3{%
  \hyper@linkurl{#3}{\##2}%
}
\def\hyper@image#1#2{%
  \begingroup
    \hyper@chars
    \special{t4ht=<img src=\hyper@quote#1\hyper@quote>}%
  \endgroup
}
\let\autoref\ref
\ifx \rEfLiNK \UnDef
  \def\rEfLiNK #1#2{#2}%
\fi
\def\backref#1{}
%</tex4ht>
%<*tex4htcfg>
\IfFileExists{\jobname.cfg}{\endinput}{}
\Preamble{html}
   \begin{document}
\EndPreamble
\def\TeX{TeX}
\def\OMEGA{Omega}
\def\LaTeX{La\TeX}
\def\LaTeXe{\LaTeX2e}
\def\eTeX{e-\TeX}
\def\MF{Metafont}
\def\MP{Metapost}
%</tex4htcfg>
%    \end{macrocode}
%
% \section{Driver-specific form support}
% \subsection{pdfmarks}
%    \begin{macrocode}
%<*pdfmark>
\long\def\@Form[#1]{%
\AtBeginDvi{%
  \headerps@out{%
[ /_objdef {pdfDocEncoding}
  /type /dict
/OBJ pdfmark
[ {pdfDocEncoding}
 << /Type /Encoding
    /Differences [ 24 /breve /caron /circumflex /dotaccent
/hungarumlaut /ogonek /ring /tilde 39 /quotesingle 96 /grave 128
/bullet /dagger /daggerdbl /ellipsis /emdash /endash /florin /fraction
/guilsinglleft /guilsinglright /minus /perthousand /quotedblbase
/quotedblleft /quotedblright /quoteleft /quoteright /quotesinglbase
/trademark /fi /fl /Lslash /OE /Scaron /Ydieresis /Zcaron /dotlessi
/lslash /oe /scaron /zcaron 164 /currency 166 /brokenbar 168 /dieresis
/copyright /ordfeminine 172 /logicalnot /.notdef /registered /macron
/degree /plusminus /twosuperior /threesuperior /acute /mu 183
/periodcentered /cedilla /onesuperior /ordmasculine 188 /onequarter
/onehalf /threequarters 192 /Agrave /Aacute /Acircumflex /Atilde
/Adieresis /Aring /AE /Ccedilla /Egrave /Eacute /Ecircumflex
/Edieresis /Igrave /Iacute /Icircumflex /Idieresis /Eth /Ntilde
/Ograve /Oacute /Ocircumflex /Otilde /Odieresis /multiply /Oslash
/Ugrave /Uacute /Ucircumflex /Udieresis /Yacute /Thorn /germandbls
/agrave /aacute /acircumflex /atilde /adieresis /aring /ae /ccedilla
/egrave /eacute /ecircumflex /edieresis /igrave /iacute /icircumflex
/idieresis /eth /ntilde /ograve /oacute /ocircumflex /otilde
/odieresis /divide /oslash /ugrave /uacute /ucircumflex /udieresis
/yacute /thorn /ydieresis ]
>>
/PUT pdfmark
[ /_objdef {ZaDb}
  /type /dict
/OBJ pdfmark
[ {ZaDb}
  <<
   /Type /Font
   /Subtype /Type1
   /Name /ZaDb
   /BaseFont /ZapfDingbats
  >>
/PUT pdfmark
[ /_objdef {Helv}
  /type /dict
/OBJ pdfmark
[ {Helv}
  <<
  /Type /Font
  /Subtype /Type1
  /Name /Helv
  /BaseFont /Helvetica
  /Encoding {pdfDocEncoding}
  >>
/PUT pdfmark
[ /_objdef {aform}
  /type /dict
/OBJ pdfmark
[ /_objdef {afields}
  /type /array
/OBJ pdfmark
[/BBox [0 0 100 100] /_objdef {Check} /BP pdfmark
1 0 0 setrgbcolor /ZapfDingbats 80 selectfont 20 20 moveto (4) show
[/EP pdfmark
[/BBox [0 0 100 100] /_objdef {Cross} /BP pdfmark
1 0 0 setrgbcolor /ZapfDingbats 80 selectfont 20 20 moveto (8) show
[/EP pdfmark
[/BBox [0 0 250 100] /_objdef {Submit} /BP pdfmark 0.6 setgray 0 0 250
100 rectfill 1 setgray 2 2 moveto 2 98 lineto 248 98 lineto 246 96
lineto 4 96 lineto 4 4 lineto fill 0.34 setgray 248 98 moveto 248 2
lineto 2 2 lineto 4 4 lineto 246 4 lineto 246 96 lineto fill
/Helvetica 76 selectfont 0 setgray 8 22.5 moveto (Submit) show
[/EP pdfmark
[/BBox [0 0 250 100] /_objdef {SubmitP} /BP pdfmark 0.6 setgray 0 0
250 100 rectfill 0.34 setgray 2 2 moveto 2 98 lineto 248 98 lineto 246
96 lineto 4 96 lineto 4 4 lineto fill 1 setgray 248 98 moveto 248 2
lineto 2 2 lineto 4 4 lineto 246 4 lineto 246 96 lineto fill
/Helvetica 76 selectfont 0 setgray 10 20.5 moveto (Submit) show
[/EP pdfmark
[ {aform}
  <<
    /Fields {afields}
    /DR << /Font << /ZaDb {ZaDb} /Helv {Helv} >> >>
    /DA (/Helv 10 Tf 0 g )
    /CO {corder}
    /NeedAppearances true
  >>
/PUT pdfmark
[ \string{Catalog\string}
  <<
    /AcroForm {aform}
  >>
/PUT pdfmark
[ /_objdef {corder} /type /array /OBJ pdfmark   % dps
}}%
 \setkeys{Form}{#1}%
}
\def\@endForm{}
\def\@Gauge[#1]#2#3#4{% parameters, label, minimum, maximum
  \typeout{Sorry, pdfmark drivers do not support FORM gauges}%
}
\def\@TextField[#1]#2{% parameters, label
  \def\Fld@name{#2}%
  \def\Fld@default{}%
  \let\Fld@value\@empty
  \def\Fld@width{\DefaultWidthofText}%
  \def\Fld@height{\DefaultHeightofText}%
  \bgroup
    \Field@toks={ }%
    \setkeys{Field}{#1}%
    \ifFld@hidden\def\Fld@width{1sp}\fi
    \ifx\Fld@value\@empty\def\Fld@value{\Fld@default}\fi
    \ifFld@multiline
      \def\Fld@height{4\DefaultHeightofText}%
    \fi
    \LayoutTextField{#2}{%
      \pdfmark[\MakeTextField{\Fld@width}{\Fld@height}]{%
        pdfmark=/ANN,Raw={\PDFForm@Text}%
      }%
    }%
  \egroup
}
\def\@ChoiceMenu[#1]#2#3{% parameters, label, choices
  \def\Fld@name{#2}%
  \def\Fld@default{}%
  \def\Fld@width{\DefaultWidthofChoiceMenu}%
  \def\Fld@height{\DefaultHeightofChoiceMenu}%
  \bgroup
    \Fld@menulength=0
    \@tempdima\z@
    \@for\@curropt:=#3\do{%
      \expandafter\Fld@checkequals\@curropt==\\%
      \Hy@StepCount\Fld@menulength
      \settowidth{\@tempdimb}{\@currDisplay}%
      \ifdim\@tempdimb>\@tempdima\@tempdima\@tempdimb\fi
    }%
    \advance\@tempdima by 15\p@
    \Field@toks={ }%
    \setkeys{Field}{#1}%
    \ifFld@hidden\def\Fld@width{1sp}\fi
    \LayoutChoiceField{#2}{%
      \ifFld@radio
        \@@Radio{#3}%
      \else
        {%
          \ifdim\Fld@width<\@tempdima
            \ifdim\@tempdima<1cm\@tempdima1cm\fi
            \edef\Fld@width{\the\@tempdima}%
          \fi
          \def\Fld@flags{}%
          \ifFld@combo\def\Fld@flags{/Ff 917504}\fi
          \ifFld@popdown\def\Fld@flags{/Ff 131072}\fi
          \ifx\Fld@flags\@empty
            \@tempdima=\the\Fld@menulength\Fld@charsize
            \advance\@tempdima by \Fld@borderwidth bp
            \advance\@tempdima by \Fld@borderwidth bp
            \edef\Fld@height{\the\@tempdima}%
          \fi
          \@@Listbox{#3}%
        }%
      \fi
    }%
  \egroup
}
\def\@@Radio#1{%
  \Fld@listcount=0
  \@for\@curropt:=#1\do{%
    \expandafter\Fld@checkequals\@curropt==\\%
    \Hy@StepCount\Fld@listcount
    \@currDisplay\space
    \pdfmark[\MakeRadioField{\Fld@width}{\Fld@height}]{%
      pdfmark=/ANN,%
      Raw={\PDFForm@Radio /AP <</N <</\@currValue\space {Check}>> >>}%
    } % deliberate space between radio buttons
  }%
}
\newcount\Fld@listcount
\def\@@Listbox#1{%
  \Choice@toks={ }%
  \Fld@listcount=0
  \@for\@curropt:=#1\do{%
    \expandafter\Fld@checkequals\@curropt==\\%
    \Hy@StepCount\Fld@listcount
    \edef\@processme{%
       \Choice@toks{\the\Choice@toks [(\@currValue) (\@currDisplay)]}%
    }\@processme
  }%
  \pdfmark[\MakeChoiceField{\Fld@width}{\Fld@height}]{%
    pdfmark=/ANN,Raw={\PDFForm@List}%
  }%
}
\def\@PushButton[#1]#2{% parameters, label
  \def\Fld@name{#2}%
  \bgroup
    \Field@toks={ }%
    \setkeys{Field}{#1}%
    \ifFld@hidden\def\Fld@width{1sp}\fi
    \LayoutPushButtonField{%
      \pdfmark[\MakeButtonField{#2}]{%
        pdfmark=/ANN,Raw={\PDFForm@Push}%
      }%
    }%
  \egroup
}
\def\@Submit[#1]#2{%
  \Field@toks={ }%
  \def\Fld@width{\DefaultWidthofSubmit}%
  \def\Fld@height{\DefaultHeightofSubmit}%
  \bgroup
    \def\Fld@name{Submit}%
    \setkeys{Field}{#1}%
    \ifFld@hidden\def\Fld@width{1sp}\fi
    \pdfmark[\MakeButtonField{#2}]{%
      pdfmark=/ANN,%
      Raw={\PDFForm@Submit /AP << /N {Submit} /D {SubmitP} >>}%
    }%
  \egroup
}
\def\@Reset[#1]#2{%
  \Field@toks={ }%
  \def\Fld@width{\DefaultWidthofReset}%
  \def\Fld@height{\DefaultHeightofReset}%
  \bgroup
    \def\Fld@name{Reset}%
    \setkeys{Field}{#1}%
    \ifFld@hidden\def\Fld@width{1sp}\fi
    \pdfmark[\MakeButtonField{#2}]{%
      pdfmark=/ANN,Raw={\PDFForm@Reset}%
    }%
  \egroup
}
\def\@CheckBox[#1]#2{% parameters, label
  \def\Fld@name{#2}%
  \def\Fld@default{0}%
  \bgroup
    \def\Fld@width{\DefaultWidthofCheckBox}%
    \def\Fld@height{\DefaultHeightofCheckBox}%
    \Field@toks={ }%
    \setkeys{Field}{#1}%
    \ifFld@hidden\def\Fld@width{1sp}\fi
    \LayoutCheckField{#2}{%
      \pdfmark[\MakeCheckField{\Fld@width}{\Fld@height}]{%
        pdfmark=/ANN,Raw={\PDFForm@Check}%
      }%
    }%
  \egroup
}
%</pdfmark>
%    \end{macrocode}
% \subsection{dvipdf}
%    \begin{macrocode}
%<*dvipdf>
\def\@Form[#1]{\typeout{Sorry, I do not support FORMs}}
\def\@endForm{}
\def\@Gauge[#1]#2#3#4{% parameters, label, minimum, maximum
  \typeout{Sorry, dvipdf does not support FORM gauges}%
}
\def\@TextField[#1]#2{% parameters, label
  \typeout{Sorry, dvipdf does not support FORM text fields}%
}
\def\@CheckBox[#1]#2{% parameters, label
  \typeout{Sorry, dvipdf does not support FORM checkboxes}%
}
\def\@ChoiceMenu[#1]#2#3{% parameters, label, choices
  \typeout{Sorry, dvipdf does not support FORM choice menus}%
}
\def\@PushButton[#1]#2{% parameters, label
  \typeout{Sorry, dvipdf does not support FORM pushbuttons}%
}
\def\@Reset[#1]#2{\typeout{Sorry, dvipdf does not support FORMs}}
\def\@Submit[#1]#2{\typeout{Sorry, dvipdf does not support FORMs}}
%</dvipdf>
%    \end{macrocode}
% \subsection{HyperTeX}
%    \begin{macrocode}
%<*hypertex>
\def\@Form[#1]{\typeout{Sorry, HyperTeX does not support FORMs}}
\def\@endForm{}
\def\@Gauge[#1]#2#3#4{% parameters, label, minimum, maximum
  \typeout{Sorry, HyperTeX does not support FORM gauges}%
}
\def\@TextField[#1]#2{% parameters, label
  \typeout{Sorry, HyperTeX does not support FORM text fields}%
}
\def\@CheckBox[#1]#2{% parameters, label
  \typeout{Sorry, HyperTeX does not support FORM checkboxes}%
}
\def\@ChoiceMenu[#1]#2#3{% parameters, label, choices
  \typeout{Sorry, HyperTeX does not support FORM choice menus}%
}
\def\@PushButton[#1]#2{% parameters, label
  \typeout{Sorry, HyperTeX does not support FORM pushbuttons}%
}
\def\@Reset[#1]#2{\typeout{Sorry, HyperTeX does not support FORMs}}
\def\@Submit[#1]#2{\typeout{Sorry, HyperTeX does not support FORMs}}
%</hypertex>
%    \end{macrocode}
% \subsection{TeX4ht}
%    \begin{macrocode}
%<*tex4ht>
\def\@Form[#1]{%
  \setkeys{Form}{#1}%
  \HCode{<form action="\Form@action" method="\Form@method">}%
}
\def\@endForm{\HCode{</form>}}
\def\@Gauge[#1]#2#3#4{% parameters, label, minimum, maximum
  \typeout{Sorry, TeX4ht does not support gauges}%
}
\def\@TextField[#1]#2{% parameters, label
  \let\Hy@reserved@a\@empty
  \def\Fld@name{#2}%
  \def\Fld@default{}%
  \bgroup
    \Field@toks={ }%
    \setkeys{Field}{#1}%
    \HCode{<label for="\Fld@name">#2</label>}%
    \ifFld@password
      \@@PasswordField
    \else
      \@@TextField
    \fi
  \egroup
}
\def\@@PasswordField{%
  \HCode{%
    <input type="password"
     id="\Fld@name"
     name="\Fld@name"
     \ifFld@hidden type="hidden"\fi
     value="\Fld@default"
     \the\Field@toks
    >%
  }%
}
\def\@@TextField{%
  \ifFld@multiline
    \HCode{<textarea
      \ifFld@readonly readonly \fi
      id="\Fld@name"
      name="\Fld@name"
      \ifFld@hidden type="hidden"\fi
      \the\Field@toks>%
    }%
    \Fld@default
    \HCode{</textarea>}%
  \else
    \HCode{<input type="textbox"
      \ifFld@readonly readonly \fi
      id="\Fld@name"
      name="\Fld@name"
      \ifFld@hidden type="hidden"\fi
      value="\Fld@default" \the\Field@toks>
    }%
  \fi
}
\def\@ChoiceMenu[#1]#2#3{% parameters, label, choices
  \def\Fld@name{#2}%
  \def\Fld@default{}%
  \let\Hy@reserved@a\relax
  \bgroup
   \expandafter\Fld@findlength#3\\%
   \Field@toks={ }%
   \setkeys{Field}{#1}%
   #2%
   \ifFld@radio
     \expandafter\@@Radio#3\\%
   \else
     \expandafter\@@Menu#3\\%
   \fi
  \egroup
}
\def\Fld@findlength#1\\{%
  \Fld@menulength=0
  \@for\@curropt:=#1\do{\Hy@StepCount\Fld@menulength}%
}
\def\@@Menu#1\\{%
  \HCode{<select size="\the\Fld@menulength"
    name="\Fld@name"  \the\Field@toks>}%
  \@for\@curropt:=#1\do{%
    \expandafter\Fld@checkequals\@curropt==\\%
    \HCode{<option
      \ifx\@curropt\Fld@default selected \fi
      value="\@currValue">\@currDisplay</option>%
    }%
  }%
  \HCode{</select>}%
}
\def\@@Radio#1\\{%
  \@for\@curropt:=#1\do{%
    \expandafter\Fld@checkequals\@curropt==\\%
    \HCode{<input type="radio"
      \ifx\@curropt\Fld@default checked \fi
      name="\Fld@name"
      value="\@currValue"
      \the\Field@toks>%
    }%
    \@currDisplay
  }%
}
\def\@PushButton[#1]#2{% parameters, label
  \def\Fld@name{#2}%
  \bgroup
    \Field@toks={ }%
    \setkeys{Field}{#1}%
    \HCode{<input type="button"
      name="\Fld@name"
      value="#2"
      \the\Field@toks>%
    }%
    \HCode{</button>}%
  \egroup
}
\def\@Submit[#1]#2{%
  \HCode{<button type="submit">#2</button>}%
}
\def\@Reset[#1]#2{%
  \HCode{<button type="reset">#2</button>}%
}
\def\@CheckBox[#1]#2{% parameters, label
  \let\Hy@reserved@a\@empty
  \def\Fld@name{#2}%
  \def\Fld@default{0}%
  \bgroup
    \Field@toks={ }%
    \setkeys{Field}{#1}%
    \HCode{<input type="checkbox"
      \ifFld@checked checked \fi
      \ifFld@disabled disabled \fi
      \ifFld@readonly readonly \fi
      name="\Fld@name"
      \ifFld@hidden type="hidden"\fi
      value="\Fld@default"
      \the\Field@toks>%
      #2%
    }%
  \egroup
}
%</tex4ht>
%    \end{macrocode}
%
% \subsection{pdfTeX}
%    \begin{macrocode}
%<*pdftex>
\def\@Gauge[#1]#2#3#4{% parameters, label, minimum, maximum
  \typeout{Sorry, pdftex does not support FORM gauges}%
}
\RequirePackage{pifont}%
\def\MakeFieldObject#1#2{\sbox0{#1}%
  \immediate\pdfxform0 %
  \expandafter\edef\csname #2Object\endcsname{%
    \the\pdflastxform\space 0 R%
  }%
% \hbox to 0pt{\hskip-\maxdimen{\pdfrefxform \the\pdflastxform}}%
}%
\def\@Form[#1]{%
  \@ifundefined{textcolor}{\let\textcolor\@gobble}{}%
  \setkeys{Form}{#1}%
  \ifnum\pdftexversion>13
    \pdfrefobj\OBJ@pdfdocencoding
    \pdfrefobj\OBJ@ZaDb
    \pdfrefobj\OBJ@Helv
    \pdfrefobj\OBJ@acroform
  \fi
  \pdfcatalog{/AcroForm \OBJ@acroform\space 0 R}%
  \MakeFieldObject{\ding{123}}{Ding}%
  \MakeFieldObject{\fbox{\textcolor{yellow}{\textsf{Submit}}}}{Submit}%
  \MakeFieldObject{\fbox{\textcolor{yellow}{\textsf{SubmitP}}}}{SubmitP}%
}
\def\@endForm{}
\def\@TextField[#1]#2{% parameters, label
  \def\Fld@name{#2}%
  \def\Fld@default{}%
  \let\Fld@value\@empty
  \def\Fld@width{\DefaultWidthofText}%
  \def\Fld@height{\DefaultHeightofText}%
  \bgroup
    \Field@toks={ }%
    \setkeys{Field}{#1}%
    \ifFld@hidden\def\Fld@width{1sp}\fi
    \ifx\Fld@value\@empty\def\Fld@value{\Fld@default}\fi
    \ifFld@multiline
      \def\Fld@height{4\DefaultHeightofText}%
    \fi
    \LayoutTextField{#2}{%
    \pdfstartlink user {\PDFForm@Text}%
    \MakeTextField{\Fld@width}{\Fld@height}\pdfendlink}%
  \egroup
}
\def\@ChoiceMenu[#1]#2#3{% parameters, label, choices
  \def\Fld@name{#2}%
  \def\Fld@default{}%
  \def\Fld@width{\DefaultWidthofChoiceMenu}%
  \def\Fld@height{\DefaultHeightofChoiceMenu}%
  \bgroup
    \Fld@menulength=0
    \@tempdima\z@
    \@for\@curropt:=#3\do{%
      \expandafter\Fld@checkequals\@curropt==\\%
      \Hy@StepCount\Fld@menulength
      \settowidth{\@tempdimb}{\@currDisplay}%
      \ifdim\@tempdimb>\@tempdima\@tempdima\@tempdimb\fi
    }%
    \advance\@tempdima by 15\p@
    \Field@toks={ }%
    \setkeys{Field}{#1}%
    \ifFld@hidden\def\Fld@width{1sp}\fi
    \LayoutChoiceField{#2}{%
      \ifFld@radio
        \@@Radio{#3}%
      \else
        {%
          \ifdim\Fld@width<\@tempdima
            \ifdim\@tempdima<1cm\@tempdima1cm\fi
            \edef\Fld@width{\the\@tempdima}%
          \fi
          \def\Fld@flags{}%
          \ifFld@combo\def\Fld@flags{/Ff 917504}\fi
          \ifFld@popdown\def\Fld@flags{/Ff 131072}\fi
          \ifx\Fld@flags\@empty
            \@tempdima=\the\Fld@menulength\Fld@charsize
            \advance\@tempdima by \Fld@borderwidth bp
            \advance\@tempdima by \Fld@borderwidth bp
            \edef\Fld@height{\the\@tempdima}%
          \fi
          \@@Listbox{#3}%
        }%
      \fi
    }%
  \egroup
}
\def\@@Radio#1{%
  \Fld@listcount=0
  \@for\@curropt:=#1\do{%
    \expandafter\Fld@checkequals\@curropt==\\%
    \Hy@StepCount\Fld@listcount
    \@currDisplay\space
    \leavevmode
    \pdfstartlink user {%
      \PDFForm@Radio
      /AP <<
        /N <<
%    \end{macrocode}
% Laurent.Guillope@math.univ-nantes.fr (Laurent Guillope)
% persuades me that this was wrong:
% |/\Fld@name\the\Fld@listcount|. But I leave it here to remind
% me that it is untested.
%    \begin{macrocode}
          /\@currValue\space \DingObject
        >>
      >>
    }%
    \MakeRadioField{\Fld@width}{\Fld@height}\pdfendlink
    \space% deliberate space between radio buttons
  }%
}
\newcount\Fld@listcount
\def\@@Listbox#1{%
  \Choice@toks={ }%
  \Fld@listcount=0
  \@for\@curropt:=#1\do{%
    \expandafter\Fld@checkequals\@curropt==\\%
    \Hy@StepCount\Fld@listcount
    \edef\@processme{%
      \Choice@toks{\the\Choice@toks [(\@currValue) (\@currDisplay)]}%
    }\@processme
  }%
  \leavevmode
  \pdfstartlink user {\PDFForm@List}%
  \MakeChoiceField{\Fld@width}{\Fld@height}%
  \pdfendlink
}
\def\@PushButton[#1]#2{% parameters, label
  \def\Fld@name{#2}%
  \bgroup
    \Field@toks={ }%
    \setkeys{Field}{#1}%
    \ifFld@hidden\def\Fld@width{1sp}\fi
    \LayoutPushButtonField{%
      \leavevmode
      \pdfstartlink user {\PDFForm@Push}%
      \MakeButtonField{#2}%
      \pdfendlink
    }%
  \egroup
}
\def\@Submit[#1]#2{%
  \Field@toks={ }%
  \def\Fld@width{\DefaultWidthofSubmit}%
  \def\Fld@height{\DefaultHeightofSubmit}%
  \bgroup
    \def\Fld@name{Submit}%
    \setkeys{Field}{#1}%
    \ifFld@hidden\def\Fld@width{1sp}\fi
    \leavevmode
    \pdfstartlink user {%
      \PDFForm@Submit
      /AP << /N \SubmitObject\space /D \SubmitPObject >>
    }%
    \MakeButtonField{#2}%
    \pdfendlink
  \egroup
}
\def\@Reset[#1]#2{%
  \Field@toks={ }%
  \def\Fld@width{\DefaultWidthofReset}%
  \def\Fld@height{\DefaultHeightofReset}%
  \bgroup
    \def\Fld@name{Reset}%
    \setkeys{Field}{#1}%
    \ifFld@hidden\def\Fld@width{1sp}\fi
    \leavevmode
    \pdfstartlink user {\PDFForm@Reset}%
    \MakeButtonField{#2}%
    \pdfendlink
  \egroup
}
\def\@CheckBox[#1]#2{% parameters, label
  \def\Fld@name{#2}%
  \def\Fld@default{0}%
  \bgroup
    \def\Fld@width{\DefaultWidthofCheckBox}%
    \def\Fld@height{\DefaultHeightofCheckBox}%
    \Field@toks={ }%
    \setkeys{Field}{#1}%
    \ifFld@hidden\def\Fld@width{1sp}\fi
    \LayoutCheckField{#2}{%
      \pdfstartlink user {\PDFForm@Check}%
      \MakeCheckField{\Fld@width}{\Fld@height}%
      \pdfendlink
    }%
  \egroup
}
\pdfobj { << /Type /Encoding /Differences [ 24 /breve /caron
/circumflex /dotaccent /hungarumlaut /ogonek /ring /tilde 39
/quotesingle 96 /grave 128 /bullet /dagger /daggerdbl /ellipsis
/emdash /endash /florin /fraction /guilsinglleft /guilsinglright
/minus /perthousand /quotedblbase /quotedblleft /quotedblright
/quoteleft /quoteright /quotesinglbase /trademark /fi /fl /Lslash /OE
/Scaron /Ydieresis /Zcaron /dotlessi /lslash /oe /scaron /zcaron 164
/currency 166 /brokenbar 168 /dieresis /copyright /ordfeminine 172
/logicalnot /.notdef /registered /macron /degree /plusminus
/twosuperior /threesuperior /acute /mu 183 /periodcentered /cedilla
/onesuperior /ordmasculine 188 /onequarter /onehalf /threequarters 192
/Agrave /Aacute /Acircumflex /Atilde /Adieresis /Aring /AE /Ccedilla
/Egrave /Eacute /Ecircumflex /Edieresis /Igrave /Iacute /Icircumflex
/Idieresis /Eth /Ntilde /Ograve /Oacute /Ocircumflex /Otilde
/Odieresis /multiply /Oslash /Ugrave /Uacute /Ucircumflex /Udieresis
/Yacute /Thorn /germandbls /agrave /aacute /acircumflex /atilde
/adieresis /aring /ae /ccedilla /egrave /eacute /ecircumflex
/edieresis /igrave /iacute /icircumflex /idieresis /eth /ntilde
/ograve /oacute /ocircumflex /otilde /odieresis /divide /oslash
/ugrave /uacute /ucircumflex /udieresis /yacute /thorn /ydieresis ] >>
}
\edef\OBJ@pdfdocencoding{\the\pdflastobj}
\pdfobj {
  <<
   /Type /Font
   /Subtype /Type1
   /Name /ZaDb
   /BaseFont /ZapfDingbats
>>
}
\edef\OBJ@ZaDb{\the\pdflastobj}
\pdfobj {  <<
  /Type /Font
  /Subtype /Type1
  /Name /Helv
  /BaseFont /Helvetica
  /Encoding \OBJ@pdfdocencoding\space 0 R
  >>
}
\edef\OBJ@Helv{\the\pdflastobj}
\pdfobj {
  <<
    /Fields []
    /DR <<
      /Font << /ZaDb \OBJ@ZaDb\space 0 R /Helv \OBJ@Helv\space0 R >>
    >>
    /DA (/Helv 10 Tf 0 g )
    /NeedAppearances true
  >>
}
\edef\OBJ@acroform{\the\pdflastobj}
%</pdftex>
%    \end{macrocode}
%
% \subsection{dvipdfm}
%    D. P. Story adapted the pdf\TeX{} forms part for dvipdfm, of which
%    version 0.12.7b or higher is required because of a bug.
%    \begin{macrocode}
%<*dvipdfm>
%    \end{macrocode}
%
%    \begin{macro}{\@Gauge}
%    \begin{macrocode}
\def\@Gauge[#1]#2#3#4{% parameters, label, minimum, maximum
  \typeout{Sorry, dvipdfm does not support FORM gauges}%
}
%    \end{macrocode}
%    \end{macro}
%
%    \begin{macro}{\@Form}
%    \begin{macrocode}
\def\@Form[#1]{%
  \@ifundefined{textcolor}{\let\textcolor\@gobble}{}%
  \setkeys{Form}{#1}%
  \@pdfm@mark{obj @afields [ ]}%
  \@pdfm@mark{obj @corder [ ]}%
  \@pdfm@mark{%
    obj @aform <<
      /Fields @afields
      /DR << /Font << /ZaDb @OBJZaDb /Helv @OBJHelv >> >>
      /DA (/Helv 10 Tf 0 g )
      /CO @corder
      /NeedAppearances true
    >>%
  }%
  \@pdfm@mark{put @catalog << /AcroForm @aform >>}%
}
%    \end{macrocode}
%    \end{macro}
%    \begin{macro}{\@endForm}
%    \begin{macrocode}
\def\@endForm{}
%    \end{macrocode}
%    \end{macro}
%
%    \begin{macro}{\dvipdfm@setdim}
%    \cmd{\dvipdfm@setdim} sets dimensions for ann using
%    \cmd{\pdfm@box}.
%    \begin{macrocode}
\def\dvipdfm@setdim{%
  height \the\ht\pdfm@box\space
  width  \the\wd\pdfm@box\space
  depth  \the\dp\pdfm@box\space
}
%    \end{macrocode}
%    \end{macro}
%
%    \begin{macro}{\@TextField}
%    \begin{macrocode}
\def\@TextField[#1]#2{% parameters, label
  \def\Fld@name{#2}%
  \def\Fld@default{}%
  \let\Fld@value\@empty
  \def\Fld@width{\DefaultWidthofText}%
  \def\Fld@height{\DefaultHeightofText}%
  \bgroup
    \Field@toks={ }%
    \setkeys{Field}{#1}%
    \ifFld@hidden\def\Fld@width{1sp}\fi
    \ifx\Fld@value\@empty\def\Fld@value{\Fld@default}\fi
    \ifFld@multiline
      \def\Fld@height{4\DefaultHeightofText}% DANGER
    \fi
    \setbox\pdfm@box=\hbox{%
      \MakeTextField{\Fld@width}{\Fld@height}%
    }%
    \LayoutTextField{#2}{%
      \@pdfm@mark{%
        ann @\Fld@name\space \dvipdfm@setdim << \PDFForm@Text >>%
      }%
    }%
    \unhbox\pdfm@box
    \@pdfm@mark{put @afields @\Fld@name}% record in @afields array
  \egroup
}
%    \end{macrocode}
%    \end{macro}
%
%    \begin{macro}{\@ChoiceMenu}
%    \begin{macrocode}
\def\@ChoiceMenu[#1]#2#3{% parameters, label, choices
  \def\Fld@name{#2}%
  \def\Fld@default{}%
  \def\Fld@width{\DefaultWidthofChoiceMenu}%
  \def\Fld@height{\DefaultHeightofChoiceMenu}%
  \bgroup
    \Fld@menulength=0
    \@tempdima\z@
    \@for\@curropt:=#3\do{%
      \expandafter\Fld@checkequals\@curropt==\\%
      \Hy@StepCount\Fld@menulength
      \settowidth{\@tempdimb}{\@currDisplay}%
      \ifdim\@tempdimb>\@tempdima\@tempdima\@tempdimb\fi
    }%
    \advance\@tempdima by 15\p@
    \Field@toks={ }%
    \setkeys{Field}{#1}%
    \ifFld@hidden\def\Fld@width{1sp}\fi
    \LayoutChoiceField{#2}{%
      \ifFld@radio
        \@@Radio{#3}%
      \else
        {%
          \ifdim\Fld@width<\@tempdima
            \ifdim\@tempdima<1cm\@tempdima1cm\fi
            \edef\Fld@width{\the\@tempdima}%
          \fi
          \def\Fld@flags{}%
          \ifFld@combo\def\Fld@flags{/Ff 917504}\fi
          \ifFld@popdown\def\Fld@flags{/Ff 131072}\fi
          \ifx\Fld@flags\@empty
            \@tempdima=\the\Fld@menulength\Fld@charsize
            \advance\@tempdima by \Fld@borderwidth bp
            \advance\@tempdima by \Fld@borderwidth bp
            \edef\Fld@height{\the\@tempdima}%
          \fi
          \@@Listbox{#3}%
        }%
      \fi
    }%
  \egroup
}
%    \end{macrocode}
%    \end{macro}
%
%    \begin{macro}{\@@Radio}
%    \begin{macrocode}
\def\@@Radio#1{%
  \Fld@listcount=0
  \setbox\pdfm@box=\hbox{\MakeRadioField{\Fld@width}{\Fld@height}}%
  \@for\@curropt:=#1\do{%
    \expandafter\Fld@checkequals\@curropt==\\%
    \Hy@StepCount\Fld@listcount
    \@currDisplay\space
    \leavevmode
    \@pdfm@mark{%
      ann \ifnum\Fld@listcount=1 @\Fld@name\space\fi
      \dvipdfm@setdim
      <<
        \PDFForm@Radio\space
        /AP << /N << /\@currValue /null >> >>
      >>%
    }%
    \unhcopy\pdfm@box\space% deliberate space between radio buttons
    \ifnum\Fld@listcount=1\@pdfm@mark{put @afields @\Fld@name}\fi
  }%
}
%    \end{macrocode}
%    \end{macro}
%
%    \begin{macro}{\Fld@listcount}
%    \begin{macrocode}
\newcount\Fld@listcount
%    \end{macrocode}
%    \end{macro}
%    \begin{macro}{\@@Listbox}
%    \begin{macrocode}
\def\@@Listbox#1{%
  \Choice@toks={ }%
  \Fld@listcount=0
  \@for\@curropt:=#1\do{%
    \expandafter\Fld@checkequals\@curropt==\\%
    \Hy@StepCount\Fld@listcount
    \edef\@processme{%
      \Choice@toks{\the\Choice@toks [(\@currValue) (\@currDisplay)]}%
    }\@processme
  }%
  \setbox\pdfm@box=\hbox{\MakeChoiceField{\Fld@width}{\Fld@height}}%
  \leavevmode
  \@pdfm@mark{%
    ann @\Fld@name\space\dvipdfm@setdim
    << \PDFForm@List >>%
  }%
  \unhbox\pdfm@box
  \@pdfm@mark{put @afields @\Fld@name}%
}
%    \end{macrocode}
%    \end{macro}
%
%    \begin{macro}{\@PushButton}
%    \begin{macrocode}
\def\@PushButton[#1]#2{% parameters, label
  \def\Fld@name{#2}%
  \bgroup
    \Field@toks={ }%
    \setkeys{Field}{#1}%
    \ifFld@hidden\def\Fld@width{1sp}\fi
    \setbox\pdfm@box=\hbox{\MakeButtonField{#2}}%
    \LayoutPushButtonField{%
      \leavevmode
      \@pdfm@mark{%
        ann @\Fld@name\space\dvipdfm@setdim
        << \PDFForm@Push >>%
      }%
    }%
    \unhbox\pdfm@box
    \@pdfm@mark{put @afields @\Fld@name}%
  \egroup
}
%    \end{macrocode}
%    \end{macro}
%
%    \begin{macro}{\@Submit}
%    \begin{macrocode}
\def\@Submit[#1]#2{%
  \Field@toks={ }%
  \def\Fld@width{\DefaultWidthofSubmit}%
  \def\Fld@height{\DefaultHeightofSubmit}%
  \bgroup
    \def\Fld@name{Submit}%
    \setkeys{Field}{#1}%
    \ifFld@hidden\def\Fld@width{1sp}\fi
    \setbox\pdfm@box=\hbox{\MakeButtonField{#2}}%
    \leavevmode
    \@pdfm@mark{%
      ann @\Fld@name\space\dvipdfm@setdim
      << \PDFForm@Submit >>%
    }%
    \unhbox\pdfm@box%
    \@pdfm@mark{put @afields @\Fld@name}%
  \egroup
}
%    \end{macrocode}
%    \end{macro}
%
%    \begin{macro}{\@Reset}
%    \begin{macrocode}
\def\@Reset[#1]#2{%
  \Field@toks={ }%
  \def\Fld@width{\DefaultWidthofReset}%
  \def\Fld@height{\DefaultHeightofReset}%
  \bgroup
    \def\Fld@name{Reset}%
    \setkeys{Field}{#1}%
    \ifFld@hidden\def\Fld@width{1sp}\fi
    \setbox\pdfm@box=\hbox{\MakeButtonField{#2}}%
    \leavevmode
    \@pdfm@mark{%
      ann @\Fld@name\space\dvipdfm@setdim
      << \PDFForm@Reset >>%
    }%
    \unhbox\pdfm@box
    \@pdfm@mark{put @afields @\Fld@name}%
  \egroup
}
%    \end{macrocode}
%    \end{macro}
%
%    \begin{macro}{\@CheckBox}
%    \begin{macrocode}
\def\@CheckBox[#1]#2{% parameters, label
  \def\Fld@name{#2}%
  \def\Fld@default{0}%
  \bgroup
    \def\Fld@width{\DefaultWidthofCheckBox}%
    \def\Fld@height{\DefaultHeightofCheckBox}%
    \Field@toks={ }%
    \setkeys{Field}{#1}%
    \ifFld@hidden\def\Fld@width{1sp}\fi
    \setbox\pdfm@box=\hbox{\MakeCheckField{\Fld@width}{\Fld@height}}%
    \LayoutCheckField{#2}{%
      \@pdfm@mark{%
        ann @\Fld@name\space\dvipdfm@setdim
        << \PDFForm@Check >>%
      }%
      \unhbox\pdfm@box
      \@pdfm@mark{put @afields @\Fld@name}%
    }%
  \egroup
}
%    \end{macrocode}
%    \end{macro}
%
%    \begin{macrocode}
\@pdfm@mark{obj @OBJpdfdocencoding << /Type /Encoding /Differences [%
   24 /breve /caron /circumflex /dotaccent /hungarumlaut /ogonek /ring
      /tilde
   39 /quotesingle
   96 /grave
  128 /bullet /dagger /daggerdbl /ellipsis /emdash /endash /florin
      /fraction /guilsinglleft /guilsinglright /minus /perthousand
      /quotedblbase /quotedblleft /quotedblright /quoteleft /quoteright
      /quotesinglbase /trademark /fi /fl /Lslash /OE /Scaron /Ydieresis
      /Zcaron /dotlessi /lslash /oe /scaron /zcaron
  164 /currency
  166 /brokenbar
  168 /dieresis /copyright /ordfeminine
  172 /logicalnot /.notdef /registered /macron /degree /plusminus
      /twosuperior /threesuperior /acute /mu
  183 /periodcentered /cedilla /onesuperior /ordmasculine
  188 /onequarter /onehalf /threequarters
  192 /Agrave /Aacute /Acircumflex /Atilde /Adieresis /Aring /AE
      /Ccedilla /Egrave /Eacute /Ecircumflex /Edieresis /Igrave /Iacute
      /Icircumflex /Idieresis /Eth /Ntilde /Ograve /Oacute /Ocircumflex
      /Otilde /Odieresis /multiply /Oslash /Ugrave /Uacute /Ucircumflex
      /Udieresis /Yacute /Thorn /germandbls /agrave /aacute /acircumflex
      /atilde /adieresis /aring /ae /ccedilla /egrave /eacute
      /ecircumflex /edieresis /igrave /iacute /icircumflex /idieresis
      /eth /ntilde /ograve /oacute /ocircumflex /otilde /odieresis
      /divide /oslash /ugrave /uacute /ucircumflex /udieresis /yacute
      /thorn /ydieresis%
  ] >>
}
\@pdfm@mark{obj @OBJZaDb
  <<
    /Type /Font
    /Subtype /Type1
    /Name /ZaDb
    /BaseFont /ZapfDingbats
  >>
}
\@pdfm@mark{obj @OBJHelv
  <<
    /Type /Font
    /Subtype /Type1
    /Name /Helv
    /BaseFont /Helvetica
    /Encoding @OBJpdfdocencoding
  >>
}
%</dvipdfm>
%    \end{macrocode}
%
% \subsection{Common forms part}
%    \begin{macrocode}
%<*pdfform>
%    \end{macrocode}
%    \begin{macro}{\PDFForm@Check}
%    \begin{macrocode}
\def\PDFForm@Check{%
  /Subtype /Widget
  \ifFld@hidden /F 6 \else /F 4 \fi
  /T (\Fld@name)
  /Q \Fld@align\space
  /BS << /W \Fld@borderwidth\space /S /\Fld@borderstyle\space >>
  /MK <<
    /BC [\Fld@bordercolor]
    \ifx\Fld@bcolor\@empty
    \else
      /BG [\Fld@bcolor]
    \fi
    /CA (\Fld@cbsymbol)
  >>
  /DA (/ZaDb \strip@pt\Fld@charsize\space Tf \Fld@color\space rg)
  /FT /Btn
  /H /P
  \ifFld@checked /V /Yes \else /V /Off \fi
}
%    \end{macrocode}
%    \end{macro}
%    \begin{macro}{\PDFForm@Push}
%    \begin{macrocode}
\def\PDFForm@Push{%
  /Subtype /Widget
  \ifFld@hidden /F 6 \else /F 4 \fi
  /T (\Fld@name)
  /FT /Btn
  /Ff 65540
  /H /P
  /BS << /W \Fld@borderwidth\space /S /\Fld@borderstyle\space >>
  /MK <<
    /BC [\Fld@bordercolor]
  >>
  /A << /S /JavaScript /JS (\Fld@onclick;) >>
}
%    \end{macrocode}
%    \end{macro}
%
%    \begin{macro}{\Fld@additionalactions}
%    \begin{macrocode}
\def\Fld@additionalactions{%
  /AA <<
%    \end{macrocode}
%    K input (keystroke) format
%    \begin{macrocode}
    \ifx\Fld@keystroke@code\@empty
    \else
      /K << /S /JavaScript /JS (\Fld@keystroke@code) >>
    \fi
%    \end{macrocode}
%    F display format
%    \begin{macrocode}
    \ifx\Fld@format@code\@empty
    \else
      /F << /S /JavaScript /JS (\Fld@format@code) >>
    \fi
%    \end{macrocode}
%    V validation
%    \begin{macrocode}
    \ifx\Fld@validate@code\@empty
    \else
      /V << /S /JavaScript /JS (\Fld@validate@code) >>
    \fi
%    \end{macrocode}
%    C calculation
%    \begin{macrocode}
    \ifx\Fld@calculate@code\@empty
    \else
      /C << /S /JavaScript /JS (\Fld@calculate@code) >>
    \fi
%    \end{macrocode}
%    Fo receiving the input focus
%    \begin{macrocode}
    \ifx\Fld@onfocus@code\@empty
    \else
      /Fo << /S /JavaScript /JS (\Fld@onfocus@code) >>
    \fi
%    \end{macrocode}
%    Bl loosing the input focus (blurred)
%    \begin{macrocode}
    \ifx\Fld@onblur@code\@empty
    \else
      /Bl << /S /JavaScript /JS (\Fld@onblur@code) >>
    \fi
%    \end{macrocode}
%    D pressing the mouse button (down)
%    \begin{macrocode}
    \ifx\Fld@onmousedown@code\@empty
    \else
      /D << /S /JavaScript /JS (\Fld@onmousedown@code) >>
    \fi
%    \end{macrocode}
%    U releasing the mouse button (up)
%    \begin{macrocode}
    \ifx\Fld@onmouseup@code\@empty
    \else
      /U << /S /JavaScript /JS (\Fld@onmouseup@code) >>
    \fi
%    \end{macrocode}
%    E cursor enters the annotation's active area.
%    \begin{macrocode}
    \ifx\Fld@onenter@code\@empty
    \else
      /E << /S /JavaScript /JS (\Fld@onenter@code) >>
    \fi
%    \end{macrocode}
%    X cursor exits the annotation's active area.
%    \begin{macrocode}
    \ifx\Fld@onexit@code\@empty
    \else
      /X << /S /JavaScript /JS (\Fld@onexit@code) >>
    \fi
  >>
}
%    \end{macrocode}
%    \end{macro}
%
%    \begin{macro}{\PDFForm@List}
%    \begin{macrocode}
\def\PDFForm@List{%
  /Subtype /Widget
  \ifFld@hidden /F 6 \else /F 4 \fi
  /T (\Fld@name)
  /FT /Ch
  /Q \Fld@align\space
  /BS << /W \Fld@borderwidth\space /S /\Fld@borderstyle\space >>
  /MK <<
    /BC [\Fld@bordercolor]
    \ifx\Fld@bcolor\@empty
    \else
      /BG [\Fld@bcolor]
    \fi
  >>
  /DA (/Helv \strip@pt\Fld@charsize\space Tf \Fld@color\space rg )
  /Opt [\the\Choice@toks]
  /DV (\Fld@default)
  \Fld@additionalactions
  \Fld@flags
}
%    \end{macrocode}
%    \end{macro}
%    \begin{macro}{\PDFForm@Radio}
%    \begin{macrocode}
\def\PDFForm@Radio{%
  /Subtype /Widget
  \ifFld@hidden /F 6 \else /F 4 \fi
  /T (\Fld@name)
  /FT /Btn
  /Ff 49152
  /H /P
  /BS << /W \Fld@borderwidth\space /S /\Fld@borderstyle\space >>
  /MK <<
    /BC [\Fld@bordercolor]
    \ifx\Fld@bcolor\@empty
    \else
      /BG [\Fld@bcolor]
    \fi
    /CA (H)
  >>
  /DA (/ZaDb \strip@pt\Fld@charsize\space Tf \Fld@color\space rg)
  \ifx\@currValue\Fld@default
    /V /\Fld@default\space
  \else
    /V /Off
  \fi
  \Fld@additionalactions
}
%    \end{macrocode}
%    \end{macro}
%    \begin{macro}{\PDFForm@Text}
%    \begin{macrocode}
\def\PDFForm@Text{%
  /Subtype /Widget
  \ifFld@hidden /F 6 \else /F 4 \fi
  /T (\Fld@name)
  /Q \Fld@align\space
  /FT /Tx
  /BS << /W \Fld@borderwidth\space /S /\Fld@borderstyle\space >>
  /MK <<
    /BC [\Fld@bordercolor]
    \ifx\Fld@bcolor\@empty
    \else
      /BG [\Fld@bcolor]
    \fi
  >>
  /DA (/Helv \strip@pt\Fld@charsize\space Tf \Fld@color\space rg )
  /DV (\Fld@default)
  /V (\Fld@value)
  \Fld@additionalactions
  \ifFld@multiline
    \ifFld@readonly /Ff 4097 \else /Ff 4096 \fi
  \else
    \ifFld@password
      \ifFld@readonly /Ff 8193 \else /Ff 8192 \fi
    \else
      \ifFld@readonly /Ff 1 \fi
    \fi
  \fi
  \ifnum\Fld@maxlen>0/MaxLen \Fld@maxlen \fi
}
%    \end{macrocode}
%    \end{macro}
%    \begin{macro}{\PDFForm@Submit}
%    \begin{macrocode}
\def\PDFForm@Submit{%
  /Subtype /Widget
  \ifFld@hidden /F 6 \else /F 4 \fi
  /T (\Fld@name)
  /FT /Btn
  /Ff 65540
  /H /P
  /BS << /W \Fld@borderwidth\space /S /\Fld@borderstyle\space >>
  /MK <<
    /BC [\Fld@bordercolor]
  >>
  /A <<
    /S /SubmitForm
    /F <<
      /FS /URL
      /F (\Form@action)
    >>
    \ifForm@html /Flags 4 \fi
  >>
}
%    \end{macrocode}
%    \end{macro}
%    \begin{macro}{\PDFForm@Reset}
%    \begin{macrocode}
\def\PDFForm@Reset{%
  /Subtype /Widget
  \ifFld@hidden /F 6 \else /F 4 \fi
  /T (\Fld@name)
  /FT /Btn
  /H /P
  /DA (/Helv \strip@pt\Fld@charsize\space Tf 0 0 1 rg)
  /Ff 65540
  /MK <<
    /BC [\Fld@bordercolor]
%    /CA (Clear)
%    /AC (Done)
  >>
  /BS << /W \Fld@borderwidth\space /S /\Fld@borderstyle\space >>
  /A << /S /ResetForm >>
}
%    \end{macrocode}
%    \end{macro}
%    \begin{macrocode}
%</pdfform>
%<*package>
%    \end{macrocode}
%
% \section{Bookmarks in the PDF file}
% This was originally developed by Yannis Haralambous
% (it was the separate |repere.sty|); it needed
% the |repere| or |makebook.pl| post-processor to work properly. Now
% redundant, as it is done entirely in \LaTeX{} macros.
%
% To write out the current section title, and its rationalized number,
% we have to intercept the |\@sect| command, which is rather
% dangerous. But how else to see the information we need?
% We do the \emph{same} for |\@ssect|, giving anchors to
% unnumbered sections. This allows things like bibliographies
% to get bookmarks when used with a manual |\addcontentsline|
%    \begin{macrocode}
\def\phantomsection{%
 \Hy@GlobalStepCount\Hy@linkcounter
 \xdef\@currentHref{section*.\the\Hy@linkcounter}%
 \Hy@raisedlink{\hyper@anchorstart{\@currentHref}\hyper@anchorend}%
}
%</package>
%    \end{macrocode}
%
% \subsection{Bookmarks}
%    \begin{macrocode}
%<*outlines>
%    \end{macrocode}
%
% This section was written by Heiko Oberdiek; the code replaces
% an earlier version by David Carlisle.
%
% The first part of bookmark code is in section \ref{sec:pdfstring}.
% Further documentation is available as paper and slides of the
% talk, that Heiko Oberdiek has given at the EuroTeX'99 meating
% in Heidelberg. See |paper.pdf| and |slides.pdf| in the
% |doc| directory of hyperref.
%
%    \begin{macrocode}
\newwrite\@outlinefile
\def\Hy@writebookmark#1#2#3#4#5{% section number, text, label, level, file
 \ifx\WriteBookmarks\relax%
 \else
  \ifnum#4>\c@tocdepth
  \else
   \@@writetorep{#1}{#2}{#3}{#4}{#5}%
  \fi
 \fi}
\def\Hy@currentbookmarklevel{0}
\def\Hy@numberline#1{#1 }
\def\@@writetorep#1#2#3#4#5{%
  \begingroup
    \def\Hy@tempa{#5}%
    \ifx\Hy@tempa\Hy@bookmarkstype
      \edef\Hy@level{#4}%
      \ifx\Hy@levelcheck Y%
        \@tempcnta\Hy@level\relax
        \advance\@tempcnta by -1
        \ifnum\Hy@currentbookmarklevel<\@tempcnta
          \advance\@tempcnta by -\Hy@currentbookmarklevel\relax
          \advance\@tempcnta by 1
          \Hy@Warning{%
            Difference (\the\@tempcnta) between bookmark levels is %
            greater \MessageBreak than one, level fixed%
          }%
          \@tempcnta\Hy@currentbookmarklevel
          \advance\@tempcnta by 1
          \edef\Hy@level{\the\@tempcnta}%
        \fi
      \else
        \global\let\Hy@levelcheck Y%
      \fi
      \global\let\Hy@currentbookmarklevel\Hy@level
      \@tempcnta\Hy@level\relax
      \expandafter\xdef\csname Parent\Hy@level\endcsname{#3}%
      \advance\@tempcnta by -1
      \edef\Hy@tempa{#3}%
      \edef\Hy@tempb{\csname Parent\the\@tempcnta\endcsname}%
      \ifx\Hy@tempa\Hy@tempb
        \Hy@Warning{%
          The anchor of a bookmark and its parent's must not%
          \MessageBreak be the same. Added a new anchor%
        }%
        \phantomsection
      \fi
      \ifHy@bookmarksnumbered
        \let\numberline\Hy@numberline
      \else
        \let\numberline\@gobble
      \fi
      \pdfstringdef\Hy@tempa{#2}%
      \protected@write\@outlinefile{}{%
        \protect\BOOKMARK
          [\Hy@level][\@bookmarkopenstatus{\Hy@level}]{#3}%
          {\Hy@tempa}{\Hy@tempb}%
      }%
    \fi
  \endgroup
}
%    \end{macrocode}
%    In the call of \cmd{\BOOKMARK} the braces around \verb|#4|
%    are omitted, because it is not likely, that the level number
%    contains \verb|]|.
%    \begin{macrocode}
\newcommand{\currentpdfbookmark}{%
  \pdfbookmark[\Hy@currentbookmarklevel]%
}
\newcommand{\subpdfbookmark}{%
  \@tempcnta\Hy@currentbookmarklevel
  \Hy@StepCount\@tempcnta
  \expandafter\pdfbookmark\expandafter[\the\@tempcnta]%
}
\newcommand{\belowpdfbookmark}[2]{%
  \@tempcnta\Hy@currentbookmarklevel
  \Hy@StepCount\@tempcnta
  \expandafter\pdfbookmark\expandafter[\the\@tempcnta]{#1}{#2}%
  \advance\@tempcnta by -1
  \xdef\Hy@currentbookmarklevel{\the\@tempcnta}%
}
%    \end{macrocode}
% Tobias Oetiker rightly points out that we need a way to
% force a bookmark entry. So we introduce |\pdfbookmark|,
% with two parameters, the title, and a symbolic name.
% By default this is at level 1, but we can reset that with the
% optional first argument.
%    \begin{macrocode}
\renewcommand\pdfbookmark[3][0]{%
  \Hy@writebookmark{}{#2}{#3.#1}{#1}{toc}%
  \hyper@anchorstart{#3.#1}\hyper@anchorend
}
\def\BOOKMARK{\@ifnextchar[{\@BOOKMARK}{\@@BOOKMARK[1][-]}}
\def\@BOOKMARK[#1]{\@ifnextchar[{\@@BOOKMARK[#1]}{\@@BOOKMARK[#1][-]}}
%    \end{macrocode}
% The macros for calculating structure of outlines
% are derived from those by  Petr Olsak used in the texinfopdf macros.
%
% The VTEX section was written originally by VTEX, but then
% amended by Denis Girou (\Email{denis.girou@idris.fr}),
% then by by Taco Hoekwater (\Email{taco.hoekwater@wkap.nl}. The problem
% is that VTEX, with its close integration of the PDF backend, does
% look at the contents of bookmarks, escaping |\| and the like.
%    \begin{macrocode}
%<*vtex>
%    \end{macrocode}
%    \begin{macrocode}
\newcount\@serial@counter\@serial@counter=1\relax
%    \end{macrocode}
%    \begin{macro}{\hv@pdf@char}
% Plain octal codes doesn't work with versions below 6.50.
% So for early versions hex numbers have to be used.
% It would be possible to program this instead of the
% large |\ifcase|, but I'm too lazy to sort that out now.
%    \begin{macrocode}
\begingroup
  \catcode`\'=12
  \ifnum\Hy@VTeXversion<650 %
    \catcode`\"=12
    \gdef\hv@pdf@char#1#2#3{%
      \char
      \ifcase'#1#2#3
         "00\or"01\or"02\or"03\or"04\or"05\or"06\or"07%
      \or"08\or"09\or"0A\or"0B\or"0C\or"0D\or"0E\or"0F%
      \or"10\or"11\or"12\or"13\or"14\or"15\or"16\or"17%
      \or"18\or"19\or"1A\or"1B\or"1C\or"1D\or"1E\or"1F%
      \or"20\or"21\or"22\or"23\or"24\or"25\or"26\or"27%
      \or"28\or"29\or"2A\or"2B\or"2C\or"2D\or"2E\or"2F%
      \or"30\or"31\or"32\or"33\or"34\or"35\or"36\or"37%
      \or"38\or"39\or"3A\or"3B\or"3C\or"3D\or"3E\or"3F%
      \or"40\or"41\or"42\or"43\or"44\or"45\or"46\or"47%
      \or"48\or"49\or"4A\or"4B\or"4C\or"4D\or"4E\or"4F%
      \or"50\or"51\or"52\or"53\or"54\or"55\or"56\or"57%
      \or"58\or"59\or"5A\or"5B\or"5C\or"5D\or"5E\or"5F%
      \or"60\or"61\or"62\or"63\or"64\or"65\or"66\or"67%
      \or"68\or"69\or"6A\or"6B\or"6C\or"6D\or"6E\or"6F%
      \or"70\or"71\or"72\or"73\or"74\or"75\or"76\or"77%
      \or"78\or"79\or"7A\or"7B\or"7C\or"7D\or"7E\or"7F%
      \or"80\or"81\or"82\or"83\or"84\or"85\or"86\or"87%
      \or"88\or"89\or"8A\or"8B\or"8C\or"8D\or"8E\or"8F%
      \or"90\or"91\or"92\or"93\or"94\or"95\or"96\or"97%
      \or"98\or"99\or"9A\or"9B\or"9C\or"9D\or"9E\or"9F%
      \or"A0\or"A1\or"A2\or"A3\or"A4\or"A5\or"A6\or"A7%
      \or"A8\or"A9\or"AA\or"AB\or"AC\or"AD\or"AE\or"AF%
      \or"B0\or"B1\or"B2\or"B3\or"B4\or"B5\or"B6\or"B7%
      \or"B8\or"B9\or"BA\or"BB\or"BC\or"BD\or"BE\or"BF%
      \or"C0\or"C1\or"C2\or"C3\or"C4\or"C5\or"C6\or"C7%
      \or"C8\or"C9\or"CA\or"CB\or"CC\or"CD\or"CE\or"CF%
      \or"D0\or"D1\or"D2\or"D3\or"D4\or"D5\or"D6\or"D7%
      \or"D8\or"D9\or"DA\or"DB\or"DC\or"DD\or"DE\or"DF%
      \or"E0\or"E1\or"E2\or"E3\or"E4\or"E5\or"E6\or"E7%
      \or"E8\or"E9\or"EA\or"EB\or"EC\or"ED\or"EE\or"EF%
      \or"F0\or"F1\or"F2\or"F3\or"F4\or"F5\or"F6\or"F7%
      \or"F8\or"F9\or"FA\or"FB\or"FC\or"FD\or"FE\or"FF%
      \fi
    }
  \else
    \gdef\hv@pdf@char{\char'}
  \fi
\endgroup
%    \end{macrocode}
%    \end{macro}
%    \begin{macro}{\@@BOOKMARK}
%    \begin{macrocode}
\def\@@BOOKMARK[#1][#2]#3#4#5{%
  \expandafter\edef\csname @count@#3\endcsname{\the\@serial@counter}%
  \edef\@mycount{\the\@serial@counter}%
  \Hy@StepCount\@serial@counter
  \edef\@parcount{%
    \expandafter\ifx\csname @count@#5\endcsname\relax
      0%
    \else
      \csname @count@#5\endcsname
    \fi
  }%
  \immediate\special{!outline #3;p=\@parcount,i=\@mycount,s=\ifx#2-c\else
o\fi,t=#4}%
}%
%    \end{macrocode}
%    \end{macro}
%    \begin{macro}{\ReadBookmarks}
%    \begin{macrocode}
\def\ReadBookmarks{%
  \begingroup
    \def\0{\hv@pdf@char 0}%
    \def\1{\hv@pdf@char 1}%
    \def\2{\hv@pdf@char 2}%
    \def\3{\hv@pdf@char 3}%
    \def\({(}%
    \def\){)}%
    \def\do##1{%
      \ifnum\the\catcode`##1=\active
        \@makeother##1%
      \fi
    }%
    \dospecials
    \InputIfFileExists{\jobname.out}{}{}%
  \endgroup
  \ifx\WriteBookmarks\relax
  \else
    \if@filesw
      \immediate\openout\@outlinefile=\jobname.out
      \ifHy@typexml
        \immediate\write\@outlinefile{<relaxxml>\relax}%
      \fi
    \fi
  \fi
}
%    \end{macrocode}
%    \end{macro}
%    \begin{macrocode}
%</vtex>
%    \end{macrocode}
%    \begin{macrocode}
%<*!vtex>
\def\ReadBookmarks{%
  \begingroup
    \escapechar=`\\%
    \let\escapechar\@gobble %
    \def\@@BOOKMARK [##1][##2]##3##4##5{\calc@bm@number{##5}}%
    \InputIfFileExists{\jobname.out}{}{}%
    \ifx\WriteBookmarks\relax
      \global\let\WriteBookmarks\relax
    \fi
    \def\@@BOOKMARK[##1][##2]##3##4##5{%
      \def\Hy@temp{##4}%
%<*pdftex>
      \pdfoutline goto
        name{##3}%
        count ##2\check@bm@number{##3}{%
          \expandafter\strip@prefix\meaning\Hy@temp
        }%
%</pdftex>
%<*pdfmark>
      \pdfmark{%
        pdfmark=/OUT,%
        Count=##2\check@bm@number{##3},%
        Dest={##3},%
        Title=\expandafter\strip@prefix\meaning\Hy@temp
      }%
%</pdfmark>
%<*dvipdfm>
      \@pdfm@mark{%
        outline ##1 <<
          /Title (\expandafter\strip@prefix\meaning\Hy@temp)
          /A <<
            /S /GoTo
            /D (##3)
          >>
        >>
      }%
%</dvipdfm>
   }%
   {%
    \def\WriteBookmarks{0}%
    \InputIfFileExists{\jobname.out}{}{}%
   }%
   %{\escapechar\m@ne\InputIfFileExists{\jobname.out}{}{}}%
   \ifx\WriteBookmarks\relax\else
     \if@filesw\immediate\openout\@outlinefile=\jobname.out
      \ifHy@typexml
       \immediate\write\@outlinefile{<relaxxml>\relax}%
      \fi
     \fi
   \fi
   \endgroup
}
\def\check@bm@number#1{%
  \expandafter\ifx\csname B_#1\endcsname\relax
    0%
  \else
    \csname B_#1\endcsname
  \fi
}
\def\calc@bm@number#1{%
  \@tempcnta=\check@bm@number{#1}\relax
  \advance\@tempcnta by1
  \expandafter\xdef\csname B_#1\endcsname{\the\@tempcnta}%
}
%</!vtex>
%    \end{macrocode}
%
%    This code is added, so that option `pdfpagenumbers' works
%    with option `implicit' (suggestion of Sebastian Rahtz).
%    \begin{macrocode}
\ifHy@implicit
\else
  \def\@begindvi{%
    \unvbox\@begindvibox
    \HyPL@EveryPage
    \global\let\@begindvi\HyPL@EveryPage
  }%
  \expandafter\endinput
\fi
%    \end{macrocode}
%
%    \begin{macrocode}
%</outlines>
%<*outlines|hypertex>
%    \end{macrocode}
%    \begin{macrocode}
\let\H@old@ssect\@ssect
\def\@ssect#1#2#3#4#5{%
  \H@old@ssect{#1}{#2}{#3}{#4}{#5}%
  \phantomsection
}
\let\H@old@schapter\@schapter
\def\@schapter#1{%
  \H@old@schapter{#1}%
  \begingroup
    \let\@mkboth\@gobbletwo
    \Hy@GlobalStepCount\Hy@linkcounter
    \xdef\@currentHref{\Hy@chapapp*.\the\Hy@linkcounter}%
    \Hy@raisedlink{\hyper@anchorstart{\@currentHref}\hyper@anchorend}%
  \endgroup
}
%    \end{macrocode}
%    If there is no chapter number (\cmd{\frontmatter} or
%    \cmd{\backmatter}) then |\refstepcounter{chapter}| is not
%    executed, so there will be no destination for \cmd{addcontentsline}.
%    So \cmd{\@chapter} is overloaded to avoid this:
%    \begin{macrocode}
\@ifundefined{@chapter}{}{%
  \let\Hy@org@chapter\@chapter
  \def\@chapter{%
    \def\Hy@next{%
      \Hy@GlobalStepCount\Hy@linkcounter
      \xdef\@currentHref{\Hy@chapapp*.\the\Hy@linkcounter}%
      \Hy@raisedlink{\hyper@anchorstart{\@currentHref}\hyper@anchorend}%
    }%
    \ifnum\c@secnumdepth>\m@ne
      \@ifundefined{if@mainmatter}%
      \iftrue{\csname if@mainmatter\endcsname}
        \let\Hy@next\relax
      \fi
    \fi
    \Hy@next
    \Hy@org@chapter
  }%
}
%    \end{macrocode}
%    \begin{macrocode}
\let\H@old@spart\@spart
\def\@spart#1{%
  \H@old@spart{#1}%
  \Hy@GlobalStepCount\Hy@linkcounter
  \xdef\@currentHref{part*.\the\Hy@linkcounter}%
  \Hy@raisedlink{\hyper@anchorstart{\@currentHref}\hyper@anchorend}%
}
\let\H@old@sect\@sect
\def\@sect#1#2#3#4#5#6[#7]#8{%
  \ifnum #2>\c@secnumdepth
    \Hy@GlobalStepCount\Hy@linkcounter
    \xdef\@currentHref{section*.\the\Hy@linkcounter}%
  \fi
  \H@old@sect{#1}{#2}{#3}{#4}{#5}{#6}[{#7}]{#8}%
  \ifnum #2>\c@secnumdepth
    \Hy@raisedlink{\hyper@anchorstart{\@currentHref}\hyper@anchorend}%
  \fi
}
%    \end{macrocode}
%    \begin{macrocode}
%</outlines|hypertex>
%<*outlines>
%    \end{macrocode}
%
%    \begin{macrocode}
\expandafter\def\csname Parent-4\endcsname{}
\expandafter\def\csname Parent-3\endcsname{}
\expandafter\def\csname Parent-2\endcsname{}
\expandafter\def\csname Parent-1\endcsname{}
\expandafter\def\csname Parent0\endcsname{}
\expandafter\def\csname Parent1\endcsname{}
\expandafter\def\csname Parent2\endcsname{}
\expandafter\def\csname Parent3\endcsname{}
\expandafter\def\csname Parent4\endcsname{}
%    \end{macrocode}
%
%    \begin{macrocode}
%</outlines>
%    \end{macrocode}
%
% \section{Compatibility with koma-script classes}
%
%    \begin{macrocode}
%<*outlines|hypertex>
%    \end{macrocode}
%
% Hard-wire in an unpleasant over-ride of komascript `scrbook' class
% for \Email{Tobias.Isenberg@gmx.de}.
% With version 6.71b the hack is also applied to `scrreprt' class
% and is removed for koma-script versions since 2001/01/01,
% because Markus Kohm supports hyperref in komascript.
%    \begin{macrocode}
\def\Hy@tempa{%
  \def\@addchap[##1]##2{%
    \typeout{##2}%
    \if@twoside
      \@mkboth{##1}{}%
    \else
      \@mkboth{}{##1}%
    \fi
    \addtocontents{lof}{\protect\addvspace{10\p@}}%
    \addtocontents{lot}{\protect\addvspace{10\p@}}%
    \Hy@GlobalStepCount\Hy@linkcounter
    \xdef\@currentHref{\Hy@chapapp*.\the\Hy@linkcounter}%
    \Hy@raisedlink{\hyper@anchorstart{\@currentHref}\hyper@anchorend}%
    \if@twocolumn
       \@topnewpage[\@makeschapterhead{##2}]%
    \else
       \@makeschapterhead{##2}%
       \@afterheading
    \fi
    \addcontentsline{toc}{chapter}{##1}%
  }%
}
\@ifclassloaded{scrbook}{%
  \@ifclasslater{scrbook}{2001/01/01}{%
    \let\Hy@tempa\@empty
  }{}%
}{%
  \@ifclassloaded{scrreprt}{%
    \@ifclasslater{scrreprt}{2001/01/01}{%
      \let\Hy@tempa\@empty
    }{}%
  }{%
    \let\Hy@tempa\@empty
  }%
}%
\Hy@tempa
%    \end{macrocode}
%
%    \begin{macrocode}
%</outlines|hypertex>
%    \end{macrocode}
%
%    \begin{macrocode}
%<*pdfmark>
%    \end{macrocode}
% We have to allow for |\baselineskip| having an optional
% stretch and shrink (you meet this in slide packages, for instance),
% so we need to strip off the junk. David Carlisle, of course,
% wrote this bit of code.
%    \begin{macrocode}
\begingroup
  \catcode`P=12
  \catcode`T=12
  \lowercase{\endgroup
  \gdef\rem@ptetc#1.#2PT#3!{#1\ifnum#2>\z@.#2\fi}%
}
\def\strip@pt@and@otherjunk#1{\expandafter\rem@ptetc\the#1!}
%</pdfmark>
%    \end{macrocode}
%
% \section{Encoding definition files for encodings of PDF strings}
% This was contributed by
% Heiko Oberdiek \Email{oberdiek@ruf.uni-freiburg.de}
%
% \subsection{PD1 encoding}
%    \begin{macrocode}
%<*pd1enc>
\DeclareFontEncoding{PD1}{}{}
%    \end{macrocode}
%    Special white space escape characters
%    not for use in bookmarks but for other PDF strings.
%    \begin{macrocode}
\DeclareTextCommand{\textLF}{PD1}{\012} % line feed
\DeclareTextCommand{\textCR}{PD1}{\015} % carriage return
\DeclareTextCommand{\textHT}{PD1}{\011} % horizontal tab
\DeclareTextCommand{\textBS}{PD1}{\010} % backspace
\DeclareTextCommand{\textFF}{PD1}{\014} % formfeed
%    \end{macrocode}
%    Accents
%    \begin{macrocode}
\DeclareTextAccent{\`}{PD1}{\textgrave}
\DeclareTextAccent{\'}{PD1}{\textacute}
\DeclareTextAccent{\^}{PD1}{\textcircumflex}
\DeclareTextAccent{\~}{PD1}{\texttilde}
\DeclareTextAccent{\"}{PD1}{\textdieresis}
\DeclareTextAccent{\r}{PD1}{\textring}
\DeclareTextAccent{\v}{PD1}{\textcaron}
\DeclareTextAccent{\.}{PD1}{\textdotaccent}
\DeclareTextAccent{\c}{PD1}{\textcedilla}
\DeclareTextCompositeCommand{\`}{PD1}{\@empty}{\textgrave}
\DeclareTextCompositeCommand{\'}{PD1}{\@empty}{\textacute}
\DeclareTextCompositeCommand{\^}{PD1}{\@empty}{\textcircumflex}
\DeclareTextCompositeCommand{\~}{PD1}{\@empty}{\texttilde}
\DeclareTextCompositeCommand{\"}{PD1}{\@empty}{\textdieresis}
\DeclareTextCompositeCommand{\r}{PD1}{\@empty}{\textring}
\DeclareTextCompositeCommand{\v}{PD1}{\@empty}{\textcaron}
\DeclareTextCompositeCommand{\.}{PD1}{\@empty}{\textdotaccent}
\DeclareTextCompositeCommand{\c}{PD1}{\@empty}{\textcedilla}
%    \end{macrocode}
%    Accent glyph names
%    \begin{macrocode}
\DeclareTextCommand{\textbreve}{PD1}{\030} % breve
\DeclareTextCommand{\textcaron}{PD1}{\031} % caron
\DeclareTextCommand{\textcircumflex}{PD1}{\032} % circumflex
\DeclareTextCommand{\textdotaccent}{PD1}{\033} % dotaccent
\DeclareTextCommand{\texthungarumlaut}{PD1}{\034} % hungarumlaut
\DeclareTextCommand{\textogonek}{PD1}{\035} % ogonek
\DeclareTextCommand{\textring}{PD1}{\036} % ring
\DeclareTextCommand{\texttilde}{PD1}{\037} % tilde
%    \end{macrocode}
%    \cs{040}: space\\
%    \cs{041}: exclam
%    \begin{macrocode}
\DeclareTextCommand{\textquotedbl}{PD1}{\string"} % quotedbl \042
\DeclareTextCommand{\textnumbersign}{PD1}{\043} % numbersign
\DeclareTextCommand{\textdollar}{PD1}{\044} % dollar
\DeclareTextCommand{\textpercent}{PD1}{\045} % percent
\DeclareTextCommand{\textampersand}{PD1}{\046} % ampersand
%    \end{macrocode}
%    \cs{047}: quotesingle
%    \begin{macrocode}
\DeclareTextCommand{\textparenleft}{PD1}{\string\(} % parenleft \050
\DeclareTextCommand{\textparenright}{PD1}{\string\)} % parenright \051
%    \end{macrocode}
%    \cs{052}: asterisk\\
%    \cs{053}: plus\\
%    \cs{054}: comma\\
%    \cs{055}: hyphen\\
%    \cs{056}: period\\
%    \cs{057}: slash\\
%    \cs{060}\dots\cs{071}: 0\dots9\\
%    \cs{072}: colon\\
%    \cs{073}: semicolon
%    \begin{macrocode}
\DeclareTextCommand{\textless}{PD1}{<} % less \074
%    \end{macrocode}
%    \cs{075}: equal
%    \begin{macrocode}
\DeclareTextCommand{\textgreater}{PD1}{>} % greater \076
%    \end{macrocode}
%    \cs{077}: question\\
%    \cs{100}: at\\
%    \cs{101}\dots\cs{132}: A\dots Z\\
%    \cs{133}: bracketleft
%    \begin{macrocode}
\DeclareTextCommand{\textbackslash}{PD1}{\134} % backslash
%    \end{macrocode}
%    \cs{135}: bracketright
%    \begin{macrocode}
\DeclareTextCommand{\textasciicircum}{PD1}{\136} % asciicircum
\DeclareTextCommand{\textunderscore}{PD1}{\137} % underscore
\DeclareTextCommand{\textgrave}{PD1}{\140} % grave
%    \end{macrocode}
%    \cs{141}\dots\cs{172}: a\dots z
%    \begin{macrocode}
\DeclareTextCompositeCommand{\.}{PD1}{i}{i} % i
\DeclareTextCommand{\textbraceleft}{PD1}{\173} % braceleft
\DeclareTextCommand{\textbar}{PD1}{|} % bar
\DeclareTextCommand{\textbraceright}{PD1}{\175} % braceright
\DeclareTextCommand{\textasciitilde}{PD1}{\176} % asciitilde
%    \end{macrocode}
%    No glyph \cs{177} in PDFDocEncoding.
%    \begin{macrocode}
\DeclareTextCommand{\textbullet}{PD1}{\200} % bullet
\DeclareTextCommand{\textdagger}{PD1}{\201} % dagger
\DeclareTextCommand{\textdaggerdbl}{PD1}{\202} % daggerdbl
\DeclareTextCommand{\textellipsis}{PD1}{\203} % ellipsis
\DeclareTextCommand{\textemdash}{PD1}{\204} % emdash
\DeclareTextCommand{\textendash}{PD1}{\205} % endash
\DeclareTextCommand{\textflorin}{PD1}{\206} % florin
\DeclareTextCommand{\textfractionmark}{PD1}{/} % fraction, \207
\DeclareTextCommand{\guilsinglleft}{PD1}{\210} % guilsinglleft
\DeclareTextCommand{\guilsinglright}{PD1}{\211} % guilsinglright
\DeclareTextCommand{\textminus}{PD1}{-} % minus, \212
\DeclareTextCommand{\textperthousand}{PD1}{\213} % perthousand
\DeclareTextCommand{\quotedblbase}{PD1}{\214} % quotedblbase
\DeclareTextCommand{\textquotedblleft}{PD1}{\215} % quotedblleft
\DeclareTextCommand{\textquotedblright}{PD1}{\216} % quotedblright
\DeclareTextCommand{\textquoteleft}{PD1}{\217} % quoteleft
\DeclareTextCommand{\textquoteright}{PD1}{\220} % quoteright
\DeclareTextCommand{\quotesinglbase}{PD1}{\221} % quotesinglbase
\DeclareTextCommand{\texttrademark}{PD1}{\222} % trademark
\DeclareTextCommand{\textfi}{PD1}{fi} % fi ?? \223
\DeclareTextCommand{\textfl}{PD1}{fl} % fl ?? \224
\DeclareTextCommand{\L}{PD1}{L} % Lslash, \225
\DeclareTextCommand{\OE}{PD1}{\226} % OE
\DeclareTextCompositeCommand{\v}{PD1}{S}{\227} % Scaron
\DeclareTextCompositeCommand{\"}{PD1}{Y}{\230} % Ydieresis
\DeclareTextCommand{\IJ}{PD1}{\230}
\DeclareTextCompositeCommand{\v}{PD1}{Z}{Z} % Zcaron, \231
\DeclareTextCommand{\i}{PD1}{i} % dotlessi, \232
\DeclareTextCommand{\l}{PD1}{l} % lslash, \233
\DeclareTextCommand{\oe}{PD1}{\234} % oe
\DeclareTextCompositeCommand{\v}{PD1}{s}{\235} % scaron
\DeclareTextCompositeCommand{\v}{PD1}{z}{z} % zcaron, 236
%    \end{macrocode}
%    No glyph \cs{237} in PDFDocEncoding.\\
%    The euro \cs{240} is inserted in version 1.3 of the pdf
%    specification.
%    \begin{macrocode}
\DeclareTextCommand{\texteuro}{PD1}{\240} % Euro
\DeclareTextCommand{\textexclamdown}{PD1}{\241} % exclamdown
\DeclareTextCommand{\textcent}{PD1}{\242} % cent
\DeclareTextCommand{\textsterling}{PD1}{\243} % sterling
\DeclareTextCommand{\textcurrency}{PD1}{\244} % currency
\DeclareTextCommand{\textyen}{PD1}{\245} % yen
\DeclareTextCommand{\textbrokenbar}{PD1}{\246} % brokenbar
\DeclareTextCommand{\textsection}{PD1}{\247} % section
\DeclareTextCommand{\textdieresis}{PD1}{\250} % dieresis
\DeclareTextCommand{\textcopyright}{PD1}{\251} % copyright
\DeclareTextCommand{\textordfeminine}{PD1}{\252} % ordfeminine
\DeclareTextCommand{\guillemotleft}{PD1}{\253} % guillemotleft
\DeclareTextCommand{\textlogicalnot}{PD1}{\254} % logicalnot
%    \end{macrocode}
%    No glyph \cs{255} in PDFDocEncoding.
%    \begin{macrocode}
\DeclareTextCommand{\textregistered}{PD1}{\256} % registered
\DeclareTextCommand{\textmacron}{PD1}{\257} % macron
\DeclareTextCommand{\textdegree}{PD1}{\260} % degree
\DeclareTextCommand{\textplusminus}{PD1}{\261} % plusminus
\DeclareTextCommand{\texttwosuperior}{PD1}{\262} % twosuperior
\DeclareTextCommand{\textthreesuperior}{PD1}{\263} % threesuperior
\DeclareTextCommand{\textacute}{PD1}{\264} % acute
\DeclareTextCommand{\textmu}{PD1}{\265} % mu
\DeclareTextCommand{\textparagraph}{PD1}{\266} % paragraph
\DeclareTextCommand{\textperiodcentered}{PD1}{\267} % periodcentered
\DeclareTextCommand{\textcedilla}{PD1}{\270} % cedilla
\DeclareTextCommand{\textonesuperior}{PD1}{\271} % onesuperior
\DeclareTextCommand{\textordmasculine}{PD1}{\272} % ordmasculine
\DeclareTextCommand{\guillemotright}{PD1}{\273} % guillemotright
\DeclareTextCommand{\textonequarter}{PD1}{\274} % onequarter
\DeclareTextCommand{\textonehalf}{PD1}{\275} % onehalf
\DeclareTextCommand{\textthreequarters}{PD1}{\276} % threequarters
\DeclareTextCommand{\textquestiondown}{PD1}{\277} % questiondown
\DeclareTextCompositeCommand{\`}{PD1}{A}{\300} % Agrave
\DeclareTextCompositeCommand{\'}{PD1}{A}{\301} % Aacute
\DeclareTextCompositeCommand{\^}{PD1}{A}{\302} % Acircumflex
\DeclareTextCompositeCommand{\~}{PD1}{A}{\303} % Atilde
\DeclareTextCompositeCommand{\"}{PD1}{A}{\304} % Adieresis
\DeclareTextCompositeCommand{\r}{PD1}{A}{\305} % Aring
\DeclareTextCommand{\AE}{PD1}{\306} % AE
\DeclareTextCompositeCommand{\c}{PD1}{C}{\307} % Ccedilla
\DeclareTextCompositeCommand{\`}{PD1}{E}{\310} % Egrave
\DeclareTextCompositeCommand{\'}{PD1}{E}{\311} % Eacute
\DeclareTextCompositeCommand{\^}{PD1}{E}{\312} % Ecircumflex
\DeclareTextCompositeCommand{\"}{PD1}{E}{\313} % Edieresis
\DeclareTextCompositeCommand{\`}{PD1}{I}{\314} % Igrave
\DeclareTextCompositeCommand{\'}{PD1}{I}{\315} % Iacute
\DeclareTextCompositeCommand{\^}{PD1}{I}{\316} % Icircumflex
\DeclareTextCompositeCommand{\"}{PD1}{I}{\317} % Idieresis
\DeclareTextCommand{\DH}{PD1}{\320} % Eth
\DeclareTextCommand{\DJ}{PD1}{\320} % Eth
\DeclareTextCompositeCommand{\~}{PD1}{N}{\321} % Ntilde
\DeclareTextCompositeCommand{\`}{PD1}{O}{\322} % Ograve
\DeclareTextCompositeCommand{\'}{PD1}{O}{\323} % Oacute
\DeclareTextCompositeCommand{\^}{PD1}{O}{\324} % Ocircumflex
\DeclareTextCompositeCommand{\~}{PD1}{O}{\325} % Otilde
\DeclareTextCompositeCommand{\"}{PD1}{O}{\326} % Odieresis
\DeclareTextCommand{\textmultiply}{PD1}{\327} % multiply
\DeclareTextCommand{\O}{PD1}{\330} % Oslash
\DeclareTextCompositeCommand{\`}{PD1}{U}{\331} % Ugrave
\DeclareTextCompositeCommand{\'}{PD1}{U}{\332} % Uacute
\DeclareTextCompositeCommand{\^}{PD1}{U}{\333} % Ucircumflex
\DeclareTextCompositeCommand{\"}{PD1}{U}{\334} % Udieresis
\DeclareTextCompositeCommand{\'}{PD1}{Y}{\335} % Yacute
\DeclareTextCommand{\TH}{PD1}{\336} % Thorn
\DeclareTextCommand{\ss}{PD1}{\337} % germandbls
\DeclareTextCompositeCommand{\`}{PD1}{a}{\340} % agrave
\DeclareTextCompositeCommand{\'}{PD1}{a}{\341} % aacute
\DeclareTextCompositeCommand{\^}{PD1}{a}{\342} % acircumflex
\DeclareTextCompositeCommand{\~}{PD1}{a}{\343} % atilde
\DeclareTextCompositeCommand{\"}{PD1}{a}{\344} % adieresis
\DeclareTextCompositeCommand{\r}{PD1}{a}{\345} % aring
\DeclareTextCommand{\ae}{PD1}{\346} % ae
\DeclareTextCompositeCommand{\c}{PD1}{c}{\347} % ccedilla
\DeclareTextCompositeCommand{\`}{PD1}{e}{\350} % egrave
\DeclareTextCompositeCommand{\'}{PD1}{e}{\351} % eacute
\DeclareTextCompositeCommand{\^}{PD1}{e}{\352} % ecircumflex
\DeclareTextCompositeCommand{\"}{PD1}{e}{\353} % edieresis
\DeclareTextCompositeCommand{\`}{PD1}{i}{\354} % igrave
\DeclareTextCompositeCommand{\`}{PD1}{\i}{\354} % igrave
\DeclareTextCompositeCommand{\'}{PD1}{i}{\355} % iacute
\DeclareTextCompositeCommand{\'}{PD1}{\i}{\355} % iacute
\DeclareTextCompositeCommand{\^}{PD1}{i}{\356} % icircumflex
\DeclareTextCompositeCommand{\^}{PD1}{\i}{\356} % icircumflex
\DeclareTextCompositeCommand{\"}{PD1}{i}{\357} % idieresis
\DeclareTextCompositeCommand{\"}{PD1}{\i}{\357} % idieresis
\DeclareTextCommand{\dh}{PD1}{\360} % eth
\DeclareTextCompositeCommand{\~}{PD1}{n}{\361} % ntilde
\DeclareTextCompositeCommand{\`}{PD1}{o}{\362} % ograve
\DeclareTextCompositeCommand{\'}{PD1}{o}{\363} % oacute
\DeclareTextCompositeCommand{\^}{PD1}{o}{\364} % ocircumflex
\DeclareTextCompositeCommand{\~}{PD1}{o}{\365} % otilde
\DeclareTextCompositeCommand{\"}{PD1}{o}{\366} % odieresis
\DeclareTextCommand{\textdivide}{PD1}{\367} % divide
\DeclareTextCommand{\o}{PD1}{\370} % oslash
\DeclareTextCompositeCommand{\`}{PD1}{u}{\371} % ugrave
\DeclareTextCompositeCommand{\'}{PD1}{u}{\372} % uacute
\DeclareTextCompositeCommand{\^}{PD1}{u}{\373} % ucircumflex
\DeclareTextCompositeCommand{\"}{PD1}{u}{\374} % udieresis
\DeclareTextCompositeCommand{\'}{PD1}{y}{\375} % yacute
\DeclareTextCommand{\th}{PD1}{\376} % thorn
\DeclareTextCompositeCommand{\"}{PD1}{y}{\377} % ydieresis
\DeclareTextCommand{\ij}{PD1}{\377}
%    \end{macrocode}
%    Glyphs that consist of several characters.
%    \begin{macrocode}
\DeclareTextCommand{\SS}{PD1}{SS}
\DeclareTextCommand{\textcelsius}{PD1}{\textdegree C}
%    \end{macrocode}
%    Aliases (german.sty)
%    \begin{macrocode}
\DeclareTextCommand{\textglqq}{PD1}{\quotedblbase}
\DeclareTextCommand{\textgrqq}{PD1}{\textquotedblleft}
\DeclareTextCommand{\textglq}{PD1}{\quotesinglbase}
\DeclareTextCommand{\textgrq}{PD1}{\textquoteleft}
\DeclareTextCommand{\textflqq}{PD1}{\guillemotleft}
\DeclareTextCommand{\textfrqq}{PD1}{\guillemotright}
\DeclareTextCommand{\textflq}{PD1}{\guilsinglleft}
\DeclareTextCommand{\textfrq}{PD1}{\guilsinglright}
%    \end{macrocode}
%    Aliases (math names)
%    \begin{macrocode}
\DeclareTextCommand{\textneg}{PD1}{\textlogicalnot}
\DeclareTextCommand{\texttimes}{PD1}{\textmultiply}
\DeclareTextCommand{\textdiv}{PD1}{\textdivide}
\DeclareTextCommand{\textpm}{PD1}{\textplusminus}
\DeclareTextCommand{\textcdot}{PD1}{\textperiodcentered}
\DeclareTextCommand{\textbeta}{PD1}{\ss}
%    \end{macrocode}
% Polish aliases. PDF encoding does not have the characters, but it
% is useful to Poles to have the plain letters regardless. Requested by
%  Wojciech Myszka <W.Myszka@immt.pwr.wroc.pl>
%    \begin{macrocode}
\DeclareTextCompositeCommand{\k}{PD1}{a}{a} % aogonek
\DeclareTextCompositeCommand{\'}{PD1}{c}{c} % cacute
\DeclareTextCompositeCommand{\k}{PD1}{e}{e} % eogonek
\DeclareTextCompositeCommand{\'}{PD1}{n}{n} % nacute
\DeclareTextCompositeCommand{\'}{PD1}{s}{s} % sacute
\DeclareTextCompositeCommand{\'}{PD1}{z}{z} % zacute
\DeclareTextCompositeCommand{\.}{PD1}{z}{z} % zdot
%    \end{macrocode}
%    \begin{macrocode}
\DeclareTextCompositeCommand{\k}{PD1}{A}{A} % Aogonek
\DeclareTextCompositeCommand{\'}{PD1}{C}{C} % Cacute
\DeclareTextCompositeCommand{\k}{PD1}{E}{E} % Eogonek
\DeclareTextCompositeCommand{\'}{PD1}{N}{N} % Nacute
\DeclareTextCompositeCommand{\'}{PD1}{S}{S} % Sacute
\DeclareTextCompositeCommand{\'}{PD1}{Z}{Z} % Zacute
\DeclareTextCompositeCommand{\.}{PD1}{Z}{Z} % Zdot
%    \end{macrocode}
%    \begin{macrocode}
%</pd1enc>
%    \end{macrocode}
%
% \subsection{PU encoding}
%    \begin{macrocode}
%<*puenc>
\DeclareFontEncoding{PU}{}{}
%    \end{macrocode}
% \subsubsection{Basic Latin}
%    Special white space escape characters
%    not for use in bookmarks but for other PDF strings.
%    \begin{macrocode}
\DeclareTextCommand{\textLF}{PU}{\80\012} % line feed
\DeclareTextCommand{\textCR}{PU}{\80\015} % carriage return
\DeclareTextCommand{\textHT}{PU}{\80\011} % horizontal tab
\DeclareTextCommand{\textBS}{PU}{\80\010} % backspace
\DeclareTextCommand{\textFF}{PU}{\80\014} % formfeed
%    \end{macrocode}
%    Accents
%    \begin{macrocode}
\DeclareTextAccent{\`}{PU}{\textgrave}
\DeclareTextAccent{\'}{PU}{\textacute}
\DeclareTextAccent{\^}{PU}{\textcircumflex}
\DeclareTextAccent{\~}{PU}{\texttilde}
\DeclareTextAccent{\"}{PU}{\textdieresis}
\DeclareTextAccent{\r}{PU}{\textring}
\DeclareTextAccent{\v}{PU}{\textcaron}
\DeclareTextAccent{\.}{PU}{\textdotaccent}
\DeclareTextAccent{\c}{PU}{\textcedilla}
\DeclareTextAccent{\U}{PU}{\textbreve}
\DeclareTextAccent{\C}{PU}{\textdoublegrave}
\DeclareTextCompositeCommand{\`}{PU}{\@empty}{\textgrave}
\DeclareTextCompositeCommand{\'}{PU}{\@empty}{\textacute}
\DeclareTextCompositeCommand{\^}{PU}{\@empty}{\textcircumflex}
\DeclareTextCompositeCommand{\~}{PU}{\@empty}{\texttilde}
\DeclareTextCompositeCommand{\"}{PU}{\@empty}{\textdieresis}
\DeclareTextCompositeCommand{\r}{PU}{\@empty}{\textring}
\DeclareTextCompositeCommand{\v}{PU}{\@empty}{\textcaron}
\DeclareTextCompositeCommand{\.}{PU}{\@empty}{\textdotaccent}
\DeclareTextCompositeCommand{\c}{PU}{\@empty}{\textcedilla}
\DeclareTextCompositeCommand{\U}{PU}{\@empty}{\textbreve}
\DeclareTextCompositeCommand{\C}{PU}{\@empty}{\textdoublegrave}
%    \end{macrocode}
%    Accent glyph names
%    \begin{macrocode}
\DeclareTextCommand{\textbreve}{PU}{\80\030} % breve
\DeclareTextCommand{\textcaron}{PU}{\80\031} % caron
\DeclareTextCommand{\textcircumflex}{PU}{\80\032} % circumflex
\DeclareTextCommand{\textdotaccent}{PU}{\80\033} % dotaccent
\DeclareTextCommand{\texthungarumlaut}{PU}{\80\034} % hungarumlaut
\DeclareTextCommand{\textogonek}{PU}{\80\035} % ogonek
\DeclareTextCommand{\textring}{PU}{\80\036} % ring
\DeclareTextCommand{\texttilde}{PU}{\80\037} % tilde
\DeclareTextCommand{\textdoublegrave}{PU}{\83\017} % double grave
%    \end{macrocode}
%    \cs{040}: space\\
%    \cs{041}: exclam
%    \begin{macrocode}
\DeclareTextCommand{\textquotedbl}{PU}{\string"} % quotedbl \80\042
\DeclareTextCommand{\textnumbersign}{PU}{\80\043} % numbersign
\DeclareTextCommand{\textdollar}{PU}{\80\044} % dollar
\DeclareTextCommand{\textpercent}{PU}{\80\045} % percent
\DeclareTextCommand{\textampersand}{PU}{\80\046} % ampersand
%    \end{macrocode}
%    \cs{047}: quotesingle
%    \begin{macrocode}
\DeclareTextCommand{\textparenleft}{PU}{\80\050} % parenleft
\DeclareTextCommand{\textparenright}{PU}{\80\051} % parenright
%    \end{macrocode}
%    \cs{052}: asterisk\\
%    \cs{053}: plus\\
%    \cs{054}: comma\\
%    \cs{055}: hyphen\\
%    \cs{056}: period\\
%    \cs{057}: slash\\
%    \cs{060}\dots\cs{071}: 0\dots9\\
%    \cs{072}: colon\\
%    \cs{073}: semicolon
%    \begin{macrocode}
\DeclareTextCommand{\textless}{PU}{<} % less \80\074
%    \end{macrocode}
%    \cs{075}: equal
%    \begin{macrocode}
\DeclareTextCommand{\textgreater}{PU}{>} % greater \80\076
%    \end{macrocode}
%    \cs{077}: question\\
%    \cs{100}: at\\
%    \cs{101}\dots\cs{132}: A\dots Z\\
%    \cs{133}: bracketleft
%    \begin{macrocode}
\DeclareTextCommand{\textbackslash}{PU}{\80\134} % backslash
%    \end{macrocode}
%    \cs{135}: bracketright
%    \begin{macrocode}
\DeclareTextCommand{\textasciicircum}{PU}{\80\136} % asciicircum
\DeclareTextCommand{\textunderscore}{PU}{\80\137} % underscore
\DeclareTextCommand{\textgrave}{PU}{\80\140} % grave
%    \end{macrocode}
%    \cs{141}\dots\cs{172}: a\dots z
%    \begin{macrocode}
\DeclareTextCompositeCommand{\.}{PU}{i}{i} % i
\DeclareTextCommand{\j}{PU}{j} % jdotless
\DeclareTextCommand{\textbraceleft}{PU}{\80\173} % braceleft
\DeclareTextCommand{\textbar}{PU}{|} % bar
\DeclareTextCommand{\textbraceright}{PU}{\80\175} % braceright
\DeclareTextCommand{\textasciitilde}{PU}{\80\176} % asciitilde
%    \end{macrocode}
%    No glyph \cs{177} in PDFDocEncoding.
%
% \subsubsection{Latin-1 Supplement}
%    \begin{macrocode}
\DeclareTextCommand{\textbullet}{PU}{\80\200} % bullet
\DeclareTextCommand{\textdagger}{PU}{\80\201} % dagger
\DeclareTextCommand{\textdaggerdbl}{PU}{\80\202} % daggerdbl
\DeclareTextCommand{\textellipsis}{PU}{\80\203} % ellipsis
\DeclareTextCommand{\textemdash}{PU}{\80\204} % emdash
\DeclareTextCommand{\textendash}{PU}{\80\205} % endash
\DeclareTextCommand{\textflorin}{PU}{\80\206} % florin
\DeclareTextCommand{\textfractionmark}{PU}{/} % fraction, \80\207
\DeclareTextCommand{\guilsinglleft}{PU}{\80\210} % guilsinglleft
\DeclareTextCommand{\guilsinglright}{PU}{\80\211} % guilsinglright
\DeclareTextCommand{\textminus}{PU}{-} % minus, \80\212
\DeclareTextCommand{\textperthousand}{PU}{\80\213} % perthousand
\DeclareTextCommand{\quotedblbase}{PU}{\80\214} % quotedblbase
\DeclareTextCommand{\textquotedblleft}{PU}{\80\215} % quotedblleft
\DeclareTextCommand{\textquotedblright}{PU}{\80\216} % quotedblright
\DeclareTextCommand{\textquoteleft}{PU}{\80\217} % quoteleft
\DeclareTextCommand{\textquoteright}{PU}{\80\220} % quoteright
\DeclareTextCommand{\quotesinglbase}{PU}{\80\221} % quotesinglbase
\DeclareTextCommand{\texttrademark}{PU}{\80\222} % trademark
\DeclareTextCommand{\textfi}{PU}{fi} % fi ?? \80\223
\DeclareTextCommand{\textfl}{PU}{fl} % fl ?? \80\224
%    \end{macrocode}
%    There are two positions for the glyphs from |\80\225| until
%    |\80\236|. See the test files |testbmoe.|, |testbml|,
%    |testbmsc|, |testbmzc|, and |testbmyi| for details.
%    Problematic are all positions, but especially \cmd{\OE},
%    \cmd{\oe}, and \cmd{\i}.
%    \begin{macrocode}
\DeclareTextCommand{\OE}{PU}{\80\226} % OE
\DeclareTextCommand{\oe}{PU}{\80\234} % oe
%    \end{macrocode}
%    The euro \cs{240} is inserted in version 1.3 of the pdf
%    specification.
%    \begin{macrocode}
\DeclareTextCommand{\texteuro}{PU}{\80\240} % Euro
\DeclareTextCommand{\textexclamdown}{PU}{\80\241} % exclamdown
\DeclareTextCommand{\textcent}{PU}{\80\242} % cent
\DeclareTextCommand{\textsterling}{PU}{\80\243} % sterling
\DeclareTextCommand{\textcurrency}{PU}{\80\244} % currency
\DeclareTextCommand{\textyen}{PU}{\80\245} % yen
\DeclareTextCommand{\textbrokenbar}{PU}{\80\246} % brokenbar
\DeclareTextCommand{\textsection}{PU}{\80\247} % section
\DeclareTextCommand{\textdieresis}{PU}{\80\250} % dieresis
\DeclareTextCommand{\textcopyright}{PU}{\80\251} % copyright
\DeclareTextCommand{\textordfeminine}{PU}{\80\252} % ordfeminine
\DeclareTextCommand{\guillemotleft}{PU}{\80\253} % guillemotleft
\DeclareTextCommand{\textlogicalnot}{PU}{\80\254} % logicalnot
%    \end{macrocode}
%    No glyph \cs{255} in PDFDocEncoding.
%    \begin{macrocode}
\DeclareTextCommand{\textregistered}{PU}{\80\256} % registered
\DeclareTextCommand{\textmacron}{PU}{\80\257} % macron
\DeclareTextCommand{\textdegree}{PU}{\80\260} % degree
\DeclareTextCommand{\textplusminus}{PU}{\80\261} % plusminus
\DeclareTextCommand{\texttwosuperior}{PU}{\80\262} % twosuperior
\DeclareTextCommand{\textthreesuperior}{PU}{\80\263} % threesuperior
\DeclareTextCommand{\textacute}{PU}{\80\264} % acute
\DeclareTextCommand{\textmu}{PU}{\80\265} % mu
\DeclareTextCommand{\textparagraph}{PU}{\80\266} % paragraph
\DeclareTextCommand{\textperiodcentered}{PU}{\80\267} % periodcentered
\DeclareTextCommand{\textcedilla}{PU}{\80\270} % cedilla
\DeclareTextCommand{\textonesuperior}{PU}{\80\271} % onesuperior
\DeclareTextCommand{\textordmasculine}{PU}{\80\272} % ordmasculine
\DeclareTextCommand{\guillemotright}{PU}{\80\273} % guillemotright
\DeclareTextCommand{\textonequarter}{PU}{\80\274} % onequarter
\DeclareTextCommand{\textonehalf}{PU}{\80\275} % onehalf
\DeclareTextCommand{\textthreequarters}{PU}{\80\276} % threequarters
\DeclareTextCommand{\textquestiondown}{PU}{\80\277} % questiondown
\DeclareTextCompositeCommand{\`}{PU}{A}{\80\300} % Agrave
\DeclareTextCompositeCommand{\'}{PU}{A}{\80\301} % Aacute
\DeclareTextCompositeCommand{\^}{PU}{A}{\80\302} % Acircumflex
\DeclareTextCompositeCommand{\~}{PU}{A}{\80\303} % Atilde
\DeclareTextCompositeCommand{\"}{PU}{A}{\80\304} % Adieresis
\DeclareTextCompositeCommand{\r}{PU}{A}{\80\305} % Aring
\DeclareTextCommand{\AE}{PU}{\80\306} % AE
\DeclareTextCompositeCommand{\c}{PU}{C}{\80\307} % Ccedilla
\DeclareTextCompositeCommand{\`}{PU}{E}{\80\310} % Egrave
\DeclareTextCompositeCommand{\'}{PU}{E}{\80\311} % Eacute
\DeclareTextCompositeCommand{\^}{PU}{E}{\80\312} % Ecircumflex
\DeclareTextCompositeCommand{\"}{PU}{E}{\80\313} % Edieresis
\DeclareTextCompositeCommand{\`}{PU}{I}{\80\314} % Igrave
\DeclareTextCompositeCommand{\'}{PU}{I}{\80\315} % Iacute
\DeclareTextCompositeCommand{\^}{PU}{I}{\80\316} % Icircumflex
\DeclareTextCompositeCommand{\"}{PU}{I}{\80\317} % Idieresis
\DeclareTextCommand{\DH}{PU}{\80\320} % Eth
\DeclareTextCompositeCommand{\~}{PU}{N}{\80\321} % Ntilde
\DeclareTextCompositeCommand{\`}{PU}{O}{\80\322} % Ograve
\DeclareTextCompositeCommand{\'}{PU}{O}{\80\323} % Oacute
\DeclareTextCompositeCommand{\^}{PU}{O}{\80\324} % Ocircumflex
\DeclareTextCompositeCommand{\~}{PU}{O}{\80\325} % Otilde
\DeclareTextCompositeCommand{\"}{PU}{O}{\80\326} % Odieresis
\DeclareTextCommand{\textmultiply}{PU}{\80\327} % multiply
\DeclareTextCommand{\O}{PU}{\80\330} % Oslash
\DeclareTextCompositeCommand{\`}{PU}{U}{\80\331} % Ugrave
\DeclareTextCompositeCommand{\'}{PU}{U}{\80\332} % Uacute
\DeclareTextCompositeCommand{\^}{PU}{U}{\80\333} % Ucircumflex
\DeclareTextCompositeCommand{\"}{PU}{U}{\80\334} % Udieresis
\DeclareTextCompositeCommand{\'}{PU}{Y}{\80\335} % Yacute
\DeclareTextCommand{\TH}{PU}{\80\336} % Thorn
\DeclareTextCommand{\ss}{PU}{\80\337} % germandbls
\DeclareTextCompositeCommand{\`}{PU}{a}{\80\340} % agrave
\DeclareTextCompositeCommand{\'}{PU}{a}{\80\341} % aacute
\DeclareTextCompositeCommand{\^}{PU}{a}{\80\342} % acircumflex
\DeclareTextCompositeCommand{\~}{PU}{a}{\80\343} % atilde
\DeclareTextCompositeCommand{\"}{PU}{a}{\80\344} % adieresis
\DeclareTextCompositeCommand{\r}{PU}{a}{\80\345} % aring
\DeclareTextCommand{\ae}{PU}{\80\346} % ae
\DeclareTextCompositeCommand{\c}{PU}{c}{\80\347} % ccedilla
\DeclareTextCompositeCommand{\`}{PU}{e}{\80\350} % egrave
\DeclareTextCompositeCommand{\'}{PU}{e}{\80\351} % eacute
\DeclareTextCompositeCommand{\^}{PU}{e}{\80\352} % ecircumflex
\DeclareTextCompositeCommand{\"}{PU}{e}{\80\353} % edieresis
\DeclareTextCompositeCommand{\`}{PU}{i}{\80\354} % igrave
\DeclareTextCompositeCommand{\`}{PU}{\i}{\80\354} % igrave
\DeclareTextCompositeCommand{\'}{PU}{i}{\80\355} % iacute
\DeclareTextCompositeCommand{\'}{PU}{\i}{\80\355} % iacute
\DeclareTextCompositeCommand{\^}{PU}{i}{\80\356} % icircumflex
\DeclareTextCompositeCommand{\^}{PU}{\i}{\80\356} % icircumflex
\DeclareTextCompositeCommand{\"}{PU}{i}{\80\357} % idieresis
\DeclareTextCompositeCommand{\"}{PU}{\i}{\80\357} % idieresis
\DeclareTextCommand{\dh}{PU}{\80\360} % eth
\DeclareTextCompositeCommand{\~}{PU}{n}{\80\361} % ntilde
\DeclareTextCompositeCommand{\`}{PU}{o}{\80\362} % ograve
\DeclareTextCompositeCommand{\'}{PU}{o}{\80\363} % oacute
\DeclareTextCompositeCommand{\^}{PU}{o}{\80\364} % ocircumflex
\DeclareTextCompositeCommand{\~}{PU}{o}{\80\365} % otilde
\DeclareTextCompositeCommand{\"}{PU}{o}{\80\366} % odieresis
\DeclareTextCommand{\textdivide}{PU}{\80\367} % divide
\DeclareTextCommand{\o}{PU}{\80\370} % oslash
\DeclareTextCompositeCommand{\`}{PU}{u}{\80\371} % ugrave
\DeclareTextCompositeCommand{\'}{PU}{u}{\80\372} % uacute
\DeclareTextCompositeCommand{\^}{PU}{u}{\80\373} % ucircumflex
\DeclareTextCompositeCommand{\"}{PU}{u}{\80\374} % udieresis
\DeclareTextCompositeCommand{\'}{PU}{y}{\80\375} % yacute
\DeclareTextCommand{\th}{PU}{\80\376} % thorn
\DeclareTextCompositeCommand{\"}{PU}{y}{\80\377} % ydieresis
\DeclareTextCommand{\ij}{PU}{\80\377}
%    \end{macrocode}
%    Glyphs that consist of several characters.
%    \begin{macrocode}
\DeclareTextCommand{\SS}{PU}{SS}
\DeclareTextCommand{\textcelsius}{PU}{\textdegree C}
%    \end{macrocode}
%    Aliases (german.sty)
%    \begin{macrocode}
\DeclareTextCommand{\textglqq}{PU}{\quotedblbase}
\DeclareTextCommand{\textgrqq}{PU}{\textquotedblleft}
\DeclareTextCommand{\textglq}{PU}{\quotesinglbase}
\DeclareTextCommand{\textgrq}{PU}{\textquoteleft}
\DeclareTextCommand{\textflqq}{PU}{\guillemotleft}
\DeclareTextCommand{\textfrqq}{PU}{\guillemotright}
\DeclareTextCommand{\textflq}{PU}{\guilsinglleft}
\DeclareTextCommand{\textfrq}{PU}{\guilsinglright}
%    \end{macrocode}
%    Aliases (math names)
%    \begin{macrocode}
\DeclareTextCommand{\textneg}{PU}{\textlogicalnot}
\DeclareTextCommand{\texttimes}{PU}{\textmultiply}
\DeclareTextCommand{\textdiv}{PU}{\textdivide}
\DeclareTextCommand{\textpm}{PU}{\textplusminus}
\DeclareTextCommand{\textcdot}{PU}{\textperiodcentered}
\DeclareTextCommand{\textbeta}{PU}{\ss}
%    \end{macrocode}
%
% \subsubsection{Latin Extended-A}
%    \begin{macrocode}
\DeclareTextCompositeCommand{\=}{PU}{A}{\81\000}% Amacron
\DeclareTextCompositeCommand{\=}{PU}{a}{\81\001}% amacron
\DeclareTextCompositeCommand{\u}{PU}{A}{\81\002}% Abreve
\DeclareTextCompositeCommand{\u}{PU}{a}{\81\003}% abreve
\DeclareTextCompositeCommand{\k}{PU}{A}{\81\004} % Aogonek
\DeclareTextCompositeCommand{\k}{PU}{a}{\81\005} % aogonek
\DeclareTextCompositeCommand{\'}{PU}{C}{\81\006} % Cacute
\DeclareTextCompositeCommand{\'}{PU}{c}{\81\007} % cacute
\DeclareTextCompositeCommand{\^}{PU}{C}{\81\010} % Ccircumflex
\DeclareTextCompositeCommand{\^}{PU}{c}{\81\011} % ccircumflex
\DeclareTextCompositeCommand{\.}{PU}{C}{\81\012} % Cdot
\DeclareTextCompositeCommand{\.}{PU}{c}{\81\013} % cdot
\DeclareTextCompositeCommand{\v}{PU}{C}{\81\014} % Ccaron
\DeclareTextCompositeCommand{\v}{PU}{c}{\81\015} % ccaron
\DeclareTextCompositeCommand{\v}{PU}{D}{\81\016} % Dcaron
\DeclareTextCompositeCommand{\v}{PU}{d}{\81\017} % dcaron
\DeclareTextCommand{\DJ}{PU}{\81\020} % Dslash
\DeclareTextCommand{\dj}{PU}{\81\021} % dslash
\DeclareTextCompositeCommand{\=}{PU}{E}{\81\022} % Emacron
\DeclareTextCompositeCommand{\=}{PU}{e}{\81\023} % emacron
\DeclareTextCompositeCommand{\u}{PU}{E}{\81\024} % Ebreve
\DeclareTextCompositeCommand{\u}{PU}{e}{\81\025} % ebreve
\DeclareTextCompositeCommand{\.}{PU}{E}{\81\026} % Edot
\DeclareTextCompositeCommand{\.}{PU}{e}{\81\027} % edot
\DeclareTextCompositeCommand{\k}{PU}{E}{\81\030} % Eogonek
\DeclareTextCompositeCommand{\k}{PU}{e}{\81\031} % eogonek
\DeclareTextCompositeCommand{\v}{PU}{E}{\81\032} % Ecaron
\DeclareTextCompositeCommand{\v}{PU}{e}{\81\033} % ecaron
\DeclareTextCompositeCommand{\^}{PU}{G}{\81\034} % Gcircumflex
\DeclareTextCompositeCommand{\^}{PU}{g}{\81\035} % gcircumflex
\DeclareTextCompositeCommand{\u}{PU}{G}{\81\036} % Gbreve
\DeclareTextCompositeCommand{\u}{PU}{g}{\81\037} % gbreve
\DeclareTextCompositeCommand{\.}{PU}{G}{\81\040} % Gdot
\DeclareTextCompositeCommand{\.}{PU}{g}{\81\041} % gdot
\DeclareTextCompositeCommand{\c}{PU}{G}{\81\042} % Gcedilla
\DeclareTextCompositeCommand{\c}{PU}{g}{\81\043} % gcedilla
\DeclareTextCompositeCommand{\^}{PU}{H}{\81\044} % Hcircumflex
\DeclareTextCompositeCommand{\^}{PU}{h}{\81\045} % hcircumflex
\DeclareTextCommand{\textHslash}{PU}{\81\046} % Hslash
\DeclareTextCommand{\texthslash}{PU}{\81\047} % hslash
\DeclareTextCompositeCommand{\~}{PU}{I}{\81\050} % Itilde
\DeclareTextCompositeCommand{\~}{PU}{i}{\81\051} % itilde
\DeclareTextCompositeCommand{\~}{PU}{\i}{\81\051} % itilde
\DeclareTextCompositeCommand{\=}{PU}{I}{\81\052} % Imacron
\DeclareTextCompositeCommand{\=}{PU}{i}{\81\053} % imacron
\DeclareTextCompositeCommand{\=}{PU}{\i}{\81\053} % imacron
\DeclareTextCompositeCommand{\u}{PU}{I}{\81\054} % Ibreve
\DeclareTextCompositeCommand{\u}{PU}{i}{\81\055} % ibreve
\DeclareTextCompositeCommand{\u}{PU}{\i}{\81\055} % ibreve
\DeclareTextCompositeCommand{\k}{PU}{I}{\81\056} % Iogonek
\DeclareTextCompositeCommand{\k}{PU}{i}{\81\057} % iogonek
\DeclareTextCompositeCommand{\k}{PU}{\i}{\81\057} % iogonek
\DeclareTextCompositeCommand{\.}{PU}{I}{\81\060} % Idot
\DeclareTextCommand{\i}{PU}{\81\061} % idotless
% IJlig
% ijlig
\DeclareTextCompositeCommand{\^}{PU}{J}{\81\064} % Jcircumflex
\DeclareTextCompositeCommand{\^}{PU}{j}{\81\065} % jcircumflex
\DeclareTextCompositeCommand{\^}{PU}{\j}{\81\065} % jcircumflex
\DeclareTextCompositeCommand{\c}{PU}{K}{\81\066} % Kcedilla
\DeclareTextCompositeCommand{\c}{PU}{k}{\81\067} % kcedilla
% kgreen
\DeclareTextCompositeCommand{\'}{PU}{L}{\81\071} % Lacute
\DeclareTextCompositeCommand{\'}{PU}{l}{\81\072} % lacute
\DeclareTextCompositeCommand{\c}{PU}{L}{\81\073} % Lcedilla
\DeclareTextCompositeCommand{\c}{PU}{l}{\81\074} % lcedilla
\DeclareTextCompositeCommand{\v}{PU}{L}{\81\075} % Lcaron
\DeclareTextCompositeCommand{\v}{PU}{l}{\81\076} % lcaron
% L middle dot
% l middle dot
\DeclareTextCommand{\L}{PU}{\81\101} % Lslash
\DeclareTextCommand{\l}{PU}{\81\102} % lslash
\DeclareTextCompositeCommand{\'}{PU}{N}{\81\103} % Nacute
\DeclareTextCompositeCommand{\'}{PU}{n}{\81\104} % nacute
\DeclareTextCompositeCommand{\c}{PU}{N}{\81\105} % Ncedilla
\DeclareTextCompositeCommand{\c}{PU}{n}{\81\106} % ncedilla
\DeclareTextCompositeCommand{\v}{PU}{N}{\81\107} % Ncaron
\DeclareTextCompositeCommand{\v}{PU}{n}{\81\110} % ncaron
% n apostrophe
\DeclareTextCommand{\NG}{PU}{NG} % \81\112
\DeclareTextCommand{\ng}{PU}{ng} % \81\113
\DeclareTextCompositeCommand{\=}{PU}{O}{\81\114} % Omacron
\DeclareTextCompositeCommand{\=}{PU}{o}{\81\115} % omacron
\DeclareTextCompositeCommand{\u}{PU}{O}{\81\116} % Obreve
\DeclareTextCompositeCommand{\u}{PU}{o}{\81\117} % obreve
\DeclareTextCompositeCommand{\H}{PU}{O}{\81\120} % Odoubleacute
\DeclareTextCompositeCommand{\H}{PU}{o}{\81\121} % odoubleacute
%\DeclareTextCommand{\OE}{PU}{\81\122} % OE
%\DeclareTextCommand{\oe}{PU}{\81\123} % oe
\DeclareTextCompositeCommand{\'}{PU}{R}{\81\124} % Racute
\DeclareTextCompositeCommand{\'}{PU}{r}{\81\125} % racute
\DeclareTextCompositeCommand{\c}{PU}{R}{\81\126} % Rcedilla
\DeclareTextCompositeCommand{\c}{PU}{r}{\81\127} % rcedilla
\DeclareTextCompositeCommand{\v}{PU}{R}{\81\130} % Rcaron
\DeclareTextCompositeCommand{\v}{PU}{r}{\81\131} % rcaron
\DeclareTextCompositeCommand{\'}{PU}{S}{\81\132} % Sacute
\DeclareTextCompositeCommand{\'}{PU}{s}{\81\133} % sacute
\DeclareTextCompositeCommand{\^}{PU}{S}{\81\134} % Scircumflex
\DeclareTextCompositeCommand{\^}{PU}{s}{\81\135} % scircumflex
\DeclareTextCompositeCommand{\c}{PU}{S}{\81\136} % Scedilla
\DeclareTextCompositeCommand{\c}{PU}{s}{\81\137} % scedilla
\DeclareTextCompositeCommand{\v}{PU}{S}{\81\140} % Scaron
\DeclareTextCompositeCommand{\v}{PU}{s}{\81\141} % scaron
\DeclareTextCompositeCommand{\c}{PU}{T}{\81\142} % Tcedilla
\DeclareTextCompositeCommand{\c}{PU}{t}{\81\143} % tcedilla
\DeclareTextCompositeCommand{\v}{PU}{T}{\81\144} % Tcaron
\DeclareTextCompositeCommand{\v}{PU}{t}{\81\145} % tcaron
\DeclareTextCommand{\textTslash}{PU}{\81\146} % Tslash
\DeclareTextCommand{\texttslash}{PU}{\81\147} % tslash
\DeclareTextCompositeCommand{\~}{PU}{U}{\81\150} % Utilde
\DeclareTextCompositeCommand{\~}{PU}{u}{\81\151} % utilde
\DeclareTextCompositeCommand{\=}{PU}{U}{\81\152} % Umacron
\DeclareTextCompositeCommand{\=}{PU}{u}{\81\153} % umacron
\DeclareTextCompositeCommand{\u}{PU}{U}{\81\154} % Ubreve
\DeclareTextCompositeCommand{\u}{PU}{u}{\81\155} % ubreve
\DeclareTextCompositeCommand{\r}{PU}{U}{\81\156} % Uring
\DeclareTextCompositeCommand{\r}{PU}{u}{\81\157} % uring
\DeclareTextCompositeCommand{\H}{PU}{U}{\81\160} % Udoubleacute
\DeclareTextCompositeCommand{\H}{PU}{u}{\81\161} % udoubleacute
\DeclareTextCompositeCommand{\k}{PU}{U}{\81\162} % Uogonek
\DeclareTextCompositeCommand{\k}{PU}{u}{\81\163} % uogonek
\DeclareTextCompositeCommand{\^}{PU}{W}{\81\164} % Wcircumflex
\DeclareTextCompositeCommand{\^}{PU}{w}{\81\165} % wcircumflex
\DeclareTextCompositeCommand{\^}{PU}{Y}{\81\166} % Ycircumflex
\DeclareTextCompositeCommand{\^}{PU}{y}{\81\167} % ycircumflex
\DeclareTextCompositeCommand{\"}{PU}{Y}{\81\170} % Ydieresis
\DeclareTextCommand{\IJ}{PU}{\81\170}
\DeclareTextCompositeCommand{\'}{PU}{Z}{\81\171} % Zacute
\DeclareTextCompositeCommand{\'}{PU}{z}{\81\172} % zacute
\DeclareTextCompositeCommand{\.}{PU}{Z}{\81\173} % Zdot
\DeclareTextCompositeCommand{\.}{PU}{z}{\81\174} % zdot
\DeclareTextCompositeCommand{\v}{PU}{Z}{\81\175} % Zcaron
\DeclareTextCompositeCommand{\v}{PU}{z}{\81\176} % zcaron
%    \end{macrocode}
%
% \subsubsection{Latin Extended-B}
%    \begin{macrocode}
\DeclareTextCompositeCommand{\v}{PU}{A}{\81\315} % Acaron
\DeclareTextCompositeCommand{\v}{PU}{a}{\81\316} % acaron
\DeclareTextCompositeCommand{\v}{PU}{I}{\81\317} % Icaron
\DeclareTextCompositeCommand{\v}{PU}{i}{\81\320} % icaron
\DeclareTextCompositeCommand{\v}{PU}{\i}{\81\320} % icaron
\DeclareTextCompositeCommand{\v}{PU}{O}{\81\321} % Ocaron
\DeclareTextCompositeCommand{\v}{PU}{o}{\81\322} % ocaron
\DeclareTextCompositeCommand{\v}{PU}{U}{\81\323} % Ucaron
\DeclareTextCompositeCommand{\v}{PU}{u}{\81\324} % ucaron
\DeclareTextCommand{\textGslash}{PU}{\81\344} % Gslash
\DeclareTextCommand{\textgslash}{PU}{\81\345} % gslash
\DeclareTextCompositeCommand{\v}{PU}{G}{\81\346} % Gcaron
\DeclareTextCompositeCommand{\v}{PU}{g}{\81\347} % gcaron
\DeclareTextCompositeCommand{\v}{PU}{K}{\81\350} % Kcaron
\DeclareTextCompositeCommand{\v}{PU}{k}{\81\351} % kcaron
\DeclareTextCompositeCommand{\k}{PU}{O}{\81\352} % Oogonek
\DeclareTextCompositeCommand{\k}{PU}{o}{\81\353} % oogonek
%    \end{macrocode}
%
% \subsubsection{Greek}
%    \begin{macrocode}
\DeclareTextCompositeCommand{\'}{PU}{\textAlpha}{\83\206}
\DeclareTextCompositeCommand{\'}{PU}{\textEpsilon}{\83\210}
\DeclareTextCompositeCommand{\'}{PU}{\textEta}{\83\211}
\DeclareTextCompositeCommand{\'}{PU}{\textIota}{\83\212}
\DeclareTextCompositeCommand{\'}{PU}{\textOmicron}{\83\214}
\DeclareTextCompositeCommand{\'}{PU}{\textUpsilon}{\83\216}
\DeclareTextCompositeCommand{\'}{PU}{\textOmega}{\83\217}
\DeclareTextCommand{\textIotadieresis}{PU}{\83\252}
\DeclareTextCompositeCommand{\'}{PU}{\textIotadieresis}{\83\220}
\DeclareTextCommand{\textAlpha}{PU}{\83\221}
\DeclareTextCommand{\textBeta}{PU}{\83\222}
\DeclareTextCommand{\textGamma}{PU}{\83\223}
\DeclareTextCommand{\textDelta}{PU}{\83\224}
\DeclareTextCommand{\textEpsilon}{PU}{\83\225}
\DeclareTextCommand{\textZeta}{PU}{\83\226}
\DeclareTextCommand{\textEta}{PU}{\83\227}
\DeclareTextCommand{\textTheta}{PU}{\83\230}
\DeclareTextCommand{\textIota}{PU}{\83\231}
\DeclareTextCommand{\textKappa}{PU}{\83\232}
\DeclareTextCommand{\textLambda}{PU}{\83\233}
\DeclareTextCommand{\textMu}{PU}{\83\234}
\DeclareTextCommand{\textNu}{PU}{\83\235}
\DeclareTextCommand{\textXi}{PU}{\83\236}
\DeclareTextCommand{\textOmicron}{PU}{\83\237}
\DeclareTextCommand{\textPi}{PU}{\83\240}
\DeclareTextCommand{\textRho}{PU}{\83\241}
\DeclareTextCommand{\textSigma}{PU}{\83\243}
\DeclareTextCommand{\textTau}{PU}{\83\244}
\DeclareTextCommand{\textUpsilon}{PU}{\83\245}
\DeclareTextCommand{\textPhi}{PU}{\83\246}
\DeclareTextCommand{\textChi}{PU}{\83\247}
\DeclareTextCommand{\textPsi}{PU}{\83\250}
\DeclareTextCommand{\textOmega}{PU}{\83\251}
\DeclareTextCompositeCommand{\"}{PU}{\textIota}{\83\252}
\DeclareTextCompositeCommand{\"}{PU}{\textUpsilon}{\83\253}
\DeclareTextCompositeCommand{\'}{PU}{\textalpha}{\83\254}
\DeclareTextCompositeCommand{\'}{PU}{\textepsilon}{\83\255}
\DeclareTextCompositeCommand{\'}{PU}{\texteta}{\83\256}
\DeclareTextCompositeCommand{\'}{PU}{\textiota}{\83\257}
\DeclareTextCommand{\textupsilonacute}{PU}{\83\315}
\DeclareTextCompositeCommand{\"}{PU}{\textupsilonacute}{\83\260}
\DeclareTextCommand{\textalpha}{PU}{\83\261}
\DeclareTextCommand{\textbeta}{PU}{\83\262}
\DeclareTextCommand{\textgamma}{PU}{\83\263}
\DeclareTextCommand{\textdelta}{PU}{\83\264}
\DeclareTextCommand{\textepsilon}{PU}{\83\265}
\DeclareTextCommand{\textzeta}{PU}{\83\266}
\DeclareTextCommand{\texteta}{PU}{\83\267}
\DeclareTextCommand{\texttheta}{PU}{\83\270}
\DeclareTextCommand{\textiota}{PU}{\83\271}
\DeclareTextCommand{\textkappa}{PU}{\83\272}
\DeclareTextCommand{\textlambda}{PU}{\83\273}
\DeclareTextCommand{\textmu}{PU}{\83\274}
\DeclareTextCommand{\textnu}{PU}{\83\275}
\DeclareTextCommand{\textxi}{PU}{\83\276}
\DeclareTextCommand{\textomicron}{PU}{\83\277}
\DeclareTextCommand{\textpi}{PU}{\83\300}
\DeclareTextCommand{\textrho}{PU}{\83\301}
\DeclareTextCommand{\textvarsigma}{PU}{\83\302}
\DeclareTextCommand{\textsigma}{PU}{\83\303}
\DeclareTextCommand{\texttau}{PU}{\83\304}
\DeclareTextCommand{\textupsilon}{PU}{\83\305}
\DeclareTextCommand{\textphi}{PU}{\83\306}
\DeclareTextCommand{\textchi}{PU}{\83\307}
\DeclareTextCommand{\textpsi}{PU}{\83\310}
\DeclareTextCommand{\textomega}{PU}{\83\311}
\DeclareTextCompositeCommand{\"}{PU}{\textiota}{\83\312}
\DeclareTextCompositeCommand{\"}{PU}{\textupsilon}{\83\313}
\DeclareTextCompositeCommand{\'}{PU}{\textomicron}{\83\314}
\DeclareTextCompositeCommand{\'}{PU}{\textupsilon}{\83\315}
\DeclareTextCompositeCommand{\'}{PU}{\textomega}{\83\316}
%\DeclareTextCommand{\textvartheta}{PU}{\83\321}
%\DeclareTextCommand{\textvarphi}{PU}{\83\325}
%\DeclareTextCommand{\textvarpi}{PU}{\83\326}
%\DeclareTextCommand{\textdigamma}{PU}{\83\334}
%\DeclareTextCommand{\textvarkappa}{PU}{\83\360}
%\DeclareTextCommand{\textvarrho}{PU}{\83\361}
%    \end{macrocode}
%
% \subsubsection{Cyrillic}
%    Thanks to Vladimir Volovich (\Email{vvv@vvv.vsu.ru}) for
%    the help with the Cyrillic glyph names.
%    \begin{macrocode}
\DeclareTextCommand{\CYRYO}{PU}{\84\001}% IO
\DeclareTextCompositeCommand{\"}{PU}{\CYRE}{\84\001}%
\DeclareTextCommand{\CYRDJE}{PU}{\84\002}% DJE
\DeclareTextCompositeCommand{\'}{PU}{\CYRG}{\84\003}% GJE
\DeclareTextCommand{\CYRIE}{PU}{\84\004}% ukrainian IE
\DeclareTextCommand{\CYRDZE}{PU}{\84\005}% DZE
\DeclareTextCommand{\CYRII}{PU}{\84\006}% byelorussian-ukrainian I
\DeclareTextCommand{\CYRYI}{PU}{\84\007}% YI
\DeclareTextCommand{\CYRJE}{PU}{\84\010}% JE
\DeclareTextCommand{\CYRLJE}{PU}{\84\011}% LJE
\DeclareTextCommand{\CYRNJE}{PU}{\84\012}% NJE
\DeclareTextCommand{\CYRTSHE}{PU}{\84\013}% TSHE
\DeclareTextCompositeCommand{\'}{PU}{\CYRK}{\84\014}% KJE
\DeclareTextCommand{\CYRUSHRT}{PU}{\84\016}% short U
\DeclareTextCommand{\CYRDZHE}{PU}{\84\017}% DZHE
\DeclareTextCommand{\CYRA}{PU}{\84\020}% A
\DeclareTextCommand{\CYRB}{PU}{\84\021}% BE
\DeclareTextCommand{\CYRV}{PU}{\84\022}% VE
\DeclareTextCommand{\CYRG}{PU}{\84\023}% GHE
\DeclareTextCommand{\CYRD}{PU}{\84\024}% DE
\DeclareTextCommand{\CYRE}{PU}{\84\025}% IE
\DeclareTextCommand{\CYRZH}{PU}{\84\026}% ZHE
\DeclareTextCommand{\CYRZ}{PU}{\84\027}% ZE
\DeclareTextCommand{\CYRI}{PU}{\84\030}% I
\DeclareTextCommand{\CYRISHRT}{PU}{\84\031}% short I
\DeclareTextCompositeCommand{\U}{PU}{\CYRI}{\84\031}%
\DeclareTextCommand{\CYRK}{PU}{\84\032}% KA
\DeclareTextCommand{\CYRL}{PU}{\84\033}% EL
\DeclareTextCommand{\CYRM}{PU}{\84\034}% EM
\DeclareTextCommand{\CYRN}{PU}{\84\035}% EN
\DeclareTextCommand{\CYRO}{PU}{\84\036}% O
\DeclareTextCommand{\CYRP}{PU}{\84\037}% PE
\DeclareTextCommand{\CYRR}{PU}{\84\040}% ER
\DeclareTextCommand{\CYRS}{PU}{\84\041}% ES
\DeclareTextCommand{\CYRT}{PU}{\84\042}% TE
\DeclareTextCommand{\CYRU}{PU}{\84\043}% U
\DeclareTextCommand{\CYRF}{PU}{\84\044}% EF
\DeclareTextCommand{\CYRH}{PU}{\84\045}% HA
\DeclareTextCommand{\CYRC}{PU}{\84\046}% TSE
\DeclareTextCommand{\CYRCH}{PU}{\84\047}% CHE
\DeclareTextCommand{\CYRSH}{PU}{\84\050}% SHA
\DeclareTextCommand{\CYRSHCH}{PU}{\84\051}% SHCHA
\DeclareTextCommand{\CYRHRDSN}{PU}{\84\052}% HARD SIGN
\DeclareTextCommand{\CYRERY}{PU}{\84\053}% YERU
\DeclareTextCommand{\CYRSFTSN}{PU}{\84\054}% SOFT SIGN
\DeclareTextCommand{\CYREREV}{PU}{\84\055}% E
\DeclareTextCommand{\CYRYU}{PU}{\84\056}% YU
\DeclareTextCommand{\CYRYA}{PU}{\84\057}% YA
\DeclareTextCommand{\cyra}{PU}{\84\060}% a
\DeclareTextCommand{\cyrb}{PU}{\84\061}% be
\DeclareTextCommand{\cyrv}{PU}{\84\062}% ve
\DeclareTextCommand{\cyrg}{PU}{\84\063}% ghe
\DeclareTextCommand{\cyrd}{PU}{\84\064}% de
\DeclareTextCommand{\cyre}{PU}{\84\065}% ie
\DeclareTextCommand{\cyrzh}{PU}{\84\066}% zhe
\DeclareTextCommand{\cyrz}{PU}{\84\067}% ze
\DeclareTextCommand{\cyri}{PU}{\84\070}% i
\DeclareTextCommand{\cyrishrt}{PU}{\84\071}% short i
\DeclareTextCompositeCommand{\U}{PU}{\cyri}{\84\071}%
\DeclareTextCommand{\cyrk}{PU}{\84\072}% ka
\DeclareTextCommand{\cyrl}{PU}{\84\073}% el
\DeclareTextCommand{\cyrm}{PU}{\84\074}% em
\DeclareTextCommand{\cyrn}{PU}{\84\075}% en
\DeclareTextCommand{\cyro}{PU}{\84\076}% o
\DeclareTextCommand{\cyrp}{PU}{\84\077}% pe
\DeclareTextCommand{\cyrr}{PU}{\84\100}% er
\DeclareTextCommand{\cyrs}{PU}{\84\101}% es
\DeclareTextCommand{\cyrt}{PU}{\84\102}% te
\DeclareTextCommand{\cyru}{PU}{\84\103}% u
\DeclareTextCommand{\cyrf}{PU}{\84\104}% ef
\DeclareTextCommand{\cyrh}{PU}{\84\105}% ha
\DeclareTextCommand{\cyrc}{PU}{\84\106}% tse
\DeclareTextCommand{\cyrch}{PU}{\84\107}% che
\DeclareTextCommand{\cyrsh}{PU}{\84\110}% sha
\DeclareTextCommand{\cyrshch}{PU}{\84\111}% shcha
\DeclareTextCommand{\cyrhrdsn}{PU}{\84\112}% hard sign
\DeclareTextCommand{\cyrery}{PU}{\84\113}% yeru
\DeclareTextCommand{\cyrsftsn}{PU}{\84\114}% soft sign
\DeclareTextCommand{\cyrerev}{PU}{\84\115}% e
\DeclareTextCommand{\cyryu}{PU}{\84\116}% yu
\DeclareTextCommand{\cyrya}{PU}{\84\117}% ya
\DeclareTextCommand{\cyryo}{PU}{\84\121}% io
\DeclareTextCompositeCommand{\"}{PU}{\cyre}{\84\121}%
\DeclareTextCommand{\cyrdje}{PU}{\84\122}% dje
\DeclareTextCompositeCommand{\'}{PU}{\cyrg}{\84\123}% gje
\DeclareTextCommand{\cyrie}{PU}{\84\124}% ukrainian ie
\DeclareTextCommand{\cyrdze}{PU}{\84\125}% dze
\DeclareTextCommand{\cyrii}{PU}{\84\126}% byelorussian-ukrainian i
\DeclareTextCommand{\cyryi}{PU}{\84\127}% yi
\DeclareTextCommand{\cyrje}{PU}{\84\130}% je
\DeclareTextCommand{\cyrlje}{PU}{\84\131}% lje
\DeclareTextCommand{\cyrnje}{PU}{\84\132}% nje
\DeclareTextCommand{\cyrtshe}{PU}{\84\133}% tshe
\DeclareTextCompositeCommand{\'}{PU}{\cyrk}{\84\134}% kje
\DeclareTextCommand{\cyrushrt}{PU}{\84\136}% short u
\DeclareTextCommand{\cyrdzhe}{PU}{\84\137}% dzhe
\DeclareTextCommand{\CYROMEGA}{PU}{\84\140}% OMEGA
\DeclareTextCommand{\cyromega}{PU}{\84\141}% omega
\DeclareTextCommand{\CYRYAT}{PU}{\84\142}% YAT
\DeclareTextCommand{\cyryat}{PU}{\84\143}% yat
\DeclareTextCommand{\CYRIOTE}{PU}{\84\144}% iotified E
\DeclareTextCommand{\cyriote}{PU}{\84\145}% iotified e
\DeclareTextCommand{\CYRLYUS}{PU}{\84\146}% little YUS
\DeclareTextCommand{\cyrlyus}{PU}{\84\147}% little yus
\DeclareTextCommand{\CYRIOTLYUS}{PU}{\84\150}% iotified little YUS
\DeclareTextCommand{\cyriotlyus}{PU}{\84\151}% iotified little yus
\DeclareTextCommand{\CYRBYUS}{PU}{\84\152}% big YUS
\DeclareTextCommand{\cyrbyus}{PU}{\84\153}% big yus
\DeclareTextCommand{\CYRIOTBYUS}{PU}{\84\154}% iotified big YUS
\DeclareTextCommand{\cyriotbyus}{PU}{\84\155}% iotified big yus
\DeclareTextCommand{\CYRKSI}{PU}{\84\156}% KSI
\DeclareTextCommand{\cyrksi}{PU}{\84\157}% ksi
\DeclareTextCommand{\CYRPSI}{PU}{\84\160}% PSI
\DeclareTextCommand{\cyrpsi}{PU}{\84\161}% psi
\DeclareTextCommand{\CYRFITA}{PU}{\84\162}% FITA
\DeclareTextCommand{\cyrfita}{PU}{\84\163}% fita
\DeclareTextCommand{\CYRIZH}{PU}{\84\164}% IZHITSA
\DeclareTextCommand{\cyrizh}{PU}{\84\165}% izhitsa
\DeclareTextCompositeCommand{\C}{PU}{\CYRIZH}{\84\166}% IZHITSA double grave
\DeclareTextCompositeCommand{\C}{PU}{\cyrizh}{\84\167}% izhitsa double grave
\DeclareTextCommand{\CYRUK}{PU}{\84\170}% UK
\DeclareTextCommand{\cyruk}{PU}{\84\171}% uk
\DeclareTextCommand{\CYROMEGARND}{PU}{\84\172}% round OMEGA
\DeclareTextCommand{\cyromegarnd}{PU}{\84\173}% round omega
\DeclareTextCommand{\CYROMEGATITLO}{PU}{\84\174}% OMEGA titlo
\DeclareTextCommand{\cyromegatitlo}{PU}{\84\175}% omega titlo
\DeclareTextCommand{\CYROT}{PU}{\84\176}% OT
\DeclareTextCommand{\cyrot}{PU}{\84\177}% ot
\DeclareTextCommand{\CYRKOPPA}{PU}{\84\200}% KOPPA
\DeclareTextCommand{\cyrkoppa}{PU}{\84\201}% koppa
\DeclareTextCommand{\cyrthousands}{PU}{\84\202}% thousands sign
%\DeclareTextCommand{\COMBINING TITLO}{PU}{\84\203}% COMBINING TITLO
%\DeclareTextCommand{\COMBINING PALATALIZATION}{PU}{\84\204}% COMBINING PALATALIZATION
%\DeclareTextCommand{\COMBINING DASIA PNEUMATA}{PU}{\84\205}% COMBINING DASIA PNEUMATA
%\DeclareTextCommand{\COMBINING PSILI PNEUMATA}{PU}{\84\206}% COMBINING PSILI PNEUMATA
\DeclareTextCommand{\CYRGUP}{PU}{\84\220}% GHE upturn
\DeclareTextCommand{\cyrgup}{PU}{\84\221}% ghe upturn
\DeclareTextCommand{\CYRGHCRS}{PU}{\84\222}% GHE stroke
\DeclareTextCommand{\cyrghcrs}{PU}{\84\223}% ghe stroke
\DeclareTextCommand{\CYRGHK}{PU}{\84\224}% GHE middle hook
\DeclareTextCommand{\cyrghk}{PU}{\84\225}% ghe middle hook
\DeclareTextCommand{\CYRZHDSC}{PU}{\84\226}% ZHE descender
\DeclareTextCommand{\cyrzhdsc}{PU}{\84\227}% zhe descender
\DeclareTextCommand{\CYRZDSC}{PU}{\84\230}% ZE descender
\DeclareTextCommand{\cyrzdsc}{PU}{\84\231}% ze descender
\DeclareTextCommand{\CYRKDSC}{PU}{\84\232}% KA descender
\DeclareTextCommand{\cyrkdsc}{PU}{\84\233}% ka descender
\DeclareTextCommand{\CYRKVCRS}{PU}{\84\234}% KA vertical stroke
\DeclareTextCommand{\cyrkvcrs}{PU}{\84\235}% ka vertical stroke
\DeclareTextCommand{\CYRKHCRS}{PU}{\84\236}% KA stroke
\DeclareTextCommand{\cyrkhcrs}{PU}{\84\237}% ka stroke
\DeclareTextCommand{\CYRKBEAK}{PU}{\84\240}% bashkir KA
\DeclareTextCommand{\cyrkbeak}{PU}{\84\241}% bashkir ka
\DeclareTextCommand{\CYRNDSC}{PU}{\84\242}% EN descender
\DeclareTextCommand{\cyrndsc}{PU}{\84\243}% en descender
\DeclareTextCommand{\CYRNG}{PU}{\84\244}% ligature EN GHE
\DeclareTextCommand{\cyrng}{PU}{\84\245}% ligature en ghe
\DeclareTextCommand{\CYRPHK}{PU}{\84\246}% PE middle hook
\DeclareTextCommand{\cyrphk}{PU}{\84\247}% pe middle hook
\DeclareTextCommand{\CYRABHHA}{PU}{\84\250}% abkhasian HA
\DeclareTextCommand{\cyrabhha}{PU}{\84\251}% abkhasian ha
\DeclareTextCommand{\CYRSDSC}{PU}{\84\252}% ES descender
\DeclareTextCommand{\cyrsdsc}{PU}{\84\253}% es descender
\DeclareTextCommand{\CYRTDSC}{PU}{\84\254}% TE descender
\DeclareTextCommand{\cyrtdsc}{PU}{\84\255}% te descender
\DeclareTextCommand{\CYRY}{PU}{\84\256}% straight U
\DeclareTextCommand{\cyry}{PU}{\84\257}% straight u
\DeclareTextCommand{\CYRYHCRS}{PU}{\84\260}% straight U stroke
\DeclareTextCommand{\cyryhcrs}{PU}{\84\261}% straight u stroke
\DeclareTextCommand{\CYRHDSC}{PU}{\84\262}% HA descender
\DeclareTextCommand{\cyrhdsc}{PU}{\84\263}% ha descender
\DeclareTextCommand{\CYRTETSE}{PU}{\84\264}% ligature TE TSE
\DeclareTextCommand{\cyrtetse}{PU}{\84\265}% ligature te tse
\DeclareTextCommand{\CYRCHRDSC}{PU}{\84\266}% CHE descender
\DeclareTextCommand{\cyrchrdsc}{PU}{\84\267}% che descender
\DeclareTextCommand{\CYRCHVCRS}{PU}{\84\270}% CHE vertical stroke
\DeclareTextCommand{\cyrchvcrs}{PU}{\84\271}% che vertical stroke
\DeclareTextCommand{\CYRSHHA}{PU}{\84\272}% SHHA
\DeclareTextCommand{\cyrshha}{PU}{\84\273}% shha
\DeclareTextCommand{\CYRABHCH}{PU}{\84\274}% abkhasian CHE
\DeclareTextCommand{\cyrabhch}{PU}{\84\275}% abkhasian che
\DeclareTextCommand{\CYRABHCHDSC}{PU}{\84\276}% abkhasian CHE descender
\DeclareTextCommand{\cyrabhchdsc}{PU}{\84\277}% abkhasian che descender
\DeclareTextCommand{\CYRpalochka}{PU}{\84\300}% palochka
\DeclareTextCompositeCommand{\U}{PU}{\CYRZH}{\84\301}% ZHE breve
\DeclareTextCompositeCommand{\U}{PU}{\cyrzh}{\84\302}% zhe breve
\DeclareTextCommand{\CYRKHK}{PU}{\84\303}% KA hook
\DeclareTextCommand{\cyrkhk}{PU}{\84\304}% ka hook
\DeclareTextCommand{\CYRNHK}{PU}{\84\307}% EN hook
\DeclareTextCommand{\cyrnhk}{PU}{\84\310}% en hook
\DeclareTextCommand{\CYRCHLDSC}{PU}{\84\313}% khakassian CHE
\DeclareTextCommand{\cyrchldsc}{PU}{\84\314}% khakassian che
\DeclareTextCompositeCommand{\U}{PU}{\CYRA}{\84\320}% A breve
\DeclareTextCompositeCommand{\U}{PU}{\cyra}{\84\321}% a breve
\DeclareTextCompositeCommand{\"}{PU}{\CYRA}{\84\322}% A diaeresis
\DeclareTextCompositeCommand{\"}{PU}{\cyra}{\84\323}% a diaeresis
\DeclareTextCommand{\CYRAE}{PU}{\84\324}% ligature A IE
\DeclareTextCommand{\cyrae}{PU}{\84\325}% ligature a ie
\DeclareTextCompositeCommand{\U}{PU}{\CYRE}{\84\326}% IE breve
\DeclareTextCompositeCommand{\U}{PU}{\cyre}{\84\327}% ie breve
\DeclareTextCommand{\CYRSCHWA}{PU}{\84\330}% SCHWA
\DeclareTextCommand{\cyrschwa}{PU}{\84\331}% schwa
\DeclareTextCompositeCommand{\"}{PU}{\CYRSCHWA}{\84\332}% SCHWA diaeresis
\DeclareTextCompositeCommand{\"}{PU}{\cyrschwa}{\84\333}% schwa diaeresis
\DeclareTextCompositeCommand{\"}{PU}{\CYRZH}{\84\334}% ZHE diaeresis
\DeclareTextCompositeCommand{\"}{PU}{\cyrzh}{\84\335}% zhe diaeresis
\DeclareTextCompositeCommand{\"}{PU}{\CYRZ}{\84\336}% ZE diaeresis
\DeclareTextCompositeCommand{\"}{PU}{\cyrz}{\84\337}% ze diaeresis
\DeclareTextCommand{\CYRABHDZE}{PU}{\84\340}% abkhasian DZE
\DeclareTextCommand{\cyrabhdze}{PU}{\84\341}% abkhasian dze
\DeclareTextCompositeCommand{\=}{PU}{\CYRI}{\84\342}% I macron
\DeclareTextCompositeCommand{\=}{PU}{\cyri}{\84\343}% i macron
\DeclareTextCompositeCommand{\"}{PU}{\CYRI}{\84\344}% I diaeresis
\DeclareTextCompositeCommand{\"}{PU}{\cyri}{\84\345}% i diaeresis
\DeclareTextCompositeCommand{\"}{PU}{\CYRO}{\84\346}% O diaeresis
\DeclareTextCompositeCommand{\"}{PU}{\cyro}{\84\347}% o diaeresis
\DeclareTextCommand{\CYROTLD}{PU}{\84\350}% barred O
\DeclareTextCommand{\cyrotld}{PU}{\84\351}% barred o
\DeclareTextCompositeCommand{\"}{PU}{\CYROTLD}{\84\352}% barred O diaeresis
\DeclareTextCompositeCommand{\"}{PU}{\cyrotld}{\84\353}% barred o diaeresis
\DeclareTextCompositeCommand{\=}{PU}{\CYRU}{\84\356}% U macron
\DeclareTextCompositeCommand{\=}{PU}{\cyru}{\84\357}% u macron
\DeclareTextCompositeCommand{\"}{PU}{\CYRU}{\84\360}% U diaeresis
\DeclareTextCompositeCommand{\"}{PU}{\cyru}{\84\361}% u diaeresis
\DeclareTextCompositeCommand{\H}{PU}{\CYRU}{\84\362}% U double acute
\DeclareTextCompositeCommand{\H}{PU}{\cyru}{\84\363}% u double acute
\DeclareTextCompositeCommand{\"}{PU}{\CYRCH}{\84\364}% CHE diaeresis
\DeclareTextCompositeCommand{\"}{PU}{\cyrch}{\84\365}% che diaeresis
\DeclareTextCompositeCommand{\"}{PU}{\CYRERY}{\84\370}% YERU diaeresis
\DeclareTextCompositeCommand{\"}{PU}{\cyrery}{\84\371}% yeru diaeresis
%    \end{macrocode}
%    \begin{macrocode}
\DeclareTextCommand{\textnumero}{PU}{\9041\026}
%\DeclareTextCommand{\cyrlangle}{PU}{\9043\051}
%\DeclareTextCommand{\cyrrangle}{PU}{\9043\052}
%    \end{macrocode}
%
%    \begin{macrocode}
%</puenc>
%    \end{macrocode}
%
% \section{End of file hycheck.tex}
%
%    \begin{macrocode}
%<*check>
\typeout{}
\begin{document}
\end{document}
%</check>
%    \end{macrocode}
%
% \Finale
%
\endinput
